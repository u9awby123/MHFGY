\pagebreak
% A feladat címe automatikus számozás nélkül
\section*{MH23. feladat: Rankine--Clausius-körfolyamat számítása}

% Hozzáadás a tartalomjegyzékhez azonos címmel
\addcontentsline{toc}{section}{MH23. feladat: Rankine--Clausius-körfolyamat számítása}

% Táblázat a szerző adataival
\begin{tabular}{ | p{2cm} | p{14cm} | } 
	\hline
	Szerző & Csendes Nikoletta, AUFM6A \\ 
	\hline
	Szak & Mechatronikai mérnöki alapszak \\ 
	\hline
	Félév & 2019/2020 II. (tavaszi) félév \\ 
	\hline
\end{tabular}
\vspace{0.5cm}

% A feladat szövege
\noindent A Rankine-Clausius körfolyamat kondenzátorában 
$\dot{Q}_{K} = \SI{8.5e4}{\kilo\joule\per\second}$ 
hőt kell elvonnunk. Az átáramló közvetítőközeg mennyisége 
$\dot{m} = \SI{40}{\kilogram\per\second}$
 . A turbinában áramló gőz állapotjelzői: \\
 
 \begin{equation*}
 	p_1 = \SI{50}{\bar}, 
 	\quad 
 	T_H = \SI{500}{\celsius}, 
 	\quad
 	h_1 = \SI{3434}{\kilo\joule\per\kilogram},
 	\quad
 	s_1 = \SI{6.98}{\kilo\joule\per\kilogram\kelvin}.
 \end{equation*}

\noindent A kondenzátornyomás ${p}_{K} = \SI{0.05}{\bar}$, amelyhez a telítési állapotjelzők:\\

\begin{equation*}
	h_2' = \SI{136}{\kilo\joule\per\kilogram}, 
	\quad 
	h_2'' = \SI{2560}{\kilo\joule\per\kilogram}, 
	\quad
	s_2' = \SI{0.47}{\kilo\joule\per\kilogram\kelvin},
	\quad
	s_2'' = \SI{8.4}{\kilo\joule\per\kilogram\kelvin}.
\end{equation*}

\noindent A körfolyamatban az expanzió irreverzibilis, egyéb irreverzibilitás nincs. Rajzolja le a körfolyamatot $T-s$ diagramban és a tápszivattyúzási munka figyelmen kívül hagyásával számítsa ki az irreverzibilis körfolyamat termikus hatásfokát, a termodinamikus hatásfokot és a turbina teljesítményét! \\

\noindent ($\eta_{TD} = \SI{89.7}{\percent}$, \quad  $\eta_{T} = \SI{35.6}{\percent}$, \quad $P_{T} = \SI{46920}{\kilo\watt}$)

\noindent\hrulefill

% A feladat megoldása
\subsubsection{A termikus hatásfok kiszámítása:} 
A termikus hatásfok kiszámításához először ki kell számítani a bevezetett és az elvont hőmennyiségeket.
\noindent A kondenzátorban elvont hőmennyiség:

\begin{equation}
	q_{EL} =\dfrac {\dot{Q}_K}{\dot{m}} = \SI{2125}{\kilo\joule\per\kilogram}
\end{equation}

\noindent A bevezetett hőmennyiség:

\begin{equation}
	q_{BE} = h_1 - h_2'=\SI {3298}{\kilo\joule\per\kilogram}	
\end{equation}

\noindent Ezek ismeretében kiszámítható a technikai munka:

\begin{equation}
	w_{t,ad,irrev} = q_{BE} - q_{EL} =\SI{1173}{\kilo\joule\per\kilogram}
\end{equation}

\noindent A technikai munka ismeretében már kiszámítható a termikus hatásfok:

\begin{equation}
	\eta_{T} = \dfrac {w_t}{q_{BE}} = \SI {35.56}{\percent}
\end{equation}

\subsubsection{A turbina teljesítményének kiszámítása:} 

\begin{equation}
	P_T = w_t \cdot \dot{m} = \SI{46920}{\kilo\watt}
\end{equation}

\subsubsection{A termodinamikus hatásfok kiszámítása:} 
Reverzibilis esetben $s_2=s_1$, így az entrópiák ismeretében az alábbi egyenletből kiszámítható a fajlagos gőztartalom:

\begin{equation}
	s_2 = (1-x) \cdot s_2' + x \cdot s_2''
\end{equation}

\begin{equation}
	\SI {6.98}{\kilo\joule\per\kilogram\kelvin} = (1-x)\cdot \SI{0.47}{\kilo\joule\per\kilogram\kelvin} + x \cdot \SI{8.4}{\kilo\joule\per\kilogram\kelvin}
	\quad 
	\Rightarrow
	\quad
	x = \SI{0.82}{}
\end{equation}

\noindent A fajlagos gőztartalom ismeretében kiszámítható reverzibilis esethez tartozó entalpia:

\begin{equation}
	h_2 =  (1-x)\cdot h_2' + x\cdot h_2'' = (1-0,82)\cdot \SI{136}{\kilo\joule\per\kilogram} + 0,82 \cdot \SI{2560}{\kilo\joule\per\kilogram} = \SI{2123.68}{\kilo\joule\per\kilogram}	
\end{equation}

\noindent Innét a technikai munka reverzibilis esetben:

\begin{equation}
	w_{t,ad,rev} = h_1- h_2 = \SI{1310.32}{\kilo\joule\per\kilogram}
\end{equation}

\noindent Ezek alapján már számítható a termodinamikus hatásfok:

\begin{equation}
	\eta_{TD} = \dfrac {w_{t,ad,irrev}}{w_{t,ad,rev}} = \dfrac {1173}{1310,32} = \SI{89.54}{\percent}
\end{equation}

\subsubsection{Fajlagos gőztartalom a végállapotban:} 
A $T-s$ diagram elkészítéséhez ki kell számítani, mennyi a fajlagos gőztartalom a végállapotban. Ehhez szükség van az entalpia értékére az irreverzibilis eset végállapotában:

\begin{equation}
w_{t,ad,irrev} = h_1- h_2^* = \SI{1173}{\kilo\joule\per\kilogram}
\quad 
\Rightarrow
\quad
h_2^*=\SI{2261}{\kilo\joule\per\kilogram}
\end{equation}

\noindent A fajlagos gőztartalom kiszámítása:

\begin{equation}
x^* = x + \dfrac {1-x}{h_2''-h_2}(h_2^*-h_2) =\SI{0.876}{}
\end{equation}

\subsubsection{A körfolyamat $T-s$ diagramja:} 
A $T-s$ diagramban a $2a$ pont a reverzibilis, $2b$ pont pedig az irreverzibilis végállapotot jelöli. A fojtásos adiabatikus kiterjedés felbontható egy reverzibilis és egy tiszta fojtásos szakaszra. A reverzibilis szakasz az $1$ és $2c$ pontok között, a tiszta fojtásos szakasz pedig a pirossal jelölt izentalpikus vonal $2c$ és $2b$ pontjai között található.    

\begin{figure}[h]
	\centering
	\begin{tikzpicture}
	
	\begin{axis}[
	axis lines = middle,
	axis line style = {->},
	xlabel={$s \left(\si{\kilo\joule\per\kilogram\kelvin}\right)$},
	ylabel={$T \left(\si{\celsius}\right)$},
	xlabel style={
		at=(current axis.right of origin), 
		anchor=north east
	},
	ylabel style={
		at=(current axis.above origin), 
		anchor=north east
	},
	ytick={32.87, 263.94, 500},
	yticklabels={{$32,87$}, {$263,94$}, {$500$}},
	xtick={0.47, 2.92, 5.97, 6.98, 7.413, 8.4},
	xticklabels={{$0,47$}, {$2,92$}, {$5,97$}, {$6,98$}, {$7,41$}, {$8,4$}},
	x tick label style={rotate=90,anchor=east},
	ytick style={draw=none},
	xtick style={draw=none},
	width=15cm, height=10cm, xmin=0, xmax=10, ymin=0, ymax=600,
	]
	
	%adatsorok
	\addplot [] table {./aufm6a/T-s.txt};
	\addplot [thick, dashed] table {./aufm6a/p005.txt};
	\addplot [ultra thick] table {./aufm6a/p50.txt};
	\addplot [ultra thick, red] table {./aufm6a/h6.txt};
	\addplot [ultra thick] table {./aufm6a/wti.txt};
	
	%körfolyamat pontjai	
	\draw[fill] (axis cs:6.98, 500) circle [radius = 1mm];
	\draw[fill] (axis cs:5.97, 263.94) circle [radius = 1mm];
	\draw[fill] (axis cs:2.92, 263.94) circle [radius = 1mm];
	\draw[fill] (axis cs:0.47, 32.87) circle [radius = 1mm];
	\draw[fill] (axis cs:6.98, 32.87) circle [radius = 1mm];
	\draw[fill] (axis cs:7.413, 32.87) circle [radius = 1mm];
	\draw[fill] (axis cs:6.98, 54.14) circle [radius = 1mm];

	%vonalak
	\draw[ultra thick] (axis cs:0.47, 32.87) -- (axis cs:7.413, 32.87);
	\draw[ultra thick] (axis cs:2.92, 263.94) -- (axis cs:5.97, 263.94);
	\draw[ultra thick, dashed] (axis cs:6.98, 32.87) -- (axis cs:6.98, 500);
	
	%értékek jelölése
	\draw[thick, dashed] (axis cs:0, 32.87) -- (axis cs:0.47, 32.87);
	\draw[thick, dashed] (axis cs:0, 263.94) -- (axis cs:2.92, 263.94);
	\draw[thick, dashed] (axis cs:0, 500) -- (axis cs:6.98, 500);
	
	\draw[thick, dashed] (axis cs: 0.47, 0) -- (axis cs:0.47, 32.87);
	\draw[thick, dashed] (axis cs: 2.92, 0) -- (axis cs:2.92, 263.94);
	\draw[thick, dashed] (axis cs: 5.97, 0) -- (axis cs:5.97, 263.94);
	\draw[thick, dashed] (axis cs: 6.98, 0) -- (axis cs:6.98, 32.87);	
	\draw[thick, dashed] (axis cs: 7.413, 0) -- (axis cs:7.413, 32.87);
	
	%nevezetes pontok, értékek
	\node[thick, circle, draw, inner sep = 1mm] at (axis cs:6.98, 540) {1};
	\node[thick, circle, draw, inner sep = 1mm] at (axis cs:8, 90) {2b};
	\node[thick, circle, draw, inner sep = 1mm] at (axis cs:6.6, 120) {2c};
	\node[thick, circle, draw, inner sep = 1mm] at (axis cs:6.3, 62) {2a};
	\node[thick, circle, draw, inner sep = 1mm] at (axis cs:0.47, 80) {3};
	\node[thick, circle, draw, inner sep = 1mm] at (axis cs:2.9, 310) {4};
	\node[thick, circle, draw, inner sep = 1mm] at (axis cs:5.97, 320) {5};
	\node at (axis cs:8, 500) {$p_1 = \SI{50}{\bar}$};
	\node at (axis cs:9.2, 263) {$p_K = \SI{0.05}{\bar}$};
	
	%nyilak
	\draw[mid arrow=black] (axis cs:6.98, 500) -- (axis cs:7.413, 32.87);
	\draw[mid arrow=black] (axis cs:7.413, 32.87) -- (axis cs:0.47, 32.87);
	\draw[mid arrow=black] (axis cs:2.92, 263.94) -- (axis cs:5.97, 263.94);

\end{axis}

\end{tikzpicture}
\end{figure}

% Oldaltörés
\pagebreak
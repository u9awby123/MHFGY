\pagebreak
% A feladat címe automatikus számozás nélkül
\section*{MH23. feladat: Rankine--Clausius-körfolyamat számítása}

% Hozzáadás a tartalomjegyzékhez azonos címmel
\addcontentsline{toc}{section}{MH23. feladat: Rankine--Clausius-körfolyamat számítása}

% Táblázat a szerző adataival
\begin{tabular}{ | p{2cm} | p{14cm} | } 
	\hline
	Szerző & Csendes Nikoletta, AUFM6A \\ 
	\hline
	Szak & Mechatronikai mérnöki alapszak \\ 
	\hline
	Félév & 2019/2020 II. (tavaszi) félév \\ 
	\hline
\end{tabular}
\vspace{0.5cm}

% A feladat szövege
\noindent A Rankine-Clausius körfolyamat kondenzátorában 
$\dot{Q}_{K} = \SI{8.5e4}{\kilo\joule\per\second}$ 
hőt kell elvonnunk. Az átáramló közvetítőközeg mennyisége 
$\dot{m} = \SI{40}{\kilogram\per\second}$
 . A turbinában áramló gőz állapotjelzői: \\
 
 \begin{equation*}
 	p_1 = \SI{50}{\bar}, 
 	\quad 
 	T_H = \SI{500}{\celsius}, 
 	\quad
 	h_1 = \SI{3434}{\kilo\joule\per\kilogram},
 	\quad
 	s_1 = \SI{6.98}{\kilo\joule\per\kilogram\kelvin}.
 \end{equation*}

\noindent A kondenzátornyomás ${p}_{K} = \SI{0.05}{\bar}$, amelyhez a telítési állapotjelzők:\\

\begin{equation*}
	h' = \SI{136}{\kilo\joule\per\kilogram}, 
	\quad 
	h'' = \SI{2560}{\kilo\joule\per\kilogram}, 
	\quad
	s' = \SI{0.47}{\kilo\joule\per\kilogram\kelvin},
	\quad
	s'' = \SI{8.4}{\kilo\joule\per\kilogram\kelvin}.
\end{equation*}

\noindent A körfolyamatban az expanzió irreverzibilis, egyéb irreverzibilitás nincs. Rajzolja le a körfolyamatot T-s diagramban és a tápszivattyúzási munka figyelmen kívül hagyásával számítsa ki az irreverzibilis körfolyamat termikus hatásfokát, a termodinamikus hatásfokot és a turbina teljesítményét! \\

\noindent ($\eta_{TD} = \SI{89.7}{\percent}$, \quad  $\eta_{T} = \SI{35.6}{\percent}$, \quad $P_{T} = \SI{46920}{\kilo\watt}$)

\noindent\hrulefill

% A feladat megoldása
...

% Oldaltörés
\pagebreak

% A "book" dokumentumosztály fölösleges oldalakat szúr be a címoldal
% és a tartalomjegyzék után.

\usepackage[utf8]{inputenc}
\usepackage[magyar]{babel}

% XeLaTeX ékezetes betűk
\usepackage{fontspec}

% A margók beállítása, később a 
%	\newgeometry{left=3cm,bottom=0.1cm}
% és a 
%	\restoregeometry
% parancsokkal egyedileg módosíthatók a margók
\usepackage[left=2cm, right=2cm, top=2cm, bottom=2cm]{geometry}

\usepackage{amsmath}
\usepackage{amssymb}

% A dupla betűk betűtípusának beállítása
\AtBeginDocument{
	\DeclareSymbolFont{AMSb}{U}{msb}{m}{n}
	\DeclareSymbolFontAlphabet{\mathbb}{AMSb}
}

% A régi német betűtípusok támogatása (fraktúra, Schwabacher, gót)
% A gót betűtípus nem működik
% Az ékezetes betűk csak a Table 18: Text-mode Accents táblázat szerint működnek
\usepackage{yfonts}

% Tabu hiba kezelés
\usepackage{booktabs}% http://ctan.org/pkg/booktabs
\newcommand\tabitem{\makebox[1em][r]{\textbullet~}}
\usepackage{tabto}
\usepackage{tabu}

% Bekeretezett részek
\usepackage[framemethod=tikz]{mdframed}
% Megjegyzés
\newcommand{\comment}[2][black!10]{%
\begin{mdframed}[hidealllines=true, backgroundcolor=#1, innerleftmargin=3pt, innerrightmargin=3pt, leftmargin=-3pt, rightmargin=-3pt]
Megjegyzés: #2
\end{mdframed}
}

%\usepackage{titlesec}

\pagestyle{plain}

\usepackage{float}
\usepackage{xifthen}

% Tizedesvessző
\usepackage{icomma}

% Függelék

\usepackage[toc,titletoc,page]{appendix}
\renewcommand{\appendixtocname}{Függelékek}
\renewcommand{\appendixpagename}{Függelékek}


%\fontspec{Times New Roman}
%\setmonofont{Consolas}
%\setmainfont{Calibri}

\usepackage{commath}
% A \Dif esetén a hatványkitevők rálógnak a D-re.
% A deriváltakban a d után sok az üres hely.

% This package will provide a complete implementation of 
% unicode maths for XELATEX and LuaLATEX.
% A Cambria Math támogatott.
% Javítja a vektor jelek elhelyezését a betűk felett.
% Az alsó és felső indexek betűtípusait módosítja, és szélesebbek a betűk.
\usepackage[math-style=TeX]{unicode-math}
%\setmathfont{XITS Math} % Túl vastagok a betűk
%\setmathfont{Cambria Math}

\usepackage{siunitx}
\sisetup{
	per-mode = fraction,
	fraction-function = \dfrac,
	detect-family = true,
	output-decimal-marker = {,},
	space-before-unit = true,
	use-xspace = true,
	exponent-product = \cdot,
	sticky-per = true
}

\usepackage[makeroom]{cancel}

% Többrészes ábrák létrehozása, subfigure környezet
\usepackage[justification=centering]{caption}
\usepackage{subcaption}

% A Hyperref csomagot lehetőleg utolsóként kell betölteni.
\usepackage[colorlinks=false]{hyperref}

% A címsorok másolása a kereszthivatkozásokba
\usepackage{nameref}

% Bekarikázott számok
\usepackage{pifont}

% Kiemelés színnel
\usepackage{xcolor}

% Az oszlopvektorok és mátrixok jelölésére
\usepackage{accents}
\newcommand{\ubar}[1]{\underaccent{\bar}{#1}}

% Színes táblázat, a tabuhoz nem szükséges
%\usepackage{colortbl}

% Fekvő oldalak
% If you are using pdfLaTeX, you should use pdflscape instead. The pdflscape package adds PDF support to the landscape environment of package lscape, by setting the PDF/Rotate page attribute. Pages with this attribute will be displayed in landscape orientation by conforming PDF viewers:
\usepackage{pdflscape}

% Kiemelés
% Több sorba osztásnál %-ezni kell a sortöréseket
%\newcommand{\highlight}[2]{\colorbox{#1}{$\displaystyle#2$}}

% Függőleges távolságok
\newcommand{\reducedstrut}{\vrule width 0pt height 1.2\ht\strutbox depth 0.95\dp\strutbox\relax}
\newcommand{\highlight}[2]{%
  \begingroup
  \setlength{\fboxsep}{1pt}% Vízszintes távolság
  \colorbox{#1}{\reducedstrut$\displaystyle#2$\/}%
  \endgroup
}

% Sortáv módosítás
\usepackage{setspace}
%\singlespacing
%\onehalfspacing
%\doublespacing

% Felsorolások
\usepackage{enumitem}

% Megjegyzés létrehozás
%\newlist{notes}{enumerate}{1}
%\setlist[notes]{label=Megjegyzés: , leftmargin=*}

% Külső fájlok írása
% Elvileg szükségtelen
%\usepackage{filecontents}

% Forráskezelés
%When using babel or polyglossia with biblatex, loading csquotes is recommended to ensure that quoted texts are typeset according to the rules of your main language.
\usepackage{csquotes}
\usepackage[ 
	backend=biber, 
	url=false, 
	citestyle=numeric,%draft,
	bibstyle=numeric,%style=numeric,%authoryear,%numeric, 
	giveninits=true, 
	clearlang=true, 
	bibencoding=auto,
	sorting=none]{biblatex}
	
\makeatletter

\newrobustcmd*{\parentexttrack}[1]{%
  \begingroup
  \blx@blxinit
  \blx@setsfcodes
  \blx@bibopenparen#1\blx@bibcloseparen
  \endgroup}

\AtEveryCite{%
  \let\parentext=\parentexttrack%
  \let\bibopenparen=\bibopenbracket%
  \let\bibcloseparen=\bibclosebracket}

\makeatother


% Az egyenletek előtti és utáni térközök beállítása
% A TikZ blokkdiagamban a 3. szinttől lefele fölösleges bal margót hoz létre (???)
%
% => NE LEGYEN ÜRES SOR AZ EGYENLETEK ELŐTT!
%
%\makeatletter
%\g@addto@macro\normalsize{%
%	\setlength\abovedisplayskip{6pt}
%	\setlength\belowdisplayskip{12pt}
%	\setlength\abovedisplayshortskip{0pt}
%	\setlength\belowdisplayshortskip{12pt}
%}
%\makeatother

% Az árva- és özvegysorok büntetése (nem biztos, hogy nem lesznek...)
\widowpenalty10000
\clubpenalty10000


% SAJÁT PARANCSOK LÉTREHOZÁSA
% ===========================

\DeclareMathOperator{\tr}{tr}

% Gradiens, forráserősség, örvényerősség
\newcommand{\Grad}[1] {\mathrm{grad\,} #1}
\newcommand{\Div}[1] {\mathrm{div\,} #1}
\newcommand{\Rot}[1] {\mathrm{rot\,} #1}
\newcommand{\Diag}[1] {\mathrm{diag\,} #1}

% Függvények
\newcommand{\Det}[1] {\mathrm{det} #1}
\DeclareMathOperator{\tg}{tg}
\DeclareMathOperator{\sgn}{sgn}
\DeclareMathOperator{\arctg}{arctg}
\DeclareMathOperator{\arctgNN}{arctg2}

% Mátrix belső közei
\makeatletter
\renewcommand*\env@matrix[1][\arraystretch]{%
	\edef\arraystretch{#1}%
	\hskip -\arraycolsep
	\let\@ifnextchar\new@ifnextchar
	\array{*\c@MaxMatrixCols c}}
\makeatother

% Oszlopvektor, sorvektor és mátrix jelölések
\newcommand{\mat}[1] {\underline{\underline{#1}}\reducedstrut}
\newcommand{\cvec}[1] {\underline{#1}\reducedstrut}
%\def\rvec#1{\underline{#1}^{T}}
%\def\mat#1{\underline{\underline{#1}}}
%\def\mat3#1{\underline{\underline{\underline{#1}}}}

\newcommand{\vektor}[2][1]{\begin{bmatrix}[#1] #2 \end{bmatrix}}
\newcommand{\tmb}[2][2]{\begin{bmatrix}[#1] #2 \end{bmatrix}}
\newcommand{\vek}[4][1]{\begin{bmatrix}[#1] #2 \\ #3 \\ #4 \end{bmatrix}}
\newcommand{\mtrx}[1]{\begin{bmatrix}#1\end{bmatrix}}



% FÜGGELÉK TARTALOMJEGYZÉK
% Appendix csomag (appendix.sty)
%\newcommand{\@redotocentry@pp}[1]{%
%  \let\oldacl@pp=\addcontentsline
%  \def\addcontentsline##1##2##3{%
%    \def\@pptempa{##1}\def\@pptempb{toc}%
%    \ifx\@pptempa\@pptempb
%      \def\@pptempa{##2}\def\@pptempb{#1}%
%      \ifx\@pptempa\@pptempb
%\oldacl@pp{##1}{##2}{##3}%A kísérleti változat: \oldacl@pp{##1}{##2}{\appendixname\space [##1][##2][##3] (\@car \splitlist{##3}\@nil) (\@cdr \splitlist{##3}\@nil)}%
%      \else
%        \oldacl@pp{##1}{##2}{##3}%
%      \fi
%    \else
%      \oldacl@pp{##1}{##2}{##3}%
%    \fi}
%}


% VÁLTOZÓK
\makeatletter
\let\JakiTitle\@title
\let\JakiAuthor\@author
\makeatother

% OLDALSZÁMOZÁS
% Oldalszámozás váltás a kezdő, a fő és a befejező szakasz határán
% A "report" osztályban nincs, a "book" osztályból másolva...
%https://tex.stackexchange.com/questions/154646/is-there-an-easy-way-to-get-the-frontmatter-mainmatter-and-backmatter-in-a-l
\makeatletter
\newcommand\frontmatterreport{%
	\cleardoublepage
	%\@mainmatterfalse
	\pagenumbering{Roman}}

\newcommand\mainmatterreport{%
	\cleardoublepage
	%\@mainmattertrue
	\pagenumbering{arabic}}

\newcommand\backmatterreport{%
	\if@openright
		\cleardoublepage
	\else
		\clearpage
	\fi
	% \@mainmatterfalse
	}
\makeatother


% HYPERREF
\hypersetup{
	pdftitle={\JakiTitle}, 
	pdfauthor={\JakiAuthor}, 
	colorlinks = true,
	citecolor = orange,
	filecolor = red,
	linkcolor = blue,
	urlcolor = red
}


% Az átmérőjel alapvonalra emelése
%\makeatletter
%\let\olddiameter\diameter
%\renewcommand{\diameter}[0]{\raisebox{\depth}{\olddiameter}}
%\makeatother


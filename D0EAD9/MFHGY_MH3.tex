

\section*{MH3. feladat}
\addcontentsline{toc}{section}{MH3. feladat}
\begin{tabular}{ | p{2cm} | p{14cm} | } 
	\hline
	Szerző & Török Zoltán D0EAD9 \\ 
	\hline
	Szak & Gépészmérnök alapszak\\ 
	\hline
	Félév & 2019/2020 II. (tavaszi) félév \\ 
	\hline
\end{tabular}
\vspace{0.5cm}

% A feladat szövege
\noindent  Mennyi munkát kaphatunk abból a körfolyamatból, amelynek a termikus hatásfoka $\eta_T=0,37$ és az elvezetett hőmennyiség  \SI{45}{\kilo\joule}?&&( Az eredmény: $W_{kinyert}=\SI{26,43e3}{\joule}$ )
\\
\\
\noindent Elvezetett hőmennyiség: ${Q_{el}=-\SI{45}{\kilo\joule}$ (ez azért lesz negatív előjelű, mert ez a hőmennyiség távozik a rendszerünkből)
\\
Körfolyamat munkája: W
\\
Bevezetett hőmennyiség: ${Q_{be}

\begin{equation}
W=Q_{be}+Q_{el} 
\end{equation}

\begin{equation}
\eta_T=\dfrac{\omega}{Q_{be}}
\end{equation}

 A (3.1)-es és a (3.2)-es egyenleteket felhasználva megkapjuk az egyenletünket mely által kifejezhetjük a körfolyamatba bevezetett hőmennyiség  értékét.\\
\\
\begin{equation*}
\eta_T=\dfrac{\omega}{Q_{be}}=\dfrac{Q_{be}+Q_{el}}{Q_{be}}=1+\dfrac{-\SI{45}{\kilo\joule}}{Q_{be}}
\end{equation*}
Az egyenletünk a termikus hatásfok behelyettesítése után:\\
\begin{equation*}
0,37=1+\dfrac{-\SI{45}{\kilo\joule}}{Q_{be}}
\end{equation*}

Az egyenletünket kirendezzük a ${Q_{be}$-re:\\
\begin{equation*}
{Q_{be}=\dfrac{-\SI{45}{\kilo\joule}}{0,37-1}=\dfrac{-\SI{45}{\kilo\joule}}{-0,63}=\SI{71,429}{\kilo\joule} 
\end{equation*}
A kezdeti egyenletünkbe ezután vissza helyettesítjük a ${Q_{be}$ értékét, és ezáltal kifejezhetjük a körfolyamat  kinyerhető munkát.
	\begin{equation*}
	W=Q_{be}+Q_{el}=\SI{71,429}{\kilo\joule}-\SI{45}{\kilo\joule}=\SI{26,429}{\kilo\joule}
	\end{equation*}
	\noident A körfolyamatból kinyerhető munka: $W=\SI{26,429}{\kilo\joule}$
	 \\
\begin{figure}[h]
	\centering
	\begin{tikzpicture}
	% A tengelykeresztet az axis környezet hozza létre
	\begin{axis}[
	width=8cm, height=8cm,
	xmin=0, xmax=8,
	ymin=260, ymax=320, 
	axis lines = middle,
	axis line style={->},
	xlabel=$s \left(\si{\kilo\joule\per\kilogram\kelvin}\right)$, 
	xlabel style={
		at=(current axis.right of origin), 
		anchor=north east
	}, 
	ylabel=$T \left(\si{\degree\kelvin}\right)$, 
	ylabel style={at=(current axis.above origin), anchor=north east
	},
	xtick={2,5},
	xticklabels={$s_1$,$s_2$},
	ytick={278,300},
	yticklabels={$T_1$,$T_2$},
	]
	\draw[dashed](axis cs:2,0)--(axis cs:2,320);
	\draw[dashed](axis cs:5,0)--(axis cs:5,320);
	\draw[dashed](axis cs:0,278)--(axis cs:8,278);
	\draw[dashed](axis cs:0,300)--(axis cs:8,300);
\draw[mid arrow=red, red, ultra thick](axis cs:2,278) -- (axis cs:2,300);
\draw[mid arrow=red, red, ultra thick](axis cs:2,300) -- (axis cs:5,300);
\draw[mid arrow=red, red, ultra thick](axis cs:5,300) -- (axis cs:5,278); 
\draw[mid arrow=red, red, ultra thick](axis cs:5,278) -- (axis cs:2,278);

	
	\end{axis}
	\end{tikzpicture}
\end{figure}
\\
\\
\\

\noident Az entrópia, illetve a hőmérséklet értékeket nem ismerjük, de a szemléltetés kedvéért ábrázolhatjuk Carnot-körfolyamatként.\\
\\
 A körfolyamat körbejárási iránya negatív, azaz óramutató járásával megegyező forgásirányú, mert fenntartása energia befektetést nem igényel(ha igényelne energia befektetést pozitív körbejárású lenne).


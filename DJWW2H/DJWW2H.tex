\section*{K3/1. feladat: Rankine-Clausius körfolyamat termodinamikai hatásfoka}
    \addcontentsline{toc}{section}{K3/1. feladat}
        \begin{tabular}{ | p{2cm} | p{14cm} | } 
            \hline
            Szerző & Vetési Tamás DJWW2H \\
            \hline
            Szak & Mechatronikai mérnöki \\
            \hline
            Félév & 2019/2020 II. (tavaszi) félév \\
            \hline
        \end{tabular}
            \vspace{0.5cm}
            %\noindent
            \\
A feladat leírása:
A vízgőz állapotjelzői a turbina előtt $60 bar$, és $420 °C$, a kondenzátornyomás $0,04 bar$. Határozzuk meg a gőz nedvességtartalmát a turbina után és a termodinamikai hatásfokot, ha a belső veszteségek okozta irreverzibilitás miatt az adott nyomásviszonyhoz tartoz elméleti hőesés $230 kJ/kg$ értékkel csökken.

\subsubsection{Adatok} 
			\begin{equation*}
				p_1= \SI{60}{\bar},
				\quad
				p_2= \SI{0,04}{\bar}
				\quad
				\end{equation*}
				\begin{equation*}
				T_1=\SI{420}{\celsius},
				\quad
				\Delta h_{veszt.} = \SI{230}{\kilo\joule\per\kilogram},
				\quad
				\end{equation*}
				\begin{equation*}
				h_4= \SI{3225}{\kilo\joule\per\kilogram},
				\quad
				h_{5A} = \SI{1991}{\kilo\joule\per\kilogram\kelvin},
				\quad
				\end{equation*}
				\begin{equation*}
				h"_5 = \SI{2554}{\kilo\joule\per\kilogram\kelvin},
				\quad
				x_{5A} = \SI{0,7686}
				\quad
				\quad
				\quad
				\quad
				\end{equation*}
				\begin{equation*}
				\eta_{TD} = \text{?},
				\quad
				1-x_5 = \text{?}
			\end{equation*}
			\noindent\hrulefill

\subsubsection{A termodinamikai hatásfok:}
A $\eta_{TD}$ termodinamikai hatásfok, amely az irreverzibilis (nem megfordítható)
 valós állapotváltozások hatásait veszi figyelembe.
				\begin{equation*}
				\quad
				\quad
				\quad
				\eta_{TD} = \dfrac{w_{t,ad,irrev}}{w_{t,ad,rev}}
				\quad
			\end{equation*}

\noindent
Ahol:
\begin{itemize}
	\item  \text{w_{t,ad,rev}: \text{ideális, reverzibilis, adiabatikus kiterjedés technikai munkája, amely egyben izentrópikus is}}
\end{itemize}
\begin{itemize}
	\item  \text{w_{t,ad,irrev}: \text{valós, irreverzibilis, adiabatikus kiterjedés technikai munkája, ami nem izentrópikus}}
\end{itemize}

\pagebreak
\subsubsection{A hőtartalom az irreverzibilis esetben:}
Ahhoz, hogy ki tudjuk számolni a termodinamikai hatásfokot, a meglévő adatok mellé ki kell számítani az irreverzibilis esethez tartozó hőtartalmat:
\begin{equation*}
h_5 = h_{5A} + \Delta h_{veszt.} = \SI{2221}{\kilo\joule\per\kilogram}
\end{equation*}
\noindent
A $h_5$ kiszámítása után ki tudjuk számolni a termodinamikai hatásfokot:

\begin{equation*}
\eta_{TD} = \dfrac{w_{t,ad,irrev}}{w_{t,ad,rev}} =  \dfrac{h_4 - h_5}{h_4-h_{5A}} = \text{0,8136}
\end{equation*}

\subsubsection{A nedvességtartalom a turbina után:}
\begin{equation*}
 \dfrac{x_5-x_{5A}}{h_5-h_{5A}} =  \dfrac{1 - x_{5A}}{h"_5-h_{5A}}
\end{equation*}

\noindent
Az egyenlet átrendezése után:
\begin{equation*}
x_5 = \dfrac{h_5-h_{5A}}{h"_5-h_{5A}}\left( 1 - x_{5A}\right) +x_{5A} = \text{0,8631}
\end{equation*}

\noindent
A nedvességtartalom:
\begin{equation*}
1 - x_5 = \text{0,1369} = \text{13,69\%}
\end{equation*}

\section*{MH28. feladat: Hőerőmű termikus jellemzői és fajlagos mutatói}
\addcontentsline{toc}{section}{MH28. feladat: Hőerőmű termikus jellemzői és fajlagos mutatói}

\begin{tabular}{ | p{2cm} | p{14cm} | } 
	\hline
	Név & Jobban Bátor \\ 
	\hline
	Szak & Mechatronikai mérnöki alapszak \\ 
	\hline
	Félév & 2019/2020 II. (tavaszi) félév \\ 
	\hline
\end{tabular}
\vspace{0.5cm}

\noindent Határozza meg, egy kondenzációs hőerőmű legfőbb termikus jellemzőit, fajlagos mutatóit.

\noindent Alapadatok:

	 $G_{g} = \SI{125}{\tonne\per\hour}$ --- gőzfogyasztás 
	
	 $p_{0} = \SI{28.29}{\bar}$; $t_{0} = \SI{535}{\celsius}$ --- friss gőz állapot
			
	 $p_{k} = \SI{0.044}{\bar}$; $t_{k} = \SI{30.7}{\celsius}$ --- kondenzátor nyomás illetve hőmérséklet	
	 
	 $\Delta T_{\textit{víz}} = \SI{10}{\celsius}$ --- hűtővíz hőfokváltozása a kondenzátorban
	 
	 $\eta_{\textit{eff}} = \SI{0.92}{}$ --- turbina eff. hatásfoka ($\eta_{td}$) gőzbeömlő szervek okozta nyomás veszteség: ${6}{\%}$
	 
	 $\eta_{k} = \SI{0.85}{}$ --- kazánhatásfok	
	 
	 $H_{i} = \SI{12.56}{\mega\J\per\kilogram}$ --- szén fűtőérték
	 
	 $\eta_{\textit{mech}} = \SI{0.89}{}$ --- mech. hatásfok

\begin{enumerate}
	\setlength\itemsep{0em}
	\item Számítsa ki a valóságos exp. végpontjában a gőz állapotjelzőit, közben ábrázolja a teljes körfolyamatot $t-s$ diagramban!
	\item Mennyi a turb. eff. teljesítménye, fajlagos gőzfogyasztása, fajlagos hőfogyasztása,
	\item a hőerőmű gazdasági összhatásfoka, az óránkénti eltüzelt szén mennyisége,
	\item a kondenzáláshoz felhasznált hűtővíz mennyisége?
\end{enumerate}

\noindent\hrulefill

\vspace{2mm}

\noindent A turbina eff. telj.-e.

\vspace{2mm}

\noindent A ${6}{\%}$-os nyomásveszteség miatt az expanzió $p_0^{x} = \SI{82.99}{\bar}$-ról indul és $p_k = p_2 = \SI{0.044}{\bar}$ vonalig tart.

\begin{equation*}
	i_o^{x} = \SI{3480}{\kJ\per\kilogram} \quad 
	i_2 = \SI{2062}{\kJ\per\kilogram} \quad 
	x_2 = \SI{0.797}{}
\end{equation*}

\vspace{2mm}

\noindent Az izentrópikus hőesés: 
\begin{equation}
	h_0 = i_0^{x} - i_2 = \SI{831} - \SI{492.5} = \SI{1418}{\kJ\per\kilogram}
\end{equation}

\vspace{2mm}

\noindent A valóságos hőesés:
\begin{equation}
	h_{\textit{eff}} = h_0 \cdot \eta_{\textit{eff}} = \SI{1418} \cdot \SI{0.92} = \SI{1304.5}{\kJ\per\kilogram}
\end{equation}

\vspace{2mm}

\noindent A valóságos expanzió végpontjában a gőz állapota:

\vspace{2mm}

\begin{equation*}
i_{\textit{2val}} = \SI{2019.7}{\kJ\per\kilogram} \quad 
p_2 = \SI{0.044}{\bar} \quad 
t_2 = \SI{30.7}{\celsius} \quad 
x_{\textit{2val}} = \SI{0.81}{}
\end{equation*}

\vspace{2mm}

\noindent Eff. telj.:

\begin{equation}
	P_{\textit{eff}} = \dfrac{G_g \cdot h_{eff}}{\SI{3600}{}} = \dfrac{\SI{125e3}{} \cdot \SI{1304.5}{}}{\SI{3600}{}} = \SI{45.26e3}{\kW}
\end{equation}

\vspace{2mm}

\noindent Belső fajl. gőzfogyasztás:

\begin{equation}
	g_0^x = \dfrac{\SI{3600}{}}{h_{eff}} = \dfrac{\SI{3600}{}}{\SI{1304.5}{}} =  \SI{2.761}{\dfrac {\kilogram_{\textit{gőz}}} {\kWh}}
\end{equation}

\noindent A fajl. hőfogyasztás (egységnyi villamosenergia termeléséhez szükséges hőenergia):

\vspace{2mm}

$i_{k} = \SI{128.5}{\kJ\per\kilogram}$ --- a tápvíz hőtartalma

\vspace{2mm}

\begin{equation}
	q = \dfrac {{g_0^x}{(i_0^x - i_k)}} {\eta_{k} \cdot \eta_{\textit{mech}}} = \dfrac{\SI{2.76} (\SI{3480} - \SI{128.5})} {\SI{0.85}{} \cdot \SI{0.89}{}} = \SI{3.398}{\dfrac {\kJ_{\textit{hő}}} {\kJ_{\textit{villamos energia}}}}
\end{equation}

\noindent mivel
\vspace{2mm}

\begin{equation}
	q = \dfrac {\SI{1}{}} {\eta_{ö}} = \dfrac {\SI{1}{}} {\eta_{T} \cdot \eta_{\textit{td}} \cdot \eta_{k} \cdot \eta_{m}}
\end{equation}

\vspace{2mm}

\begin{equation}
	\eta_{T} \cdot \eta_{\textit{td}} = \dfrac {i_0^{x} - i_{\textit{2val}}} {i_0^{x} - i_{k}} = \dfrac {h_{\textit{eff}}} {i_0^{x} - i_{k}}
\end{equation}

\vspace{2mm}

\noindent Hőerőmű gazdasági hatásfoka (össz-):

\begin{equation}
	\eta_{\textit{gazd}} = \dfrac {\SI{1}{}} {q} = \SI{29.43}{\%}
\end{equation}

\vspace{2mm}

\noindent Az óránkénti eltüzelt szén mennyisége:

\begin{equation}
	G_{\textit{szén}} = \dfrac{{G_{g}}{(i_0^x - i_{k})}} {H_{i} \cdot \eta_{k} \cdot \eta_{\textit{mech}}} = \dfrac{\SI{125e3} (\SI{3480} - \SI{128.5})} {\SI{12.56e3}{} \cdot \SI{0.85}{} \cdot \SI{0.89}{}} = \SI{44.08}{\tonne\per\hour}
\end{equation}

\vspace{2mm}

\noindent A kondenzáláshoz szükséges víz mennyiséges: $ c_{v} = \SI{4.187}{\kJ\per\kilogram\celsius}$

\vspace{2mm}

\begin{equation}
	G_{\textit{víz}} = \dfrac{{G_{g}}{(i_{\textit{2val}} - i_{k})}} {c_{v} \cdot  \Delta t_{\textit{víz}}} = \dfrac{\SI{125e3} (\SI{2091.7} - \SI{128.5})} {\SI{4.187}{} \cdot \SI{10}{}} = \SI{5861.2}{\tonne\per\hour}
\end{equation}

\vspace{2mm}

kb. $\SI{47}{}$-szerese a gőztermelésnek

%%%%%%%%%%%%%%%%%%%%%%%%

\begin{figure}[h]
	\centering
	\begin{tikzpicture}
	
	\begin{axis}[
	axis lines = middle,
	axis line style = {->},
	xlabel={$s \left(\si{\kilo\joule\per\kilogram\kelvin}\right)$},
	ylabel={$T \left(\si{\celsius}\right)$},
	xlabel style={
		at=(current axis.right of origin), 
		anchor=north east
	},
	ylabel style={
		at=(current axis.above origin), 
		anchor=north east
	},
	ytick={30.847, 230.057, 535},
	yticklabels={{$30.847$}, {$230.057$}, {$535$}},
	xtick={0.445, 2.61, 6.212, 7.368},
	xticklabels={{$0.445$}, {$2.61$}, {$6.212$}, {$7.368$}},
	x tick label style={rotate=90,anchor=east},
	ytick style={draw=none},
	xtick style={draw=none},
	width=15cm, height=10cm, xmin=0, xmax=10, ymin=0, ymax=700,
	]
	
	
	\addplot [] table {./ocr5ny/ts.txt}; %T-s
	\addplot [ultra thick, red] table {./ocr5ny/p28.txt};

	%körfolyamat ábrázolása	
	\draw[fill, red] (axis cs:7.368, 535) circle [radius = 1mm];
	\draw[fill, red] (axis cs:6.221, 230.057) circle [radius = 1mm];
	\draw[fill, red] (axis cs:2.61, 230.057) circle [radius = 1mm];
	\draw[fill, red] (axis cs:0.445, 30.847) circle [radius = 1mm];
	\draw[fill, red] (axis cs:7.368, 30.847) circle [radius = 1mm];

	\draw[ultra thick, dashed, red] (axis cs:0.445, 30.847) -- (axis cs:7.368, 30.847);
	\draw[ultra thick, red] (axis cs:2.61, 230.057) -- (axis cs:6.212, 230.057);
	\draw[ultra thick, red] (axis cs:7.368, 30.847) -- (axis cs:7.368, 535);
	
	%a körfolyamat iránya
	\draw[mid arrow=red, dashed, red] (axis cs:7.368, 30.847) -- (axis cs:0.445, 30.847);
	\draw[mid arrow=red, red] (axis cs:2.61, 230.057) -- (axis cs:6.212, 230.057);
	\draw[mid arrow=red, red] (axis cs:7.368, 535) -- (axis cs:7.368, 30.847);
	
	%értékek jelölése
	\draw[thick, dashed, red] (axis cs:0, 30.847) -- (axis cs:0.445, 30.847);
	\draw[thick, dashed, red] (axis cs:0, 230.057) -- (axis cs:2.61, 230.057);
	\draw[thick, dashed, red] (axis cs:0, 535) -- (axis cs:7.368, 535);

	\draw[thick, dashed, red] (axis cs: 0.445, 0) -- (axis cs:0.445, 30.847);
	\draw[thick, dashed, red] (axis cs: 2.61, 0) -- (axis cs:2.61, 230.057);
	\draw[thick, dashed, red] (axis cs: 6.212, 0) -- (axis cs:6.212, 230.057);
	\draw[thick, dashed, red] (axis cs: 7.368, 0) -- (axis cs:7.368, 535);	
	\draw[thick, dashed, red] (axis cs:7.368, 0) -- (axis cs:7.368, 30.847);

		
	\end{axis}
	
	\end{tikzpicture}
		\caption{Víz-gőz $T-s$ diagram}
\end{figure}


\pagebreak


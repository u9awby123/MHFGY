\section*{K2/3. feladat: Rankine-Clausius-körfolyamat termikus hatásfoka}
\addcontentsline{toc}{section}{K2/3. feladat}
Állapítsa meg az alábbi adatokkal felvett idealizált Rankine-Clausius körfolyamat terimikus hatásfokát, valamint a részkörfolyamatok termikus hatásfokát! Rajzolja meg a körfolyamatot T-s diagramban!
\subsubsection{Adatok}
\begin{equation*}
      p=\SI{78.45}{\bar},
      \quad                            
      p_K=\SI{0.0392}{\bar},
      \quad
      T_T=\SI{500}{\celsius}=\SI{773.15}{\kelvin}
\end{equation*}
\subsubsection{A következő adatokat a táblázatból olvashatjuk ki}
\begin{equation*}
	T_K=\SI{28.6}{\celsius}=\SI{301.75}{\kelvin}
	\quad 
	T_S=\SI{293.6}{\celsius}=\SI{566.75}{\kelvin}
	\quad
	h_0=\SI{120.03}{\kilo\joule\per\kilogram}
	\quad
	h_1=\SI{1309.6}{\kilo\joule\per\kilogram}
	\quad
	h_2=\SI{2760.4}{\kilo\joule\per\kilogram}
	\quad
\end{equation*}

\begin{equation*}
	h_3=\SI{3398.8}{\kilo\joule\per\kilogram}
	\quad
	s_0=\SI{0.4178}{\kilo\joule\per\kilogram\kelvin}
	\quad
	s_1=\SI{3.1949}{\kilo\joule\per\kilogram\kelvin}
	\quad
	s_2=\SI{5.7548}{\kilo\joule\per\kilogram\kelvin}
	\quad
	s_3=\SI{6.7332}{\kilo\joule\per\kilogram\kelvin}
\end{equation*}

\subsubsection{Rankine-Clausius körfolyamat}
Állandó nyomáson történő hőkezelés (0-1-2-3 vonal)
\\adiabatikus expanzió (3-4 vonal), munka kinyerés
\\végül állandó hőmérsékleten való hőelvonás (4-5-6-0 vonal).
\pagebreak
\begin{figure}[h]
	\centering
	\begin{tikzpicture}
	% Rács és vágómaszk
	\draw[step=1cm, gray!25, very thin] (-1.5, -1) grid (15, 12.5);
	%\clip (-1.5, -1) rectangle (14.5, 11);
	
	% A tengelykeresztet az axis környezet hozza létre
	\begin{axis}[
	width=16cm, height=14cm,
	xmin=0, xmax=10.8,
	ymin=0, ymax=600, 
	axis lines = middle,
	axis line style={->},
	xlabel=$s \left(\si{\kilo\joule\per\kilogram\kelvin}\right)$, 
	xlabel style={
		at=(current axis.right of origin), 
		anchor=north east
	}, 
	ylabel=$T \left(\si{\degreeCelsius}\right)$, 
	ylabel style={
		at=(current axis.above origin), 
		anchor=north east},
	xtick={1, 2, 3, 4, 5, 6, 7, 8, 9},
	ytick={20,100, 200, 300, 400, 500}
	]
	\node[anchor=base west] at (1,28.6) {0};
	\node[anchor=base east] at (31,293) {1};
	\node[anchor=base west] at (60,293) {2};
	\node[anchor=base west] at (68,498) {3};
	\node[anchor=base west] at (68,33) {4};
	\node[anchor=base west] at (58,33) {5};
	\node[anchor=base west] at (28,33) {6};	
	\node[anchor=base west] at (38,150) {$w_{\textit{körfolyamat}}$};
	\node[anchor=base west] at (70,554) {$p$};
	\node[anchor=base west] at (96,212) {$p_{ki}$};

	\addplot[thick] table {./X3CGOS/ts.txt};
	

	\addplot[thick, dashed] table {./X3CGOS/ts_x50.txt};
	

	\addplot[ultra thick, dashed,red] table {./X3CGOS/ts_x75_373_473.txt};
	

	\addplot[ultra thick, dashed,blue] table {./X3CGOS/ts_p50.txt};
	
	\end{axis}
	
	\end{tikzpicture}
	\caption{Rankine--Clausius-körfolyamat T-s diagramja.}
\end{figure}

\subsubsection{A körfolyamatba bevitt hő}
\begin{equation*}
      q_{be} = {h_3}-{h_0} = \SI{3278.77}{\kilo\joule\per\kilogram}
\end{equation*}
Az $h_4$-et a táblázatban nem találjuk meg, ki kell számítani.
Az $h_4$ meghatározásához ismerni kell az $x_4$-et (az expanzió végén a gőz fajlagos gőztartalmát).
Vegyük figyelembe, hogy az expanzió (ideális esetben) izentrópikus tehát $s_3$=$s_4$. Az $s_4$ a 4-es pontbeli folyadék és gőz elegy entrópiája.
\begin{equation*}
s_4 = \left (1 - x_4 \right)s_0'+x_4 s_0'' 
\end{equation*}
\begin{equation*}
s_4 = s_0'-x_4 s_0'+x_4 s_0'' 
\end{equation*}

\begin{equation*}
 s_4 =  s_0'+ x_4 \left(s_0'' - s_0' \right)
\quad 
\Rightarrow
\quad 
x_4
= 
\dfrac{s_4 - s_0'}{s_0'' - s_0'} = \SI{0.7833}{}
\end{equation*}

\begin{equation*}
	 \textrm{$p_K$,$T_K$ állapotú víz entrópiája:} 
	 \quad 
	 s_0=\SI{0.4178}{\kilo\joule\per\kilogram\kelvin}
\end{equation*}


\begin{equation*}
	\textrm{$p_K$,$T_K$ állapotú gőz entrópiája:}
	\quad
s_0'' = \SI{8.4804}{\kilo\joule\per\kilogram\kelvin}
\end{equation*}

\begin{equation*}
\textrm{$p_K$,$T_K$ állapotú gőz entalpiája:}
\quad
h_0'' = \SI{3.11}{\kilo\joule\per\kilogram}
\end{equation*}
\\Ezek után $h_4$:
\begin{equation*}
	h_4 =  \left (1 - x_4 \right)h_4'+x_4 h_4''=\SI{2025.85}{\kilo\joule\per\kilogram}
\end{equation*}
Az elvezetett hő:
\begin{equation*}
	q_{el} =h_4 - h_0 =
	\SI{1905.8}{\kilo\joule\per\kilogram} 
\end{equation*}
Végül
\begin{equation*}
	w={q_{be}} - {q_{el}} =
	\SI{1373}{\kilo\joule\per\kilogram} 
\end{equation*}
\begin{equation*}
	\eta_T=\dfrac{w} {q_{be}}=
	\SI{0.419}
	=
	\SI{41.9}{\%}
\end{equation*}
\subsubsection{a) Vizsgáljuk meg a részkörfolyamatokat!}
Ezek termikus hatásfokát is a részkörfolyamatból kinyert munka és a bevezetett hő aránya adja.
\begin{figure}[h]
	\centering
	\begin{tikzpicture}
	% Rács és vágómaszk
	\draw[step=1cm, gray!25, very thin] (-1.5, -1) grid (16.5, 14);
	%\clip (-1.5, -1) rectangle (14.5, 11);
	
	% A tengelykeresztet az axis környezet hozza létre
	\begin{axis}[
	width=16cm, height=14cm,
	xmin=0, xmax=10.8,
	ymin=0, ymax=600, 
	axis lines = middle,
	axis line style={->},
	xlabel=$s \left(\si{\kilo\joule\per\kilogram\kelvin}\right)$, 
	xlabel style={
		at=(current axis.right of origin), 
		anchor=north east
	}, 
	ylabel=$T \left(\si{\degreeCelsius}\right)$, 
	ylabel style={
		at=(current axis.above origin), 
		anchor=north east},
	xtick={1, 2, 3, 4, 5, 6, 7, 8, 9},
	ytick={20,100, 200, 300, 400, 500}
	]
	\node[anchor=base west] at (1,28.6) {0};
	\node[anchor=base east] at (31,293) {1};
	\node[anchor=base west] at (60,293) {2};
	\node[anchor=base west] at (68,500) {3};
	\node[anchor=base west] at (68,33) {4};
	\node[anchor=base west] at (58,33) {5};
	\node[anchor=base west] at (28,33) {6};	
	\node[anchor=base west] at (14,80) {1.részkörf.};
	\node[anchor=base west] at (36,80) {2.részkörf.};
	\node[anchor=base west] at (59.5,80) {3.részkörf.};
	\node[anchor=base west] at (71,575) {$p$};
	\node[anchor=base west] at (96,220) {$p_{ki}$};

	\addplot[thick] table {./X3CGOS/ts_2.txt};
	

	\addplot[thick, dashed] table {./X3CGOS/ts_x50_2.txt};
	

	\addplot[thick, dashed] table {./X3CGOS/ts_x75_373_473_2.txt};
	

	\addplot[thick, dashed] table {./X3CGOS/ts_p50_2.txt};
	

	\addplot[ultra thick,dashed, green] table {./X3CGOS/elso_reszkorfolyamat_2.txt};

	\addplot[ultra thick,dashed, red] table {./X3CGOS/masodik_reszkorfolyamat_2.txt};	

	\addplot[ultra thick,dashed, blue] table {./X3CGOS/harmadik_reszkorfolyamat_2.txt};		

	\end{axis}
	
	\end{tikzpicture}
	\caption{Rankine--Clausius összesített részkörfolyamat T-s diagram.}
\end{figure}
\subsubsection{Az első részkörfolyamat (jele ')}
A kinyert munka w' arányos a 0-1-6 területtel, a bevezetett hő pedig az $s_0$-0-1-$s_1$-el. Itt azonban hőközlés közben hőfok változik, ezért területintegrálással tudjuk kiszámítani. Elemi Carnot-körfolyamatra bontjuk a területet.
\\Ez esetben  $\eta_{TC}=1-\dfrac{T_A}{T_F}$
\\Ez alapján	$\dif w=\eta_{TC}  \dif q= \left (1-\dfrac{T_A}{T} \right) c \dif T$
\\\\A hőváltozás miatt a 0 és az 1-es pont között bevezetett hő:
\begin{equation*}
	T_{be \textit{átl}}'=\dfrac{q_{0 - 1}} {s_1 - s_0}=\dfrac{h_1-h_0}{s_1-s_0}=\SI{428.35}{\kelvin}	
\end{equation*}
\begin{equation*}
	q_{be}'=h_1 - h_0=T_{be\textit{átl}}' \left (s_1-s_0 \right)=\SI{1189.57}{\kilo\joule\per\kilogram}	
\end{equation*}
Megkeressük a $T_{be \textit{átl}}'$-hoz tartozó fajhő értéket $c=\SI{4.278}{\kilo\joule\per\kilogram\kelvin}$
\begin{equation*}
	w'=\left (1-\dfrac{T_A}{T_{be\textit{átl}}'} \right) c \!\ \left(T_F-T_A \right)=\SI{335.1}{\kilo\joule\per\kilogram}	
\end{equation*}
\begin{equation*}
	\eta_T'=\dfrac{w'}{q_{be}'}=\SI{0.282}
    =
	\SI{28.2}{\%}	
\end{equation*}
\subsubsection{A második részkörfolyamat}
Ez tiszta Carnot-ciklus, tehát termikus hatásfoka egyszerűen számítható.
\begin{equation*}
	\eta_{TC}''=1-\dfrac{T_A}{T_F}=\SI{0.4675}
	=
	\SI{46.75}{\%}	
\end{equation*}
\subsubsection{A harmadik részkörfolyamat}
(Az elsőhöz hasonlóan számoljuk.)
\begin{equation*}
	T_{be\textit{átl}}'''=\dfrac{h_3-h_2}{s_3-s_2}=\SI{652.5}{\kelvin}
	=
	\SI{379.35}{\celsius}	
\end{equation*}
A $T_{be\textit{átl}}'''$ -hoz tartozó fajhő értéke 
    $c=\SI{2.097}{\kilo\joule\per\kilogram\kelvin}$
\begin{equation*}
	q_{be}'''=T_{be\textit{átl}}''' \!\ \left(s_3-s_2 \right) =\SI{638.4}{\kilo\joule\per\kilogram}
\end{equation*}
\begin{equation*}
	w'''=\left (1-\dfrac{T_A}{T_{be\textit{átl}}'''} \right) c \left(T_f-T_a \right)=\SI{532.4}{\kilo\joule\per\kilogram}	
\end{equation*}
\begin{equation*}
	\eta_T'''=1-\dfrac{w'''}{q_{be}'''}=\SI{0.8323}
	=
	\SI{83.23}{\%}	
\end{equation*}
\\Theát $\eta'={28.2}{\%}$, $\eta''={46.75}{\%}$, $\eta'''={83.23}{\%}$

\subsubsection{Ha a $T_T$ és $T_K$ között Carnot-ciklust tudnánk megvalósítani akkor a termikus hatásfok}
\begin{equation*}
	\eta_{TC}=1-\dfrac{T_A}{T_F}=\SI{0.6097}
	=
	\SI{60.97}{\%}	
\end{equation*}
A két hatásfok hányadosa a Carnot-fok.
\begin{equation*}
	C=1-\dfrac{\eta_T}{\eta_{TC}}=\SI{0.688}
	=
	\SI{68.8}{\%}	
\end{equation*}
 %2. számítás
 \subsubsection{b) Végezzünk entrópia-entalpia vizsgálatot hogy megkapjuk a termikus hatásfokot.}
 \vspace{0.5cm}
  \subsubsection{Az első részkörfolyamat}
  Az alábbi mennyiségek egyenlőek, mert egy vízszintes görbén helyezkednek el.
  \begin{equation*}
  	s_6=s_1=\SI{3.1949}{\kilo\joule\per\kilogram\kelvin},
  	\quad
  	s_6'=s_0'=s_0,
  	\quad
  	s_0''=s_4''=s_6''	
  \end{equation*}
   \begin{equation*}
 	h_6'=h_0'=h_0,
 	\quad
 	h_6''=h_0''
 	 \end{equation*}

\begin{equation*}
   s_6=\left (1-x_6 \right) s_6' + x_6 s_6''
   \quad
   \Rightarrow
   \quad
   x_6=\dfrac{s_6 - s_6'} {s_6'' - s_6'}=0,344
\end{equation*}
  \begin{equation*}
	h_6=\left (1-x_6 \right)h_6'+x_6 h_6''=\SI{957}{\kilo\joule\per\kilogram}
\end{equation*}
  \begin{equation*}
	q_{be}'=h_1-h_0=\SI{1189.57}{\kilo\joule\per\kilogram}
\end{equation*}
  \begin{equation*}
	q_{el}'=h_6-h_0=\SI{836.97}{\kilo\joule\per\kilogram}
\end{equation*}
  \begin{equation*}
	w'=q_{be}'-q_{el}'=\SI{352.6}{\kilo\joule\per\kilogram}
\end{equation*}
  \begin{equation*}
	\eta_T'=\dfrac{w'}{q_{be}'}=\SI{0.29}
    =
	\SI{29}{\%}
\end{equation*}
 \subsubsection{A második részkörfolyamat}
  \begin{equation*}
  	s_5=s_2=\SI{5.7548}{\kilo\joule\per\kilogram\kelvin},
 	\quad
 	s_5'=s_0,
 	\quad
 	h_5'=h_0	
 \end{equation*}
  \begin{equation*}
 	s_5''=s_0'',
 	\quad
 	h_5''=h_0''	
 \end{equation*}
\begin{equation*}
s_5=\left (1-x_5 \right) s_5' + x_5 s_5''
\quad
\Rightarrow
\quad
x_5=\dfrac{s_5 - s_5'} {s_5'' - s_5'}=0,662
\end{equation*}
 \begin{equation*}
	h_5=\left (1-x_5 \right)h_5'+x_5 h_5''=\SI{1730.72}{\kilo\joule\per\kilogram}
\end{equation*}
   \begin{equation*}
 	\eta_T''=1-\dfrac{T_A}{T_F}=\SI{0.4675}
    =
 	\SI{46.75}{\%}
 \end{equation*}
 \subsubsection{A harmadik részkörfolyamat}
   \begin{equation*}
 	q_{be}'''=h_3-h_2=\SI{638.4}{\kilo\joule\per\kilogram}
 \end{equation*}
   \begin{equation*}
	q_{el}'''=h_4-h_5=\SI{295.13}{\kilo\joule\per\kilogram}
\end{equation*}
  \begin{equation*}
	w'''=q_{be}'''-q_{el}'''=\SI{343.27}{\kilo\joule\per\kilogram}
\end{equation*}
   \begin{equation*}
	\eta_T'''=1-\dfrac{w'''}{q_{be}'''}=\SI{0.5377}
	=
	\SI{53.77}{\%}
\end{equation*}
\subsubsection{c) Eddig elhanyagoltuk a tápszivattyú kompressziós munkáját, amellyel a tápvíz nyomását $p_K$-ról $p$-re emeli. Ábrázoljuk ezt a folyamatot is h-s-ben! Számoljuk ki a tápszivattyú teljesítményét!}
\begin{figure}[h]
	\centering
	\begin{tikzpicture}
	% Rács és vágómaszk
	\draw[step=1cm, gray!25, very thin] (-1.5, -1) grid (16.5, 14);
	%\clip (-1.5, -1) rectangle (14.5, 11);
	
	% A tengelykeresztet az axis környezet hozza létre
	\begin{axis}[
	width=16cm, height=15.5cm,
	xmin=0, xmax=4,
	ymin=0, ymax=190, 
	axis lines = middle,
	axis line style={->},
	xlabel=$s \left(\si{\kilo\joule\per\kilogram\kelvin}\right)$, 
	xlabel style={
		at=(current axis.right of origin), 
		anchor=north east
	}, 
	ylabel=$h \left(\si{\kilo\joule\per\kilogram}\right)$, 
	ylabel style={
		at=(current axis.above origin), 
		anchor=north east},
	xtick={0.417,1, 2, 3, 3.195},
	ytick={120,128.75}
	]
	\node[anchor=base west] at (41,115) {0};
	\node[anchor=base west] at (38,132) {1*};
	\node[anchor=base west] at (310,171) {1};
	\node[anchor=base west] at (330,174) {$p$};
	\node[anchor=base west] at (330,165) {$p_k$};
	\node[anchor=base east] at (180,171) {1309.6};


	\addplot[thick, dashed] table {./X3CGOS/szivattyu_1.txt};
	

	\addplot[thick, dashed] table {./X3CGOS/szivattyu_2.txt};
	

	\addplot[ ultra thick, dashed,red] table {./X3CGOS/szivattyu_3.txt};
	

	\addplot[ultra thick, dashed, blue] table {./X3CGOS/szivattyu_4.txt};
	
	
	\addplot[thick, dashed] table {./X3CGOS/szivattyu_5.txt};

	\end{axis}
	
	\end{tikzpicture}
	\caption{Tápszivattyú kompressziós munkájának nem méretarányos diagramja. }
\end{figure}
\vspace{1.5cm}
A kompressziós munka kiszámításához szükségünk van az $h^*_1$-ra amit az alábbi módon számolhatunk:

\begin{equation*}
    \dfrac{s_0-\SI{0.2943}{\kilo\joule\per\kilogram\kelvin}}{\SI{0.569}{\kilo\joule\per\kilogram\kelvin}\SI{0.2943}{\kilo\joule\per\kilogram\kelvin}}
    =
    \dfrac{h^*_1\SI{-91.3}{\kilo\joule\per\kilogram}}{\SI{174.17}{\kilo\joule\per\kilogram}-\SI{91.3}{\kilo\joule\per\kilogram}}
\end{equation*}
   \begin{equation*}
	0.449=\dfrac{h^*_1-\SI{-91.3}{\kilo\joule\per\kilogram}}{\SI{82.87}{\kilo\joule\per\kilogram}}
	\quad
	\Rightarrow
	\quad
	h^*_1=\SI{128.5422}{\kilo\joule\per\kilogram}
\end{equation*}
  \begin{equation*}
	w_{komp}=h^*_1-h_0=\SI{8.5122}{\kilo\joule\per\kilogram}=w_{sziv}
\end{equation*}
 \begin{equation*}
q^*_{be}=h_3-h^*_1=\SI{3270.28}{\kilo\joule\per\kilogram}=w_{sziv}
\end{equation*}
 \begin{equation*}
	\eta_{Tkomp}=\dfrac{w-w_{komp}}{q_{be*}}=\SI{0.417}
    =
	\SI{41.7}{\%}
\end{equation*}
 \begin{equation*}
	w_{turb}=h_3-h_4=\SI{1372.95}{\kilo\joule\per\kilogram}
\end{equation*}
\begin{equation*}
	\dfrac{w_{sziv}}{w_{turb}}=\SI{0.0062}
    =
	\SI{0.62}{\%}
\end{equation*}
\\A szivattyú teljesítménye:
\begin{equation*}
    P_{sziv}=\dfrac{100000}{3600}\SI {8.5122}=
    \SI{236.45}{\kilo\watt}
\end{equation*}
\\

\begin{figure}[h]
	\centering
	\begin{tikzpicture}
	% Rács és vágómaszk
	\draw[step=1cm, gray!25, very thin] (-1.5, -1) grid (15, 13);
	%\clip (-1.5, -1) rectangle (14.5, 11);
	
	% A tengelykeresztet az axis környezet hozza létre
	\begin{axis}[
	width=16cm, height=14cm,
	xmin=0, xmax=1,
	ymin=0, ymax=220, 
	axis lines = middle,
	axis line style={->},
	xlabel=$s \left(\si{\kilo\joule\per\kilogram\kelvin}\right)$, 
	xlabel style={
		at=(current axis.right of origin), 
		anchor=north east
	}, 
	ylabel=$h \left(\si{\kilo\joule\per\kilogram}\right)$, 
	ylabel style={
		at=(current axis.above origin), 
		anchor=north east},
	xtick={0.2943,0.4178,0.569},
	ytick={91.3,174.17}
	]
	\node[anchor=base west] at (17,97) {$T=\SI{20}{\celsius}$};
	\node[anchor=base west] at (29,129) {$T=\SI{28}{\celsius}$};
	\node[anchor=base west] at (44,179) {$T=\SI{40}{\celsius}$};
	\node[anchor=base west] at (52,208) {$p=\SI{78.45}{\bar}$};
	\node[anchor=base west] at (1,129) {$h^*_1$};
	

	\addplot[thick,dashed] table {./X3CGOS/h_s_1.txt};
	

	\addplot[thick, dashed] table {./X3CGOS/h_s_2.txt};
	
	\end{axis}
	
	\end{tikzpicture}
	\caption{h-s diagram $h^*_1$ meghatározásához.}
\end{figure}
\pagebreak
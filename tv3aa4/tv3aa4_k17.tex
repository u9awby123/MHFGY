% !TeX TS-program = xelatex

\documentclass[11pt, a4paper]{report}

% Dokumentum adatok
% =================
\author{}
\title{Műszaki hőtan feladatgyűjtemény}

% A közös fájlok beszúrása
% ========================
\newcommand*{\JakiFolder}{./JAKI}%

% A közös fájlok a JAKI tárolóban vannak, amit az MHFGY-vel 
% közös mappába kell letölteni (clone/pull).
\input{JakiAlap.tex}				% Formázás, csomagok
\input{JakiTikz.tex}				% Rajzoló parancsok

% A dokumentum kezdete
% ====================
\begin{document}
	\section*{K1/7. feladat: Túlhevített vízgőz sűrítése állandó hőmérsékleten}
	\addcontentsline{toc}{section}{K1/7. feladat}
		\begin{tabular}{ | p{2cm} | p{14cm} | } 
			\hline
			Szerző & Hadabás Márk István TV3AA4 \\ 
			\hline
			Szak & Vegyészmérnöki \\ 
			\hline
			Félév & 2019/2020 II. (tavaszi) félév \\ 
			\hline
		\end{tabular}
			\vspace{0.5cm} 		
			%\noindent
			\\
		Izotermikus folyamat során 1 kg $p_1$=10 bar nyomású és $t_1$=200 °C hőmérsékletű vízgőz \textit{száraz telített állapotúvá válik}. Határozza meg a végállapotban a nyomást, az állapotváltozással kapcsolatos a munkát és a  gőz belső energiájának megváltozását. Ábrázolja a változást T-s rendszerben! \\
		\subsubsection{Ismert jellemzők a kezdeti állapotban} 
			\begin{equation*}
				m= \SI{1}{\kilogram}, 
				\quad
				p_1= \SI{10}{\bar},
				\quad
				p_2= \SI{15,55}{\bar},
				\quad
				t_1=\SI{200}{\celsius} = konst.
			\end{equation*}
		\subsubsection{Gőztáblázatból származó adatok}
			\begin{equation*}
				h_1 = \SI{2827}{\kilo\joule\per\kilogram}, 
				\quad
				h_2= \SI{2793}{\kilo\joule\per\kilogram}
			\end{equation*}
			\begin{equation*}
				s_1 = \SI{6,662}{\kilo\joule\per\kilogram\kelvin},
				\quad
				s_2 = \SI{6,430}{\kilo\joule\per\kilogram\kelvin}
			\end{equation*}
			\begin{equation*}
				v_1 = \SI{0,206}{\meter\cubed\per\kilogram},
				\quad
				v_2 = \SI{0,1272}{\meter\cubed\per\kilogram}
			\end{equation*}
			\noindent\hrulefill

		\subsubsection{Belső energia megváltozása}
			Termodinamika I. főtétele szerint a belső energia egy állapotjelző, tehát teljes differenciál.
			\[ \int_{1}^{2} du = u_1 - u_1 = \Delta u_{12}
			\]  
			Ugyan akkor ez zárt rendszerre igaz, de nyitott rendszerre az entalpiát alkalmazzuk. Gibbs egyenlet:
			\[ H=U+pv\]  
			\[U=H-pv\] 
			\begin{equation}
				\Delta u_{12} = u_2 - u_1 = \left(h_2 -p_2 \cdot v_2\right) - \left(h_1 -p_1 \cdot v_1\right) 				 
			\end{equation}
			\begin{equation}
				\Delta u_{12} = \left( \SI{2793}{\kilo\joule\per\kilogram} - \SI{15,55e5}{\pascal} \cdot \SI{0,1272}{\meter\cubed\per\kilogram} \right) - \left( \SI{2827}{\kilo\joule\per\kilogram} - \SI{10e5}{\pascal} \cdot \SI{0,206}{\meter\cubed\per\kilogram} \right) 
			\end{equation}
			\begin{equation}
				\Delta u_{12} = \SI{-25,76}{\kilo\joule\per\kilogram} 
			\end{equation}
                                Figyeljünk a nyomás mértékegységére!
		\subsubsection{Hőforgalom számítása}
			Termodinamika II. főtétele szerint:
			\[ ds=(1/T) \cdot  \delta q\] 
			Ezt q-ra rendezve:
			\[ q=T \cdot ds \]
			teljes differenciál az s-nél:
			\[ \int_{1}^{2} Tds= T(s_2-s_1) \]
			mert T kivihető integrál jel elé:
			\begin{equation}
				q_{12} = T_1 \cdot \left( s_2 - s_1\right) 
			\end{equation}   
			\begin{equation}
				q_{12} = \SI{473,15}{\kelvin} \left( \SI{6,430}{\kilo\joule\per\kilogram\kelvin} -\SI{6,662}{\kilo\joule\per\kilogram\kelvin}   \right) 
			\end{equation}
			\begin{equation}
				q_{12} = \SI{-123,13}{\kilo\joule\per\kilogram}
			\end{equation}	
                                 Figyeljünk a hőmérséklet mértékegységére!
		\subsubsection{Munka forgalom}
			Termodinamika I. főtétele alapján:
				\[ du = \delta q - \delta w \]
			Ezt átrendezve és integrálva:					 
			\begin{equation}
				w_{12} = q_{12}- \Delta U_{12} 
			\end{equation}	
			\begin{equation}
				w_{12} = \SI{-123,13}{\kilo\joule\per\kilogram} - \SI{-25,76}{\kilo\joule\per\kilogram} 
			\end{equation}	
			\begin{equation}
				w_{12} = \SI{-97,3}{\kilo\joule\per\kilogram}
			\end{equation}
		\subsubsection{T-S diagram}	
		\begin{figure}[h]
			\centering
			\label{figure:guh7ud-vgtsd}
			\begin{tikzpicture}
			% Rács és vágómaszk
			\draw[step=1cm, gray, very thin] (-1.5, -1) grid (14.5, 11);

			\begin{axis}[
			width=16cm, height=12cm,
			xmin=0, xmax=10.8,
			ymin=0, ymax=475, 
			axis lines = middle,
			axis line style={->},
			xlabel=$s \left(\si{\kilo\joule\per\kilogram\kelvin}\right)$, 
			xlabel style={
				at=(current axis.right of origin), 
				anchor=north east
			}, 
			ylabel=$T \left(\si{\degreeCelsius}\right)$, 
			ylabel style={
				at=(current axis.above origin), 
				anchor=north east
			},
			xtick={1, 2, 3, 4, 5, 6, 7, 8, 9},
			ytick={100, 200, 300, 400},
			]					

			\addplot[thick] table {ts-diagram-fázishatár.txt};

			\node[anchor=south east] at (axis cs: 6.430, 200) {\pgfcircled{$2$}};
			\node[anchor=south west] at (axis cs: 6.662, 200) {\pgfcircled{$1$}};	
			\node[circle,fill=red,inner sep=0pt,minimum size=3pt] (a) at (axis cs: 6.662, 200) {};
			\node[circle,fill=red,inner sep=0pt,minimum size=3pt] (a) at (axis cs: 6.430, 200) {};											
			\draw[black, thick, dashed] (axis cs: 6.662, 200) -|(axis cs: 6.662, 0);
			\draw[black, thick, dashed] (axis cs: 6.430, 200) -|(axis cs: 6.430, 0);
			\draw[black, thick]	(axis cs: 6.430, 200)-|	(axis cs: 6.662, 200);	
			\end{axis}									
			\end{tikzpicture}
			\caption{Vízgőz $T-s$ diagram}
		\end{figure}		
\end{document}


	\section*{K1/7. feladat: Túlhevített vízgőz sűrítése állandó hőmérsékleten}
	\addcontentsline{toc}{section}{K1/7. feladat}
		\begin{tabular}{ | p{2cm} | p{14cm} | } 
			\hline
			Szerző & Hadabás Márk István TV3AA4 \\ 
			\hline
			Szak & Vegyészmérnöki \\ 
			\hline
			Félév & 2019/2020 II. (tavaszi) félév \\ 
			\hline
		\end{tabular}
			\vspace{0.5cm} 		
			%\noindent
			\\
		Izotermikus folyamat során 1 kg $p_1=10 bar$ nyomású és $T_1=200 °C$ hőmérsékletű vízgőz \textit{száraz telített állapotúvá válik}. Határozza meg a végállapotban a nyomást, az állapotváltozással kapcsolatos a munkát és a  gőz belső energiájának megváltozását. Ábrázolja a változást $T-s$ rendszerben! \\
		\subsubsection{Ismert jellemzők a kezdeti állapotban} 
			\begin{equation*}
				m= \SI{1}{\kilogram}, 
				\quad
				p_1= \SI{10}{\bar},
				\quad
				p_2= \SI{15,55}{\bar},
				\quad
				T_1=\SI{200}{\celsius} = \text{állandó}
			\end{equation*}
		\subsubsection{Gőztáblázatból származó adatok}
			\begin{equation*}
				h_1 = \SI{2827}{\kilo\joule\per\kilogram}, 
				\quad
				h_2= \SI{2793}{\kilo\joule\per\kilogram}
			\end{equation*}
			\begin{equation*}
				s_1 = \SI{6,662}{\kilo\joule\per\kilogram\kelvin},
				\quad
				s_2 = \SI{6,430}{\kilo\joule\per\kilogram\kelvin}
			\end{equation*}
			\begin{equation*}
				v_1 = \SI{0,206}{\meter\cubed\per\kilogram},
				\quad
				v_2 = \SI{0,1272}{\meter\cubed\per\kilogram}
			\end{equation*}
			\noindent\hrulefill

		\subsubsection{Belső energia megváltozása}
			Termodinamika I. főtétele szerint a belső energia egy állapotjelző, tehát teljes differenciál.
			\[ \int_{1}^{2} \dif u = u_1 - u_1 = \Delta u_{12}
			\]  
			Ugyan akkor ez zárt rendszerre igaz, de nyitott rendszerre az entalpiát alkalmazzuk. Gibbs egyenlet:
			\[ h=u+pv\]  
			\[u=h-pv\] 
			\begin{equation*}
				\Delta u_{12} = u_2 - u_1 = \left(h_2 -p_2  v_2\right) - \left(h_1 -p_1  v_1\right) 				 
			\end{equation*}
			\begin{equation}
				\Delta u_{12} = \left( \SI{2793}{\kilo\joule\per\kilogram} - \SI{15,55e5}{\pascal} \cdot \SI{0,1272}{\meter\cubed\per\kilogram} \right) - \left( \SI{2827}{\kilo\joule\per\kilogram} - \SI{10e5}{\pascal} \cdot \SI{0,206}{\meter\cubed\per\kilogram} \right) 
			\end{equation}
			\begin{equation}
				\Delta u_{12} = \SI{-25,76}{\kilo\joule\per\kilogram} 
			\end{equation}
                                Figyeljünk a nyomás mértékegységére!
		\subsubsection{Hőforgalom számítása}
			Termodinamika II. főtétele szerint:
			\[ \dif s=\dfrac {1}{T} \delta q\] 
			Felszorzunk $T$-vel, majd $q$-ra rendezünk:
			\[ \delta q=T\dif s \]
			Teljes differenciált képzünk mindkét oldalnál:
			\[\int_{1}^{2} \delta q = \int_{1}^{2} T\dif s \]
                                Elvégezzük az integrálást, $T$ kivihető az integrál jel elé, mert $T=\text{állandó}$:
			\begin{equation*}
				q_{12} = T_1\left( s_2 - s_1\right) 
			\end{equation*}   
			\begin{equation}
				q_{12} = \SI{473,15}{\kelvin} \left( \SI{6,430}{\kilo\joule\per\kilogram\kelvin} -\SI{6,662}{\kilo\joule\per\kilogram\kelvin}   \right) 
			\end{equation}
			\begin{equation}
				q_{12} = \SI{-123,13}{\kilo\joule\per\kilogram}
			\end{equation}	
                                 Figyeljünk a helyes hőmérséklet skálára!
		\subsubsection{Munka forgalom}
			Termodinamika I. főtétele alapján:
				\[ \dif u = \delta q - \delta w \]
			Ezt átrendezzük $\delta w$-re:
				\[ \delta w = \delta q - \dif u \]	 
			Teljes differenciált képzünk:
				\[\int_{1}^{2} \delta w = \int_{1}^{2} \delta q - \int_{1}^{2} \delta u \]
			Majd integrálunk:
			\begin{equation*}
				w_{12} = q_{12}- \Delta u_{12} 
			\end{equation*}
			Előzőleg számolt értékeinkkel kiszámoljuk a $w_{12}$ értékét:
			\begin{equation}
				w_{12} = \SI{-123,13}{\kilo\joule\per\kilogram} - \SI{-25,76}{\kilo\joule\per\kilogram} 
			\end{equation}	
			\begin{equation}
				w_{12} = \SI{-97,3}{\kilo\joule\per\kilogram}
			\end{equation}
\pagebreak
		\subsubsection{T-s diagram}	
		\begin{figure}[h]
			\centering
			\label{figure:guh7ud-vgtsd}
			\begin{tikzpicture}

			\begin{axis}[
			width=16cm, height=12cm,
			xmin=0, xmax=10.8,
			ymin=0, ymax=475, 
			axis lines = middle,
			axis line style={->},
			xlabel=$s \left(\si{\kilo\joule\per\kilogram\kelvin}\right)$, 
			xlabel style={
				at=(current axis.right of origin), 
				anchor=north east
			}, 
			ylabel=$T \left(\si{\degreeCelsius}\right)$, 
			ylabel style={
				at=(current axis.above origin), 
				anchor=north east
			},
			xtick={1, 2, 3, 4, 5, 6, 7, 8, 9},
			ytick={100, 200, 300, 400},
			]					

			\addplot[thick] table {./tv3aa4/ts-diagram-fázishatár.txt};
			\addplot[ultra thick, dashed, blue] table {./tv3aa4/izobar10bar.txt};
			\addplot[ultra thick, dashed, red] table {./tv3aa4/izobar1555.txt};
			\node[anchor=south east] at (axis cs: 6.430, 200) {\pgfcircled{$2$}};
			\node[anchor=south west] at (axis cs: 6.662, 200) {\pgfcircled{$1$}};	
			\node[circle,fill=red,inner sep=0pt,minimum size=3pt] (a) at (axis cs: 6.662, 200) {};
			\node[circle,fill=red,inner sep=0pt,minimum size=3pt] (a) at (axis cs: 6.430, 200) {};											
			\draw[black, thick, dashed] (axis cs: 6.662, 200) -|(axis cs: 6.662, 0);
			\draw[black, thick, dashed] (axis cs: 6.430, 200) -|(axis cs: 6.430, 0);
			\draw[black, thick]	(axis cs: 6.430, 200)-|	(axis cs: 6.662, 200);
			\end{axis}									
			\end{tikzpicture}
			\caption{Vízgőz $T-s$ diagram}
		\end{figure}		
\pagebreak



\section*{K6/3. feladat: Hőátadási tényező számítása}
\addcontentsline{toc}{section}{K6/3. feladat: Hőátadási tényező számítása}
	\begin{tabular}{ | p{2cm} | p{14cm} | } 
		\hline
		Szerző & Hadabás Márk István TV3AA4 \\ 
		\hline
		Szak & Vegyészmérnöki \\ 
		\hline
		Félév & 2019/2020 II. (tavaszi) félév \\ 
		\hline
	\end{tabular} \\
	\vspace{0.5cm} 
		
	\noindent
	15 °C-os víz áramlik a $d$=0,015 m átmérőjű, $L$=0,5 m hosszú csőben. A közepes áramlási sebesség $w$=0,1 m/s. A cső belső falának hőmérséklete $t_w$ =50 °C. Számítsa ki a hőátadási tényező értéket!
	\\ \\
	Hausen képlet:
		\[ Nu = \left[ 3,65 + \frac{0,19 \left( Re \cdot Pr \cdot \frac{d}{L} \right)^{0,8} }{1 + 0,117 \left( Re \cdot Pr \cdot \frac{d}{L} \right)^{0,467} } \right] \cdot \left( \frac{Pr}{Pr_w} \right)^{0,11}  \]
	Érvényes, ha $Re$ < 2320 és 0,1 < $Re \cdot Pr \cdot \frac{d}{L}$; (di=belső átmérő) folyadékra.
	\\ 
	\subsubsection{Ismert jellemzők a kezdeti állapotban}
		\begin{equation*}	
			T= \SI{15}{\celsius}, 
			\quad
			d= \SI{0,015}{\meter},
			\quad
			L= \SI{0,5}{\meter},
			\quad
			w= \SI{0,1}{\meter\per\second},
			\quad
			t_w= \SI{50}{\celsius},			
		\end{equation*}
		\begin{equation*}
			\lambda= \SI{0,595}{\watt\per\meter\kelvin},
			\quad
			\nu = \SI{1,15 10}{\meter\squared\per\second },
			\quad
			Pr_{15}= 8,11,
			\quad
			Pr_{50}= 3,54,			
		\end{equation*}
		\noindent\hrulefill
	\subsubsection{Jelzőszámok számítása}
	A $Re$ és a $Nu$ jelzőszámok számítása a megadott adatok alapján. 
		\begin{equation}
			Re = \frac{w \cdot d}{\nu} = \frac{\SI{0,1}{\meter\per\second} \cdot \SI{0,015}{\meter}}{\SI{1,15 10}{\meter\squared\per\second }}=1304,35
		\end{equation}
          A kapott Reynolds jelzőszám értéke benne van a megadott peremfeltételekben, emiatt behelyettesíthetünk a Hausen képletbe.
		\begin{equation}
			Nu = \left[ 3,65 + \frac{ 0,19 \cdot \left( 1304,35 \cdot 8,11 \cdot\frac{0,015}{0,5}\right)^{0,8}}{1 + 0,0117 \left( 1304,35 \cdot 8,11 \cdot \frac{0,015}{0,5}\right)^{0,467}}\right] \cdot \left( \frac{8,11}{3,54}\right)^{0,11}  = 11,66
		\end{equation}
	\subsubsection{Hőátadási tényező számítása}
		A Nusselt szám defínició szerint.
                      \begin{equation*}
                     Nu = \frac{\text{konvektív hőátadás}}{\text{konduktív hőátadás}} = \alpha \cdot \frac{d}{\lambda}
                       \end{equation}
                     Rendezzük át az egyenletet $\alpha$-ra és helyettesítsünk be.
                      \begin{equation}
			\alpha = Nu \cdot \frac{\lambda}{d} = 11,66 \cdot \frac{\SI{0,595}{\watt\per\meter\kelvin}}{\SI{0,015}{\meter}}= \SI{462,51}{\watt\per\meter\squared\kelvin} 
		\end{equation}		
\pagebreak




		
\section*{HS 33.: Ellenáramú hőcserélő}	
\addcontentsline{toc}{section}{HS 33.}

\begin{tabular}{ | p{2cm} | p{14cm} | } 
	\hline
	Név & Molnár Annna \\ 
	\hline
	Szak & Vegyészmérnök alapszak \\ 
	\hline
	Félév & 2019/2020 II. (tavaszi) félév \\ 
	\hline
\end{tabular}
\vspace{0.5cm}

% A feladat szövege
\noindent
 Egy ellenáramú hőcserélőről az alábbi adatokat ismerjük:
$T_{1k} = \SI{115}{\celsius}$ meleg közeg belépési(kezdeti) hőmérsékletét,
$T_{2k} = \SI{18}{\celsius}$ hideg közeg belépési hőmérsékletét,
$\dot{w}_1 = \dot{w}_2 =\SI{50000}{\watt\per\kelvin}$ konvektív vízértékét,
$\kappa$ $= \SI{185}{\watt\per\meter\squared\kelvin}$ hőátszármaztatási tényezőjét,
$A_{\ddot{O}} = \SI{85}{\meter\squared}$ hőátadó felületét.
\vspace{5mm}

\underline{Feladat veszteségmentes esetben:}
\begin{itemize}
\item a,Számítsa ki a kilépési hőmérsékletet! 
\item b, Határozza meg $\kappa$ * értékét, ha $T_{1v} = T_{2v}$ (a kilépési hőmérsékletek azonosak)!
\item c,Rajzolja le a hőmérséklet-hely függvényt!
\end{itemize}
\vspace{5mm}
% A feladat megoldása
\subsubsection*{a) A kilépési hőmérsékletek kiszámítása}

 Az ellenáramú hőcsere esetén a melegebb és a hidegebb közeg az elválasztó felület két oldalán egymással párhuzamosan, ellentétes irányba haladnak. Feltételezzük, hogy a hőcserefolyamatban csak a két áramló közeg vesz részt, és a környezet felé nincs hőveszteség. Így az energiamegmaradás tétele következtében azt írhatjuk, hogy a felmelegedő közeg által felvett hő egyenlő a csökkenő hőmérsékletű közeg által leadott hővel $\Delta \dot{Q} = \dot{m}_1{c}_1 \left(T_{1v} - T_{1k}\right)=\dot{m}_2{c}_2 \left(T_{2v} - T_{2k}\right)$.A konvektív hőáramokkal történő  hőterjedést tehát a belépő(k) és kilépő(v) hőáramok különbségével felírható: $\Delta \dot{Q} = \dot{w}_1 \left(T_{1v} - T_{1k}\right)=\dot{w}_2 \left(T_{2v} - T_{2k}\right)$.
 Az $\dot{w}_1 = \dot{w}_2$ egyenlő,ami azt jeleni, hogy a T(A) függvények között lineáris viszony van. Illetve, ha a hőmérsékletkülönbségeket átírjuk $\Delta T_N =\left(T_{1k} - T_{2v}\right)$ és $\Delta T_k= \left(T_{1v} - T_{2k}\right)$ miatt a állandó a hőmérsékletkülönbség $\Delta T_N = \Delta T_k = \Delta T = $állandó. A hőmérséklet felírható így a vég-és kezdeti $\Delta T$ alapján:
 
\begin{equation}
  \dot{w}_1\left(T_{1k} - T_{1v}\right) = \kappa A_{\ddot{0}} \Delta T = \kappa A_{\ddot{0}} \left(T_{1v} - T_{2k}\right)
 \end{equation}
Mert a hőáram felírható a hőátszármaztatási tényező, a hőcserélő felülete és a hőmérsékletkülönbség szorzataként.
Felbontva a zárójeleket:

 \begin{equation}
\dot{w}_1 T_{1k} - \dot{w}_1 T_{1v} =  \kappa A_{\ddot{0}} T_{1v} -  \kappa A_{\ddot{0}} T_{2k}
 \end{equation}


Ezután ha rendezzük az egyenletet kezdeti és végső oldalra, $T_{1v}$ kifejezhető.

\begin{equation}
T_{1v} = \frac{\kappa A_{\ddot{O}} T_{2k} + \dot{w} T_{1k}}{\kappa A_{\ddot{O}} + \dot{w}} =\frac{\SI{185}{\watt\per\meter\squared\kelvin}\cdot \SI{85}{\meter\squared}\cdot \SI{18}{\celsius} + \SI{50000}{\watt\per\kelvin} \cdot \SI{115}{\celsius}}{\SI{185}{\watt\per\meter\squared\kelvin}\cdot \SI{85}{\meter\squared} + \SI{50000}{\watt\per\kelvin}}= \SI{91.79}{\celsius}
\end{equation}
$T_{2v}$ meghatározása pedig:
\begin{equation}
\dot{w}_1\left(T_{1v} - T_{1k}\right) = \dot{w}_2\left(T_{2v} - T_{2k}\right)
\end{equation}

De mivel a konvektív vezetés megegyezik($\dot{w}_1 = \dot{w}_2$), azzal egyszerűsítve, majd átrendezzük $T_{2v}$-re.

\begin{equation}
T_{2v} = T_{2k} + T_{1k} - T_{1v} = \SI{18}{\celsius} + \SI{115}{\celsius} - \SI{91.79}{\celsius} = \SI{49.375}{\celsius}
\end{equation}

\subsubsection*{b) $\kappa$* értékének meghatározása, ha a kilépési hőmérsékletek egyenlőek( $T_{1v} = T_{2v}$) }
Az a, feladatban használt egyenletből kifejezhető a $\kappa$* értéke mert már kiszámoltuk a kilépő közegek hőmérsékletét. A $\Delta \dot{Q}= \kappa^* A_{\ddot{0}} \Delta T$ alapegyenletnek a $\Delta T$ tényezője az ellenáramú hőcserélőnek a vízértékei egyenlősége miatt ($\dot{w}_1 = \dot{w}_2$) ,  $\Delta T_N = \Delta T_k = \Delta T = $állandó ezért:
\vspace{1mm}
\begin{equation}
\Delta \dot{Q}=\dot{w}\left(T_{1k} - T_{1v}\right) = \kappa^* A_{\ddot{0}} \Delta T = \kappa A_{\ddot{0}} \left(T_{1v} - T_{2k}\right)
\end{equation}

\begin{equation}
\kappa^*=\dfrac{\dot{w} \left(T_{1k} - T_{1v}\right)}{A_{\ddot{O}} \left(T_{1v} - T_{2k}\right)} =\dfrac{\dot{w} \Delta T}{A_{\ddot{O}} \Delta T}=\dfrac{\dot{w}}{A_\ddot{O}}=\dfrac{\SI{50000}{\watt\per\kelvin}}{\SI{85}{\meter\squared}}=\SI{588.23}{\watt\per\meter\squared\kelvin}
\end{equation}

\vspace{5mm}
\subsubsection*{c)Hőmérséklet-hely függvény}
  A hőmérséklet-hely függvényéből kiszámítható az úgynevezett  logaritmikus középhőmérséklet-különbség, a $\Delta T_{\textit{köz}}$   
\begin{equation}
\Delta T_{\textit{köz}}=\dfrac{\Delta T_N - \Delta T_k}{\ln\frac{\Delta T_N}{\Delta T_k}}
\end{equation}
Ehhez kell a kisebb és a nagyobb hőmérsékletkülönbség, amik egyenlőek jelen esetben. Tehát ez nem meghatározható. Itt, ha a hőmérsékletkülönbségeket behelyettesítjük, ln1-et kapunk, ami 0, illetve a számláló is 0.
\begin{equation}
\Delta T(A) = \Delta T_{N}\mathrm{e}^{0} = \Delta T
\end{equation}
\begin{figure}[h]
\centering
\label{figure:sm}
\begin{tikzpicture}
\pgfmathsetmacro{\L}{4}
\pgfmathsetmacro{\AÖ}{8}
		
\pgfmathsetmacro{\kelvin}{34}
\pgfmathsetmacro{\TAK}{115/\kelvin}
\pgfmathsetmacro{\TAV}{91.79/\kelvin}
\pgfmathsetmacro{\TBK}{18/\kelvin}
\pgfmathsetmacro{\TBV}{41.21/\kelvin}
		
% Tengelyek
\draw[->] (0,-1) -- (0,\L+1) node[anchor=north east]{$T$};
\draw[->] (-1.25,0) -- (\AÖ+1,0) node[anchor=base east, shift={(0,-0.5)}]{$A$};
	
% Az Aö
\draw[gray, dashed] (\AÖ,0) -- (\AÖ,\L+0.5);
\draw (\AÖ,-0.1) -- (\AÖ,0.1);
\node[anchor=base, shift={(0,-0.5)}] at (\AÖ,0) {$A_{\ddot{O}}$};
		
% A két T(A)
\draw[mid arrow=yellow, yellow, ultra thick] (0,\TAK) -- (\AÖ,\TAV);
\draw[mid arrow=green, green, ultra thick] (\AÖ,\TBK) -- (0,\TBV);
		
% A hőmérsékletek
\draw (-0.1,\TAK) -- (0.1,\TAK);
\node[anchor=base east] at (0,\TAK) {$T_{1k}$};
\node[anchor=north east] at (0,\TAK) {$\SI{115}{\celsius}$};
		
\draw (-0.1,\TBV) -- (0.1,\TBV);
\node[anchor=base east] at (0,\TBV) {$T_{2v}$};
\node[anchor=north east] at (0,\TBV) {$\SI{41.21}{\celsius}$};
		
\draw (-0.1+\AÖ,\TBK) -- (0.1+\AÖ,\TBK);
\node[anchor=base west] at (\AÖ,\TBK) {$T_{2k}$};
\node[anchor=north west] at (\AÖ,\TBK) {$\SI{18}{\celsius}$};
		
\draw (-0.1+\AÖ,\TAV) -- (0.1+\AÖ,\TAV);
\node[anchor=base west] at (\AÖ,\TAV) {$T_{1v}$};
\node[anchor=north west] at (\AÖ,\TAV) {$\SI{91.79}{\celsius}$};
		
% A hőmérsékletkülönbség
\pgflength[xb={\AÖ*0.35}, yb={0.75*\TBV+0.35*\TBK}, xa={\AÖ*0.35}, ya={0.65*\TAK+0.35*\TAV}, alim=0, blim=0, ra=0, ny=0]{$\Delta T$};
		
\end{tikzpicture}
\caption{A hőmérséklet-hely függvények.}
\end{figure}


Az ábrán is látszik, hogy a $\Delta T_{N}= \Delta T_{k}= \Delta T$.   Ha $\dot{w}_1 > \dot{w}_2$, akkor $\Delta T_{N} < \Delta T_{k}$ és ha $\dot{w}_1 < \dot{w}_2$, akkor $\Delta T_{N} > \Delta T_{k}$ igaz. Ezekben az esetekben viszont meghatározható a $\Delta T_{\textit{köz}}$.
\pagebreak	





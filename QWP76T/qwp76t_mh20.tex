\section*{MH20}
\begin{tabular}{ | p{2cm} | p{14cm} | } 
	\hline
	Név & Rezi Bence \\ 
	\hline
	Szak & Gépészmérnök \\ 
	\hline
	Félév & 2019/2020 II. (tavaszi) félév \\ 
	\hline
\end{tabular}
\vspace{0.5cm}

\noindent 	Számítsa ki a hőerőmű tápszivattyújának villamos teljesítményét egy \SI{200}{\bar} nyomású és \SI{600}{\celsius} hőmérsékletű, 160 t/h mennyiségű gőzt termelő kazánnál,ha a kompresszió adiabatikus és reverzibilis valamint a tápvíz \SI{0,05}{\bar} nyomású telített folyadék.Határozza meg az előbbi paraméterekkel rendelkező körfolyamat esetén hogy a tápszivattyú teljesítménye hány százaléka a turbina teljesítményének.

\vspace{2mm}
\noindent Mutassa be T-s diagramban a tápszivattyú munkájának megfelelő területet! 

\vspace{5mm}
\noindent Adatok : p0,05-nél :
\begin{equation*}
	i' = \SI{137,83}{\kilo\joule\per\kilogram},
	\quad
	s' = \SI{0,4761}{\kilo\joule\per\kilogram\kelvin}
\end{equation*}
\begin{equation*}
	i'' = \SI{2561}{\kilo\joule\per\kilogram},
	\quad
	s'' = \SI{8,393}{\kilo\joule\per\kilogram\kelvin}
\end{equation*}

\vspace{2mm}
\noindent Adatok : p200-nál :
\begin{equation*}
	T  =\SI{30}{\celsius}
	\quad
	i = \SI{143,9}{\kilo\joule\per\kilogram},
	\quad
	s' = \SI{0,4303}{\kilo\joule\per\kilogram\kelvin}
\end{equation*}
\begin{equation*}
	T  =\SI{40}{\celsius}
	\quad
	i = \SI{185,3}{\kilo\joule\per\kilogram},
	\quad
	s' = \SI{0,5640}{\kilo\joule\per\kilogram\kelvin}
\end{equation*}
\begin{equation*}
	T  =\SI{600}{\celsius}
	\quad
	i = \SI{3530}{\kilo\joule\per\kilogram},
	\quad
	s' = \SI{6,508}{\kilo\joule\per\kilogram\kelvin}
\end{equation*}
\vspace{5mm}

\noindent Megoldás:
\vspace{2mm}

\noindent Adiabatikus tehát s_1=s_2.

\begin{equation}
x_2 = \frac{s_2-s'}{s''-s'} =0,7619
\end{equation}

\begin{equation}
i_2=x_2\cdot i''+(1-x_2)\cdot i_2'=i = \SI{1984,3}{\kilo\joule\per\kilogram}
\end{equation}

\begin{equation}
W_{komp}=i_1-i_2=\SI{1545,7}{\kilo\joule\per\kilogram}
\end{equation}

\begin{equation}
q_{be}=i_1-i_2'=\SI{3392,17}{\kilo\joule\per\kilogram}
\end{equation}

\begin{equation}
q_{el}=i_2-i_2'=\SI{1846,47}{\kilo\joule\per\kilogram}
\end{equation}

\begin{equation}
W_{exp}=q_{be}-q_{el}=\SI{1546,23}{\kilo\joule\per\kilogram}
\end{equation}

\begin{equation}
\frac{W_{komp}}{W_{exp}}=1,3 \%
\end{equation}

\begin{equation}
m= \SI{1,84}{\kilogram\per\second}
\end{equation}

\begin{equation}
P_{szivo} = W \cdot m =\SI{837,5}{\kilo\watt}
\end{equation}






\noindent T-s diagram :
\vspace{2mm}


	\section*{MH20: Tápszivattyú villamos teljesítménye}
	\addcontentsline{toc}{section}{MH20-as feladat}
	\begin{tabular}{ | p{2cm} | p{14cm} | } 
		\hline
		Név & Rezi Bence \\ 
		\hline
		Szak & Gépészmérnöki alapszak \\ 
		\hline
		Félév & 2019/2020 II. (tavaszi) félév \\ 
		\hline
	\end{tabular}
	\vspace{0.5cm}

	\noindent 	Számítsa ki a hőerőmű tápszivattyújának villamos teljesítményét egy \SI{200}{\bar} nyomású és \SI{600}{\celsius} hőmérsékletű, 160 t/h mennyiségű gőzt termelő kazánnál,ha a kompresszió adiabatikus és reverzibilis valamint a tápvíz \SI{0,05}{\bar} nyomású telített folyadék. Határozza meg az előbbi paraméterekkel rendelkező körfolyamat esetén hogy a tápszivattyú teljesítménye hány százaléka a turbina teljesítményének.

	\vspace{2mm}
	\noindent Mutassa be T-s diagramban a tápszivattyú munkájának megfelelő területet! 

	\vspace{5mm}
	\noindent Adatok: $p_L = \SI{0,05}{\bar}$ -nál :
	\begin{equation*}
	h' = \SI{137,83}{\kilo\joule\per\kilogram},
	\quad
	s' = \SI{0,4761}{\kilo\joule\per\kilogram\kelvin}
	\end{equation*}
	\begin{equation*}
	h'' = \SI{2561}{\kilo\joule\per\kilogram},
	\quad
	s'' = \SI{8,393}{\kilo\joule\per\kilogram\kelvin}
	\end{equation*}

	\vspace{2mm}
	\noindent Adatok: $p_F = \SI{200}{\bar}$ -nál :
	\begin{equation*}
	T  =\SI{30}{\celsius}
	\quad
	h_3 = \SI{143,9}{\kilo\joule\per\kilogram},
	\quad
	s' = \SI{0,4303}{\kilo\joule\per\kilogram\kelvin}
	\end{equation*}
	\begin{equation*}
	T  =\SI{40}{\celsius}
	\quad
	h_4 = \SI{185,3}{\kilo\joule\per\kilogram},
	\quad
	s' = \SI{0,5640}{\kilo\joule\per\kilogram\kelvin}
	\end{equation*}
	\begin{equation*}
	T  =\SI{600}{\celsius}
	\quad
	h_1 = \SI{3530}{\kilo\joule\per\kilogram},
	\quad
	s' = \SI{6,508}{\kilo\joule\per\kilogram\kelvin}
	\end{equation*}
	\vspace{5mm}

	\noindent Megoldás:
	\vspace{2mm}

	\noindent A végállapot állapotjelzőinek számolásához szükségünk van az ismert szélsőértékek mellett az $x_2$ fajlagos gőztartalomra is. Az állapotváltozás adiabatikus jellegű, emiatt $s_1 \approx s_2$ (ha reverzibilisnek tekintjük az állapotváltozást, akkor $s_1 = s_2$):

	\begin{equation*}
	x_2 = \frac{s_2-s'}{s''-s'} =0,7619
	\end{equation*}
 

 
	\begin{equation*}
	h_2=x_2\cdot h''+(1-x_2)\cdot h_2'=h = \SI{1984,3}{\kilo\joule\per\kilogram}
	\end{equation*}



	\begin{equation*}
	w_{Sz}=h_1-h_2=\SI{1545,7}{\kilo\joule\per\kilogram}
	\end{equation*}

	 \noindent A végállapotbeli állapotjelzőket felhasználva  meghatározhatjuk a tömegfajlagos munkát melyet  turbina a végez, a tápszivattyú  tömegfajlagos munkájának meghatározásához   szükséges meghatározni a $q_{be}$ és $q_{el}$ -t a kezdeti és végállapotbeli állapotjelzők segítségével:
 
	\begin{equation*}
	q_{be}=h_1-h_2'=\SI{3392,17}{\kilo\joule\per\kilogram}
	\end{equation*}

	\begin{equation*}
	q_{el}=h_2-h_2'=\SI{1846,47}{\kilo\joule\per\kilogram}
	\end{equation*}

	\begin{equation*}
	w_{T}=q_{be}-q_{el}=\SI{1546,23}{\kilo\joule\per\kilogram}
	\end{equation*}

	\noindent A tápszivattyúra és a turbinára is ugyan az a  tömegáram vonatkoztatható így a tömegfajlgos munkák ismeretében számíthatható a teljesítmény illetve a teljesítmények aránya:

	\begin{equation*}
	\frac{w_{Sz}}{w_{T}}=1,3 \%
	\end{equation*}

	\begin{equation*}
	\dot{m}= \SI{1,84}{\kilogram\per\second}
	\end{equation*}

	\begin{equation*}
	P_{szivo} = w_{Sz} \cdot \dot{m} =\SI{837,5}{\kilo\watt}
	\end{equation*}


	\noindent T-s diagram :
	\vspace{2mm}

	\begin{figure}[h]
	\centering
	\label{figure:guh7ud-vgtsd}
	\begin{tikzpicture}

	% A tengelykeresztet az axis környezet hozza létre
	\begin{axis}[
	width=16cm, height=12cm,
	xmin=0, xmax=10.8,
	ymin=0, ymax=650, 
	axis lines = middle,
	axis line style={->},
	xlabel=$s \left(\si{\kilo\joule\per\kilogram\kelvin}\right)$, 
	xlabel style={
		at=(current axis.right of origin), 
		anchor=north east
	}, 
	ylabel=$T \left(\si{\degreeCelsius}\right)$, 
	ylabel style={
		at=(current axis.above origin), 
		anchor=north east
	},
	xtick={1, 2, 3, 4, 5, 6, 7, 8, 9},
	ytick={100, 200, 300, 400 , 500 , 600}
	]
	\node[anchor=south east] at (axis cs: 4.40696, 373.919) {\pgfcircled{$K$}};
	\filldraw[black, fill=black] (axis cs: 4.40696, 373.919) circle (1mm);
	% Az adat az MHFGY Wolfram-jegyzetfüzetből származik
	
	% A nedves gőzmező fázishatárai
	\addplot[thick] table {./QWP76T/ts.txt};
	\addplot[ultra thin] table {./QWP76T/005bar.txt};
	\addplot[ultra thin] table {./QWP76T/200bar.txt};
	
    \node[anchor=south east] at (axis cs: 6.508, 600) {\pgfcircled{$1$}};
	\filldraw[black, fill=black] (axis cs: 6.508, 600) circle (1mm);
	
	\node[anchor=south east] at (axis cs: 0.4761, 32.874) {\pgfcircled{$2'$}};
	\filldraw[black, fill=black] (axis cs: 0.4761, 32.874) circle (1mm);
	
	\node[anchor=south east] at (axis cs: 6.508, 32.874) {\pgfcircled{$2$}};
	\filldraw[black, fill=black] (axis cs: 6.508, 32.874) circle (1mm);
	
	\node[anchor=west] at (axis cs: 8.394, 32.874) {\pgfcircled{$2''$}};
	\filldraw[black, fill=black] (axis cs: 8.394, 32.874) circle (1mm);
	
	
	\addplot[ ultra thin,mid arrow,name path=a] table {./QWP76T/ts2.txt};
	
	\draw[ultra thin,mid arrow] (axis cs:6.508, 32.874  ) -- (axis cs:0.4761, 32.874  );
	\draw[ultra thin,mid arrow] (axis cs:6.508, 600 ) -- (axis cs:  6.508, 32.874 );
	\draw[ultra thin] (axis cs:0.4761, 32.874  ) -- (axis cs: 6.508, 32.874 );
	\draw[ultra thin] (axis cs:0.4761, 32.874  ) -- (axis cs: 0.4761, 40  );
    %\draw[ultra thin,] (axis cs: 3.989 ,364.327) -- (axis cs:  4.966, 364.327 );
	%\addplot[ultra thin,mid arrow, name path=b] table {./QWP76T/ts1.txt};
	%\addplot[gray!30] fill between[of=a and b];
	
	\end{axis}
	
	
	\end{tikzpicture}
	\caption{Víz-gőz $T-s$ diagram}	
	\end{figure}
	\begin{figure}[h]
	\centering
	\label{figure:guh7ud-vgtsd}
	\begin{tikzpicture}
	
	% A tengelykeresztet az axis környezet hozza létre
	\begin{axis}[
	width=16cm, height=12cm,
	xmin=0, xmax=2,
	ymin=0, ymax=50, 
	axis lines = middle,
	axis line style={->},
	xlabel=$s \left(\si{\kilo\joule\per\kilogram\kelvin}\right)$, 
	xlabel style={
		at=(current axis.right of origin), 
		anchor=north east
	}, 
	ylabel=$T \left(\si{\degreeCelsius}\right)$, 
	ylabel style={
		at=(current axis.above origin), 
		anchor=north east
	},
	xtick={0.5,1,1.5, 2},
	ytick={10,20,30,40,60}
	]
	\node[anchor=south east] at (axis cs: 4.40696, 373.919) {\pgfcircled{$K$}};
	\filldraw[black, fill=black] (axis cs: 4.40696, 373.919) circle (1mm);
	% Az adat az MHFGY Wolfram-jegyzetfüzetből származik
	
	% A nedves gőzmező fázishatárai
	\addplot[very thin,] table {./QWP76T/ts.txt};
	\addplot[thick,mid arrow,] table {./QWP76T/ts1.txt};
	\addplot[thick,mid arrow] table {./QWP76T/ts3.txt};
	\addplot[very thin,mid arrow] table {./QWP76T/200bar.txt};
	\node[anchor=south east] at (axis cs:  0.4761, 40) {\pgfcircled{$4$}};
	\filldraw[black, fill=black] (axis cs:  0.4761, 40) circle (1mm);
	\node[anchor=south east] at (axis cs: 0.4761, 32.874) {\pgfcircled{$2'$}};
	\filldraw[black, fill=black] (axis cs: 0.4761, 32.874) circle (1mm);
	\draw[ultra thin,] (axis cs:6.508, 32.874  ) -- (axis cs:  0.4761, 32.874 );
	\draw[thick,mid arrow,name path=a] (axis cs:0.4761, 32.874  ) -- (axis cs: 0.4761, 40  );
	\draw[ultra thin,name path=b] (axis cs: 3.989 ,364.327) -- (axis cs:  4.966, 364.327 );
	\addplot[gray!30] fill between[of=a and b];
	
	
	
	\end{axis}
	
	
	\end{tikzpicture}
	\caption{Tápszivattyú munkája a Víz-gőz $T-s$ diagramon}
\end{figure}

\begin{document}
\section*{7/4. Csőfal hőszigetelésének számítása}


\addcontentsline{toc}{section}{7/4. Csőfal hőszigetelésének számítása}

\begin{tabular}{ | p{3cm} | p{12cm} | } 
	\hline
	Zelovics Szabolcs NS9IZH  & \\  \\
	\hline
	Vegyészmérnök BSC  & \\  \\
	\hline
	Félév & 2019/2020 II. (tavaszi) félév \\ 
	\hline
\end{tabular}
\vspace{0.5cm}

\noindent Pincében falon kívül felszerelt $\SI{1/2}{\inch}$ melegvíz vezetéket samottal szigetelnek. Hogyan változik a hőveszteség a szigetelés vastagságával?$D_k_r_i_t$ = ? . Hogyan változik üvegszálas szigetelés használatakor?

\subsubsection{Adatok}
\begin{equation*}
	$d_k = \SI{0,0215}{\meter}$, $t_f_l = \SI{65}{\celsius}$
	\quad 
	 $\alpha_1 = \SI{11,6}{\watt\per\meter\cubed\kelvin}$, 
	\quad
	$t_l_e_v = \SI{5}{\celsius}$,
	\quad
	$\lambda_s = \SI{0,47}{\watt\per\meter\kelvin},
	\quad
	$l_u = {0,07}{\inch}$
\end{equation*}

\noindent Gyakorlatban a csőfal termikus ellenállása a szigetelés mellet elhanyagoltható, ennek figyelembevételével a lineáris hőáram:
\begin{equation*}
	 \dot{q_l} = \frac{T_1 - T_2}{\frac{ln(\frac{D}{dk})}{2\pi\lambda}+\frac{1}{\alpha D\pi}}
\end{equation}

\noindent A kritikus átmérő 
\begin{equation*}
D_k_r_i_t = \frac{2\lambda}{\alpha}
\end{equation*}

$D_K_s_a_m = \SI{0,081034}{\meter}$

$D_K_u_v = \SI{0,00687}{\meter}$



\begin{tabular}{ | p{2cm} | p{2cm} | p{3cm} | p{3cm} | } 
	\hline
	D & \szigma & q(W/m) & q(W/m) \\
	\hline
	mm & mm & samott & üvegszál \\ 
	\hline
	21,5 & 0 & 47,0164 & 47,0128 \\
	\hline
	41 & 19,5 & 67,5943 & 18,5275 \\
	\hline
	61 & 39,5 & 74,7213 & 13,0449 \\
	\hline
	81 & 59,5 & 76,1487 & 10,6827 \\
	\hline
	101 & 79,5 & 75,4187 & 9,3342 \\
	\hline
	121 & 99,5 & 73,9071 & 8,4485 \\
	\hline
	221 & 199,5 & 65,7028 & 6,3854 \\
	\hline
	321 & 299,5 & 59,9463 & 5,5336 \\
	\hline
	421 & 399,5 & 55,9479 & 5,0413 \\
	\hline
\end{tabular}




\pagebreak

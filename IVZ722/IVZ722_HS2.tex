
\section*{HS 2. feladat: Vízforraló üst acélfalának hőforgalma}
\addcontentsline{toc}{section}{HS 2. feladat}

\begin{tabular}{ | p{2cm} | p{14cm} | } 
	\hline
	Név & Kovács Bence \\ 
	\hline
	Szak &  Mechatronikai Mérnöki\\
	\hline
	Félév & 2019/2020 II. (tavaszi) félév \\ 
	\hline
\end{tabular}
\vspace{0.5cm}


Egy vízforraló üst $\SI{20}{\milli\meter}$ vastag acélfalának vízoldali hőfoka $\SI{100}{\celsius}$ s a falban kialakuló áramsűrüség $\SI{60000}{\watt\per\meter^2}$.

\noindent Adatok:    ($\lambda_a_c = \SI{45}{\watt\per\meter\kelvin}$ ;
                     $\lambda_v_i_z_k_o = \SI{1,8}{\watt\per\meter\kelvin}$)
                     
\vspace{1mm}

\vspace{1mm}
\noindent a: Határozza meg az üst füstgázoldali hőmérsékletét!

\vspace{1mm}
\noindent b: Számítsa ki a falban lévő hőáramsűrüség változását, ha a vízoldalon $\SI{2}{\milli\meter}$ vastag vízkőréteg képződik és a fal két oldalán a hőmérséklet változatlan marad!

\vspace{1mm}
\noindent c: Határozza meg a két felület érintkezésnél kialakult hőmérsékletet
\vspace{1mm}

Adatok:
($ T_v = \SI{100}{\celsius} $ , $\lambda = \SI{45.4}{\watt\per\meter\kelvin}$ , $\delta = \SI{20}{\milli\meter}$ ,  $ q = \SI{60000}{\watt\per\meter^2} $)

\vspace{2mm}

Számolás:


Először felírjuk az áramsűrüségre vonatkozó egyenletet:

\begin{equation}
	 \dot{q} = \frac{\lambda_1}{\delta_1} (T_1 - T_2)
\end{equation}


Majd behelyettesítjük az értékeket:
\begin{equation}
60 \cdot \SI{1000}{\watt\per\meter^2} =  \frac{\SI{45.4}{\watt\per\meter\kelvin}}{\SI{0.02}{\meter}} (T_f_g - T_v)
\end{equation}

Mivel tudjuk, hogy $T_v = \SI{373.15}{\kelvin}$ , behelyettesítjül ezt is.
A zárójel felbontása után, a $T_2$-őt jobb oldalra rendezve az alábbi egyenletet kapjuk:

\begin{equation}
60 \cdot \SI{1000}{\watt\per\meter^2}+  \frac{\SI{45.4}{\watt\per\meter\kelvin}}{\SI{0.02}{\meter}} \cdot \SI{373.15}{\kelvin}=  \frac{\SI{45.4}{\watt\per\meter\kelvin}}{\SI{0.02}{\meter}} \cdot T_2
\end{equation}

\begin{equation}
60 \cdot \SI{1000}{\watt\per\meter^2} + \SI{847050.5}{\watt\per\meter^2} = \SI{2270}{\watt\per\meter^2} \cdot T_2
\end{equation}
\vspace{1mm}

Ebből $T_2$-őt ki tudjuk fejezni, ezzel meg is kapjuk az 'a' kérdésre a választ:

\begin{equation}
T_2 = \SI{399.58}{\kelvin} = \SI{126.43}{\celsius}
\end{equation}

\vspace{1mm}


A 'b' illetve a 'c' feladat megoldásához ismételten felírjuk az áramsűrűség egyenletét:

\begin{equation}
	 \dot{q} = \frac{\lambda_1}{\delta_1} (T_1 - T_w)
\end{equation}

Majd felírjuk az $\dot{q}_1$ illetve $\dot{q}_2$-re vonatkozó egyenleteket. Mivel a vízkőréteg és a fal érintkezésénél nem ismerjük a hőmérsékletet, ezt paraméteresen $T_w$-vel fogjuk jelölni

\begin{equation}
	 \dot{q}_1 =  \frac{\SI{45.4}{\watt\per\meter\kelvin}}{\SI{0.02}{\meter}} (\SI{399.58}{\kelvin} - T_w)
\end{equation}


\begin{equation}
	 \dot{q}_2 =  \frac{\SI{1.8}{\watt\per\meter\kelvin}}{\SI{0.002}{\meter}} (T_w - \SI{373.15}{\kelvin})
\end{equation}

Majd egyszerűsítések után az alábbi két egyenletet kapjuk:

\begin{equation}     
     \dot{q}_1 = \SI{2270}{\watt\per\kelvin\meter^2} \cdot (\SI{399.58}{\kelvin} - T_w)
\end{equation}


 \begin{equation}   
    \dot{q}_2 = \SI{900}{\watt\per\kelvin\meter^2} \cdot (T_w - \SI{373.15}{\kelvin})
\end{equation}

A kapott egyenleteket egyenlővé tesszük egymással
    
\begin{equation}    
     \SI{2270}{\watt\per\kelvin\meter^2} \cdot (\SI{399.58}{\kelvin}-T_2) =  \SI{900}{\watt\per\kelvin\meter^2}(T_w-\SI{373.15}{\kelvin})
\end{equation}

Majd kiszámoljuk és rendezzük az egyenletet:

\begin{equation}
    \SI{1007}{\kelvin}-3 \cdot T_w = T_w-\SI{373.15}{\kelvin}
\end{equation}
        
$T_w$-t kifejezve az egyenletből megkapjuk a felületi érintkezésnél kialakul hőmérsékletet, ami a 'c' kérdésre adott válasz:

\begin{equation}
    T_w = \frac\SI{1380}{\kelvin}3,52} = \SI{392.08}{\kelvin} = \SI{118,93}{\celsius}
\end{equation}
    
Behelyettesítve a kezdeti egyenletbe a kapott $T_w$ értéket, ki tudjuk számolni az áramsűrüséget is, amire a következőt kapjuk:
    
\begin{equation}
     \dot{q} = \SI{900}{\watt\per\K}} \cdot (\SI{392.08}{\kelvin} - \SI{373.15}{\kelvin}) = \SI{17046}{\watt\per\meter^2}
\end{equation}

Kivonva a kapott értéket az erdeteti feladat áramsűrüségéből, megkapjuk, hogy az eredeti feladathoz képest mennyivel változott ez az érték:

\begin{equation}
            60 \cdot 10^3 - 17046 =  \SI{42954}{\watt\per\meter^2}
\end{equation}



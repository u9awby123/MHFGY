
\section*{HS 2. feladat: Vízforraló üst acélfalának hőforgalma}
\addcontentsline{toc}{section}{HS 2. feladat}

\begin{tabular}{ | p{2cm} | p{14cm} | } 
	\hline
	Név & Kovács Bence \\ 
	\hline
	Szak &  Mechatronikai Mérnöki\\
	\hline
	Félév & 2019/2020 II. (tavaszi) félév \\ 
	\hline
\end{tabular}
\vspace{0.5cm}


Egy vízforraló üst $\SI{20}{\milli\meter}$ vastag acélfalának vízoldali hőfoka $\SI{100}{\celsius}$ s a falban kialakuló áramsűrüség $\SI{60000}{\watt\per\meter^2}$.

\noindent Adatok:    ($\lambda_a_c = \SI{45}{\watt\per\meter\K}$ ;
                     $\lambda_v_i_z_k_o = \SI{1,8}{\watt\per\meter\K}$)
                     
\vspace{1mm}

\vspace{1mm}
\noindent a: Határozza meg az üst füstgázoldali hőmérsékletét!


Adatok:
($ T_v = 100 °C $ , $\lambda = \SI{45.4}{\watt\per\meter\K}$ , $\delta = \SI{20}{\milli\meter}$ ,  $ q = \SI{60000}{\watt\per\meter^2} $)

Számolás:

\begin{equation}
60 \cdot 10^3 = \frac{45,4}{0,02} (T_f_g - T_v)
\end{equation}

\begin{equation}
60 \cdot 10^3 + \frac{45,4}{0,02} \cdot 373,15= \frac{45,4}{0,02} \cdot T_2
\end{equation}

\begin{equation}
60 \cdot 10^3 + 847050,5 = 2270 \cdot T_2
\end{equation}

\begin{equation}
T_2 = 399.58K = 126,43 °C
\end{equation}

\vspace{1mm}
\noindent b: Számítsa ki a falban lévő hőáramsűrüség változását, ha a vízoldalon $\SI{2}{\milli\meter}$ vastag vízkőréteg képződik és a fal két oldalán a hőmérséklet változatlan marad!

\vspace{1mm}
\noindent c: Határozza meg a két felületi érintkezésnél kialakult hőmérsékletet

\vspace{1mm}


\begin{equation}
	 {q} = \frac{\lambda_1}{\delta_1} (T_1 - T_w)
\end{equation}


\begin{equation}
	 {q}_1 = \frac{45,4}{0,02} (399,58 - T_w)
\end{equation}


\begin{equation}
	 {q}_2 = \frac{1,8}{0,002} (T_w - 373,15)
\end{equation}


\begin{equation}     
     {q}_1 = \SI{2270}{\watt\per\K} \cdot (399,58 - T_w)
\end{equation}


 \begin{equation}   
    {q}_2 = \SI{900}{\watt\per\K}} \cdot (T_w - 373,15)
\end{equation}

    
\begin{equation}    
    2270 \cdot (399,58-T_2) = 900(T_w-373,15)
\end{equation}


\begin{equation}
    1007-3 \cdot T_w = T_w-373,15
\end{equation}
        
    Felületi érintkezésnél kialakult hőmérséklet (C):

\begin{equation}
    T_w = \frac{1380}{3,52} = \SI{392.08}{\Kelvin} = \SI{118,93}{\celsius}
\end{equation}
    
     Hőáramsűrüség változása (B):
    
\begin{equation}
     {q} = \SI{900}{\watt\per\K}} \cdot (392.08 - 373,15) = \SI{17046}{\watt\per\meter^2}
\end{equation}

\begin{equation}
            60 \cdot 10^3 - 17046 =  \SI{42954}{\watt\per\meter^2}
\end{equation}

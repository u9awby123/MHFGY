\section*{K1/2. Feladat: Levegő melegítése állandó nyomáson}
\addcontentsline{toc}{section}{K1/2. feladat: Levegő melegítése állandó nyomáson}
\begin{tabular}{ | p{2cm} | p{14cm} | } 
	\hline
	Szerző & Kustán Gergely UNXNSY \\ 
	\hline
	Szak & Mechatronikai mérnök  \\ 
	\hline
	Félév & 2019/2020 II. (tavaszi) félév \\ 
	\hline
\end{tabular}
\vspace{0.5cm}

\noindent
Határozzuk meg azt a hőmennyiséget, amely $\SI{180}{\kilogram}$ levegő állandó nyomáson történő felmelegítéséhez szükséges $\SI{300}{\celsius}$ és $\SI{1200}{\celsius}$ hőmérsékletek között.
\subsection*{Adatok}
\begin{equation*}
	m = \SI{180}{\kilogram},
	\quad
	T_1 = \SI{300}{\celsius},
	\quad
	T_2 = \SI{1200}{\celsius} 
\end{equation*}
\noindent A számításhoz szükséges még a kérdezett hőintervallumhoz tartozó közepes fajhőre amit közvetlenül a gőztáblázatból nem lehet leolvasni. A táblázatban csak a $(0-T)$ $\SIUnitSymbolCelsius$ közötti intervallumra vonatkozó értékek találhatók, ill. egy-egy adott hőfokhoz tartozó érték.
A feladathoz szükséges értékek:
\begin{equation*}
	c_{p(0-300)} = c_{p1} = \SI{1,019}{\kilo\joule\per\kilogram\kelvin} 
\end{equation*}

\begin{equation*}
	c_{p(0-1200)} = c_{p2} = \SI{1,108}{\kilo\joule\per\kilogram\kelvin} 
\end{equation*}

\subsection*{Megoldás}
A $Q_{1,2}$ hőmennyiséghez ki először számítani a $300-1200$ $\SIUnitSymbolCelsius$ közötti közepes fajhőt, ami a táblázatból kiolvasott értékek segítségével meghatározható. Mivel az állapotváltozás \textbf{állandó nyomáson történik (izobar)}  ezért fenn áll a következő összefüggés amit átrendezve kiszámolható a közepes fajhő:

\begin{equation}
	\delta{q}=\dif{h}-v\dif{p} 
	\Rightarrow
	\dif{p}=\SI{0}{\pascal} 
\end{equation}

\begin{equation*}
	\delta{q}=c_p \dif{T}
\end{equation*}

\begin{equation}
	q_{1,2} = \int_{T_1}^{T_2}c_{p1,2}dT 
	= 
	c_{p1,2}(T_2-T_1) 
	\Rightarrow 
	c_{p1,2} =\dfrac{q_{1,2}}{T_2-T_1} 
\end{equation}

\noindent
Az $q_{1,2}$ tömegfajlagos hőmennyiség (1.2)-es egyenletből:
\begin{equation}
	q_{1,2} = \int_{0}^{T_2}c_{p2}dT - \int_{0}^{T_1}c_{p1}dT
	= 
	c_{p2}T_2 - c_{p1}T_1
\end{equation}

\subsection*{A közepes fajhő}
\begin{equation}
	c_{p1,2} =\dfrac{c_{p2}T_2 - c_{p1}T_1}{T_2-T_1}
	=
	\SI{1,1377}{\kilo\joule\per\kilogram\kelvin}
\end{equation}

%T-c_p

\usepgfplotslibrary{fillbetween}

\begin{figure}[h]
	\centering
	\label{figure:vgtsd}
	\begin{tikzpicture}
	
	% A tengelykeresztet az axis környezet hozza létre
	\begin{axis}[
	width=16cm, height=12cm,
	xmin=0, xmax=1650,
	ymin=900, ymax=1300, 
	axis lines = middle,
	axis line style={->},
	xlabel=$T \left(\si{\degreeCelsius}\right)$, 
	xlabel style={
		at=(current axis.right of origin), 
		anchor=north east
	}, 
	ylabel=$c_p \left(\si{\joule\per\kilogram\kelvin}\right)$, 
	ylabel style={
		at=(current axis.above origin), 
		anchor=north east
	},
	xtick={0, 200, 400, 600, 800, 1000, 1400},
	ytick={900, 950, 1000, 1050, 1150, 1200, 1250},
	extra x ticks={300, 1200},
	extra x tick labels={$T_1$,$T_2$},
	extra y ticks={1019, 1100, 1108},
	extra y tick labels={$c_{p1}$, $ $, $c_{p2}$},
	]
	
	% Az adat az MHFGY Wolfram-jegyzetfüzetből származik
	
	\addplot[very thick, name path=A] table {./unxnsy/c_p-T 1.txt};
	\addplot[very thick, name path=C] table {./unxnsy/c_p-T 2.txt};
	\addplot[very thick] table {./unxnsy/c_p-T.txt};
	
	% 1-es pont
	\node[anchor=south east] at (axis cs: 300, 1045.11) {\pgfcircled{$1$}};
	\filldraw[black, fill=white] (axis cs: 300, 1045.11) circle (1mm);
	
	%2-es pont
	\node[anchor=south east] at (axis cs: 1200, 1207.37) {\pgfcircled{$2$}};
	\filldraw[black, fill=white] (axis cs: 1200, 1207.37) circle (1mm);
	
	

	%T_1 és T_2 Függőleges vonalak 
	\draw[very thin] (axis cs: 300, 1137.7) -- (axis cs:  300, 0);
	\draw[very thin] (axis cs: 1200, 1207.37) -- (axis cs:  1200, 0);
	
	%Vízszintes fajhő vonalak
	\draw[very thin, name path=B] (axis cs: 0, 1019) -- (axis cs:  300, 1019);
	\draw[very thin] (axis cs: 0, 1108) -- (axis cs:  1200, 1108);
	\draw[very thin, name path=D] (axis cs: 300, 1137.7) -- (axis cs:  1200, 1137.7);
	
	
	\addplot[gray!30] fill between[of=A and B];
	\addplot[gray!30] fill between[of=C and D];

	\node[anchor=south east] at (axis cs: 800, 1200) {\pgf{$c_p(T)$}};
	\draw[very thin, <-] (axis cs: 850, 1162.50) -- (axis cs: 800, 1210);
	


\end{axis}

\end{tikzpicture}
\caption{Izobar $T-c_p$}
\end{figure}


\subsection*{A felmelegítéshez szükséges hőmennyiség}
\begin{equation}
	Q_{1,2} = m \int_{T_1}^{T_2}c_{p1,2}dT = mc_{p1,2}(T_2-T_1) 
	=
	\SI{184307,4}{\kilo\joule}
\end{equation}
\pagebreak

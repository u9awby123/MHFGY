\section*{MH8 Feladat: Csővezetékben áramló vízgőz nyomása}
\addcontentsline{toc}{section}{MH8 Feladat: Csővezetékben áramló vízgőz nyomása}
\begin{tabular}{ | p{2cm} | p{14cm} | } 
	\hline
	Szerző & Kustán Gergely UNXNSY \\ 
	\hline
	Szak & Mechatronikai mérnök  \\ 
	\hline
	Félév & 2019/2020 II. (tavaszi) félév \\ 
	\hline
\end{tabular}
\vspace{0.5cm}

\noindent
Egy csővezetékben $\SI{230}{\celsius}$ hőmérsékletű telített vízgőz áramlik. Milyen nyomást mutat a vezetékre kapcsolt manométer, ha a barométerállás $b=\SI{650}{\milli\meter} Hg$?(kb. $\SI{1200}{\meter}$ tengerszint feletti magasságban)

\subsection*{Adatok}
\begin{equation*}
	T=\SI{230}{\celsius}
	\quad
	b=\SI{650}{\milli\meter} Hg
	\quad
	(\SI{1}{\milli\meter}Hg=\SI{133,32}{\pascal})
\end{equation*}
\noindent
Adatok a vízgőztáblázatból:

\begin{equation*}
	p_1=\SI{27}{\bar}
	\quad
	T_1=\SI{228,06}{\celsius}
\end{equation*}

\begin{equation*}
	p_2=\SI{28}{\bar}
	\quad
	T_2=\SI{230,04}{\celsius}
\end{equation*}
\subsection*{Megoldás}
Ahhoz hogy megtudjuk mennyit mutat a manométer először meg kell határozni a nyomás változást $\SI{1}{\celsius}$-onként:

\begin{equation}
	\Delta{T_1}=T_2-T_1=\SI{1,98}{\celsius}
\end{equation}

\noindent
$\SI{1,98}{\celsius}$ alatt változik a nyomás $\SI{1}{\bar}$-t, ez egy egyszerű arányossággal kiszámítható a $\SI{1}{\celsius}$-onkénti nyomás változás:

\begin{equation}
	\Delta{p_x}=\frac{1}{\Delta{T_1}}1=\SI{0,505}{\bar}
\end{equation}

\noindent
A $\SI{228,06}{\celsius}$-os hőmérséklethez képest $\SI{1,94}{\celsius}$-kal tér el a kérdezett hőfoktól és már tudjuk mennyivel nő a nyomás hőfokonként így ezekből kiszámítható a $p_1$-es értékhez tartozó növekedés:

\begin{equation}
	\Delta{p_y}=\frac{\Delta{T_2}}{1}\Delta{p_x}
	= 
	\SI{0,9797}{\bar}
\end{equation}

\noindent
$\SI{230}{\celsius}$-on a nyomás:
\begin{equation}
	p_{\SI{230}{\celsius}}
	=
	p_1+\Delta{p_y}=\SI{27,9797}{\bar}
\end{equation}

\noindent
A manométeren mutatott nyomás:
\begin{equation*}
	b=\SI{650}{\milli\meter} Hg
 	= 
 	\SI{0,86658}{\bar}
\end{equation*}

\begin{equation}
	p_m=p_{\SI{230}{\celsius}}-b
	=
	\SI{27,11312}{\bar}
\end{equation}

\pagebreak


%p-v izoterm


\usepgfplotslibrary{fillbetween}

\begin{figure}[h]
	\centering
	\label{figure:guh7ud-vgpvd}
	\begin{tikzpicture}
	
	% A tengelykeresztet az axis környezet hozza létre
	\begin{loglogaxis}[
	width=16cm, height=12cm,
	xmin=0.0003, xmax=10000,
	ymin=0.006, ymax=5000, 
	axis lines = middle,
	axis line style={->},
	log origin x=infty,
	log origin y=infty,
	xlabel=$v \left(\si{\meter\cubed\per\kilogram}\right)$, 
	xlabel style={
		at=(current axis.right of origin), 
		anchor=north east
	}, 
	ylabel=$p \left(\si{\bar}\right)$, 
	ylabel style={
		at=(current axis.above origin), 
		anchor=north east
	},
	xtick={0.001, 0.01, 0.1, 1, 10, 1000},
	ytick={0.01, 0.1, 1, 10, 100, 1000},
	extra x ticks={0.0031056,0.05,1.672373,20.540213},
	extra x tick labels={$v_K$,$v_1$,$v_2$,$v_3$},
	extra y ticks={220.64},
	extra y tick labels={$p_K$},
	]
	
	% Az adat az MHFGY Wolfram-jegyzetfüzetből származik
	
	% A nedves gőzmező fázishatárai
	\addplot[thick] table {./guh7ud/pv.txt};
	\addplot[thick] table {./guh7ud/adat50.txt};
	
	%izoterma (100C)
	\addplot[ultra thick,name path=A,,->] table {./guh7ud/100C izoterma.txt};	
	
	% A kritikus pont
	\node[anchor=south] at (axis cs: 0.0031056, 220.64) {$K$};
	\fill[fill=black] (axis cs: 0.0031056, 220.64) circle (0.75mm);
	
	% x jelölések
	\node[anchor=south east] at (axis cs: 0.005, 5) {\pgf{$x=0$}};
	\node[anchor=south east] at (axis cs: 1.2, 5) {\pgf{$x=1$}};
	
	% 1-es pont
	\node[anchor=south east] at (axis cs: 0.05, 1.014180) {\pgfcircled{$1$}};
	\filldraw[black, fill=white] (axis cs: 0.05, 1.014180) circle (1mm);
	
	% 2-es pont
	\node[anchor=south west] at (axis cs: 1.672373, 1.014180) {\pgfcircled{$2$}};
	\filldraw[black, fill=white] (axis cs: 1.672373, 1.014180) circle (1mm);
	
	% 3-as pont
	\node[anchor=south west] at (axis cs: 20.540213, 0.083741) {\pgfcircled{$3$}};
	\filldraw[black, fill=white] (axis cs: 20.540213, 0.083741) circle (1mm);

	% Függőleges vonal
	\draw[very thin] (axis cs: 0.05, 1.014180) -- (axis cs:  0.05, 0.006);
	\draw[very thin] (axis cs: 1.672373, 1.014180) -- (axis cs:  1.672373, 0.006);
	\draw[very thin] (axis cs: 20.540213, 0.083741) -- (axis cs:  20.540213, 0.006);
		
	%vonalkazas segedvonal
	\draw[very thin,name path=B,ultra thin] (axis cs: 0.05, 0.006) -- (axis cs: 20.540213, 0.006);

	%vonalkazas
	\addplot[grey!30] fill between[of=A and B];
	
	%W1-2,W2-3 pontok
	\node[anchor=north east] at (axis cs: 0.55, 0.1) {\pgf{$w_1$}};
	\node[anchor=north] at (axis cs: 6, 0.1) {\pgf{$w_2$}};
	
	
	
1	\end{loglogaxis}
	
	\end{tikzpicture}
	\caption{Példa ábra}
\end{figure}
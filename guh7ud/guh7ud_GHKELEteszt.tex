

\section*{K7/2. feladat: Olajipari hűtő}
\addcontentsline{toc}{section}{K7/2. feladat}

\begin{tabular}{ | p{2cm} | p{14cm} | } 
	\hline
	Név & Szalay István \\ 
	\hline
	Szak & \\ 
	\hline
	Félév & 2019/2020 II. (tavaszi) félév \\ 
	\hline
\end{tabular}
\vspace{0.5cm}

\noindent Egy olajipari hűtőnél mérés útján határozzuk meg a hőátszármaztatási tényezőt, a környezeti hatást, és rajzoljuk le axonometrikusan a hőcserélőt!

A hőcserélő 1-4-es (azaz köpenyoldalon 1-szeres, csőoldalon 4-szeres) átfutású, a csőkötegben hűtővíz, a köpenyoldalon benzin  áramlik. Hőszigetelés nincs, a "hőveszteség", a benzinből a környezetbe távozó hő valójában nyereség, ennyivel kevesebb hűtővíz szükséges. A benzin a hűtött közeg, ezért ez lesz az \ding{172}-es közeg, a víz pedig a \ding{173}-es.

A benzin tömegárama $\dot{m}_B = \SI{66000}{\kilogram\per\hour}$, sűrűsége $\varrho_B = \SI{740}{\kilogram\per\meter\squared}$, fajhője $c_B = \SI{2.26}{\kilo\joule\per\kilogram\kelvin}$. A benzin kezdeti hőmérséklete $T_{1k} = \SI{68}{\celsius}$, véghőmérséklete $T_{1v} = \SI{47}{\celsius}$ (a változás $\SI{21}{\celsius}$).

A víz tömegárama $\dot{m}_V = \SI{45400}{\kilogram\per\hour}$, sűrűsége $\varrho_V = \SI{997}{\kilogram\per\meter\squared}$ ($\SI{23}{\celsius}$ hőmérsékletes), fajhője $c_V = \SI{4.179}{\kilo\joule\per\kilogram\kelvin}$. A víz kezdeti hőmérséklete $T_{2k} = \SI{15}{\celsius}$, véghőmérséklete $T_{2v} = \SI{31}{\celsius}$ (a változás $\SI{16}{\celsius}$).

A névleges hőátadó felület $A_{\ddot{O}} = \SI{100}{\meter\squared}$.

\subsubsection*{a) Készítse el a hőcserélő mérési vázlatát és rajzolja meg a hőmérséklet-hely függvényt!}
A mérési vázlat a hőcserélő főbb jellemzőit ábrázolja, nem feltétlenül a térbeli elhelyezkedésük szerint, hanem a lehető legegyszerűbben. A négyszeres köpenyoldali átfutást például elegendő kiterítve, egy-egy csővel jelölni.

\vspace{-5mm}

\begin{figure}[h]
	\centering
	\begin{tikzpicture}[every path/.style={line cap=rect}]
		\pgfmathsetmacro{\D}{3.5}		% A köpeny átmérője
		\pgfmathsetmacro{\L}{8}			% A köpeny hossza
		\pgfmathsetmacro{\p}{0.2}
		
		\pgfmathsetmacro{\NA}{1}		% Csőoldali átfutás (NEM CSINÁLTAM MEG)
		\pgfmathsetmacro{\NB}{4}		% Köpenyoldali átfutás
		\pgfmathsetmacro{\NT}{4}		% Terelőlemez (LEGYEN PÁROS)
		\pgfmathparse{floor(\NT/2)*2}
		\pgfmathsetmacro{\NT}{\pgfmathresult}
		
		\pgfmathsetmacro{\DCS}{0.3}		% Csonkátmérő
		\pgfmathsetmacro{\RS}{\D/\NB}	% Saroksugár
		\pgfmathsetmacro{\RK}{\D/(\NB+1)/2/3}		% Csősugár
		
		\pgfmathsetmacro{\kOldal}{0}	% Köpeny átfolyás [balra(↑←↰)]/jobbra(↱→↑) [0]/1 (mindig felfele)
		
		% Színek
		\colorlet{Hideg}{blue}
		\colorlet{Meleg}{red}
		
		% Fejek
		\fill[Hideg!25] (0,0) -- (-\D/2+\RS,0) arc[start angle=-90, end angle=-180, radius={\RS}] -- (-\D/2, \D-\RS) arc[start angle=180, end angle=90, radius={\RS}] -- (0, \D);
		\draw[-, ultra thick] (-\p, -\p) -- (-\p,0) -- (-\D/2+\RS,0) arc[start angle=-90, end angle=-180, radius={\RS}] -- (-\D/2, \D-\RS)  arc[start angle=180, end angle=90, radius={\RS}] -- (-\p, \D) -- (-\p, \D+\p);
		
		\fill[Hideg!25] (\L, 0) -- (\L+\D/2-\RS,0) arc[start angle=-90, end angle=0, radius={\RS}] -- (\L+\D/2, \D-\RS)  arc[start angle=0, end angle=90, radius={\RS}] -- (\L, \D);
		\draw[-, ultra thick] (\L+\p, -\p) -- (\L+\p,0) -- (\L+\D/2-\RS,0) arc[start angle=-90, end angle=0, radius={\RS}] -- (\L+\D/2, \D-\RS)  arc[start angle=0, end angle=90, radius={\RS}] -- (\L+\p, \D) -- (\L+\p, \D+\p);
		
		% Csőoldali csonkok
		\fill[Hideg!25, xshift={-\D/2*1cm}, yshift={\D/\NB*(1-0.5)*1cm}, rotate=90] ({-\DCS/2}, -\D/4) -- ({-\DCS/2}, 2*\DCS) -- ({\DCS/2}, 2*\DCS) -- ({\DCS/2}, -\D/4);
		\fill[Hideg!25, xshift={-\D/2*1cm}, yshift={\D/\NB*(\NB-0.5)*1cm}, rotate=90] ({-\DCS/2}, -\D/4) -- ({-\DCS/2}, 2*\DCS) -- ({\DCS/2}, 2*\DCS) -- ({\DCS/2}, -\D/4);
		
		\begin{scope}
			% Körülvágás és a globális stílusok összeakadnak
			\tikzset{every path/.style={}}
			\pgfmathsetmacro{\o}{2*\pgflinewidth}
			\clip (-\D cm,\o pt) -- (-\D/2*1cm+\RS*1cm, \o pt) arc[start angle=-90, end angle=-180, radius={\RS*1cm-\o}] -- (-\D/2*1cm + \o, {\D*1cm-\RS*1cm}) arc[start angle=180, end angle=90, radius={\RS*1cm-\o}] -- (-\D cm, {\D*1cm-\o}) -- cycle;
			
			\draw[ultra thick, xshift={-\D/2*1cm}, yshift={\D/\NB*(1-0.5)*1cm}, rotate=90] ({-\DCS/2}, -\D/4) -- ({-\DCS/2}, 2*\DCS) -- ({-\DCS/2 -\p}, 2*\DCS) -- ({\DCS/2 + \p}, 2*\DCS) -- ({\DCS/2}, 2*\DCS) -- ({\DCS/2}, -\D/4);
			\draw[ultra thick, xshift={-\D/2*1cm}, yshift={\D/\NB*(\NB-0.5)*1cm}, rotate=90] ({-\DCS/2}, -\D/4) -- ({-\DCS/2}, 2*\DCS) -- ({-\DCS/2 -\p}, 2*\DCS) -- ({\DCS/2 + \p}, 2*\DCS) -- ({\DCS/2}, 2*\DCS) -- ({\DCS/2}, -\D/4);
		\end{scope}
		
		% Köpeny
		\fill[Meleg!25] (0,0) -- (\L,0) -- (\L, \D) -- (0, \D) -- cycle;
		\draw[-, ultra thick] (0, -\p) -- (0,0) -- (\L,0) -- (\L, -\p) -- (\L, \D+\p) -- (\L, \D) -- (0, \D) -- (0, \D+\p) -- cycle;
		
		% Köpenyoldali terelőlemezek
		\foreach \i in {1,...,\NT}{
			\pgfmathparse{Mod(\i,2)==\kOldal?1:0}
			\ifnum\pgfmathresult>0
				\draw[ultra thick] ({\L/(1+\NT)*\i}, 0) -- ({\L/(1+\NT)*\i}, 2/3*\D);
			\else
				\draw[ultra thick] ({\L/(1+\NT)*\i}, 1/3*\D) -- ({\L/(1+\NT)*\i}, \D);
			\fi
			}
		
		% Köpenyoldali csonkok + FELIRATOK
		\ifnum\kOldal=0
			% balra(↑←↰)
			\pgfmathsetmacro{\dxf}{\L/(1+\NT)/2*1cm}
			\pgfmathsetmacro{\dxa}{\L cm -\L/(1+\NT)/2*1cm}
		\else
			% jobbra(↱→↑)
			\pgfmathsetmacro{\dxf}{(\L-\L/(1+\NT)/2)*1cm}
			\pgfmathsetmacro{\dxa}{\L/(1+\NT)/2 * 1cm}
		\fi
		
		\fill[Meleg!25, xshift=\dxf, yshift=\D cm] ({-\DCS/2}, -\p) -- ({-\DCS/2}, 2*\DCS) -- ({\DCS/2}, 2*\DCS) -- ({\DCS/2}, -\p);
		\draw[ultra thick, xshift=\dxf, yshift=\D cm] ({-\DCS/2}, 0) -- ({-\DCS/2}, 2*\DCS) -- ({-\DCS/2 -\p}, 2*\DCS) -- ({\DCS/2 + \p}, 2*\DCS) -- ({\DCS/2}, 2*\DCS) -- ({\DCS/2}, 0);
		
		\fill[Meleg!25, xshift=\dxa, rotate=180] ({-\DCS/2}, -\p) -- ({-\DCS/2}, 2*\DCS) -- ({\DCS/2}, 2*\DCS) -- ({\DCS/2}, -\p);
		\draw[ultra thick, xshift=\dxa, rotate=180] ({-\DCS/2}, 0) -- ({-\DCS/2}, 2*\DCS) -- ({-\DCS/2 -\p}, 2*\DCS) -- ({\DCS/2 + \p}, 2*\DCS) -- ({\DCS/2}, 2*\DCS) -- ({\DCS/2}, 0);
		
		% Csőkötegfalak
		\draw[-, ultra thick] (-\p/2, -\p) -- (-\p/2, \D+\p);
		\draw[-, ultra thick] (\L+\p/2, -\p) -- (\L+\p/2, \D+\p);
		
		% Csövek
		\foreach \i in {1,...,\NB}{
			\fill[Hideg!25, yshift={\D/\NB*(\i-0.5)*1cm}] (-\p, -\RK) -- (\L+\p, -\RK) -- (\L+\p, \RK) -- (-\p, \RK);
			\draw[ultra thick, yshift={\D/\NB*(\i-0.5)*1cm}] (0, -\RK) -- (\L, -\RK);
			\draw[ultra thick, yshift={\D/\NB*(\i-0.5)*1cm}] (\L, \RK) -- (0, \RK);
			}
		
		% Csőoldali terelőlemezek
		\foreach \i in {2,...,\NB}{
			\pgfmathparse{Mod(\i,2)==0}
			\ifnum\pgfmathresult=1
				\draw[ultra thick, yshift={\D/\NB*(\i-1)*1cm}] (0, 0) -- ({-\D/2}, 0);
			\else
				\draw[ultra thick, yshift={\D/\NB*(\i-1)*1cm}] (\L, 0) -- ({\L+\D/2}, 0);
			\fi
			}
		
		% Csőoldali átfutás
		\draw[->, Hideg, dashed, ultra thick, yshift={\D/\NB/2*1cm}] (-3*\D/8, 0) -- (0, 0);
		\draw[<-, Hideg, dashed, ultra thick, yshift={\D/\NB/2*(2*\NB-1)*1cm}] (-3*\D/8, 0) -- (0, 0);
		\foreach \i in {2,...,\NB}{
			\pgfmathparse{Mod(\i,2)==0}
			\ifnum\pgfmathresult=0
				\draw[<-, Hideg, dashed, ultra thick, yshift={\D/\NB*(\i-1)*1cm}, rotate=180] (0, -\D/\NB/2) -- (\D/10, -\D/\NB/2) arc[radius={\D/\NB/2}, start angle=-90, end angle=90];
			\else
				\draw[->, Hideg, dashed, ultra thick, xshift={\L cm}, yshift={\D/\NB*(\i-1)*1cm}] (\D/10, -\D/\NB/2) arc[radius={\D/\NB/2}, start angle=-90, end angle=90] -- (0, \D/\NB/2);
			\fi
			}
		
		% Köpenyoldali átfutás
		\ifnum\kOldal=1
			\foreach \i in {1,...,\NT}{
				\pgfmathparse{Mod(\i,2)==1}
				\ifnum\pgfmathresult=1
					\draw[<-, Meleg, dashed, ultra thick, xshift={\L/(1+\NT)*\i * 1cm}, yshift={2/3*\D*1cm}] ({\L/(1+\NT)/2}, {-\L/(1+\NT)/2}) -- ({\L/(1+\NT)/2}, 0) arc[radius={\L/(1+\NT)/2}, start angle=0, end angle=180];
				\else
					\draw[->, Meleg, dashed, ultra thick, xshift={\L/(1+\NT)*\i * 1cm}, yshift={1/3*\D*1cm}, rotate=180] ({\L/(1+\NT)/2}, 0) arc[radius={\L/(1+\NT)/2}, start angle=0, end angle=180] -- ({-\L/(1+\NT)/2}, {-\L/(1+\NT)/2});
				\fi
				}
		\else
			\foreach \i in {1,...,\NT}{
				\pgfmathparse{Mod(\i,2)==0}
				\ifnum\pgfmathresult=1
					\draw[->, Meleg, dashed, ultra thick, xshift={\L/(1+\NT)*\i * 1cm}, yshift={2/3*\D*1cm}] ({\L/(1+\NT)/2}, 0) arc[radius={\L/(1+\NT)/2}, start angle=0, end angle=180] -- ({-\L/(1+\NT)/2}, {-\L/(1+\NT)/2});
				\else
					\draw[<-, Meleg, dashed, ultra thick, xshift={\L/(1+\NT)*\i * 1cm}, yshift={1/3*\D*1cm}, rotate=180] ({\L/(1+\NT)/2}, {-\L/(1+\NT)/2}) -- ({\L/(1+\NT)/2}, 0) arc[radius={\L/(1+\NT)/2}, start angle=0, end angle=180];
				\fi
				}
		\fi
		
		% FELIRATOZÁS
		
		% Csőoldal
		\draw[<-,Hideg, ultra thick, xshift={-\D/2*1cm-2*\DCS*1cm-\p*1cm}, yshift={\D/\NB/2*(2*\NB-1)*1cm}] (-3*\D/8, 0) -- (0, 0);
		\node[anchor=south, xshift={-\D/2*1cm-2*\DCS*1cm-\p*1cm}, yshift={\D/\NB/2*(2*\NB-1)*1cm}] at (-3*\D/16, 0) {\ding{173}};
		\node[anchor=north, xshift={-\D/2*1cm-2*\DCS*1cm-\p*1cm}, yshift={\D/\NB/2*(2*\NB-1)*1cm}] at (-3*\D/16, 0) {$T_{2v}$};
		
		\draw[->,Hideg, ultra thick, xshift={-\D/2*1cm-2*\DCS*1cm}, yshift={\D/\NB/2*1cm}] (-3*\D/8, 0) -- (0, 0);
		\node[anchor=south, xshift={-\D/2*1cm-2*\DCS*1cm-\p*1cm}, yshift={\D/\NB/2*1cm}] at (-3*\D/16, 0) {\ding{173}};
		\node[anchor=north, xshift={-\D/2*1cm-2*\DCS*1cm-\p*1cm}, yshift={\D/\NB/2*1cm}] at (-3*\D/16, 0) {$T_{2k}$};
		
		% Köpenyoldal
		% Baloldal fent
		\draw[->, Meleg, ultra thick, xshift=\dxf, yshift={\D cm+2*\DCS cm+\p cm}] (0, 0) -- (0, 1);
		\node[anchor=north west, xshift=\dxf, yshift={\D cm+2*\DCS cm+\p cm+0.5cm}] at (0, 0) {$T_{1v}$};
		\node[anchor=north east, xshift=\dxf, yshift={\D cm+2*\DCS cm+\p cm+0.5cm}] at (0, 0) {\ding{172}};
		
		% Jobboldal lent
		\draw[->, Meleg, ultra thick, xshift=\dxa, yshift={-2*\DCS cm-\p cm - 1cm}] (0, 0) -- (0, 1);
		\node[anchor=north west, xshift=\dxa, yshift={-2*\DCS cm-\p cm - 0.5cm}] at (0, 0) {$T_{1k}$};
		\node[anchor=north east, xshift=\dxa, yshift={-2*\DCS cm-\p cm - 0.5cm}] at (0, 0) {\ding{172}};
		
		% TA HŐMÉRSÉKLET-HELY FÜGGVÉNY
		\pgfmathsetmacro{\L}{4.75}
		\pgfmathsetmacro{\AÖ}{8}
		
		\pgfmathsetmacro{\DY}{-7.5 cm}
		
		\pgfmathsetmacro{\kelvin}{13}
		\pgfmathsetmacro{\TAK}{68/\kelvin}
		\pgfmathsetmacro{\TAV}{47/\kelvin}
		\pgfmathsetmacro{\TBK}{15/\kelvin}
		\pgfmathsetmacro{\TBV}{31/\kelvin}
		
		\begin{scope}
			\tikzset{every path/.style={yshift=\DY}}
			
			% Tengelyek
			\draw[->] (0,-1) -- (0,\L+1.5) node[anchor=north east]{$T$};
			\draw[->] (-1.25,0) -- (\AÖ+1,0) node[anchor=base east, shift={(0,-0.5)}]{$A$};
			
			% Az összes felület
			\draw[gray, dashed] (\AÖ,0) -- (\AÖ,\L+0.5);
			\draw (\AÖ,-0.1) -- (\AÖ,0.1);
			\node[anchor=base, shift={(0,-0.5)}] at (\AÖ,0) {$A_{\ddot{O}}$};
			
			% A hőmérséklet értékek
			\draw[gray, dashed] (0,\TAK) -- (\AÖ+0.5,\TAK);
			\draw (-0.1,\TAK) -- (0.1,\TAK);
			\node[anchor=base east] at (0,\TAK) {$T_{1k}$};
			\node[anchor=north east] at (0,\TAK) {$\SI{68}{\celsius}$};
			
			\draw[gray, dashed] (0,\TAV) -- (\AÖ+0.5,\TAV);
			\draw (-0.1,\TAV) -- (0.1,\TAV);
			\node[anchor=base east] at (0,\TAV) {$T_{1v}$};
			\node[anchor=north east] at (0,\TAV) {$\SI{48}{\celsius}$};
			
			\draw[gray, dashed] (0,\TBK) -- (\AÖ+0.5,\TBK);
			\draw (-0.1,\TBK) -- (0.1,\TBK);
			\node[anchor=base east] at (0,\TBK) {$T_{2k}$};
			\node[anchor=north east] at (0,\TBK) {$\SI{15}{\celsius}$};
			
			\draw[gray, dashed] (0,\TBV) -- (\AÖ+0.5,\TBV);
			\draw (-0.1,\TBV) -- (0.1,\TBV);
			\node[anchor=base east] at (0,\TBV) {$T_{2v}$};
			\node[anchor=north east] at (0,\TBV) {$\SI{31}{\celsius}$};
			
			\pgflength[xb=\AÖ, yb=\TBV, xa=\AÖ, ya=\TAK, ra=-0.65, ny=0]{$\Delta T_N$}
			\pgflength[xa=-\AÖ/8, ya=\TBK, xb=-\AÖ/8, yb=\TAV, ra=-0.65, ny=0]{$\Delta T_K$}
			
		\end{scope}
		
		% A többszörös átfutás
		\foreach \i in {1,...,\NB}{
			\pgfmathparse{Mod(\i,2)==0}
			\pgfmathsetmacro{\j}{2*\AÖ/\NB * floor(\i/2)}
			\ifnum\pgfmathresult=0
				\draw[dashed, ultra thick, color=cyan, mid arrow=cyan, domain=0:\AÖ, smooth, variable=\A, yshift=\DY] plot 
				(\A, {\TBV - (\TAK-\TBV)*3.5*(1 - exp(-0.00133*(\AÖ-(\j + \A/\NB))*100/\AÖ) )});
			\else
				\draw[dashed, ultra thick, color=cyan, mid arrow=cyan, domain=\AÖ:0, smooth, variable=\A, yshift=\DY] plot 
				(\A, {\TBV - (\TAK-\TBV)*3.5*(1 - exp(-0.00133*(\AÖ-(\j - \A/\NB))*100/\AÖ) )});
			\fi
			}
		
		% A két T(A)
		\draw[ultra thick, color=red, mid arrow=red, domain=\AÖ:0, smooth, variable=\A, yshift=\DY] plot (\A, {\TAK - (\TAK-\TBV)/0.213816*(1 - exp(-0.00133*(\AÖ-\A)*100/\AÖ) )});
		\draw[ultra thick, color=blue, mid arrow=blue, domain=0:\AÖ, smooth, variable=\A, yshift=\DY] plot (\A, {\TBV - (\TAK-\TBV)*3.5*(1 - exp(-0.00133*(\AÖ-\A)*100/\AÖ) )});
		
		% FELIRATOK
		\node[draw=cyan, ultra thick, dashed, yshift=\DY] at (\AÖ/4, \TBV+0.5) {valós $T_2\!\left(A\right)$};
		\node[draw=blue, ultra thick, yshift=\DY, text width=2.2cm, align=center] at (3*\AÖ/4, \TBV+0.65) {ellenáramú helyettesítés};
		
		
	\end{tikzpicture}
	\caption{A hőcserélő mérési vázlata és a hőmérséklet-hely függvények.}
	\label{figure:K72A}
\end{figure}

A többszörös csőoldali átfutás miatt a hőcserélő vegyesáramú, de a hőmérséklet-hely függvény helyettesíthető egy ellenáramúval.

%\subsubsection*{b) Rajzolja le a hőcserélő robbantott ábráját!}
%[KÉSŐBB]

\subsubsection*{c) A vegyesáramú hőcserélő átszármaztatott hőárama}
A vegyesáramú hőcserélő átszármaztatott hőáram az alábbi összefüggéssel számolható:
\begin{equation}
	\dot{Q}_{\acute{a}tsz} = \kappa A_{\ddot{O}} \Delta T_{k\ddot{o}z,ln} F_t
\end{equation}
ahol $F_t$ a logaritmikus közepes hőmérsékletkülönbség vegyesáram esetén szükséges helyesbítő tényezője.

\subsubsection*{d) A logaritmikus közepes hőmérsékletkülönbség (LMTD)}
A logaritmikus közepes hőmérsékletkülönbség (LMTD: logarithmic mean temperature difference) számításához szükség van a kisebb és nagyobb hőmérsékletkülönbségre. Ezek leolvasásához a $T_2\!\left(A\right)$ hőmérséklet-hely függvényt a megfelelő ellenáramúval kell helyettesíteni (lásd \hyperref[figure:K72A]{\ref{figure:K72A}. ábra}).
\begin{equation}
	\Delta T_{k\ddot{o}z,ln} 
	= 
	\dfrac{\Delta T_N - \Delta T_K}{\ln\dfrac{\Delta T_N}{\Delta T_K}} 
	= 
	\dfrac{\SI{37}{\celsius} - \SI{33}{\celsius}}{\ln\dfrac{\SI{37}{\celsius}}{\SI{33}{\celsius}}} 
	= 
	\SI{34.44}{\celsius} = \SI{34.44}{\kelvin}
\end{equation}

A logaritmikus közepes hőmérsékletkülönbség egy hőmérsékletkülönbség, ezért Celsius-fokban és kelvinben azonos az értéke.

\subsubsection*{e) Az $F_t$ helyesbítő tényező}
Az $F_t$ leolvasásához ki kell számítani a hőmérsékletviszonyokat jellemző $R$ és $S$ hányadosokat. Az $R$ a meleg és hideg közeg hőmérsékletváltozásának hányadosa:
\begin{equation}
	R = \dfrac{\Delta T_1}{\Delta T_2} = \dfrac{T_{1k} - T_{1v}}{T_{2v} - T_{2k}} = \SI{1.31}{}
\end{equation}

Az $S$ a hidegebb közeg hőmérsékletváltozásának és a legnagyobb hőmérsékletkülönbségnek (általában a belépő hőmérsékletek különbségének) hányadosa:
\begin{equation}
	S = \dfrac{\Delta T_2}{\Delta T_{max}} = \dfrac{T_{2v} - T_{2k}}{T_{1k} - T_{2k}} = \SI{0.3}{}
\end{equation}

Az $F_t$ leolvasható értéke 0,95.

\subsubsection*{f) Az átszármaztatott hőáram és a hőátszármaztatási tényező}
Az átszármaztatott hőáramot a hűtővíz által felvett hőárammal tekinthetjük azonosnak, a benzin által leadott hőáram ennél több, mivel a benzin hőjének egy része a környezetbe távozik.
\begin{equation}
	\dot{Q}_{\acute{a}tsz} = \Delta \dot{Q}_{V} = \dot{m}_V c_V \Delta T_2 = \dot{m}_V c_V \left(T_{2v} - T_{2k}\right) = \SI{843.23}{\kilo\watt}
\end{equation}

Innen a hőátszármaztatási tényező:
\begin{equation}
	\kappa = \dfrac{\dot{Q}_{\acute{a}tsz}}{A_{\ddot{O}} \Delta T_{k\ddot{o}z, ln} F_t} = \SI{257.73}{\watt\per\meter\squared\kelvin}
\end{equation}

\subsubsection*{g) A környezeti hatás}
A "hasznos hőveszteségként" megnyilvánuló környezeti hatást a benzin által leadott és a víz által felvett hőáramok különbségeként tudjuk számolni. A benzin által leadott hőáram:
\begin{equation}
	\Delta \dot{Q}_{B} = \dot{m}_B c_B \Delta T_1 = \dot{m}_B c_B \left(T_{1k} - T_{1v}\right) = \SI{870.1}{\kilo\watt}
\end{equation}

A környezeti hatás:
\begin{equation}
	\Delta \dot{Q}_{B} - \Delta \dot{Q}_{V} = \SI{26.9}{\kilo\watt}
\end{equation}

\pagebreak

% A feladat címe automatikus számozás nélkül
\section*{K1/1. feladat: Levegő térfogatárama}

% Hozzáadás a tartalomjegyzékhez azonos címmel
\addcontentsline{toc}{section}{K1/1. feladat: Levegő térfogatárama}


% Táblázat a szerző adataival
\begin{tabular}{ | p{2cm} | p{14cm} | } 
	\hline
	Szerző & Kocsor Péter Ernő B00RZ7 \\ 
	\hline
	Szak & Mechatronikai mérnöki alapszak \\ 
	\hline
	Félév & 2019/2020 II. (tavaszi) félév \\ 
	\hline
\end{tabular}
\vspace{0.5cm}

% A feladat szövege
\noindent Áramló gáz mennyiségét a következő termodinamikai módszerrel is mérhetjük: a mérendő gázáramba egy villamos fűtőtestet építünk be, mely a gázzal ismert hőmennyiséget közöl és közben - a fűtött szakasz előtt és után - termoelemmel mérjük a gáz hőfokváltozását. A termoelem " hideg" pontját a fűtött szakasz előtt, a "meleg" pontját a fűtött szakasz után helyezzük el.

\subsubsection*{a) Számítsuk ki egy $A_{} = \SI{3e-3}{\meter\squared}$  keresztmetszetű csövön /$ \diameter Q_{}= \SI{61,8}{\milli\meter}$/ átáramló \textit{levegő sebességét és térfogatáramát} , ha a fűtőtest teljesítményfelvétele $\SI{0,57}{\kilo\watt}$, a termoelemmel mért feszültség-különbség $\SI{0,63}{\milli\volt}$ ($1 fok=\SI{0,042}{\milli\volt}$). A fűtött szakasz előtt $\SI{42,7}{\celsius} $ a hőmérséklet. Az áramlás a levegő állandó, $\SI{0.981}{\bar}$ nyomása mellett történik.}


\begin{figure}[h]
	\centering
	\label{figure:sm}
	\begin{tikzpicture}
	%\draw[step=1cm, gray, very thin] (-6, -2) grid (11, 8);
	%felső téglalap
	\draw[ultra thick]  (-5, 4.5) -- (5, 4.5);
	\draw[ultra thick]  (5, 5) -- (-5, 5);
%	\fill[pattern={Lines[angle=45, distance=2mm]}] (-5, 5) rectangle (5, 4.5);
%	\fill[pattern={Lines[angle=-45, distance=2mm]}] (-5, 5) rectangle (5, 4.5);
	%alsó téglalap
	\draw[ultra thick]  (-5, 0.5) -- (5, 0.5);
	\draw[ultra thick]  (5, 0) -- (-5, 0);
%	\fill[pattern={Lines[angle=45, distance=2mm]}] (-5, 0) rectangle (5, 0.5);
%	\fill[pattern={Lines[angle=-45, distance=2mm]}] (-5, 0) rectangle (5, -0.5);
	\draw[dash pattern={on 8mm off 1mm on 0.5mm off 1mm}] (-6, 2.5) -- (6, 2.5);
	%fűtőberendezés
	\draw[ultra thick] (0.5,0.5)--(0.5,4);
	\draw[ultra thick] (0.5,4)--(-0.5,4);
	\draw[ultra thick] (-0.5,4)--(0,3.5);
	\draw[ultra thick] (-0.5,3)--(0,3.5);
	\draw[ultra thick] (-0.5,3)--(0,2.5);
	\draw[ultra thick] (-0.5,2)--(0,2.5);
	\draw[ultra thick] (-0.5,2)--(0,1.5);
	\draw[ultra thick] (-0.5,1)--(0,1.5);
	\draw[ultra thick] (-0.5,1)--(0,0.5);
	
	\draw[ultra thick] (0,0.5)--(0,-0.75);
	\draw[ultra thick] (0.5,0.5)--(0.5,-0.75);
	\draw[ultra thick] (0.3,-0.75)--(0.5,-0.75);
	\draw[ultra thick] (0.2,-0.75)--(0,-0.75);
	\draw[ultra thick] (0.2,-0.5)--(0.2,-1);
	\draw[ultra thick] (0.3,-0.25)--(0.3,-1.25);
	%termoelem
	\draw[ultra thick] (0, 7) circle (1);
	\draw (1,7)--(2,7);
	\draw (2,7)--(2,2.75);
	\draw (2,2.5) circle(0.25);
	\node[anchor=north] at (-2.5, 3) {$H$};
	\node[anchor=north] at (2.5, 3) {$M$};
	
	\draw (-1,7)--(-2,7);
	\draw (-2,7)--(-2,2.75);
	\draw (-2,2.5) circle(0.25);
	
	\draw[->, ultra thick] (-1, 6) -- (1, 8) node[midway, anchor=north west, xshift={20mm}, yshift={8mm}]{$\Delta U_{}$};
	
	\draw[ultra thick] (-2+0.17677669,2.5+0.17677669)--(-1,5.5);
	\draw[ultra thick] (2+0.17677669,2.5+0.17677669)--(3,5.5);
	\draw[ultra thick] (3,5.5)--(-1,5.5);
	
	%random feliratok :'(
	\draw[<->] (4, 0.5) -- (4, 4.5);
	\node[anchor=north] at (3.5, 2) {$\diameter Q$};
	\draw (4.25,3)--(5,4);
	\node[anchor=north] at (5.25, 4.25) {$A$};
	
	\draw[->, ultra thick] (-5, 2) -- (-3, 2) node[midway, anchor=north west, xshift={0mm}, yshift={-5mm}]{$ w_{}$};
	
	\draw (-0.25,1.5)--(-1.5,1);
	\node[anchor=east] at (-1.5,1) {$\dot{Q}$};
	
	%oldalsó cuccosok
	\draw [ultra thick] (-4.5, 4) arc [radius=0.5, start angle=0, end angle=180];
	\draw [ultra thick] (-4.5, 4) arc [radius=2.5, start angle=0, end angle=-36.8698976458];
	\draw [ultra thick] (-5, 2.5) arc [radius=2.5, start angle=180-36.8698976458, end angle=180];
	\draw [ultra thick] (-5.5, 1) arc [radius=0.5, start angle=180, end angle=270];
	\draw [ultra thick] (-5.5, 4) arc [radius=2.5, start angle=180, end angle=180+36.8698976458];
	
	\draw [ultra thick] (4.5, 1) arc [radius=0.5, start angle=180, end angle=360];
	\draw [ultra thick] (4.5, 1) arc [radius=2.5, start angle=180, end angle=180-36.8698976458];
	\draw [ultra thick] (5, 2.5) arc [radius=2.5, start angle=36.8698976458, end angle=0];
	\draw [ultra thick] (5, 2.5) arc [radius=2.5, start angle=0-36.8698976458, end angle=0];
	\draw [ultra thick] (5.5, 4) arc [radius=0.5, start angle=0, end angle=90];
	
	%adatok1
	\node[anchor=west] at (5.5,7) {Adatok:};
	\node[anchor=west] at (7,7) {$\dot{Q_{}}=\SI{0.57}{\kilo\watt}$};
	\node[anchor=west] at (7,6.5) {$A_{}=\SI{3e-3}{\meter\squared}$};
	\node[anchor=west] at (7,6) {$\Delta U_{}=\SI{0.63}{\milli\volt}$};
	\node[anchor=west] at (7,5.5) {(1 fok $=\SI{0.042}{\milli\volt}$)};
	\node[anchor=west] at (7,5) {$ T_{H}=\SI{42.7}{\celsius}$};
	\node[anchor=west] at (7,4.5) {$p_{} = \SI{0.981}{\bar}= áll.$};
	
	%adatok2
	\node[anchor=west] at (6,3.5) {levegő fajhője (táblázatból):};
	\node[anchor=west] at (6,2.5) {$c_{p_{0}}= \SI{1.003}{\kilo\joule\per\kilo\gram\kelvin} .. \SI{0}{\celsius}-on $};
	\node[anchor=west] at (6,1.5) {$c_{p_{100}}= \SI{1.010}{\kilo\joule\per\kilo\gram\kelvin} .. \SI{100}{\celsius}$};
	
	%adatok3
	\node[anchor=west] at (6,0.5) {levegő sűrűsége:};
	\node[anchor=west] at (5,-0.25) {$\varrho_{0}= \SI{1.2928}{\kilo\gram\per\meter\cubed}/ \SI{0}{\celsius},\SI{760}{\milli\meter}  Hg/$};
	\node[anchor=west] at (6,-1.25) { $ w_{} = $ ? $  \si{\meter\per\second} $ };
	\node[anchor=west] at (8,-1.25) { $ \dot{V_{}}= $ ? $ \si{\meter\cubed\per\second} $};
	
	\end{tikzpicture}
\end{figure}


% Oldaltörés
\pagebreak
\noindent A megoldáshoz először kiszámoljuk a $\Delta T$-t, majd a mért levegő közepes hőkapacitását és sűrűségét.
\begin{equation*}
\Delta T=\dfrac{\SI{0.63}{\milli\volt}}{\SI{0.042}{\milli\volt}}=\SI{15}{\celsius}
\end{equation*}
\begin{equation*}
T_{M}=T_{H}+ \Delta T = \SI{57.7}{\celsius} ( \SI{330,85}{\kelvin})
\end{equation*}
\noindent A hőkapacitás hőmérséklet függő változását vehetjük közelítően lineáris változásnak, így a mért levegő közepes hőkapacitását a következő módon számoljuk:
\begin{equation*}
c_{kzp}=\dfrac{c_{p_{100}}-c_{p_{0}}}{\SI{100}{\celsius}} \cdot \dfrac{T_{M}+T_{H}}{2} + c_{p_{0}}= \SI{1.006514}{\kilo\joule\per\kilo\gram\kelvin}
\end{equation*}
\noindent A sűrűségszámításhoz összegyűjtjük az összes állapot jellemzőit (nyomás és hőmérséklet), majd az egyesített gáztörvényt, valamint a sűrűség és térfogat közötti fordított arányosságot felhasználva megadjuk az egyes állapotokhoz tartozó sűrűséget. 

\begin{tabular}{ | p{3 cm} | p{6 cm} | p{5.5 cm} | } 
	\hline
	Sűrűségállapot & Hőmérséklet & Nyomás \\ 
	\hline
	$\varrho_{0}$ & $\SI{0}{\celsius} (\SI{273.15}{\kelvin}) $ & $ \SI{760}{\milli\meter}Hg (\SI{101325}{\pascal}) $ \\ 
	\hline
	$\varrho_{1}$ & $\SI{42.7}{\celsius} (\SI{315.85}{\kelvin}) $ & $ \SI{0.981}{\bar} (\SI{98100}{\pascal}) $ \\ 
	\hline
	$\varrho_{2}$ & $\SI{57.7}{\celsius} (\SI{330.85}{\kelvin}) $ & $ \SI{0.981}{\bar} (\SI{98100}{\pascal}) $ \\ 
	\hline
\end{tabular}
\vspace{0.5cm}
\begin{equation*}
\dfrac{p_{1} \cdot V_{1}}{T_{1}}=\dfrac{p_{2} \cdot V_{2}}{T_{2}}
\end{equation*}
\noindent Mivel $m_{}= V_{} \cdot \varrho_{}$  és $m_{1}=m_{2}$, így:
\begin{equation*}
\dfrac{1}{T_{p_{1}}\cdot \varrho_{1}}=\dfrac{p_{2}}{T_{2}\cdot \varrho_{2}}
\end{equation*}
\begin{equation*}
\dfrac{p_{1} \cdot T_{2} }{T_{1} \cdot p_{2}}=\dfrac{\varrho_{1}}{\varrho_{2}}
\end{equation*}
\noindent Az így kapott eredmények:
\begin{equation*}
\varrho_{1}=\SI{1.08244}{\kilo\gram\per\meter\cubed}
\end{equation*}
\begin{equation*}
\varrho_{2}=\SI{1.03337}{\kilo\gram\per\meter\cubed}
\end{equation*}
\begin{equation*}
\varrho_{kzp}=\SI{1.0579}{\kilo\gram\per\meter\cubed}
\end{equation*}
\noindent Ezután felírjuk a levegő szállításos hőáramának növekedésére vonatkozó egyenletet:
\begin{equation*}
\dot{Q}= \Delta T \cdot \dot{m} \cdot c_{kzp}
\end{equation*}
\noindent Kifejezzük a $\dot{m}-et$:
\begin{equation*}
\dot{m}= \dfrac{\dot{Q}}{\Delta T \cdot c_{kzp}}= \SI{0.037754}{\kilo\gram\per\second}
\end{equation*}
\noindent A tömegáram értékét leosztva a sűrűséggel megkapjuk  a térfogatáramot, majd azt elosztva a cső keresztmetszetével pedig a sebességet:
\begin{equation*}
\dot{V}= \dfrac{\dot{m}}{\varrho_{kzp}}= \SI{0.03568775}{\meter\cubed\per\second}
\end{equation*}
\begin{equation*}
w_{}= \dfrac{\dot{V}}{A_{}}= \SI{11.8959}{\meter\per\second}
\end{equation*}
\subsubsection*{b) Mire fordítódik a közölt hő? }
\noindent A közölt hő egy része a levegő melegedését okozza, a másik része pedig a melegedés miatti kitágulás munkáját fedezi.
\pagebreak
% A feladat címe automatikus számozás nélkül
\section*{A vízgőz állapotváltozások ábrái}

% Hozzáadás a tartalomjegyzékhez azonos címmel
\addcontentsline{toc}{section}{A vízgőz állapotváltozások ábrái}

% Táblázat a szerző adataival
\begin{tabular}{ | p{2cm} | p{14cm} | } 
	\hline
	Szerző & Egressy Levente NY03MT és Lőkös Gábor G69PUV \\ 
	\hline
	Szak & Mechatronikai mérnöki alapszak \\ 
	\hline
	Félév & 2019/2020 II. (tavaszi) félév \\ 
	\hline
\end{tabular}
\vspace{0.5cm}

% A feladat szövege

\noindent A vízgőz állapotváltozások megértése során sokat segíthet azok ábráinak ismerete. Az alábbiakban az izochor, izobár, izoterm és adiabatikus állapotváltozások p-v és T-s diagramjai vannak bemutatva.

Az állapotábrát három részre bontja a fázishatárgörbe. Ennek baloldala az alsó határgörbe (a telített vagy forrponti folyadék állapothoz tartozó görbe) jobboldala pedig a felső határgörbe (a száraz telített gőz állapothoz tartozó görbe). Az alsó határgörbétől balra folyékony halmazállapotú víz van, a két határgörbe alatt nedvesgőz, a felső határgörbétől jobbra pedig túlhevített gőz.

A két görbe a kritikus pontban találkozik. A kritikus pont hőmérsékleténél magasabb hőmérsékletű légnemű fázist gáznak, az annál alacsonyabb hőmérsékletűt gőznek nevezzük.

% A feladat megoldása

\subsection*{a) Az izochor állapotváltozás p-v és T-s diagramja:}

Izochor állapotváltozás során a $v$ fajtérfogat állandó, fizikai munkavégzés nincs, viszont van nyomásváltozás. Ez pedig azt jelenti hogy technikai munkavégzés történik.

A $w_t$ technikai munkát az alábbi módon számolhatjuk ki a p-v diagram felhasználásával:

\begin{equation*}
	w_t = v (p_2 - p_1)
\end{equation*}

A $q$ tömegfajlagos hőtartalom számítására pedig az alábbi képletet használhatjuk:

\begin{equation*}
p_1_3 = u_2-_1 = h_2-h_1-v(p_2-p_1)
\end{equation*}


%izochor
\begin{figure}[h]
	\centering
	\begin{subfigure}[b]{0.545\textwidth} 
		\centering
		\begin{tikzpicture}
		
		% A tengelykeresztet az axis környezet hozza létre
	\begin{loglogaxis}[
	width=9.6cm, height=8cm,
	xmin=0.0003, xmax=10000,
	ymin=0.006, ymax=5000, 
	axis lines = middle,
	axis line style={->},
	log origin x=infty,
	log origin y=infty,
	xlabel=$v \left(\si{\meter\cubed\per\kilogram}\right)$, 
	xlabel style={
		at=(current axis.right of origin), 
		anchor=north east, xshift={8mm}
	}, 
	ylabel=$p \left(\si{\bar}\right)$, 
	ylabel style={
		at=(current axis.above origin), 
		anchor=north east
	},
	xtick={0.001, 0.01, 1, 10,100, 1000},
	ytick={0.01, 0.1, 10, 100, 1000},
	extra x ticks={0.0031056,0.1 ,0.25},
	extra x tick labels={$ $,$ $, $v_1$},
	extra y ticks={220.64,0.2,1.75,50,1},
	extra y tick labels={$p_K$,$p_1$,$p_2$,$p_3$,$ $},
	]
	
	% Az adat az MHFGY Wolfram-jegyzetfüzetből származik
	
	% A nedves gőzmező fázishatárai
	\addplot[thick] table {./ny03mt_g69puv/pv.txt};
	
	% x jelölések
	\node[anchor=west] at (axis cs: 0.001, 5) {\pgf{$x=0$}};
	\node[anchor=west] at (axis cs: 0.75, 5) {\pgf{$x=1$}};
	
	% 1-es pont
	\node[anchor=south east] at (axis cs: 0.25, 0.2) {\pgfcircled{$1$}};
	\filldraw[black, fill=white] (axis cs: 0.25, 0.2) circle (1mm);
	
	% 2-es pont
	\node[anchor=south east] at (axis cs: 0.25, 1.75) {\pgfcircled{$2$}};
	\filldraw[black, fill=white] (axis cs: 0.25, 1.75) circle (1mm);
	
	% 3-as pont
	\node[anchor=south east] at (axis cs: 0.25, 50) {\pgfcircled{$3$}};
	\filldraw[black, fill=white] (axis cs: 0.25, 50) circle (1mm);
	
	% Izochor vonal
	\draw[->, very thick] (axis cs: 0.25, 0.2) -- (axis cs:  0.25, 50);
	
	% Függőleges vonal
	\draw[very thin] (axis cs: 0.25, 0.2) -- (axis cs:  0.25, 0.006);
	
	% Vízszintes vonalak
	\draw[very thin] (axis cs: 0.0003, 0.2) -- (axis cs:  0.25, 0.2);
	\draw[very thin] (axis cs: 0.0003, 1.75) -- (axis cs:  0.25, 1.75);
	\draw[very thin] (axis cs: 0.0003, 50) -- (axis cs:  0.25, 50);	
	
	% A kritikus pont
	\node[anchor=south] at (axis cs: 0.0031056, 220.64) {\pgfcircled{$K$}};
	\fill[fill=black] (axis cs: 0.0031056, 220.64) circle (0.75mm);
	
	% A két görbe metszéspontja
	%\node[anchor=north east] at (axis cs: 0.836481, 1.01418) {\pgfcircled{$1$}};
	%\filldraw[black, fill=white] (axis cs: 0.836481, 1.01418) circle (1mm);
	
	\end{loglogaxis}
		
		\end{tikzpicture}
		\caption{Izochor p-v diagram}
		
	\end{subfigure}
	\begin{subfigure}[b]{0.435\textwidth}
		\centering
		\begin{tikzpicture}
		
		% A tengelykeresztet az axis környezet hozza létre
	\begin{axis}[
	width=9.6cm, height=8cm,
	xmin=-1, xmax=10.8,
	ymin=-45, ymax=475, 
	axis lines = middle,
	axis line style={->},
	xlabel=$s \left(\si{\kilo\joule\per\kilogram\kelvin}\right)$, 
	xlabel style={
		at=(current axis.right of origin), 
		anchor=north east, xshift={8mm}
	}, 
	ylabel=$T \left(\si{\degreeCelsius}\right)$, 
	ylabel style={
		at=(current axis.above origin), 
		anchor=north east
	},
	xtick={1, 2, 3, 4, 5, 7, 8, 9},
	ytick={100, 200, 300, 400},
	extra x ticks={6},
	extra x tick labels={$ $},
	]
	
	% Az adat az MHFGY Wolfram-jegyzetfüzetből származik
	
	% A nedves gőzmező fázishatárai
	\addplot[thick] table {./ny03mt_g69puv/ts.txt};
	
	% A kritikus pont
	\node[anchor=south east] at (axis cs: 4.40696, 373.919) {\pgfcircled{$K$}};
	\filldraw[black, fill=black] (axis cs: 4.40696, 373.919) circle (1mm);
	
	% 1-es pont
	\node[anchor=south east] at (axis cs: 3.5, 150) {\pgfcircled{$1$}};
	\filldraw[black, fill=white] (axis cs: 3.5, 150) circle (1mm);
	
	%2-es pont
	\node[anchor=south east] at (axis cs: 5.9731, 263.941) {\pgfcircled{$2$}};
	\filldraw[black, fill=white] (axis cs: 5.9731, 263.941) circle (1mm);
	
	%3-as pont
	\node[anchor=south east] at (axis cs: 6.497, 360.850) {\pgfcircled{$3$}};
	\filldraw[black, fill=white] (axis cs: 6.497, 360.850) circle (1mm);
	
	
	%Izochor_gorbe
	\addplot[name path=A, ultra thick] table {./ny03mt_g69puv/izochor_gorbe.txt}; 
	
	%Függőleges vonalak és s jelölések
	\draw[very thin] (axis cs: 3.5, 150) -- (axis cs:  3.5, 0);
	\draw[very thin] (axis cs: 5.9731, 263.941) -- (axis cs:  5.9731, 0);
	\draw[very thin] (axis cs: 6.497, 360.850) -- (axis cs:  6.497, 0);
	\node[anchor=north] at (axis cs: 3.5, -5) {\pgf{$s_1$}};
	\filldraw[black, fill=white] (axis cs: 3.5, 0) circle (0.75mm);
	\node[anchor=north] at (axis cs: 6, -5) {\pgf{$s_2$}};
	\filldraw[black, fill=white] (axis cs: 5.9731, 0) circle (0.75mm);
	\node[anchor=north] at (axis cs: 6.557, -5) {\pgf{$s_3$}};
	\filldraw[black, fill=white] (axis cs: 6.497, 0) circle (0.75mm);
	
	
	
	%x jelölések
	\node[anchor=south east] at (axis cs: 2.4, 200) {\pgf{$x=0$}};
	\node[anchor=west] at (axis cs: 7.2, 120) {\pgf{$x=1$}};
	
	%Tjelolesek
		%T_1jelölése
	\draw[very thin] (axis cs: 0, 150) -- (axis cs:  3.5, 150);
	\node[anchor=south east] at (axis cs: 0, 137) {\pgf{$T_1$}};
	\filldraw[black, fill=white] (axis cs: 0, 150) circle (0.75mm);
	
		%T_2jelolese
	\draw[very thin] (axis cs: 0, 263.941) -- (axis cs:  5.9731, 263.941);
	\node[anchor=north east] at (axis cs: 0, 278.941) {\pgf{$T_2$}};
	\filldraw[black, fill=white] (axis cs: 0, 263.941) circle (0.75mm);
	
		%T_3jelolese
	\draw[very thin] (axis cs: 0, 360.850) -- (axis cs:  6.497, 360.850);
	\node[anchor=north east] at (axis cs: 0, 375.850) {\pgf{$T_3$}};
	\filldraw[black, fill=white] (axis cs: 0, 360.850) circle (0.75mm);
	
	% q jeloles
	\node[anchor=south east] at (axis cs: 5.3, 100) {\pgf{$q_{12}$}};
	\node[anchor=south] at (axis cs: 7.5, 330) {\pgf{$q_{23}$}};
	\draw[very thin] (axis cs: 6.35, 285) -- (axis cs: 7, 330);
	\draw[very thin] (axis cs: 7, 330) -- (axis cs:  8, 330);
	
	%vonalkazas segedvonal
	\draw[name path=B, ultra thin] (axis cs: 3.5, 0) -- (axis cs:  6.497, 0);
	%vonalkazas
	\addplot[gray!30] fill between[of=A and B];
	
	
	\end{axis}
		
		\end{tikzpicture}
		\caption{Izochor T-s diagram}
		
	\end{subfigure}
\end{figure}

% Oldaltörés
\pagebreak

\subsection*{b) Az izobár állapotváltozás p-v és T-s diagramja:}

Az izobár állapotváltozás során az állapotváltozás egy $p=constans$ nyomáson megy végbe. Az izobár-vonal nedves és száraz gőzmezőbe eső szakaszait külön vizsgáljuk.

A tömegfajlagos hőmennyiség számítása:

\begin{equation*}
q_1 = T_s (s_2-s_1) = (1-x_1)r = h_2-h_1
\end{equation*}

\begin{equation*}
q_2 = h_3-h_2
\end{equation*}

\begin{equation*}
q_1_3 = q_1+q_2 = h_3-h_1
\end{equation*}

A tömegfajlagos fizikai munka számítása:

\begin{equation*}
w_1_3 = w_1+w_2 = p(v_2-v_1)
\end{equation*}

%izobár
\begin{figure}[h]
	\centering
	\begin{subfigure}[b]{0.545\textwidth} 
		\centering
		\begin{tikzpicture}
		
		% A tengelykeresztet az axis környezet hozza létre
	\begin{loglogaxis}[
	width=9.6cm, height=8cm,
	xmin=0.0003, xmax=10000,
	ymin=0.006, ymax=5000, 
	axis lines = middle,
	axis line style={->},
	log origin x=infty,
	log origin y=infty,
	xlabel=$v \left(\si{\meter\cubed\per\kilogram}\right)$, 
	xlabel style={
		at=(current axis.right of origin), 
		anchor=north east, xshift={8mm}
	}, 
	ylabel=$p \left(\si{\bar}\right)$, 
	ylabel style={
		at=(current axis.above origin), 
		anchor=north east
	},
	xtick={0.001, 0.01, 1, 10,100, 1000},
	ytick={0.01, 0.1, 10, 100, 1000},
	extra x ticks={0.05,4.130687,25, 0.1},
	extra x tick labels={$v_1$,$v_2$,$v_3$, $ $},
	extra y ticks={220.64,0.385954,1},
	extra y tick labels={$p_K$,$p_1$,$10^0$},
	]
	
	% Az adat az MHFGY Wolfram-jegyzetfüzetből származik
	
	% A nedves gőzmező fázishatárai
	\addplot[thick] table {./ny03mt_g69puv/pv.txt};
	
	% A kritikus pont
	\node[anchor=south] at (axis cs: 0.0031056, 220.64) {\pgfcircled{$K$}};
	\fill[fill=black] (axis cs: 0.0031056, 220.64) circle (0.75mm);
	
	% x jelölések
	\node[anchor=west] at (axis cs: 0.001, 5) {\pgf{$x=0$}};
	\node[anchor=west] at (axis cs: 0.75, 5) {\pgf{$x=1$}};
	
	% 1-es pont
	\node[anchor=south] at (axis cs: 0.05, 0.385954) {\pgfcircled{$1$}};
	\filldraw[black, fill=white] (axis cs: 0.05, 0.385954) circle (1mm);
	
	% 2-es pont
	\node[anchor=south] at (axis cs: 4.130687, 0.385954) {\pgfcircled{$2$}};
	\filldraw[black, fill=white] (axis cs: 4.130687, 0.385954) circle (1mm);
	
	% 3-as pont
	\node[anchor=south] at (axis cs: 25, 0.385954) {\pgfcircled{$3$}};
	\filldraw[black, fill=white] (axis cs: 25, 0.385954) circle (1mm);
	
	% Izobar vonal (150C0)
	\draw[mid arrow,name path=A,->, very thick] (axis cs: 0.05, 0.385954) -- (axis cs:  25, 0.385954);
	
	% Függőleges vonal
	\draw[very thin] (axis cs: 0.05, 0.385954) -- (axis cs:  0.05, 0.006);
	\draw[very thin] (axis cs: 4.130687, 0.385954) -- (axis cs:  4.130687, 0.006);
	\draw[very thin] (axis cs: 25, 0.385954) -- (axis cs:  25, 0.006);
	
	% Vízszintes vonalak
	\draw[very thin] (axis cs: 0.0003, 0.385954) -- (axis cs:  0.05, 0.385954);
	
	%vonalkazas segedvonal
	\draw[very thin,name path=B,ultra thin] (axis cs: 0.05, 0.006) -- (axis cs:  25, 0.006);
	%\addplot+ [name path=B, ultra thin, domain= 0.05: 25, samples=2] {0.006};
	
	%vonalkazas
	\addplot[gray!30] fill between[of=A and B];
	
	%W1-2,W2-3 pontok
	\node[anchor=north east] at (axis cs: 0.75, 0.07) {\pgf{$w_1$}};
	\node[anchor=north] at (axis cs: 10, 0.07) {\pgf{$w_2$}};
	
	
	
	\end{loglogaxis}
		
		\end{tikzpicture}
		\caption{Izobár p-v diagram}
		
	\end{subfigure}
	\begin{subfigure}[b]{0.435\textwidth}
		\centering
		\begin{tikzpicture}
		
		% A tengelykeresztet az axis környezet hozza létre
	\begin{axis}[
	width=9.6cm, height=8cm,
	xmin=-1.2, xmax=10.8,
	ymin=-45, ymax=475, 
	axis lines = middle,
	axis line style={->},
	xlabel=$s \left(\si{\kilo\joule\per\kilogram\kelvin}\right)$, 
	xlabel style={
		at=(current axis.right of origin), 
		anchor=north east, xshift={8mm}
	}, 
	ylabel=$T \left(\si{\degreeCelsius}\right)$, 
	ylabel style={
		at=(current axis.above origin), 
		anchor=north east
	},
	xtick={1, 2, 3, 4, 5, 6, 8, 9},
	ytick={100, 200, 300, 400},
	extra x ticks={7},
	extra x tick labels={$ $}
	]
	
	% Az adat az MHFGY Wolfram-jegyzetfüzetből származik
	
	% A nedves gőzmező fázishatárai
	\addplot[thick] table {./ny03mt_g69puv/ts.txt};
	
	% x jelölések
	\node[anchor=south east] at (axis cs: 2.4, 200) {\pgf{$x=0$}};
	\node[anchor=west] at (axis cs: 7.2, 120) {\pgf{$x=1$}};
	
	% A kritikus pont
	\node[anchor=north] at (axis cs: 4.40696, 373.919) {\pgfcircled{$K$}};
	\filldraw[black, fill=black] (axis cs: 4.40696, 373.919) circle (1mm);
	
	% 50bar-os izobar vonal
	\addplot[->, name path=A, ultra thick] table {./ny03mt_g69puv/50Bar_gorbe_1-3.txt}; 
	\addplot[thick] table {./ny03mt_g69puv/50Bar_gorbe_maradek.txt};
	
	% 1-es pont
	\node[anchor=south ] at (axis cs: 3.5, 263.941) {\pgfcircled{$1$}};
	\filldraw[black, fill=white] (axis cs: 3.5, 263.941) circle (1mm);
	
	% 2-es pont
	\node[anchor=south east] at (axis cs: 5.9, 263.941) {\pgfcircled{$2$}};
	\filldraw[black, fill=white] (axis cs: 5.9731, 263.941) circle (1mm);
	
	% 3-as pont
	\node[anchor=south east] at (axis cs: 6.614, 390.850) {\pgfcircled{$3$}};
	\filldraw[black, fill=white] (axis cs: 6.614, 390.850) circle (1mm);
	
	% Függőleges vonalak és s jelölések
	\draw[very thin] (axis cs: 3.5, 263.941) -- (axis cs:  3.5, 0);
	\draw[very thin] (axis cs: 5.9731, 263.941) -- (axis cs:  5.9731, 0);
	\draw[very thin] (axis cs: 6.614, 390.850) -- (axis cs:  6.614, 0);
	\node[anchor=north] at (axis cs: 3.5, -5) {\pgf{$s_1$}};
	\filldraw[black, fill=white] (axis cs: 3.5, 0) circle (0.75mm);
	\node[anchor=north east] at (axis cs: 6.05, -5) {\pgf{$s_2$}};
	\filldraw[black, fill=white] (axis cs: 5.9731, 0) circle (0.75mm);
	\node[anchor=north] at (axis cs: 6.75, -5) {\pgf{$s_3$}};
	\filldraw[black, fill=white] (axis cs: 6.614, 0) circle (0.75mm);
	
	% T jelölése
	\draw[very thin] (axis cs: 0, 263.941) -- (axis cs:  3.5, 263.941);
	\node[anchor=east] at (axis cs: 0, 263.941) {\pgf{$T_{1,2}$}};
	\filldraw[black, fill=white] (axis cs: 0, 263.941) circle (0.75mm);
	
	\draw[very thin] (axis cs: 0, 390.850) -- (axis cs:  6.614, 390.850);
	\node[anchor=north east] at (axis cs: 0, 390.850) {\pgf{$T_3$}};
	\filldraw[black, fill=white] (axis cs: 0, 390.850) circle (0.75mm);
	
	% p jelölése
	\node[anchor=south west] at (axis cs: 6.614, 390.850) {\pgf{$p=kons.$}};
	
	% q jeloles
	\node[anchor=south east] at (axis cs: 5.2, 135) {\pgf{$q_{12}$}};
	\node[anchor=south] at (axis cs: 7.5, 330) {\pgf{$q_{23}$}};
	\draw[very thin] (axis cs: 6.35, 285) -- (axis cs: 7, 330);
	\draw[very thin] (axis cs: 7, 330) -- (axis cs:  8, 330);
	
	%vonalkazas segedvonal
	\addplot+ [name path=B, ultra thin, domain=3.5:6.614, samples=2] {0};
	
	%vonalkazas
	\addplot[gray!30] fill between[of=A and B];
		
	\end{axis}
		
		\end{tikzpicture}
		\caption{Izobár T-s diagram}
		
	\end{subfigure}
\end{figure}

\pagebreak

\subsection*{C) Az izoterm állapotváltozás p-v és T-s diagramja:}

Az izoterm állapotváltozás során az állapotváltozás egy $T=constans$ hőmérsékleten megy végbe. Az állapotváltozást itt is két szakaszra bontjuk, külön vizsgálva a nedves és a száraz gőzmezőbe tartozó szakaszokat.

A tömegfajlagos hőmennyiség számítása:

\begin{equation*}
q_1 = T_1_2_3 (s_2-s_1) = (1-x_1)r = h_2-h_1
\end{equation*}

\begin{equation*}
q_2 = T_1_2_3(s_3-s_2)
\end{equation*}

\begin{equation*}
q_1_3 = q_1+q_2 = T_1_2_3(s_3-s_2)
\end{equation*}

A tömegfajlagos fizikai munka számítása:

\begin{equation*}
w_1 = p_1_2(v_2-v_1)
\end{equation*}

\begin{equation*}
w_2 = q_2 = T_1_2_3(s_3-s_2) 
\end{equation*}

\begin{equation*}
w_1_3 = w_1+w_2 = p(v_3-v_1)
\end{equation*}

A tömegfajlagos belső energiaváltozás számítása:

\begin{equation*}
\Delta u_1 = u_2-u_1 = (1-x_1)\rho
\end{equation*}

\begin{equation*}
\Delta u_2 =0
\end{equation*}

%izoterm

\begin{figure}[h]
	\centering
	\begin{subfigure}[b]{0.545\textwidth} 
		\centering
		\begin{tikzpicture}
		
			% A tengelykeresztet az axis környezet hozza létre
	\begin{loglogaxis}[
	width=9.6cm, height=8cm,
	xmin=0.0003, xmax=10000,
	ymin=0.006, ymax=5000, 
	axis lines = middle,
	axis line style={->},
	log origin x=infty,
	log origin y=infty,
	xlabel=$v \left(\si{\meter\cubed\per\kilogram}\right)$, 
	xlabel style={
		at=(current axis.right of origin), 
		anchor=north
	}, 
	ylabel=$p \left(\si{\bar}\right)$, 
	ylabel style={
		at=(current axis.above origin), 
		anchor=north east
	},
	xtick={0.001, 0.01, 1, 10,100, 1000},
	ytick={0.01, 0.1, 10, 100, 1000},
	extra x ticks={0.1, 0.05},
	extra x tick labels={$ $, $v_1$},
	extra y ticks={220.64, 1},
	extra y tick labels={$p_K$, $10^0$},
	]
	
	% Az adat az MHFGY Wolfram-jegyzetfüzetből származik
	
	% A nedves gőzmező fázishatárai
	\addplot[thick] table {./ny03mt_g69puv/pv.txt};
	
	%izoterma (100C)
	\addplot[ultra thick,name path=A,,->] table {./ny03mt_g69puv/100C izoterma.txt};	
	
	% A kritikus pont
	\node[anchor=south] at (axis cs: 0.0031056, 220.64) {\pgfcircled{$K$}};
	\fill[fill=black] (axis cs: 0.0031056, 220.64) circle (0.75mm);
	
	% x jelölések
	\node[anchor=west] at (axis cs: 0.001, 5) {\pgf{$x=0$}};
	\node[anchor=west] at (axis cs: 0.75, 5) {\pgf{$x=1$}};
	
	% 1-es pont
	\node[anchor=south east] at (axis cs: 0.05, 1.014180) {\pgfcircled{$1$}};
	\filldraw[black, fill=white] (axis cs: 0.05, 1.014180) circle (1mm);
	
	% 2-es pont
	\node[anchor=south west] at (axis cs: 1.672373, 1.014180) {\pgfcircled{$2$}};
	\filldraw[black, fill=white] (axis cs: 1.672373, 1.014180) circle (1mm);
	
	% 3-as pont
	\node[anchor=south west] at (axis cs: 20.540213, 0.083741) {\pgfcircled{$3$}};
	\filldraw[black, fill=white] (axis cs: 20.540213, 0.083741) circle (1mm);

	% Függőleges vonal
	\draw[very thin] (axis cs: 0.05, 1.014180) -- (axis cs:  0.05, 0.006);
	\draw[very thin] (axis cs: 1.672373, 1.014180) -- (axis cs:  1.672373, 0.006);
	\draw[very thin] (axis cs: 20.540213, 0.083741) -- (axis cs:  20.540213, 0.006);
		
	%vonalkazas segedvonal
	\draw[very thin,name path=B,ultra thin] (axis cs: 0.05, 0.006) -- (axis cs: 20.540213, 0.006);

	%vonalkazas
	\addplot[gray!30] fill between[of=A and B];
	
	%W1-2,W2-3 pontok
	\node[anchor=north east] at (axis cs: 0.55, 0.1) {\pgf{$w_1$}};
	\node[anchor=north] at (axis cs: 6, 0.1) {\pgf{$w_2$}};
	
	
	
	\end{loglogaxis}
		
		\end{tikzpicture}
		\caption{Izoterm p-v diagram}
		
	\end{subfigure}
	\begin{subfigure}[b]{0.435\textwidth}
		\centering
		\begin{tikzpicture}
		
		% A tengelykeresztet az axis környezet hozza létre
	\begin{axis}[
	width=9.6cm, height=8cm,
	xmin=-1, xmax=10.8,
	ymin=-45, ymax=475, 
	axis lines = middle,
	axis line style={->},
	xlabel=$s \left(\si{\kilo\joule\per\kilogram\kelvin}\right)$, 
	xlabel style={
		at=(current axis.right of origin), 
		anchor=north east, xshift={8mm}
	}, 
	ylabel=$T \left(\si{\degreeCelsius}\right)$, 
	ylabel style={
		at=(current axis.above origin), 
		anchor=north east
	},
	xtick={1, 2, 3, 4, 5, 6, 7, 8, 9},
	ytick={100, 200, 300, 400},
	extra y ticks={263.941},
	extra y tick labels={$T_{123}$}
	]
	
	% Az adat az MHFGY Wolfram-jegyzetfüzetből származik
	
	% A nedves gőzmező fázishatárai
	\addplot[thick] table {./ny03mt_g69puv/ts.txt};
	
	% A kritikus pont
	\node[anchor=south] at (axis cs: 4.40696, 373.919) {\pgfcircled{$K$}};
	\filldraw[black, fill=black] (axis cs: 4.40696, 373.919) circle (1mm);
	
	% 1-es pont
	\node[anchor=south east] at (axis cs: 4.5, 263.941) {\pgfcircled{$1$}};
	\filldraw[black, fill=white] (axis cs: 4.5, 263.941) circle (1mm);
	
	%2-es pont
	\node[anchor=south east] at (axis cs: 5.9, 263.941) {\pgfcircled{$2$}};
	\filldraw[black, fill=white] (axis cs: 5.9731, 263.941) circle (1mm);
	
	%3-as pont
	\node[anchor=south east] at (axis cs: 7.5, 263.941) {\pgfcircled{$3$}};
	\filldraw[black, fill=white] (axis cs: 7.5, 263.941) circle (1mm);
	
	
	%Izoterma
	\draw[->,name path=A, ultra thick] (axis cs: 4.5, 263.941) -- (axis cs: 7.5, 263.941);
	
	%Függőleges vonalak és s jelölések
	\draw[very thin] (axis cs: 4.5, 263.941) -- (axis cs:  4.5, 0);
	\draw[very thin] (axis cs: 5.9731, 263.941) -- (axis cs:  5.9731, 0);
	\draw[very thin] (axis cs: 7.5, 263.941) -- (axis cs:  7.5, 0);
	\node[anchor=north] at (axis cs: 4.5, -5) {\pgf{$s_1$}};
	\filldraw[black, fill=white] (axis cs: 4.5, 0) circle (0.75mm);
	\node[anchor=north west] at (axis cs: 5.9731, -5) {\pgf{$s_2$}};
	\filldraw[black, fill=white] (axis cs: 5.9731, 0) circle (0.75mm);
	\node[anchor=north] at (axis cs: 7.5, -5) {\pgf{$s_3$}};
	\filldraw[black, fill=white] (axis cs: 7.5, 0) circle (0.75mm);
	
	
	
	%x jelölések
	\node[anchor=south east] at (axis cs: 2.4, 200) {\pgf{$x=0$}};
	\node[anchor=west] at (axis cs: 7.6, 105) {\pgf{$x=1$}};
	
	%Tjelölése
	\draw[very thin] (axis cs: 0, 263.941) -- (axis cs:  4.5, 263.941);
	\filldraw[black, fill=white] (axis cs: 0, 263.941) circle (0.75mm);
	
	% q jeloles
	\node[anchor=south] at (axis cs: 5.35, 80) {\pgf{$q_{12}$}};
	\node[anchor=south] at (axis cs: 6.75, 80) {\pgf{$q_{23}$}};
	
	%vonalkazas segedvonal
	\draw[name path=B, ultra thin] (axis cs: 4.5, 0) -- (axis cs:  7.5, 0);
	%vonalkazas
	\addplot[gray!30] fill between[of=A and B];
	
	
	\end{axis}
		
		\end{tikzpicture}
		\caption{Izoterm T-s diagram}
		
	\end{subfigure}
\end{figure}

\pagebreak

\subsection*{D) Az adiabatikus állapotváltozás p-v és T-s diagramja:}

Vízgőz adiabatikus állapotváltozása során a termodinamikai rendszerünk a környezetével nem cserél hőt, vagyis a rendszer entrópiája állandó ($s=constans$). A hőmérséklet megváltozását a vízgőz összenyomásából eredő melegedés, vagy kitágulásából eredő lehűlés okozza. Emellett befolyásolja még a vízgőz kicsapódásakor felszabaduló látens hő, illetve a kicsapódott víz saját hőtartalma. Expanzió (tágulás) esetén a belső energia csökken, kompresszió (sűrítés) esetén nő.

A tömegfajlagos fizikai munka számítása:

\begin{equation*}
w = u_3-u_1 = h_3-h_1-(p_1v_1+p_3v_3)
\end{equation*}

A technikai munka számítása:

\begin{equation*}
w_t = h_3-h_1
\end{equation*}


%adiabatikus

\begin{figure}[h]
	\centering
	\begin{subfigure}[b]{0.545\textwidth} 
		\centering
		\begin{tikzpicture}
		
		% A tengelykeresztet az axis környezet hozza létre
	\begin{loglogaxis}[
	width=9.6cm, height=8cm,
	xmin=0.0003, xmax=10000,
	ymin=0.006, ymax=5000, 
	axis lines = middle,
	axis line style={->},
	log origin x=infty,
	log origin y=infty,
	xlabel=$v \left(\si{\meter\cubed\per\kilogram}\right)$, 
	xlabel style={
		at=(current axis.right of origin), 
		anchor=north east, xshift={8mm}
	}, 
	ylabel=$p \left(\si{\bar}\right)$, 
	ylabel style={
		at=(current axis.above origin), 
		anchor=north east yshift={1mm}
	},
	xtick={0.001, 0.01, 1, 10,100, 1000},
	ytick={0.01, 0.1, 10, 1000},
	extra x ticks={2.429802,0.043,0.013260, 0.1},
	extra x tick labels={$v_1$,$v_2$, $ $},
	extra y ticks={0.5,45.886615,200, 1, 100},
	extra y tick labels={$p_1$,$p_2$,$p_3$, $10^0$ $ $},
	]
	
	% Az adat az MHFGY Wolfram-jegyzetfüzetből származik
	
	% A nedves gőzmező fázishatárai
	\addplot[thick] table {./ny03mt_g69puv/pv.txt};
	
	% A kritikus pont
	\node[anchor=south] at (axis cs: 0.0031056, 220.64) {\pgfcircled{$K$}};
	\fill[fill=black] (axis cs: 0.0031056, 220.64) circle (0.75mm);
	
	% x jelölések
	\node[anchor=west] at (axis cs: 0.001, 5) {\pgf{$x=0$}};
	\node[anchor=west] at (axis cs: 0.75, 5) {\pgf{$x=1$}};
	
	% 1-es pont
	\node[anchor=north east] at (axis cs: 2.429802, 0.500000) {\pgfcircled{$1$}};
	\filldraw[black, fill=white] (axis cs: 2.429802, 0.500000) circle (1mm);
	
	% 2-es pont
	\node[anchor=south west] at (axis cs: 0.05,  45.86615) {\pgfcircled{$2$}};
	\filldraw[black, fill=white] (axis cs: 0.043, 45.886615) circle (1mm);
	
	% 3-as pont
	\node[anchor=south] at (axis cs: 0.013260, 200.000000) {\pgfcircled{$3$}};
	\filldraw[black, fill=white] (axis cs: 0.013260, 200.000000) circle (1mm);
	
	% adiabata
	\addplot[ultra thick, name path=A] table {./ny03mt_g69puv/adiabata.txt}; 
	
	% Függőleges méretvonalak
	\draw[very thin] (axis cs: 2.429802, 0.500000) -- (axis cs: 2.429802, 0.006);
	\draw[very thin] (axis cs: 0.043, 45.886615) -- (axis cs:  0.043, 0.006);
	\draw[very thin] (axis cs: 0.013260, 200.000000) -- (axis cs: 0.013260, 0.006);
	
	% Vízszintes vonalak
	\draw[very thin] (axis cs: 0.0003, 0.500000) -- (axis cs: 2.429802, 0.500000);
	\draw[very thin] (axis cs: 0.0003, 45.886615) -- (axis cs: 0.043, 45.886615);
	\draw[very thin] (axis cs: 0.0003, 200) -- (axis cs: 0.013260, 200.000000);
	
	%v_3_felirat
	\node[anchor=north east] at (axis cs:0.015360 , 0.018) {\pgf{$v_3$}};
	
	%vonalkazas segedvonal
	\draw[name path=B,ultra thin] (axis cs:0.013260 , 0.006) -- (axis cs: 2.429802, 0.006);

	
	%vonalkazas
	\addplot[gray!30] fill between[of=A and B];
	
	%W1-2,W2-3 pontok
	\node[anchor=north] at (axis cs: 0.25, 0.4) {\pgf{$w$}};
	%\node[anchor=north] at (axis cs: 0.0135, 1) {\pgf{$w_{23}$}};
	
	
	
	\end{loglogaxis}
		
		\end{tikzpicture}
		\caption{Adiabatikus p-v diagram}
		
	\end{subfigure}
	\begin{subfigure}[b]{0.435\textwidth}
		\centering
		\begin{tikzpicture}
		
		
	% Rács és vágómaszk
	%\draw[step=1cm, gray, very thin] (-1.5, -0.5) grid (14.5, 10.5);
	%\clip (-1.5, -1) rectangle (14.5, 11);
	
	% A tengelykeresztet az axis környezet hozza létre
	\begin{axis}[
	width=9.6cm, height=8cm,
	xmin=-1, xmax=10.8,
	ymin=-45, ymax=475, 
	axis lines = middle,
	axis line style={->},
	xlabel=$s \left(\si{\kilo\joule\per\kilogram\kelvin}\right)$, 
	xlabel style={
		at=(current axis.right of origin), 
		anchor=north east, xshift={8mm}
	}, 
	ylabel=$T \left(\si{\degreeCelsius}\right)$, 
	ylabel style={
		at=(current axis.above origin), 
		anchor=north east
	},
	xtick={1, 2, 3, 4, 5, 6, 8, 9},
	ytick={100, 200, 300, 400},
	extra x ticks={7},
	extra x tick labels={$ $},
	]
	
	% Az adat az MHFGY Wolfram-jegyzetfüzetből származik
	
	% A nedves gőzmező fázishatárai
	\addplot[thick] table {./ny03mt_g69puv/ts.txt};
	
	% A kritikus pont
	\node[anchor=south east] at (axis cs: 4.40696, 373.919) {\pgfcircled{$K$}};
	\filldraw[black, fill=black] (axis cs: 4.40696, 373.919) circle (1mm);
	
	% 1-es pont
	\node[anchor=north east] at (axis cs: 6.497, 151.831) {\pgfcircled{$1$}};
	\filldraw[black, fill=white] (axis cs: 6.497, 151.831) circle (1mm);
	
	%2-es pont
	\node[anchor=south west] at (axis cs: 6.497, 185.850) {\pgfcircled{$2$}};
	\filldraw[black, fill=white] (axis cs: 6.497, 190.850) circle (1mm);
	
	%3-as pont
	\node[anchor=south east] at (axis cs: 6.497, 360.850) {\pgfcircled{$3$}};
	\filldraw[black, fill=white] (axis cs: 6.497, 360.850) circle (1mm);
	
	
	%p1_gorbe
	\addplot[thick] table {./ny03mt_g69puv/adiabatikus_p1.txt}; 
	\node[anchor=south] at (axis cs: 4, 151.831) {\pgf{$p_1=const.$}};
	
	%p3_gorbe
	\addplot[thick] table {./ny03mt_g69puv/50Bar görbe.txt};
	\node[anchor=south] at (axis cs: 4.5, 263.941) {\pgf{$p_3$}};
	
	%Függőleges vonalak és s jelölések
	\draw[very thin] (axis cs: 6.497, 360.850) -- (axis cs:  6.497, 0);
	\node[anchor=north west] at (axis cs: 6.2, -5) {\pgf{$s_{123}$}};
	\filldraw[black, fill=white] (axis cs: 6.497, 0) circle (0.75mm);
	
	
	
	%x jelölések
	\node[anchor=south east] at (axis cs: 2.4, 200) {\pgf{$x=0$}};
	\node[anchor=west] at (axis cs: 7.2, 120) {\pgf{$x=1$}};
	
	%Tjelolesek
		%T_1jelölése
	\draw[very thin] (axis cs: 0, 151.831) -- (axis cs:  1.860, 151.831);
	\node[anchor=east] at (axis cs: 0, 151.831) {\pgf{$T_1$}};
	\filldraw[black, fill=white] (axis cs: 0, 151.831) circle (0.75mm);
	
		%T_2jelolese
	%\draw[very thin] (axis cs: 0, 185.850) -- (axis cs:  6.497, 185.850);
	%\node[anchor=north east] at (axis cs: 0, 278.941) {\pgf{$T_2$}};
	%\filldraw[black, fill=white] (axis cs: 0, 263.941) circle (0.75mm);
	
		%T_3jelolese
	\draw[very thin] (axis cs: 0, 360.850) -- (axis cs:  6.497, 360.850);
	\node[anchor=east] at (axis cs: 0, 360.850) {\pgf{$T_3$}};
	\filldraw[black, fill=white] (axis cs: 0, 360.850) circle (0.75mm);
	
	%1-3_szakasz
	\draw[ultra thick, mid arrow] (axis cs: 6.497, 151.831) -- (axis cs: 6.497, 360.850);
	
	\end{axis}
	

		
		\end{tikzpicture}
		\caption{Adiabatikus T-s diagram}
		
	\end{subfigure}
\end{figure}

\pagebreak
% A feladat címe automatikus számozás nélkül
\section*{A vízgőz állapotváltozások ábrái}

% Hozzáadás a tartalomjegyzékhez azonos címmel
\addcontentsline{toc}{section}{A vízgőz állapotváltozások ábrái}

% Táblázat a szerző adataival
\begin{tabular}{ | p{2cm} | p{14cm} | } 
	\hline
	Szerző & Egressy Levente NY03MT és Lőkös Gábor G69PUV \\ 
	\hline
	Szak & Mechatronikaimérnök alapszak \\ 
	\hline
	Félév & 2019/2020 II. (tavaszi) félév \\ 
	\hline
\end{tabular}
\vspace{0.5cm}

% A feladat szövege


% A feladat megoldása


%p-v izochor

\begin{figure}[h]
	\centering
	\label{figure:guh7ud-vgpvd}
	\begin{tikzpicture}
	% Rács és vágómaszk
	%\draw[step=1cm, gray, very thin] (-1.5, -1) grid (14.5, 11);
	%\clip (-1.5, -1) rectangle (14.5, 11);
	
	% A tengelykeresztet az axis környezet hozza létre
	\begin{loglogaxis}[
	width=16cm, height=12cm,
	xmin=0.0003, xmax=10000,
	ymin=0.006, ymax=5000, 
	axis lines = middle,
	axis line style={->},
	log origin x=infty,
	log origin y=infty,
	xlabel=$v \left(\si{\meter\cubed\per\kilogram}\right)$, 
	xlabel style={
		at=(current axis.right of origin), 
		anchor=north east
	}, 
	ylabel=$p \left(\si{\bar}\right)$, 
	ylabel style={
		at=(current axis.above origin), 
		anchor=north east
	},
	xtick={0.001, 0.01, 0.1, 1, 10, 1000},
	ytick={0.01, 0.1, 1, 10, 100, 1000},
	extra x ticks={0.0031056,0.25},
	extra x tick labels={$v_K$,$v_1$},
	extra y ticks={220.64,0.2,1.75,50},
	extra y tick labels={$p_K$,$p_1$,$p_2$,$p_3$},
	]
	
	% Az adat az MHFGY Wolfram-jegyzetfüzetből származik
	
	% A nedves gőzmező fázishatárai
	\addplot[thick] table {./ny03mt_g69puv/pv.txt};
	
	% x jelölések
	\node[anchor=south east] at (axis cs: 0.005, 5) {\pgf{$x=0$}};
	\node[anchor=south east] at (axis cs: 1.2
	, 5) {\pgf{$x=1$}};
	
	% 1-es pont
	\node[anchor=south east] at (axis cs: 0.25, 0.2) {\pgfcircled{$1$}};
	\filldraw[black, fill=white] (axis cs: 0.25, 0.2) circle (1mm);
	
	% 2-es pont
	\node[anchor=south east] at (axis cs: 0.25, 1.75) {\pgfcircled{$2$}};
	\filldraw[black, fill=white] (axis cs: 0.25, 1.75) circle (1mm);
	
	% 3-as pont
	\node[anchor=south east] at (axis cs: 0.25, 50) {\pgfcircled{$3$}};
	\filldraw[black, fill=white] (axis cs: 0.25, 50) circle (1mm);
	
	% Izochor vonal
	\draw[->, very thick] (axis cs: 0.25, 0.2) -- (axis cs:  0.25, 50);
	
	% Függőleges vonal
	\draw[very thin] (axis cs: 0.25, 0.2) -- (axis cs:  0.25, 0.006);
	
	% Vízszintes vonalak
	\draw[very thin] (axis cs: 0.0003, 0.2) -- (axis cs:  0.25, 0.2);
	\draw[very thin] (axis cs: 0.0003, 1.75) -- (axis cs:  0.25, 1.75);
	\draw[very thin] (axis cs: 0.0003, 50) -- (axis cs:  0.25, 50);	
	
	% A kritikus pont
	\node[anchor=south] at (axis cs: 0.0031056, 220.64) {$K$};
	\fill[fill=black] (axis cs: 0.0031056, 220.64) circle (0.75mm);
	
	% A két görbe metszéspontja
	%\node[anchor=north east] at (axis cs: 0.836481, 1.01418) {\pgfcircled{$1$}};
	%\filldraw[black, fill=white] (axis cs: 0.836481, 1.01418) circle (1mm);
	
	\end{loglogaxis}
	
	\end{tikzpicture}
	\caption{Izochor p-v diagram}
\end{figure}

\pagebreak

%T-s izochor

\usepgfplotslibrary{fillbetween}

\begin{figure}[h]
	\centering
	\label{figure:vgtsd}
	\begin{tikzpicture}
	
	% A tengelykeresztet az axis környezet hozza létre
	\begin{axis}[
	width=16cm, height=12cm,
	xmin=-1, xmax=10.8,
	ymin=-30, ymax=475, 
	axis lines = middle,
	axis line style={->},
	xlabel=$s \left(\si{\kilo\joule\per\kilogram\kelvin}\right)$, 
	xlabel style={
		at=(current axis.right of origin), 
		anchor=north east
	}, 
	ylabel=$T \left(\si{\degreeCelsius}\right)$, 
	ylabel style={
		at=(current axis.above origin), 
		anchor=north east
	},
	xtick={1, 2, 3, 4, 5, 6, 7, 8, 9},
	ytick={100, 200, 300, 400}
	]
	
	% Az adat az MHFGY Wolfram-jegyzetfüzetből származik
	
	% A nedves gőzmező fázishatárai
	\addplot[thick] table {./ny03mt_g69puv/ts.txt};
	
	% A kritikus pont
	\node[anchor=south east] at (axis cs: 4.40696, 373.919) {\pgfcircled{$K$}};
	\filldraw[black, fill=black] (axis cs: 4.40696, 373.919) circle (1mm);
	
	% 1-es pont
	\node[anchor=south east] at (axis cs: 3.5, 150) {\pgfcircled{$1$}};
	\filldraw[black, fill=white] (axis cs: 3.5, 150) circle (1mm);
	
	%2-es pont
	\node[anchor=south east] at (axis cs: 5.9731, 263.941) {\pgfcircled{$2$}};
	\filldraw[black, fill=white] (axis cs: 5.9731, 263.941) circle (1mm);
	
	%3-as pont
	\node[anchor=south east] at (axis cs: 6.497, 360.850) {\pgfcircled{$3$}};
	\filldraw[black, fill=white] (axis cs: 6.497, 360.850) circle (1mm);
	
	
	%Izochor_gorbe
	\addplot[name path=A, ultra thick] table {./ny03mt_g69puv/izochor_gorbe.txt}; 
	
	%Függőleges vonalak és s jelölések
	\draw[very thin] (axis cs: 3.5, 150) -- (axis cs:  3.5, 0);
	\draw[very thin] (axis cs: 5.9731, 263.941) -- (axis cs:  5.9731, 0);
	\draw[very thin] (axis cs: 6.497, 360.850) -- (axis cs:  6.497, 0);
	\node[anchor=north west] at (axis cs: 3.3, -5) {\pgf{$s_1$}};
	\filldraw[black, fill=white] (axis cs: 3.5, 0) circle (0.75mm);
	\node[anchor=north west] at (axis cs: 5.9731, -5) {\pgf{$s_2$}};
	\filldraw[black, fill=white] (axis cs: 5.9731, 0) circle (0.75mm);
	\node[anchor=north west] at (axis cs: 6.357, -5) {\pgf{$s_3$}};
	\filldraw[black, fill=white] (axis cs: 6.497, 0) circle (0.75mm);
	
	
	
	%x jelölések
	\node[anchor=south east] at (axis cs: 3.5, 320) {\pgf{$x=0$}};
	\node[anchor=south east] at (axis cs: 6.4, 320) {\pgf{$x=1$}};
	
	%Tjelolesek
	%T_1jelölése
	\draw[very thin] (axis cs: 0, 150) -- (axis cs:  3.5, 150);
	\node[anchor=south east] at (axis cs: 0, 137) {\pgf{$T_1$}};
	\filldraw[black, fill=white] (axis cs: 0, 150) circle (0.75mm);
	
	%T_2jelolese
	\draw[very thin] (axis cs: 0, 263.941) -- (axis cs:  5.9731, 263.941);
	\node[anchor=north east] at (axis cs: 0, 278.941) {\pgf{$T_2$}};
	\filldraw[black, fill=white] (axis cs: 0, 263.941) circle (0.75mm);
	
	%T_3jelolese
	\draw[very thin] (axis cs: 0, 360.850) -- (axis cs:  6.497, 360.850);
	\node[anchor=north east] at (axis cs: 0, 375.850) {\pgf{$T_3$}};
	\filldraw[black, fill=white] (axis cs: 0, 360.850) circle (0.75mm);
	
	% q jeloles
	\node[anchor=south east] at (axis cs: 5.3, 100) {\pgf{$q_{12}$}};
	\node[anchor=south east] at (axis cs: 6.57, 100) {\pgf{$q_{23}$}};
	
	%vonalkazas segedvonal
	\draw[name path=B, ultra thin] (axis cs: 3.5, 0) -- (axis cs:  6.497, 0);
	%vonalkazas
	\addplot[grey!30] fill between[of=A and B];
	
	
	\end{axis}
	
	\end{tikzpicture}
	\caption{Izochor T-s diagram}
\end{figure}

\pagebreak

%p-v izobar


\usepgfplotslibrary{fillbetween}

\begin{figure}[h]
	\centering
	\label{figure:guh7ud-vgpvd}
	\begin{tikzpicture}
	
	% A tengelykeresztet az axis környezet hozza létre
	\begin{loglogaxis}[
	width=16cm, height=12cm,
	xmin=0.0003, xmax=10000,
	ymin=0.006, ymax=5000, 
	axis lines = middle,
	axis line style={->},
	log origin x=infty,
	log origin y=infty,
	xlabel=$v \left(\si{\meter\cubed\per\kilogram}\right)$, 
	xlabel style={
		at=(current axis.right of origin), 
		anchor=north east
	}, 
	ylabel=$p \left(\si{\bar}\right)$, 
	ylabel style={
		at=(current axis.above origin), 
		anchor=north east
	},
	xtick={0.001, 0.01, 0.1, 1, 10, 1000},
	ytick={0.01, 0.1, 1, 10, 100, 1000},
	extra x ticks={0.0031056,0.05,4.130687,25},
	extra x tick labels={$v_K$,$v_1$,$v_2$,$v_3$},
	extra y ticks={220.64,0.385954},
	extra y tick labels={$p_K$,$p_1$},
	]
	
	% Az adat az MHFGY Wolfram-jegyzetfüzetből származik
	
	% A nedves gőzmező fázishatárai
	\addplot[thick] table {./ny03mt_g69puv/pv.txt};
	
	% A kritikus pont
	\node[anchor=south] at (axis cs: 0.0031056, 220.64) {$K$};
	\fill[fill=black] (axis cs: 0.0031056, 220.64) circle (0.75mm);
	
	% x jelölések
	\node[anchor=south east] at (axis cs: 0.005, 5) {\pgf{$x=0$}};
	\node[anchor=south east] at (axis cs: 1.2, 5) {\pgf{$x=1$}};
	
	% 1-es pont
	\node[anchor=south east] at (axis cs: 0.05, 0.385954) {\pgfcircled{$1$}};
	\filldraw[black, fill=white] (axis cs: 0.05, 0.385954) circle (1mm);
	
	% 2-es pont
	\node[anchor=south west] at (axis cs: 4.130687, 0.385954) {\pgfcircled{$2$}};
	\filldraw[black, fill=white] (axis cs: 4.130687, 0.385954) circle (1mm);
	
	% 3-as pont
	\node[anchor=south east] at (axis cs: 25, 0.385954) {\pgfcircled{$3$}};
	\filldraw[black, fill=white] (axis cs: 25, 0.385954) circle (1mm);
	
	% Izobar vonal (150C0)
	\draw[mid arrow,name path=A,->, very thick] (axis cs: 0.05, 0.385954) -- (axis cs:  25, 0.385954);
	
	% Függőleges vonal
	\draw[very thin] (axis cs: 0.05, 0.385954) -- (axis cs:  0.05, 0.006);
	\draw[very thin] (axis cs: 4.130687, 0.385954) -- (axis cs:  4.130687, 0.006);
	\draw[very thin] (axis cs: 25, 0.385954) -- (axis cs:  25, 0.006);
	
	% Vízszintes vonalak
	\draw[very thin] (axis cs: 0.0003, 0.385954) -- (axis cs:  0.05, 0.385954);
	
	%vonalkazas segedvonal
	\draw[very thin,name path=B,ultra thin] (axis cs: 0.05, 0.006) -- (axis cs:  25, 0.006);
	%\addplot+ [name path=B, ultra thin, domain= 0.05: 25, samples=2] {0.006};
	
	%vonalkazas
	\addplot[grey!30] fill between[of=A and B];
	
	%W1-2,W2-3 pontok
	\node[anchor=north east] at (axis cs: 0.75, 0.07) {\pgf{$w_1$}};
	\node[anchor=north] at (axis cs: 10, 0.07) {\pgf{$w_2$}};
	
	
	
	\end{loglogaxis}
	
	\end{tikzpicture}
	\caption{Izobár p-v diagram}
\end{figure}

\pagebreak

%T-s izobár

\usepgfplotslibrary{fillbetween}

\begin{figure}[h]
	\centering
	\label{figure:vgtsd}
	\begin{tikzpicture}
	% Rács és vágómaszk
	%\draw[step=1cm, gray, very thin] (-1.5, -0.5) grid (14.5, 10.5);
	%\clip (-1.5, -1) rectangle (14.5, 11);
	
	% A tengelykeresztet az axis környezet hozza létre
	\begin{axis}[
	width=16cm, height=12cm,
	xmin=-1, xmax=10.8,
	ymin=-30, ymax=475, 
	axis lines = middle,
	axis line style={->},
	xlabel=$s \left(\si{\kilo\joule\per\kilogram\kelvin}\right)$, 
	xlabel style={
		at=(current axis.right of origin), 
		anchor=north east
	}, 
	ylabel=$T \left(\si{\degreeCelsius}\right)$, 
	ylabel style={
		at=(current axis.above origin), 
		anchor=north east
	},
	xtick={1, 2, 3, 4, 5, 6, 7, 8, 9},
	ytick={100, 200, 300, 400}
	]
	
	% Az adat az MHFGY Wolfram-jegyzetfüzetből származik
	
	% A nedves gőzmező fázishatárai
	\addplot[thick] table {./ny03mt_g69puv/ts.txt};
	
	% x jelölések
	\node[anchor=south east] at (axis cs: 3.5, 320) {\pgf{$x=0$}};
	\node[anchor=south east] at (axis cs: 6.4, 320) {\pgf{$x=1$}};
	
	% A kritikus pont
	\node[anchor=north west] at (axis cs: 4.40696, 373.919) {\pgfcircled{$K$}};
	\filldraw[black, fill=black] (axis cs: 4.40696, 373.919) circle (1mm);
	
	% 50bar-os izobar vonal
	\addplot[->, name path=A, ultra thick] table {./ny03mt_g69puv/50Bar_gorbe_1-3.txt}; 
	\addplot[thick] table {./ny03mt_g69puv/50Bar_gorbe_maradek.txt};
	
	% 1-es pont
	\node[anchor=south east] at (axis cs: 3.5, 263.941) {\pgfcircled{$1$}};
	\filldraw[black, fill=white] (axis cs: 3.5, 263.941) circle (1mm);
	
	% 2-es pont
	\node[anchor=south east] at (axis cs: 5.9731, 263.941) {\pgfcircled{$2$}};
	\filldraw[black, fill=white] (axis cs: 5.9731, 263.941) circle (1mm);
	
	% 3-as pont
	\node[anchor=south east] at (axis cs: 6.614, 390.850) {\pgfcircled{$3$}};
	\filldraw[black, fill=white] (axis cs: 6.614, 390.850) circle (1mm);
	
	% Függőleges vonalak és s jelölések
	\draw[very thin] (axis cs: 3.5, 263.941) -- (axis cs:  3.5, 0);
	\draw[very thin] (axis cs: 5.9731, 263.941) -- (axis cs:  5.9731, 0);
	\draw[very thin] (axis cs: 6.614, 390.850) -- (axis cs:  6.614, 0);
	\node[anchor=north west] at (axis cs: 3.3, -5) {\pgf{$s_1$}};
	\filldraw[black, fill=white] (axis cs: 3.5, 0) circle (0.75mm);
	\node[anchor=north west] at (axis cs: 5.9731, -5) {\pgf{$s_2$}};
	\filldraw[black, fill=white] (axis cs: 5.9731, 0) circle (0.75mm);
	\node[anchor=north west] at (axis cs: 6.414, -5) {\pgf{$s_3$}};
	\filldraw[black, fill=white] (axis cs: 6.614, 0) circle (0.75mm);
	
	% T jelölése
	\draw[very thin] (axis cs: 0, 263.941) -- (axis cs:  3.5, 263.941);
	\node[anchor=south east] at (axis cs: 0, 246.941) {\pgf{$T_{1,2}$}};
	\filldraw[black, fill=white] (axis cs: 0, 263.941) circle (0.75mm);
	
	\draw[very thin] (axis cs: 0, 390.850) -- (axis cs:  6.614, 390.850);
	\node[anchor=north east] at (axis cs: 0, 390.850) {\pgf{$T_3$}};
	\filldraw[black, fill=white] (axis cs: 0, 390.850) circle (0.75mm);
	
	% p jelölése
	\node[anchor=south west] at (axis cs: 6.614, 390.850) {\pgf{$p=kons.$}};
	
	% q jeloles
	\node[anchor=south east] at (axis cs: 5.2, 135) {\pgf{$q_{12}$}};
	\node[anchor=south east] at (axis cs: 6.65, 135) {\pgf{$q_{23}$}};
	
	%vonalkazas segedvonal
	\addplot+ [name path=B, ultra thin, domain=3.5:6.614, samples=2] {0};
	
	%vonalkazas
	\addplot[grey!30] fill between[of=A and B];
	
	\end{axis}
	
	\end{tikzpicture}
	\caption{Izobár T-s diagram}
\end{figure}

\pagebreak

%p-v izoterm

\usepgfplotslibrary{fillbetween}

\begin{figure}[h]
	\centering
	\label{figure:guh7ud-vgpvd}
	\begin{tikzpicture}
	
	% A tengelykeresztet az axis környezet hozza létre
	\begin{loglogaxis}[
	width=16cm, height=12cm,
	xmin=0.0003, xmax=10000,
	ymin=0.006, ymax=5000, 
	axis lines = middle,
	axis line style={->},
	log origin x=infty,
	log origin y=infty,
	xlabel=$v \left(\si{\meter\cubed\per\kilogram}\right)$, 
	xlabel style={
		at=(current axis.right of origin), 
		anchor=north east
	}, 
	ylabel=$p \left(\si{\bar}\right)$, 
	ylabel style={
		at=(current axis.above origin), 
		anchor=north east
	},
	xtick={0.001, 0.01, 0.1, 1, 10, 1000},
	ytick={0.01, 0.1, 1, 10, 100, 1000},
	extra x ticks={0.0031056,0.05,1.672373,20.540213},
	extra x tick labels={$v_K$,$v_1$,$v_2$,$v_3$},
	extra y ticks={220.64},
	extra y tick labels={$p_K$},
	]
	
	% Az adat az MHFGY Wolfram-jegyzetfüzetből származik
	
	% A nedves gőzmező fázishatárai
	\addplot[thick] table {./ny03mt_g69puv/pv.txt};
	
	%izoterma (100C)
	\addplot[ultra thick,name path=A,,->] table {./ny03mt_g69puv/100C izoterma.txt};	
	
	% A kritikus pont
	\node[anchor=south] at (axis cs: 0.0031056, 220.64) {$K$};
	\fill[fill=black] (axis cs: 0.0031056, 220.64) circle (0.75mm);
	
	% x jelölések
	\node[anchor=south east] at (axis cs: 0.005, 5) {\pgf{$x=0$}};
	\node[anchor=south east] at (axis cs: 1.2, 5) {\pgf{$x=1$}};
	
	% 1-es pont
	\node[anchor=south east] at (axis cs: 0.05, 1.014180) {\pgfcircled{$1$}};
	\filldraw[black, fill=white] (axis cs: 0.05, 1.014180) circle (1mm);
	
	% 2-es pont
	\node[anchor=south west] at (axis cs: 1.672373, 1.014180) {\pgfcircled{$2$}};
	\filldraw[black, fill=white] (axis cs: 1.672373, 1.014180) circle (1mm);
	
	% 3-as pont
	\node[anchor=south west] at (axis cs: 20.540213, 0.083741) {\pgfcircled{$3$}};
	\filldraw[black, fill=white] (axis cs: 20.540213, 0.083741) circle (1mm);
	
	% Függőleges vonal
	\draw[very thin] (axis cs: 0.05, 1.014180) -- (axis cs:  0.05, 0.006);
	\draw[very thin] (axis cs: 1.672373, 1.014180) -- (axis cs:  1.672373, 0.006);
	\draw[very thin] (axis cs: 20.540213, 0.083741) -- (axis cs:  20.540213, 0.006);
	
	%vonalkazas segedvonal
	\draw[very thin,name path=B,ultra thin] (axis cs: 0.05, 0.006) -- (axis cs: 20.540213, 0.006);
	
	%vonalkazas
	\addplot[grey!30] fill between[of=A and B];
	
	%W1-2,W2-3 pontok
	\node[anchor=north east] at (axis cs: 0.55, 0.1) {\pgf{$w_1$}};
	\node[anchor=north] at (axis cs: 6, 0.1) {\pgf{$w_2$}};
	
	
	\end{loglogaxis}
	
	\end{tikzpicture}
	\caption{Izoterm p-v diagram}
\end{figure}

\pagebreak

%T-s izoterm

\usepgfplotslibrary{fillbetween}

\begin{figure}[h]
	\centering
	\label{figure:vgtsd}
	\begin{tikzpicture}
	% Rács és vágómaszk
	%\draw[step=1cm, gray, very thin] (-1.5, -0.5) grid (14.5, 10.5);
	%\clip (-1.5, -1) rectangle (14.5, 11);
	
	% A tengelykeresztet az axis környezet hozza létre
	\begin{axis}[
	width=16cm, height=12cm,
	xmin=-1, xmax=10.8,
	ymin=-30, ymax=475, 
	axis lines = middle,
	axis line style={->},
	xlabel=$s \left(\si{\kilo\joule\per\kilogram\kelvin}\right)$, 
	xlabel style={
		at=(current axis.right of origin), 
		anchor=north east
	}, 
	ylabel=$T \left(\si{\degreeCelsius}\right)$, 
	ylabel style={
		at=(current axis.above origin), 
		anchor=north east
	},
	xtick={1, 2, 3, 4, 5, 6, 7, 8, 9},
	ytick={100, 200, 300, 400}
	]
	
	% Az adat az MHFGY Wolfram-jegyzetfüzetből származik
	
	% A nedves gőzmező fázishatárai
	\addplot[thick] table {./ny03mt_g69puv/ts.txt};
	
	% A kritikus pont
	\node[anchor=south east] at (axis cs: 4.40696, 373.919) {\pgfcircled{$K$}};
	\filldraw[black, fill=black] (axis cs: 4.40696, 373.919) circle (1mm);
	
	% 1-es pont
	\node[anchor=south east] at (axis cs: 4.5, 263.941) {\pgfcircled{$1$}};
	\filldraw[black, fill=white] (axis cs: 4.5, 263.941) circle (1mm);
	
	%2-es pont
	\node[anchor=south east] at (axis cs: 5.9731, 263.941) {\pgfcircled{$2$}};
	\filldraw[black, fill=white] (axis cs: 5.9731, 263.941) circle (1mm);
	
	%3-as pont
	\node[anchor=south east] at (axis cs: 7.5, 263.941) {\pgfcircled{$3$}};
	\filldraw[black, fill=white] (axis cs: 7.5, 263.941) circle (1mm);
	
	
	%Izoterma
	\draw[->,name path=A, ultra thick] (axis cs: 4.5, 263.941) -- (axis cs: 7.5, 263.941);
	
	%Függőleges vonalak és s jelölések
	\draw[very thin] (axis cs: 4.5, 263.941) -- (axis cs:  4.5, 0);
	\draw[very thin] (axis cs: 5.9731, 263.941) -- (axis cs:  5.9731, 0);
	\draw[very thin] (axis cs: 7.5, 263.941) -- (axis cs:  7.5, 0);
	\node[anchor=north west] at (axis cs: 4.3, -5) {\pgf{$s_1$}};
	\filldraw[black, fill=white] (axis cs: 4.5, 0) circle (0.75mm);
	\node[anchor=north west] at (axis cs: 5.9731, -5) {\pgf{$s_2$}};
	\filldraw[black, fill=white] (axis cs: 5.9731, 0) circle (0.75mm);
	\node[anchor=north west] at (axis cs: 7.3, -5) {\pgf{$s_3$}};
	\filldraw[black, fill=white] (axis cs: 7.5, 0) circle (0.75mm);
	
	
	
	%x jelölések
	\node[anchor=south east] at (axis cs: 3.5, 320) {\pgf{$x=0$}};
	\node[anchor=south east] at (axis cs: 6.4, 320) {\pgf{$x=1$}};
	
	%Tjelölése
	\draw[very thin] (axis cs: 0, 263.941) -- (axis cs:  4.5, 263.941);
	\node[anchor=south east] at (axis cs: 0, 263.941) {\pgf{$T_{1,2,3}$}};
	\filldraw[black, fill=white] (axis cs: 0, 263.941) circle (0.75mm);
	
	% q jeloles
	\node[anchor=south east] at (axis cs: 5.6, 100) {\pgf{$q_{12}$}};
	\node[anchor=south east] at (axis cs: 7.1, 100) {\pgf{$q_{23}$}};
	
	%vonalkazas segedvonal
	\draw[name path=B, ultra thin] (axis cs: 4.5, 0) -- (axis cs:  7.5, 0);
	%vonalkazas
	\addplot[grey!30] fill between[of=A and B];
	
	
	\end{axis}
	
	\end{tikzpicture}
	\caption{Izoterm T-s diagram}
\end{figure}

\pagebreak

%p-v izobar


\usepgfplotslibrary{fillbetween}

\begin{figure}[h]
	\centering
	\label{figure:guh7ud-vgpvd}
	\begin{tikzpicture}
	
	% A tengelykeresztet az axis környezet hozza létre
	\begin{loglogaxis}[
	width=16cm, height=12cm,
	xmin=0.0003, xmax=10000,
	ymin=0.006, ymax=5000, 
	axis lines = middle,
	axis line style={->},
	log origin x=infty,
	log origin y=infty,
	xlabel=$v \left(\si{\meter\cubed\per\kilogram}\right)$, 
	xlabel style={
		at=(current axis.right of origin), 
		anchor=north east
	}, 
	ylabel=$p \left(\si{\bar}\right)$, 
	ylabel style={
		at=(current axis.above origin), 
		anchor=north east
	},
	xtick={0.001, 0.01, 0.1, 1, 10, 1000},
	ytick={0.01, 0.1, 1, 10, 100, 1000},
	extra x ticks={0.0031056,2.429802,0.043,0.013260},
	extra x tick labels={$v_K$,$v_1$,$v_2$},
	extra y ticks={0.5,45.886615,200},
	extra y tick labels={$p_1$,$p_2$,$p_3$},
	]
	
	% Az adat az MHFGY Wolfram-jegyzetfüzetből származik
	
	% A nedves gőzmező fázishatárai
	\addplot[thick] table {./ny03mt_g69puv/pv.txt};
	
	% A kritikus pont
	\node[anchor=south] at (axis cs: 0.0031056, 220.64) {$K$};
	\fill[fill=black] (axis cs: 0.0031056, 220.64) circle (0.75mm);
	
	% x jelölések
	\node[anchor=south east] at (axis cs: 0.005, 5) {\pgf{$x=0$}};
	\node[anchor=south east] at (axis cs: 1.2, 5) {\pgf{$x=1$}};
	
	% 1-es pont
	\node[anchor=south east] at (axis cs: 2.229802, 0.26) {\pgfcircled{$1$}};
	\filldraw[black, fill=white] (axis cs: 2.429802, 0.500000) circle (1mm);
	
	% 2-es pont
	\node[anchor=south west] at (axis cs: 0.05,  45.86615) {\pgfcircled{$2$}};
	\filldraw[black, fill=white] (axis cs: 0.043, 45.886615) circle (1mm);
	
	% 3-as pont
	\node[anchor=south east] at (axis cs: 0.013260, 200.000000) {\pgfcircled{$3$}};
	\filldraw[black, fill=white] (axis cs: 0.013260, 200.000000) circle (1mm);
	
	% adiabata
	\addplot[ultra thick, name path=A] table {./ny03mt_g69puv/pv_adiabata.txt}; 
	
	% Függőleges méretvonalak
	\draw[very thin] (axis cs: 2.429802, 0.500000) -- (axis cs: 2.429802, 0.006);
	\draw[very thin] (axis cs: 0.043, 45.886615) -- (axis cs:  0.043, 0.006);
	\draw[very thin] (axis cs: 0.013260, 200.000000) -- (axis cs: 0.013260, 0.006);
	
	% Vízszintes vonalak
	\draw[very thin] (axis cs: 0.0003, 0.500000) -- (axis cs: 2.429802, 0.500000);
	\draw[very thin] (axis cs: 0.0003, 45.886615) -- (axis cs: 0.043, 45.886615);
	\draw[very thin] (axis cs: 0.0003, 200) -- (axis cs: 0.013260, 200.000000);
	
	%v_3_felirat
	\node[anchor=north west] at (axis cs:0.013260 , 0.011) {\pgf{$v_3$}};
	
	%vonalkazas segedvonal
	\draw[name path=B,ultra thin] (axis cs:0.013260 , 0.006) -- (axis cs: 2.429802, 0.006);
	
	
	%vonalkazas
	\addplot[grey!30] fill between[of=A and B];
	
	%W1-2,W2-3 pontok
	\node[anchor=north east] at (axis cs: 0.2, 0.5) {\pgf{$w$}};
	%\node[anchor=north] at (axis cs: 0.0135, 1) {\pgf{$w_{23}$}};
	
	
	
	\end{loglogaxis}
	
	\end{tikzpicture}
	\caption{Adiabatikus p-v diagram}
\end{figure}

\pagebreak

%T-s adiabatikus

\usepgfplotslibrary{fillbetween}

\begin{figure}[h]
	\centering
	\label{figure:vgtsd}
	\begin{tikzpicture}
	% Rács és vágómaszk
	%\draw[step=1cm, gray, very thin] (-1.5, -0.5) grid (14.5, 10.5);
	%\clip (-1.5, -1) rectangle (14.5, 11);
	
	% A tengelykeresztet az axis környezet hozza létre
	\begin{axis}[
	width=16cm, height=12cm,
	xmin=-1, xmax=10.8,
	ymin=-30, ymax=475, 
	axis lines = middle,
	axis line style={->},
	xlabel=$s \left(\si{\kilo\joule\per\kilogram\kelvin}\right)$, 
	xlabel style={
		at=(current axis.right of origin), 
		anchor=north east
	}, 
	ylabel=$T \left(\si{\degreeCelsius}\right)$, 
	ylabel style={
		at=(current axis.above origin), 
		anchor=north east
	},
	xtick={1, 2, 3, 4, 5, 6, 7, 8, 9},
	ytick={100, 200, 300, 400}
	]
	
	% Az adat az MHFGY Wolfram-jegyzetfüzetből származik
	
	% A nedves gőzmező fázishatárai
	\addplot[thick] table {./ny03mt_g69puv/ts.txt};
	
	% A kritikus pont
	\node[anchor=south east] at (axis cs: 4.40696, 373.919) {\pgfcircled{$K$}};
	\filldraw[black, fill=black] (axis cs: 4.40696, 373.919) circle (1mm);
	
	% 1-es pont
	\node[anchor=south east] at (axis cs: 6.497, 151.831) {\pgfcircled{$1$}};
	\filldraw[black, fill=white] (axis cs: 6.497, 151.831) circle (1mm);
	
	%2-es pont
	\node[anchor=south east] at (axis cs: 6.497, 185.850) {\pgfcircled{$2$}};
	\filldraw[black, fill=white] (axis cs: 6.497, 190.850) circle (1mm);
	
	%3-as pont
	\node[anchor=south east] at (axis cs: 6.497, 360.850) {\pgfcircled{$3$}};
	\filldraw[black, fill=white] (axis cs: 6.497, 360.850) circle (1mm);
	
	
	%p1_gorbe
	\addplot[thick] table {./ny03mt_g69puv/adiabatikus_p1.txt}; 
	\node[anchor=south east] at (axis cs: 5.4, 151.831) {\pgf{$p_1=const.$}};
	
	%p3_gorbe
	\addplot[thick] table {./ny03mt_g69puv/50Bar görbe.txt};
	\node[anchor=south east] at (axis cs: 4.7, 263.941) {\pgf{$p_3$}};
	
	%Függőleges vonalak és s jelölések
	\draw[very thin] (axis cs: 6.497, 360.850) -- (axis cs:  6.497, 0);
	\node[anchor=north west] at (axis cs: 6.2, -5) {\pgf{$s_{123}$}};
	\filldraw[black, fill=white] (axis cs: 6.497, 0) circle (0.75mm);
	
	
	
	%x jelölések
	\node[anchor=south east] at (axis cs: 3.5, 320) {\pgf{$x=0$}};
	\node[anchor=south east] at (axis cs: 6.4, 320) {\pgf{$x=1$}};
	
	%Tjelolesek
	%T_1jelölése
	\draw[very thin] (axis cs: 0, 151.831) -- (axis cs:  1.860, 151.831);
	\node[anchor=south east] at (axis cs: 0, 139) {\pgf{$T_1$}};
	\filldraw[black, fill=white] (axis cs: 0, 151.831) circle (0.75mm);
	
	%T_2jelolese
	%\draw[very thin] (axis cs: 0, 185.850) -- (axis cs:  6.497, 185.850);
	%\node[anchor=north east] at (axis cs: 0, 278.941) {\pgf{$T_2$}};
	%\filldraw[black, fill=white] (axis cs: 0, 263.941) circle (0.75mm);
	
	%T_3jelolese
	\draw[very thin] (axis cs: 0, 360.850) -- (axis cs:  6.497, 360.850);
	\node[anchor=north east] at (axis cs: 0, 375.850) {\pgf{$T_3$}};
	\filldraw[black, fill=white] (axis cs: 0, 360.850) circle (0.75mm);
	
	%1-3_szakasz
	\draw[ultra thick, mid arrow] (axis cs: 6.497, 151.831) -- (axis cs: 6.497, 360.850);
	
	\end{axis}
	
	\end{tikzpicture}
	\caption{Adiabatikus T-s diagram}
\end{figure}


% Oldaltörés
\pagebreak

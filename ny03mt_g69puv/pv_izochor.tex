% A tengelykeresztet az axis környezet hozza létre
	\begin{loglogaxis}[
	width=9.6cm, height=8cm,
	xmin=0.0003, xmax=10000,
	ymin=0.006, ymax=5000, 
	axis lines = middle,
	axis line style={->},
	log origin x=infty,
	log origin y=infty,
	xlabel=$v \left(\si{\meter\cubed\per\kilogram}\right)$, 
	xlabel style={
		at=(current axis.right of origin), 
		anchor=north east, xshift={8mm}
	}, 
	ylabel=$p \left(\si{\bar}\right)$, 
	ylabel style={
		at=(current axis.above origin), 
		anchor=north east
	},
	xtick={0.001, 0.01, 1, 10,100, 1000},
	ytick={0.01, 0.1, 10, 100, 1000},
	extra x ticks={0.0031056,0.1 ,0.25},
	extra x tick labels={$ $,$ $, $v_1$},
	extra y ticks={220.64,0.2,1.75,50,1},
	extra y tick labels={$p_K$,$p_1$,$p_2$,$p_3$,$ $},
	]
	
	% Az adat az MHFGY Wolfram-jegyzetfüzetből származik
	
	% A nedves gőzmező fázishatárai
	\addplot[thick] table {./ny03mt_g69puv/pv.txt};
	
	% x jelölések
	\node[anchor=west] at (axis cs: 0.001, 5) {\pgf{$x=0$}};
	\node[anchor=west] at (axis cs: 0.75, 5) {\pgf{$x=1$}};
	
	% 1-es pont
	\node[anchor=south east] at (axis cs: 0.25, 0.2) {\pgfcircled{$1$}};
	\filldraw[black, fill=white] (axis cs: 0.25, 0.2) circle (1mm);
	
	% 2-es pont
	\node[anchor=south east] at (axis cs: 0.25, 1.75) {\pgfcircled{$2$}};
	\filldraw[black, fill=white] (axis cs: 0.25, 1.75) circle (1mm);
	
	% 3-as pont
	\node[anchor=south east] at (axis cs: 0.25, 50) {\pgfcircled{$3$}};
	\filldraw[black, fill=white] (axis cs: 0.25, 50) circle (1mm);
	
	% Izochor vonal
	\draw[->, very thick] (axis cs: 0.25, 0.2) -- (axis cs:  0.25, 50);
	
	% Függőleges vonal
	\draw[very thin] (axis cs: 0.25, 0.2) -- (axis cs:  0.25, 0.006);
	
	% Vízszintes vonalak
	\draw[very thin] (axis cs: 0.0003, 0.2) -- (axis cs:  0.25, 0.2);
	\draw[very thin] (axis cs: 0.0003, 1.75) -- (axis cs:  0.25, 1.75);
	\draw[very thin] (axis cs: 0.0003, 50) -- (axis cs:  0.25, 50);	
	
	% A kritikus pont
	\node[anchor=south] at (axis cs: 0.0031056, 220.64) {\pgfcircled{$K$}};
	\fill[fill=black] (axis cs: 0.0031056, 220.64) circle (0.75mm);
	
	% A két görbe metszéspontja
	%\node[anchor=north east] at (axis cs: 0.836481, 1.01418) {\pgfcircled{$1$}};
	%\filldraw[black, fill=white] (axis cs: 0.836481, 1.01418) circle (1mm);
	
	\end{loglogaxis}
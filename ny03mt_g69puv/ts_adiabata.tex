
	% Rács és vágómaszk
	%\draw[step=1cm, gray, very thin] (-1.5, -0.5) grid (14.5, 10.5);
	%\clip (-1.5, -1) rectangle (14.5, 11);
	
	% A tengelykeresztet az axis környezet hozza létre
	\begin{axis}[
	width=9.6cm, height=8cm,
	xmin=-1, xmax=10.8,
	ymin=-45, ymax=475, 
	axis lines = middle,
	axis line style={->},
	xlabel=$s \left(\si{\kilo\joule\per\kilogram\kelvin}\right)$, 
	xlabel style={
		at=(current axis.right of origin), 
		anchor=north east, xshift={8mm}
	}, 
	ylabel=$T \left(\si{\degreeCelsius}\right)$, 
	ylabel style={
		at=(current axis.above origin), 
		anchor=north east
	},
	xtick={1, 2, 3, 4, 5, 6, 8, 9},
	ytick={100, 200, 300, 400},
	extra x ticks={7},
	extra x tick labels={$ $},
	]
	
	% Az adat az MHFGY Wolfram-jegyzetfüzetből származik
	
	% A nedves gőzmező fázishatárai
	\addplot[thick] table {./ny03mt_g69puv/ts.txt};
	
	% A kritikus pont
	\node[anchor=south east] at (axis cs: 4.40696, 373.919) {\pgfcircled{$K$}};
	\filldraw[black, fill=black] (axis cs: 4.40696, 373.919) circle (1mm);
	
	% 1-es pont
	\node[anchor=north east] at (axis cs: 6.497, 151.831) {\pgfcircled{$1$}};
	\filldraw[black, fill=white] (axis cs: 6.497, 151.831) circle (1mm);
	
	%2-es pont
	\node[anchor=south west] at (axis cs: 6.497, 185.850) {\pgfcircled{$2$}};
	\filldraw[black, fill=white] (axis cs: 6.497, 190.850) circle (1mm);
	
	%3-as pont
	\node[anchor=south east] at (axis cs: 6.497, 360.850) {\pgfcircled{$3$}};
	\filldraw[black, fill=white] (axis cs: 6.497, 360.850) circle (1mm);
	
	
	%p1_gorbe
	\addplot[thick] table {./ny03mt_g69puv/adiabatikus_p1.txt}; 
	\node[anchor=south] at (axis cs: 4, 151.831) {\pgf{$p_1=const.$}};
	
	%p3_gorbe
	\addplot[thick] table {./ny03mt_g69puv/50Bar görbe.txt};
	\node[anchor=south] at (axis cs: 4.5, 263.941) {\pgf{$p_3$}};
	
	%Függőleges vonalak és s jelölések
	\draw[very thin] (axis cs: 6.497, 360.850) -- (axis cs:  6.497, 0);
	\node[anchor=north west] at (axis cs: 6.2, -5) {\pgf{$s_{123}$}};
	\filldraw[black, fill=white] (axis cs: 6.497, 0) circle (0.75mm);
	
	
	
	%x jelölések
	\node[anchor=south east] at (axis cs: 2.4, 200) {\pgf{$x=0$}};
	\node[anchor=west] at (axis cs: 7.2, 120) {\pgf{$x=1$}};
	
	%Tjelolesek
		%T_1jelölése
	\draw[very thin] (axis cs: 0, 151.831) -- (axis cs:  1.860, 151.831);
	\node[anchor=east] at (axis cs: 0, 151.831) {\pgf{$T_1$}};
	\filldraw[black, fill=white] (axis cs: 0, 151.831) circle (0.75mm);
	
		%T_2jelolese
	%\draw[very thin] (axis cs: 0, 185.850) -- (axis cs:  6.497, 185.850);
	%\node[anchor=north east] at (axis cs: 0, 278.941) {\pgf{$T_2$}};
	%\filldraw[black, fill=white] (axis cs: 0, 263.941) circle (0.75mm);
	
		%T_3jelolese
	\draw[very thin] (axis cs: 0, 360.850) -- (axis cs:  6.497, 360.850);
	\node[anchor=east] at (axis cs: 0, 360.850) {\pgf{$T_3$}};
	\filldraw[black, fill=white] (axis cs: 0, 360.850) circle (0.75mm);
	
	%1-3_szakasz
	\draw[ultra thick, mid arrow] (axis cs: 6.497, 151.831) -- (axis cs: 6.497, 360.850);
	
	\end{axis}
	

% A tengelykeresztet az axis környezet hozza létre
	\begin{loglogaxis}[
	width=9.6cm, height=8cm,
	xmin=0.0003, xmax=10000,
	ymin=0.006, ymax=5000, 
	axis lines = middle,
	axis line style={->},
	log origin x=infty,
	log origin y=infty,
	xlabel=$v \left(\si{\meter\cubed\per\kilogram}\right)$, 
	xlabel style={
		at=(current axis.right of origin), 
		anchor=north east, xshift={8mm}
	}, 
	ylabel=$p \left(\si{\bar}\right)$, 
	ylabel style={
		at=(current axis.above origin), 
		anchor=north east yshift={1mm}
	},
	xtick={0.001, 0.01, 1, 10,100, 1000},
	ytick={0.01, 0.1, 10, 1000},
	extra x ticks={2.429802,0.043,0.013260, 0.1},
	extra x tick labels={$v_1$,$v_2$, $ $},
	extra y ticks={0.5,45.886615,200, 1, 100},
	extra y tick labels={$p_1$,$p_2$,$p_3$, $10^0$ $ $},
	]
	
	% Az adat az MHFGY Wolfram-jegyzetfüzetből származik
	
	% A nedves gőzmező fázishatárai
	\addplot[thick] table {./ny03mt_g69puv/pv.txt};
	
	% A kritikus pont
	\node[anchor=south] at (axis cs: 0.0031056, 220.64) {\pgfcircled{$K$}};
	\fill[fill=black] (axis cs: 0.0031056, 220.64) circle (0.75mm);
	
	% x jelölések
	\node[anchor=west] at (axis cs: 0.001, 5) {\pgf{$x=0$}};
	\node[anchor=west] at (axis cs: 0.75, 5) {\pgf{$x=1$}};
	
	% 1-es pont
	\node[anchor=north east] at (axis cs: 2.429802, 0.500000) {\pgfcircled{$1$}};
	\filldraw[black, fill=white] (axis cs: 2.429802, 0.500000) circle (1mm);
	
	% 2-es pont
	\node[anchor=south west] at (axis cs: 0.05,  45.86615) {\pgfcircled{$2$}};
	\filldraw[black, fill=white] (axis cs: 0.043, 45.886615) circle (1mm);
	
	% 3-as pont
	\node[anchor=south] at (axis cs: 0.013260, 200.000000) {\pgfcircled{$3$}};
	\filldraw[black, fill=white] (axis cs: 0.013260, 200.000000) circle (1mm);
	
	% adiabata
	\addplot[ultra thick, name path=A] table {./ny03mt_g69puv/adiabata.txt}; 
	
	% Függőleges méretvonalak
	\draw[very thin] (axis cs: 2.429802, 0.500000) -- (axis cs: 2.429802, 0.006);
	\draw[very thin] (axis cs: 0.043, 45.886615) -- (axis cs:  0.043, 0.006);
	\draw[very thin] (axis cs: 0.013260, 200.000000) -- (axis cs: 0.013260, 0.006);
	
	% Vízszintes vonalak
	\draw[very thin] (axis cs: 0.0003, 0.500000) -- (axis cs: 2.429802, 0.500000);
	\draw[very thin] (axis cs: 0.0003, 45.886615) -- (axis cs: 0.043, 45.886615);
	\draw[very thin] (axis cs: 0.0003, 200) -- (axis cs: 0.013260, 200.000000);
	
	%v_3_felirat
	\node[anchor=north east] at (axis cs:0.015360 , 0.018) {\pgf{$v_3$}};
	
	%vonalkazas segedvonal
	\draw[name path=B,ultra thin] (axis cs:0.013260 , 0.006) -- (axis cs: 2.429802, 0.006);

	
	%vonalkazas
	\addplot[gray!30] fill between[of=A and B];
	
	%W1-2,W2-3 pontok
	\node[anchor=north] at (axis cs: 0.25, 0.4) {\pgf{$w$}};
	%\node[anchor=north] at (axis cs: 0.0135, 1) {\pgf{$w_{23}$}};
	
	
	
	\end{loglogaxis}
% A tengelykeresztet az axis környezet hozza létre
	\begin{axis}[
	width=9.6cm, height=8cm,
	xmin=-1, xmax=10.8,
	ymin=-45, ymax=475, 
	axis lines = middle,
	axis line style={->},
	xlabel=$s \left(\si{\kilo\joule\per\kilogram\kelvin}\right)$, 
	xlabel style={
		at=(current axis.right of origin), 
		anchor=north east, xshift={8mm}
	}, 
	ylabel=$T \left(\si{\degreeCelsius}\right)$, 
	ylabel style={
		at=(current axis.above origin), 
		anchor=north east
	},
	xtick={1, 2, 3, 4, 5, 6, 7, 8, 9},
	ytick={100, 200, 300, 400},
	extra y ticks={263.941},
	extra y tick labels={$T_{123}$}
	]
	
	% Az adat az MHFGY Wolfram-jegyzetfüzetből származik
	
	% A nedves gőzmező fázishatárai
	\addplot[thick] table {./ny03mt_g69puv/ts.txt};
	
	% A kritikus pont
	\node[anchor=south] at (axis cs: 4.40696, 373.919) {\pgfcircled{$K$}};
	\filldraw[black, fill=black] (axis cs: 4.40696, 373.919) circle (1mm);
	
	% 1-es pont
	\node[anchor=south east] at (axis cs: 4.5, 263.941) {\pgfcircled{$1$}};
	\filldraw[black, fill=white] (axis cs: 4.5, 263.941) circle (1mm);
	
	%2-es pont
	\node[anchor=south east] at (axis cs: 5.9, 263.941) {\pgfcircled{$2$}};
	\filldraw[black, fill=white] (axis cs: 5.9731, 263.941) circle (1mm);
	
	%3-as pont
	\node[anchor=south east] at (axis cs: 7.5, 263.941) {\pgfcircled{$3$}};
	\filldraw[black, fill=white] (axis cs: 7.5, 263.941) circle (1mm);
	
	
	%Izoterma
	\draw[->,name path=A, ultra thick] (axis cs: 4.5, 263.941) -- (axis cs: 7.5, 263.941);
	
	%Függőleges vonalak és s jelölések
	\draw[very thin] (axis cs: 4.5, 263.941) -- (axis cs:  4.5, 0);
	\draw[very thin] (axis cs: 5.9731, 263.941) -- (axis cs:  5.9731, 0);
	\draw[very thin] (axis cs: 7.5, 263.941) -- (axis cs:  7.5, 0);
	\node[anchor=north] at (axis cs: 4.5, -5) {\pgf{$s_1$}};
	\filldraw[black, fill=white] (axis cs: 4.5, 0) circle (0.75mm);
	\node[anchor=north west] at (axis cs: 5.9731, -5) {\pgf{$s_2$}};
	\filldraw[black, fill=white] (axis cs: 5.9731, 0) circle (0.75mm);
	\node[anchor=north] at (axis cs: 7.5, -5) {\pgf{$s_3$}};
	\filldraw[black, fill=white] (axis cs: 7.5, 0) circle (0.75mm);
	
	
	
	%x jelölések
	\node[anchor=south east] at (axis cs: 2.4, 200) {\pgf{$x=0$}};
	\node[anchor=west] at (axis cs: 7.6, 105) {\pgf{$x=1$}};
	
	%Tjelölése
	\draw[very thin] (axis cs: 0, 263.941) -- (axis cs:  4.5, 263.941);
	\filldraw[black, fill=white] (axis cs: 0, 263.941) circle (0.75mm);
	
	% q jeloles
	\node[anchor=south] at (axis cs: 5.35, 80) {\pgf{$q_{12}$}};
	\node[anchor=south] at (axis cs: 6.75, 80) {\pgf{$q_{23}$}};
	
	%vonalkazas segedvonal
	\draw[name path=B, ultra thin] (axis cs: 4.5, 0) -- (axis cs:  7.5, 0);
	%vonalkazas
	\addplot[gray!30] fill between[of=A and B];
	
	
	\end{axis}
\section*{K6/5. feladat}
\addcontentsline{toc}{section}{H1/5. feladat}

Freon-12 nevű anyag kondenzálódik $d = \SI{20}{\milli\meter}$ külső átmérőjű vízszintes csövek felületén. Egymás alatt függőleges síkban $z = 18 db$ csövet helyezünk el. A kondenzáció hőmérséklete $t_s = \SI{25}{\celsius}$. A csövekben hűtővíz áramlik, a csőfal hőmérsékletét $t_w = \SI{21}{\celsius}$-ra becsüljük. Határozzuk meg a kondenzáció hőátadási tényezőjét!

\vspace{6mm}

Nusselt képlet:

\vspace{2mm}

\begin{equation}
\alpha = c \left(\dfrac{r \varrho^2 \lambda^3 g}{\eta H \Delta T}\right)^{\tfrac{1}{4}}
\end{equation}

\vspace{2mm}

Adatok:

\vspace{2mm}
\begin{equation}
	\label{equation:k65T}
\left.
\begin{array}{l}
r = \SI{142,8}{\kilo\joule\per\kilogram} = \SI{142800}{\joule\per\kilogram}
\quad
\varrho = \SI{1318}{\kilogram\per\meter\cubed}
\\ \\
\lambda = \SI{0,0897}{\watt\per\meter\kelvin}
\quad
\eta = \SI{2,3e-4}{\newton\second\per\meter\squared\left(\kilogram\per\second\meter)\right}
\end{array}
\right\rbrace \quad \text{\SI{23}{\celsius}-nal}
\end{equation}

\vspace{6mm}
Megoldás: Az egyenletbe behelyettesítünk

\begin{equation}
\alpha = c \left(\dfrac{r \varrho^2 \lambda^3 g}{\eta H \Delta T}\right)^{\tfrac{1}{4}} = \SI{1101,7}{\watt\per\meter\squared\kelvin}
\end{equation}


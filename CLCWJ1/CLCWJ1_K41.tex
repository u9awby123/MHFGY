


\section*{K4/1. feladat:Hűtőkörfolyamatok utóhűtővel}
\addcontentsline{toc}{section}{K4/1. feladat:Hűtőkörfolyamatok utóhűtővel}

\begin{tabular}{ | p{2cm} | p{14cm} | } 
	\hline
	Szerző & Nagy Krisztina CLCWJ1 \\ 
	\hline
	Szak & Környezetmérnök \\ 
	\hline
	Félév & 2019/2020 II. (tavaszi) félév \\ 
	\hline
\end{tabular}
\vspace{0.5cm}

% A feladat szövege
\noindent Egy ammóniák közvetítő közeggel dolgozó kompresszoros hűtőgép, utóhűtővel, \SI{-20}{\celsius} elpárolgási és \SI{25}{\celsius} kondenzációs hőmérséklet mellett $3\cdot10^5 W$ hűtőteljesítményt szolgáltat, $p_2/p_1 = \SI{10,0}{\bar}$ /  $\SI{1,90}{\bar}$ nyomásviszonynál. A kompresszió végén száraz telített gőz  $(x=\SI{1}{}) $ van.

\vspace{2mm}

a) Határozzuk meg a kompresszió elején a gőz állapotjelzőit!

b) Kiszámítandó a keringetett hűtőközeg mennyisége $(\dot{G})$ a kompresszoros elméleti teljesítménye $(Pelm)$

c)és a fajlagos hűtőteljesítmény ($\epsilon$) !

d) Mennyi a kondenzátorban leadott hő ?

\subsubsection{Ismert jellemzők}
\begin{equation*}
	T_a = \SI{-20}{\celsius},
	\quad
	T_f = \SI{+25}{\celsius},
	\quad
	p_1 = \SI{1,9}{\bar},
	\quad
	p_2 = \SI{10}{\bar},
	\quad
	\dot{Q}_H = 3\cdot10^5 W,
	\quad
	x_2 = \SI{1}{},
\end{equation*}
\begin{equation*}
	s_1' = \SI{1,632}{\kilo\joule\per\kilogram\kelvin},
	\quad
	s_1'' = \SI{6,883}{\kilo\joule\per\kilogram\kelvin},
	\quad
	s_2 = \SI{6,3}{\kilo\joule\per\kilogram\kelvin},
	\quad
	v_1' = \SI{0,0015}{\meter\cubed\per\kilogram},
	\quad
	v_1'' = \SI{0,6239}{\meter\cubed\per\kilogram},
\end{equation*}
\begin{equation*}
	h_1' = \SI{408,8}{\kilo\joule\per\kilogram},
    \quad
    h_1'' = \SI{1737,9}{\kilo\joule\per\kilogram},
    \quad
	h_2 = \SI{1784}{\kilo\joule\per\kilogram},
    \quad
    h_3' = \SI{617}{\kilo\joule\per\kilogram},
    \quad
    h_3 = h_4 = \SI{596}{\kilo\joule\per\kilogram},
\end{equation*}

\noindent\hrulefill

\subsubsection*{a) A fajlagos gőztartalom a sűrítés elején}

\begin{equation}
   x_1
   = 
   \dfrac{s_2 - s_1'}{s_1'' - s_1'} 
   = 
   \SI{0,889}{}
\end{equation}
\begin{equation}
   h_1 = \left(1 - x_1\right) h_1' + x_1 h_1'' 
   = 
   \SI{1590}{\kilo\joule\per\kilogram}
\end{equation}
\begin{equation}
   v_1 = \left(1 - x_1\right) v_1' + x_1 v_1'' 
   = 
   \SI{0,5548}{\meter\cubed\per\kilogram}
\end{equation}
 
\subsubsection*{b) Tömegfajlagos hőelvonás a hűtőtérből}
 
\begin{equation}
  q_H = h_1-h_4 
  = 
  \SI{994}{\kilo\joule\per\kilogram}
\end{equation}
 
 \subsubsection*{Keringetett hűtőközeg tömegárama}
 
\begin{equation}
   {\dot{m}}_{NH3}= \dfrac{\dot{Q}_H}{q_h} = \SI{0,3018}{\kilogram\per\second}
\end{equation}

\subsubsection*{Térfogatfajlagos hűtőteljesítmény}
  
\begin{equation}
    {q}_{VOL} = q_H\cdot \rho_1
    =
    \dfrac{q_H}{v_1}
    =\SI{1791,64}{\kilo\joule\per\meter\cubed}
\end{equation} 
\subsubsection*{A kompresszor óránkénti lökettérfogata, a térfogatkiszorítás hatásfokának figyelembevételével}

\noindent $ {\alpha}_{VOL} = 0,7 $

\begin{equation}
    {\dot{V}}_{H}= \dfrac{{\dot{Q}}_H}{{q}_{VOL} {\alpha}_{VOL}} = \SI{860,9}{\meter\cubed\per\hour}
\end{equation}

\begin{equation}
   n=210 \dfrac{1}{min}
\end{equation}
\begin{equation}
    V = \dfrac{{\dot{V}}_{H}}{n} = 68 l 
\end{equation}

\subsubsection*{A henger átmérője, ha a löketfurat 68 liter}

\begin{equation}
    V=\dfrac{d^2\cdot\pi}{4\cdot s} = \dfrac{d^2\cdot\pi\cdot 1,2d}{4}
\end{equation}

\begin{equation}
     d =\left(\dfrac{4V}{1,2\pi}\right)^{\dfrac{1}{3}} = 0,416 m
\end{equation}

\begin{equation}
     {P}_{elm}=\dot{m} \cdot W_K = \dot{m} (h_2 - h_1)=58,55 W
\end{equation}

\begin{equation}
     \nu_o = 0,643
     \quad
     {P}_{val} = \dfrac{{P}_{elm}}{\nu_o} = 91,06 kW
\end{equation}

\subsubsection* { c) Fajlagos hűtőteljesítmény }

\begin{equation}
     \epsilon = \dfrac{q_H}{W_K} = 5,124
\end{equation}

\subsubsection* {d) Lecsapatóban leadott hő}

\begin{equation}
     q_K = h_2 - h_ 3' = \SI{1167}{\kilo\joule\per\kilogram}
\end{equation}
\begin{equation}
     \dot{Q}_K = \dot{m} \cdot q_K = 352 kW
\end{equation}

\subsubsection*{e) Ha a lecsapatóban leadott hő is hasznosítható, akkor egy hőtranszportot kapunk, aminek a fajlagos hőteljesítménye}

\begin{equation}
    \epsilon = \dfrac{q_H + q_K}{W_K} = 11,14
\end{equation}

\subsubsection* {A hűtőgép kapcsolási vázlata és T-s diagramja}

\begin{figure}[h]
	\centering
	\label{figure:sm}
	\begin{tikzpicture}
	% Rács és vágómaszk
	%\draw[step=1cm, gray,  ultra thin] (-8, -8) grid (8, 8);
	%\clip (-8, -8) rectangle (8, 8);
	
		% Az ábra tartalma
			%alsó
			\draw (-5, -5) -- (-3.5, -5) -- (-3.5, -4.5) -- (-3, -5.5) -- (-2.5, -4.5) -- (-2, -5.5) -- (-1.5, -4.5) -- (-1, -5.5) -- (-0.5, -4.5) -- (0, -5.5) -- (0.5, -4.5) -- (1, -5.5) -- (1.5, -4.5) -- (1.5, -5)-- (5, -5);
		       	  %nyíl fej rá
		  	      \draw[-> ] (3, -5) -- (4, -5);
		  	      
		  	  \draw [ultra thick] (-4, -6) -- (-4, -4) -- (2, -4) -- (2, -6);
		  	  
		  	  %feliratok
			  	  % szám: 4
			  	  \node[anchor=mid] at (-5, -6) {$4$};
			  	  \draw(-5, -6) circle(0.5);
			  	   % szám: 1
			  	  \node[anchor=mid] at (5, -6) {$1$};
			  	  \draw(5, -6) circle(0.5);
			  	  %ELP
			  	  \node[anchor=mid] at (-1, -6) {$ELP$};
		  	      
			%jobb oldal
			\draw (5, -5) -- (5, 5);
			      %nyíl fej rá
		       	  \draw[-> ] (5, 3) -- (5, 4);
		       	  
		       	\draw (7, -1) -- (4, -1) -- (4, 1) -- (7,1);
		       	\draw (5.5, -1) -- (5.5, 1);
		       	\draw (5.5, 0) -- (7, 0);
		       	  
			%felső
			\draw (5, 5) -- (4, 5) -- (4, 4.5) -- (3.5, 5.5) --(3, 4.5) -- (2.5, 5.5)--(2, 4.5) -- (1.5, 5.5) -- (1, 4.5) -- (0.5, 5.5) -- (0, 4.5) -- (-0.5, 5.5) -- (-1, 4.5) -- (-1.5, 5.5) -- (-2, 4.5) -- (-2.5, 5.5) -- (-3, 4.5) -- (-3, 5) -- (-5, 5);
		
				  %nyíl fej rá
			      \draw[->] (-3, 5) -- (-4, 5);
			      
			       %feliratok
				      % szám: 2
				      \node[anchor=mid] at (5, 6) {$2$};
				      \draw(5, 6) circle(0.5);
				      % szám: 3
				      \node[anchor=mid] at (-5, 6) {$3$};
				      \draw(-5, 6) circle(0.5);
				      %LCS
				      \node[anchor=mid] at (0, 4) {$L.CS$};
			
			%baloldal
			   %U.H
			   \draw (-5, 5) -- (-5, 4) -- (-5.5, 4) -- (-4.5, 3.5) -- (-5.5, 3) -- (-4.5, 2.5) -- (-5.5, 2) -- (-4.5, 1.5) -- (-5.5, 1) -- (-5, 1) -- (-5, 0) -- (-5, -1);
			   
			  \draw (-6, 0) rectangle (-4, -2);
			  \draw (-4, -2) -- (-5, -1) -- (-6, -2);
			  \draw (-5, -2) -- (-5, -5);
			       %nyíl fej rá
		   	       \draw[->] (-5, -3) -- (-5, -4);
		   	       
		   	        %felirat
			   	       % szám: 3'
			   	       \node[anchor=mid] at (-6, 1) {$3'$};
			   	       \draw(-6, 1) circle(0.5);
			   	       %LCS
			   	       \node[anchor=mid] at (-3.5, 2.5) {$U.H$};
		
	\end{tikzpicture}
	\caption{Hűtőgép kapcsolási vázlata}
\end{figure}

\begin{figure}[h]
	\centering
	%\label{figure:CLCWJ1-tsd}
	\begin{tikzpicture}
	\draw[step=1cm, gray, very thin] (-1.5, -1) grid (14.5, 11);
	\clip (-1.5, -1) rectangle (14.5, 11);
	
	\begin{axis}[
	width=16cm, height=12cm,
	xmin=0, xmax=10.8,
	ymin=0, ymax=475, 
	axis lines = middle,
	axis line style={->},
	xlabel=$s \left(\si{\kilo\joule\per\kilogram\kelvin}\right)$, 
	xlabel style={
		at=(current axis.right of origin), 
		anchor=north east
	}, 
	ylabel=$T \left(\si{\degreeCelsius}\right)$, 
	ylabel style={
		at=(current axis.above origin), 
		anchor=north east
	},
	%xtick={1, 2, 3, 4, 5, 6, 7, 8, 9},
	%ytick={270, 300, 400, 500, 600, 700}
	]
	%\addplot[thick] table {./CLCWJ1/T_s_diagram.txt};
	\end{axis}
	
	
	\end{tikzpicture}
	\caption{ $T-s$ diagram}
	
\end{figure}



					
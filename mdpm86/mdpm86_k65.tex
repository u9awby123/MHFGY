\section*{K6/5. feladat: Freon-12 lecsapódása}
\addcontentsline{toc}{section}{K6/5. feladat: Freon-12 lecsapódása}

\begin{tabular}{ | p{2cm} | p{14cm} | } 
	\hline
	Név & Vasáros Mátyás \\ 
	\hline
	Szak & Mechatronikai mérnöki alapszak\\ 
	\hline
	Félév & 2019/2020 II. (tavaszi) félév \\ 
	\hline
\end{tabular}
\vspace{0.5cm}

\noindent A feladat freon-12 anyag kondenzációja esetén a hőátadási tényező meghatározása. Az anyag $ d=\SI {20}{\milli\meter}$ külső átmérőjű vízszintes csövek felületén csapódik le, melyekből egymás alatt $z= 18db$ helyezkedik el függőleges síkban és víz folyik bennük. A kondenzáció hőmérséklete $T_{k}=\SI{25}{\degreeCelsius}$, a csőfal hőmérsékletét pedig $T_f=\SI{21}{\degreeCelsius}$-ra becsüljük. Ismert $T=\SI{23}{\degreeCelsius}$-on a freon-12 $\rho$ sűrűsége, a $\lambda$ hőátszármaztatási tényező, a $\eta$ dinamikai viszkozitás és az $r$ fűtőérték.

\begin{equation*}
	\rho=\SI{1318}{\kilogram\per\meter\cubed},
	\quad 
	\lambda=\SI{0,0897}{\watt\per\meter\kelvin},
	\quad 
	\eta=\SI{2,3e-4}{\pascal\cdot\second },
	\quad 
	r=\SI{142,8}{\kilo\joule\per\kilogram}=\SI{142800}{\joule\per\kilogram}
\end{equation*}

\noindent\hrulefill

\noindent A probléma megoldására az alábbi Nusselt képlet alkalmas.Ez egy cső esetét vizsgálva ahol $h=d$ a következő.

\begin{equation}
\alpha_{v}=0,726\left(\dfrac{r g \rho^2 \lambda^3 }{\eta}\dfrac{1}{H \Delta T} \right)^\tfrac{1}{4}
\end{equation} 

\noindent A feladatmegoldás során a nehézségi gyorsulást $g=\SI{9,81}{\meter\per\second\squared}$-nek tekintjük. Az egyetlen dolog amit meg kell határozni már csak a $\Delta T$.

\begin{equation}
\Delta T = T_k-T_f=\SI{25}{\degreeCelsius}-\SI{21}{\degreeCelsius}=\SI{4}{\degreeCelsius}=\SI{4}{\kelvin}
\end{equation} 

\noindent Az értékeket a Nusselt képletbe helyettesítve a következő eredményt kapjuk.

\begin{equation}
\alpha_{v}=0,726\left(\dfrac{r g \rho^2 \lambda^3 }{\eta}\dfrac{1}{H \Delta T} \right)^\tfrac{1}{4}=\SI{1241,89}{\watt\per\meter\squared\kelvin}
\end{equation} 

\noindent A fenti egyenlet csak egyetlen csőre vonatkozik így a következő módon ki kell számolnunk az egymás alatt elhelyezkedő csövekre.

\begin{equation}
\alpha=\alpha_{v} z^{-\tfrac{1}{4}}=\SI{602,93}{\watt\per\meter\squared\kelvin}
\end{equation} 
%állapot változók
%1-es pont
\newcommand\pegy{0.176} %nyomás MPa-ban
\newcommand\TegyK{389.34} %hőmérséklet kelvinben
\newcommand\TegyC{116.19} %hőmérséklet celsiusban
\newcommand\vegy{0.699}
\newcommand\uegy{1913.2}
\newcommand\hegy{2036.3}
\newcommand\segy{5.4645}
\newcommand\xegy{0.7}
% " pont
\newcommand\pvesszo{0.2575}
\newcommand\TvesszoK{401.52}
\newcommand\TvesszoC{128.37}
\newcommand\vvesszo{0.699}
\newcommand\uvesszo{2537.3}
\newcommand\hvesszo{2717.3}
\newcommand\svesszo{7.0419}
\newcommand\xvesszo{1}
%2-es pont
\newcommand\pketto{0.3073}
\newcommand\TkettoK{473.15}
\newcommand\TkettoC{200}
\newcommand\vketto{0.699}
\newcommand\uketto{2649.5}
\newcommand\hketto{2864.3}
\newcommand\sketto{7.2992}

\newcommand\pontvastagsag{0.7 mm} %állapotjelző pontoknál a pont mérete

\section*{K2/13. feladat: Vízgőz melegítése állandó térfogaton}
\begin{tabular}{ | p{2cm} | p{14cm} | } 
	\hline
	Név & Rupert Balázs \\ 
	\hline
	Szak & Gépészmérnök\\ 
	\hline
	Félév & 2019/2020 II. (tavaszi) félév \\ 
	\hline
\end{tabular}
\vspace{0.5cm}

	Határozza meg, hogy mennyi hőt kell közölni 1 kg $p_1=\SI{1,76}{\bar}$ nyomású és $x = 0,7$ fajlagos gőztartalmú vízgőzzel \textit{állandó térfogaton} ahhoz, hogy a hőmérséklete $\SI{200}{\celsius}$ legyen. Először $h-s$ diagram segítségével oldja meg a feladatot, majd ábrázolja az állapotváltozást $T-s$ és $p-v$ diagramban is, és határozza meg a végállapotban a gőz többi állapotjelzőit is gőztáblázat segítségével! ($v_2$; $p_2$; $i_2$; $u_2$)
	\vspace{2mm}

\subsection*{a) Feladat felismerése}
	Mivel ebben a feladatban a vízgőzt állandó térfogaton melegítjük, ezért az állapotváltozás izochor.
	Izochor állapotváltozásnál a következőket ismerjük:
	
\begin{equation}
	v=const. \rightarrow dv=0
\end{equation}

\begin{equation}
	q_{1,2}=u_2-u_1=c_v(T_2-T_1)
\end{equation}

\begin{equation}
	w_{t_{1,2}}=0	
\end{equation}

\subsection*{b) Állapotváltozók meghatározása a nevezetes pontokban}
\subsubsection{- Nedves-gőz állapotjelzők $x = 0,7$ és $p = \SI{1,76}{\bar}$ esetén:}
\begin{equation}
	p_1=\SI{1,76}{\bar},
	\quad
	T_1=\SI{389,34}{\kelvin},
	\quad
	v_1=\SI{0,699}{\meter\cubed\per\kilogram},
	\quad
	u_1=\SI{1913,2}{\kilo\joule\per\kilogram},
\end{equation}
\begin{equation}
	h_1=\SI{2036,3}{\kilo\joule\per\kilogram},
	\quad
	s_1=\SI{5,4645}{\kilo\joule\per\kilogram\kelvin},
	\quad
	x_1=0,7
\end{equation}

\subsubsection{- Telített gőz vonalon az állapotjelzők:}
\begin{equation}
	p"=\SI{2,575}{\bar},
	\quad
	T"=\SI{401,52}{\kelvin},
	\quad
	v"=\SI{0,699}{\meter\cubed\per\kilogram},
	\quad
	u"=\SI{2537,3}{\kilo\joule\per\kilogram},
\end{equation}
\begin{equation}
	h"=\SI{2717,3}{\kilo\joule\per\kilogram},
	\quad
	s"=\SI{7,0419}{\kilo\joule\per\kilogram\kelvin},
	\quad
	x"=1
\end{equation}

\subsubsection{- Túlhevített gőz mezőnél a 2-es pontban az állapotjelzők:}
\begin{equation}
	p_2=\SI{3,073}{\bar},
	\quad
	T_2=\SI{473,15}{\kelvin},
	\quad
	v_2=\SI{0,699}{\meter\cubed\per\kilogram},
	\quad
	u_2=\SI{2649,5}{\kilo\joule\per\kilogram},
\end{equation}
\begin{equation}
	h_2=\SI{2864,3}{\kilo\joule\per\kilogram},
	\quad
	s_2=\SI{7,2992}{\kilo\joule\per\kilogram\kelvin}
\end{equation}

\subsection*{c) Szükséges hő számítása:}
Az a) pontban ismertetett képlet alapján számoljuk a szükséges hőt,
\begin{equation}
	q_{1,2}=u_2-u_1=\SI{2649,5}{\kilo\joule\per\kilogram}-\SI{1913,2}{\kilo\joule\per\kilogram}=\SI{736,3}{\kilo\joule\per\kilogram}
\end{equation}
\subsection*{d) Állapotváltozási diagramok megrajzolása:}
\centering
%T-s diagram
\begin{figure}[h]
	\centering
	\begin{tikzpicture}
	%sraffozás beállításai
		[
		hatch distance/.store in=\hatchdistance,
		hatch distance=6pt,
		hatch thickness/.store in=\hatchthickness,
		hatch thickness=0.5pt
		]
		\makeatletter
		\pgfdeclarepatternformonly[\hatchdistance,\hatchthickness]{flexible hatch}
		{\pgfqpoint{0pt}{0pt}}
		{\pgfqpoint{\hatchdistance}{\hatchdistance}}
		{\pgfpoint{\hatchdistance-1pt}{\hatchdistance-1pt}}
		{
			\pgfsetcolor{\tikz@pattern@color}
			\pgfsetlinewidth{\hatchthickness}
			\pgfpathmoveto{\pgfqpoint{0pt}{0pt}}
			\pgfpathlineto{\pgfqpoint{\hatchdistance}{\hatchdistance}}
			\pgfusepath{stroke}
		}
		
		%változó a tábla méreteinek arányos változtatására
		\newcommand\kicsinyites{0.6};
		
		% Tengelyek
		\draw[->] ({0}, {0}) -- ({14*\kicsinyites}, {0}) node[anchor=base east, shift={(0.4,-0.7)}]{$s$ [\si{\kilo\joule\per\kilogram\kelvin}]};
		\draw[->] ({0}, {0}) -- ({0}, {9*\kicsinyites}) node[anchor=north east]{$T$ $[\si{\celsius}]$};
		
		\begin{axis}[
			axis lines = middle,
			axis line style = {draw = none},
			xlabel={},
			ylabel={},
			ytick={\TegyC-18, \TvesszoC+10, \TkettoC},
			yticklabels={$T_1=116.19$, $T"=128.37$, $T_2=200$},
			ytick style={draw=none},
			xtick={\segy, \svesszo-0.2, \sketto+0.2},
			xticklabels={$s_1=5.46$, $s"=7.04$, $s_2=7.30$},
			x tick label style={rotate=90,anchor=east},
			xtick style={draw=none},
			width=15cm*\kicsinyites, height=10cm*\kicsinyites, xmin=0, ymin=0	
			]
			
			%T-s alap
			\addplot [thick] table {./I8FMUH/tsah.txt};
			\addplot [thick] table {./I8FMUH/tsfh.txt};
			
			%nevezetes pontok
			\draw[fill] (axis cs:\segy, \TegyC) circle [radius = \pontvastagsag];
			\draw[fill] (axis cs:\svesszo, \TvesszoC) circle [radius = \pontvastagsag];
			\draw[fill] (axis cs:\sketto, \TkettoC) circle [radius = \pontvastagsag];
			\node[thick, circle, draw, inner sep = 1mm] at (axis cs:4.9, 75) {1};
			\node[thick, circle, draw, inner sep = 1mm] at (axis cs:7.8, 230) {2};
			
			%allapotjelzo egyenesek
			\draw[dashed] (axis cs:\segy, \TegyC) -- (axis cs:0, \TegyC);
			\draw[dashed] (axis cs:\svesszo, \TvesszoC) -- (axis cs:0, \TvesszoC);
			\draw[dashed] (axis cs:\sketto, \TkettoC) -- (axis cs:0, \TkettoC);
			\draw[dashed] (axis cs:\svesszo, \TvesszoC) -- (axis cs:\svesszo, 0);
			
			%1 - 2 görbe
			\addplot[thick] table {./I8FMUH/12gorbe_ts.txt};
			
			%sraffozás
			\addplot+[mark=none,
			domain=05.47:7.29,
			samples=100,
			pattern=flexible hatch,
			draw = black,
			pattern color=black] table {./I8FMUH/12gorbe_ts.txt} \closedcycle;
		\end{axis}
	\end{tikzpicture}
	\caption{Az állapotváltozás T-s diagramja}
\end{figure}
%p-v diagram
\begin{figure}[h]
	\centering
	\begin{tikzpicture}
		\newcommand\kicsinyites{0.6};
		% Tengelyek
		\draw[->] ({0}, {0}) -- ({14*\kicsinyites}, {0}) node[anchor=base east, shift={(1,-0.5)}]{$lg(v)$ [\si{\meter\cubed\per\kilogram]}};
		\draw[->] ({0}, {0}) -- ({0}, {9*\kicsinyites}) node[anchor=north east]{$p$ [\si{\bar}]};
		
		\begin{axis}[
			%nyomás skála adatai MPa-ban vannak megadva
			axis lines = left,
			axis line style = {draw = none},
			xlabel={},
			ylabel={},
			ytick = {\pegy, \pvesszo, \pketto},
			yticklabels = {$p_1=1.76$,$p"=2.575$,$p_2=3.073$},
			xtick = {\vegy},
			xticklabels = {$v=0.699$},
			xmode = log,
			width=15cm*\kicsinyites, height=10cm*\kicsinyites, xmin=0, ymin=0
			]
			
			\addplot [thick] table {./I8FMUH/pv_jelleghelyes.txt};
			
			%nevezetes pontok
			\draw[fill] (axis cs:\vegy, \pegy) circle [radius = \pontvastagsag];
			\draw[fill] (axis cs:\vvesszo, \pvesszo) circle [radius = \pontvastagsag];
			\draw[fill] (axis cs:\vketto, \pketto) circle [radius = \pontvastagsag];
			
			%állapotváltozás iránya
			\draw[thick] (axis cs:\vegy, \pegy) -- (axis cs:0.699, 0.3073);
			
			%állapotot jelző vonalak
			\draw[dashed] (axis cs:\vegy, \pegy) -- (axis cs:\vegy, 0);
			\draw[dashed] (axis cs:\vegy, \pegy) -- (axis cs:0.3925, \pegy);
			\draw[dashed] (axis cs:\vegy, \pvesszo) -- (axis cs:0.3925, \pvesszo);
			\draw[dashed] (axis cs:\vegy, \pketto) -- (axis cs:0.3925, \pketto);
			
			\node[thick, circle, draw, inner sep = 1mm] at (axis cs:0.76, 0.176) {1};
			\node[thick, circle, draw, inner sep = 1mm] at (axis cs:0.76, 0.32) {2};
		\end{axis}
	\end{tikzpicture}
	\caption{Az állapotváltozás p-v diagramja}
\end{figure}
\begin{flushleft}
Az ábrák rajzolásához, illetve a különböző állapotok adataihoz a \textit{\url{https://www.spiraxsarco.com/resources-and-design-tools/steam-tables}} weboldal segítségét használtam fel, melynél a víz különböző állapotait kiválasztva dolgozhatunk előre definiált függvényekkel. Előre megadott adatainkat felhasználva nyerhetjük ki az állapotok értékeit, illetve függvények rajzolásához átváltva a "Table" jelölésre a köztes állapotok adatait is megkapjuk egy listában. Ennek a segítségével rajzoltam meg a $T-s$ és $p-v$ diagramot is, illetve a $T-s$ diagramnak az alapját a $github$ felületéről vettem le.
\end{flushleft}


\section*{HS 30. feladat: Ellenáramú hőcserélő}
\addcontentsline{toc}{section}{HS 30. feladat}
\begin{tabular}{ | p{2cm} | p{14cm} | } 
	\hline
	Név & Mayer Kíra \\ 
	\hline
	Szak &  Biomérnöki mérnöki alapszak\\
	\hline
	Félév & 2019/2020 II. (tavaszi) félév \\ 
	\hline
\end{tabular}
\vspace{0.5cm}
Egy hőcserélő alul $T_{1k} = \SI{15}{\celsius}$ -os levegő lép be és felül $T_{1v} = \SI{80}{\celsius}$ -on távozik. Ugyanekkor a hőcserélő másik oldalán $T_{vk} = \SI{88}{\celsius}$- os hőmérsékletű víz lép be és $T_{vv} = \SI{28}{\celsius}$-os hőmérséklettel hagyja el a berendezést. Rajzolja fel a hőmérséklet-hely függvényt és állapítsa meg a hőcsere közepes hőmérséklet különbségét logaritmikus középpel és számtani középpel!

\vspace{1mm}
\subsubsection*{Adatok:}
\vspace{1mm}
($T_{1k} = \SI{15}{\celsius}$ ;
$T_{1v} = \SI{80}{\celsius}$ ;
$T_{vk} = \SI{88}{\celsius}$ ; 
$T_{vv} = \SI{28}{\celsius}$ ; 
$\delta_T_{köz lg}= \SI{10,3}{\celsius}$;
$\delta_T_{köz}= \SI{10,5}{\celsius}
 )
\vspace{3mm}
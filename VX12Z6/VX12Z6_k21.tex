\section*{K2/1. feladat: Gőzfejlesztés állandó nyomáson}
\addcontentsline{toc}{section}{K2/1. feladat: Gőzfejlesztés állandó nyomáson}

\begin{tabular}{ | p{2cm} | p{14cm} | } 
	\hline
	Szerző & Fehér Árpád VX12Z6 \\ 
	\hline
	Szak & Vegyészmérnök alapszak \\ 
	\hline
	Félév & 2019/2020 II. (tavaszi) félév \\ 
	\hline
\end{tabular}
\vspace{0.5cm}

\noindent Számítsa ki a \textit{vízgőzfejlesztéshez szükséges hőenergiát} (külön-külön a folyadékhőt, párolgáshőt és túlhevítési hőt), illetve a \textit{hőközlés átlagos hőmérsékletét}. Ha \SI{9,81}{\bar} \SI{35}{\celsius}-os vízből állítunk elő \SI{400}{\celsius}-os túlhevített gőzt, \textbf{izobár} körülmények között. A vizsgálat 1 kg vízre vonatkozik. A feladatot gőztáblázat segítségével oldjuk meg.

\vspace{2mm}
\noindent Ábrázolja a folyamatot $T-s$ diagramban!

\subsubsection{Kezdeti és gőztáblázati adatok.}
\begin{equation*}
p = \SI{9,81}{\bar}, 
\quad 
T_a = \SI{35}{\celsius}, 
\quad
T_t = \SI{400}{\celsius},
\quad
s_1 = \SI{0,5035}{\kilo\joule\per\kilogram\kelvin},
\quad
s_4 = \SI{7,4705}{\kilo\joule\per\kilogram\kelvin}
\end{equation*}
\begin{equation*}
h_1 = \SI{147,4}{\kilo\joule\per\kilogram},
\quad
h_2 = \SI{759,07}{\kilo\joule\per\kilogram},
\quad
h_3 = \SI{2777,1}{\kilo\joule\per\kilogram},
\quad
h_4 = \SI{3263,6}{\kilo\joule\per\kilogram}
\end{equation*}

\noindent\hrulefill

\subsubsection*{a) Határozzuk meg vízgőzfejlesztéshez szűkséges energiát}

A vizsgált folyamat p = áll. nyomáson megy végbe ezért, az első főtétel megfelelő alakja alapján kapjuk, hogy:

\begin{equation}
\delta q = dh + \bcancel{\delta w_t}-\bcancel{vdp}
\end{equation}

\begin{equation}
\ q = h_2-h_1
\end{equation}

Ez alapján a folyadékhő:
\begin{equation}
\ q_{1,2} = h_2-h_1 = \SI{611,67}{\kilo\joule\per\kilogram}
\end{equation}

A párolgáshő:
\begin{equation}
\ q_{2,3} = h_3-h_2 = \SI{2018,03}{\kilo\joule\per\kilogram}
\end{equation}

A túlhevítési hő:
\begin{equation}
\ q_{3,4} = h_4-h_3 = \SI{486,5}{\kilo\joule\per\kilogram}
\end{equation}

Az összes hő
\begin{equation}
\ q_{1,4} = h_4-h_1 = \SI{3116,2}{\kilo\joule\per\kilogram}
\end{equation}

\subsubsection*{b) Határozzuk meg a hőközlés átlagos hőmérsékletét}

A $T_{\textit{köz}}$ átlagos hőmérséklet lényegében az állapotváltozás átlag hőmérséklete. Ezt egy folytonos mennyiség átlagolásával kapjuk meg. Integráljuk az adott mennyiséget és elosztjuk az intervallum nagyságával. 

\begin{equation*}
\ q_{1,4} = \int\displaylimits_{1}^{4} T_{\textit{köz}} ds = T_{\textit{köz}} \int\displaylimits_{1}^{4} ds = T_{\textit{köz}} (s_4-s_1)
\end{equation*}

Az egyenlet átrendezve $T_{\textit{köz}}$-re

\begin{equation}
\ T_{\textit{köz}} = \frac{q_{1,4}}{s_4-s_1} = \SI{447,28}{\kelvin} = \SI{174}{\celsius}
\end{equation}

\subsubsection*{c) Az állapotváltozás $T-s$ diagramban}

\begin{figure}[h]
	\centering
	\begin{tikzpicture}
	% Rács és vágómaszk
	\draw[step=1cm, gray, very thin] (-1.5, -1) grid (14.5, 11);
	%\clip (-1.5, -1) rectangle (14.5, 11);
	
	\pgfmathsetmacro{\RLEV}{220}
	\pgfmathsetmacro{\TA}{300}
	\pgfmathsetmacro{\TB}{600}
	
	% A tengelykeresztet az axis környezet hozza létre
	\begin{axis}[
	width=16cm, height=12cm,
	xmin=0, xmax=10.8,
	ymin=0, ymax=475, 
	axis lines = middle,
	axis line style={->},
	xlabel=$s \left(\si{\kilo\joule\per\kilogram\kelvin}\right)$, 
	xlabel style={
		at=(current axis.right of origin), 
		anchor=north east
	}, 
	ylabel=$T \left(\si{\degreeCelsius}\right)$, 
	ylabel style={
		at=(current axis.above origin), 
		anchor=north east
	},
	xtick={1, 2, 3, 4, 5, 6, 7, 8, 9},
	ytick={100, 200, 300, 400}
	]
	

		% A nedves gőzmező fázishatárai
		\addplot[thick] table {./guh7ud/ts.txt};

		% Az p = 9,81 bar-hoz tartozó izobár vonal
		\addplot[ultra thick, blue, mid arrow=blue] table {./VX12Z6/ts_p981.txt};
		
		%egyenesek
		\draw[dashed] (axis cs:0.5047, 35) -- (axis cs:0.5047, 0);
		\draw[dashed] (axis cs:2.129, 179) -- (axis cs:2.129, 0);
		\draw[dashed] (axis cs:6.591, 179) -- (axis cs:6.591, 0);
		\draw[dashed] (axis cs:7.47, 400) -- (axis cs:7.47, 0);
		
		%Sraffozás
		\addplot[opacity=0, name path=A, domain=0.5047:7.47, samples=25] table {./VX12Z6/ts_p981.txt};
		\addplot[opacity=0, name path=B, domain=0.5047:7.47, samples=2] {0};
		\addplot[gray!20] fill between [of=A and B];
				
		% Az állapotváltozás pontjai
		\node[anchor=south east] at (axis cs: 0.5047, 35) {\pgfcircled{$1$}};
		\filldraw[black, fill=white] (axis cs: 0.5047, 35) circle (1mm);
	
		\node[anchor=south east] at (axis cs: 2.129, 179) {\pgfcircled{$2$}};
		\filldraw[black, fill=white] (axis cs: 2.129, 179) circle (1mm);
		
		\node[anchor=south east] at (axis cs: 6.591, 179) {\pgfcircled{$3$}};
		\filldraw[black, fill=white] (axis cs: 6.591, 179) circle (1mm);
		
		\node[anchor=south east] at (axis cs: 7.47, 400) {\pgfcircled{$4$}};
		\filldraw[black, fill=white] (axis cs: 7.47, 400) circle (1mm);
		
		%Feliratok
		\node[fill=white, rounded corners] at (axis cs: 1.5, 40) {$q_{1,2}$};
		\node[fill=white, rounded corners] at (axis cs: 4.5, 100) {$q_{2,3}$};
		\node[fill=white, rounded corners] at (axis cs: 7.05, 50) {$q_{3,4}$};
		
	
	\end{axis}
	
	\end{tikzpicture}
	\caption{Víz-gőz $T-s$ diagram}
\end{figure}


\pagebreak

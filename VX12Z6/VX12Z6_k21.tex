\section*{K2/1. feladat: Gőzfejlesztés állandó nyomáson}
\addcontentsline{toc}{section}{K2/1. feladat: Gőzfejlesztés állandó nyomáson}

\begin{tabular}{ | p{2cm} | p{14cm} | } 
	\hline
	Szerző & Fehér Árpád VX12Z6 \\ 
	\hline
	Szak & Vegyészmérnök alapszak \\ 
	\hline
	Félév & 2019/2020 II. (tavaszi) félév \\ 
	\hline
\end{tabular}
\vspace{0.5cm}

\noindent Számítsa ki a \textit{vízgőzfejlesztéshez szükséges hőenergiát} (külön-külön a folyadékhőt, párolgáshőt és túlhevítési hőt), illetve a \textit{hőközlés átlagos hőmérsékletét}. Ha \SI{9,81}{\bar} \SI{35}{\celsius}-os vízből állítunk elő \SI{400}{\celsius}-os túlhevített gőzt, \textbf{izobár} körülmények között. A vizsgálat 1 kg vízre vonatkozik. A feladatot gőztáblázat segítségével oldjuk meg.

\vspace{2mm}
\noindent Ábrázolja a folyamatot $T-s$ diagramban!

\subsubsection{Kezdeti és gőztáblázati adatok.}
\begin{equation*}
p = \SI{9,81}{\bar}, 
\quad 
T_a = \SI{35}{\celsius}, 
\quad
T_t = \SI{400}{\celsius},
\quad
s_1 = \SI{0,5035}{\kilo\joule\per\kilogram\kelvin},
\quad
s_4 = \SI{7,4705}{\kilo\joule\per\kilogram\kelvin}
\end{equation*}
\begin{equation*}
i_1 = \SI{147,4}{\kilo\joule\per\kilogram},
\quad
i_2 = \SI{759,07}{\kilo\joule\per\kilogram},
\quad
i_3 = \SI{2777,1}{\kilo\joule\per\kilogram},
\quad
i_4 = \SI{3263,6}{\kilo\joule\per\kilogram}
\end{equation*}

\noindent\hrulefill

\subsubsection*{a) Határozzuk meg vízgőzfejlesztéshez szűkséges energiát}

A vizsgált folyamat p = áll. nyomáson megy végbe ezért, az első főtétel megfelelő alakja alapján kapjuk, hogy:

\begin{equation}
\delta q = di + \bcancel{\delta w_t}-\bcancel{vdp}
\end{equation}

\begin{equation}
\ q = i_2-i_1
\end{equation}

Ez alapján a folyadékhő:
\begin{equation}
\ q_{1-2} = i_2-i_1 = \SI{611,67}{\kilo\joule\per\kilogram}
\end{equation}

A párolgáshő:
\begin{equation}
\ q_{2-3} = i_3-i_2 = \SI{2018,03}{\kilo\joule\per\kilogram}
\end{equation}

A túlhevítési hő:
\begin{equation}
\ q_{3-4} = i_4-i_3 = \SI{486,5}{\kilo\joule\per\kilogram}
\end{equation}

Az összes hő
\begin{equation}
\ q_{1-4} = i_4-i_1 = \SI{3116,2}{\kilo\joule\per\kilogram}
\end{equation}

\subsubsection*{b) Határozzuk meg a hőközlés átlagos hőmérsékletét}

A $T_{\textit{köz}}$ átlagos hőmérséklet lényegében az állapotváltozás átlag hőmérséklete. Ezt egy folytonos mennyiség átlagolásával kapjuk meg. Integráljuk az adott mennyiséget és elosztjuk az intervallum nagyságával. 

\begin{equation*}
\ q_{1-4} = \int\displaylimits_{1}^{4} T_{\textit{köz}} ds = T_{\textit{köz}} \int\displaylimits_{1}^{4} ds = T_{\textit{köz}} (s_4-s_1)
\end{equation*}

Az egyenlet átrendezve $T_{\textit{köz}}$-re

\begin{equation}
\ T_{\textit{köz}} = \frac{q_{1-4}}{s_4-s_1} = \SI{447,28}{\kelvin} = \SI{174}{\celsius}
\end{equation}

\subsubsection*{c) Az állapotváltozás $T-s$ diagramban}

\pagebreak

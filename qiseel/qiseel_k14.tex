\section*{K1/4. feladat: Légsűrítő (kompresszor) teljesítménye} 
\addcontentsline{toc}{section}{K1/4. feladat}
\begin{tabular}{ | p{2cm} | p{14cm} | } 
	\hline
	Szerző & Biró Géza Konrád, 	QISEEL \\ 
	\hline
	Szak & anyagmérnök\\ 
	\hline
	Félév & 2019/2020 II. (tavaszi) félév \\ 
	\hline
\end{tabular}
\vspace{0.5cm}

\noindent  Egy légkompresszor $\dot{V} = \SI{580}{\meter\cubed\per\hour}$ levegőt $p_1=\SI{1}{\bar}$ nyomásról $p_2=\SI{8}{\bar}$ nyomásra sűrít. A levegő hőmérséklete $t_1=\SI{22}{\celsius}$, beszíváskor. Mennyi a kompresszor teljesítmény igénye a/izometrikus;b/adiabatikus  ($=\SI{1,4}{} $ ); c/politrópikus ($n = \SI{1,3}{} $) kompresszió esetén? 

\vspace{2mm}
\noindent Ábrázolja a folyamatokat $p$-$v$ és $T$-$s$ diagramokban!



\subsubsection{Teljesítmény számolása}
Technikai munka az a munkamennyiség amely egy nyitott rendszerben adott folyamatsor ismételt, ciklikus végrehajtására alkalmas technikai gép működéséhez szükséges.
Technikai munka számítása:
\begin{equation*}
	W_t=mw_t
\end{equation*}
\begin{equation*}
P\approx \dod{W_t}{t}=\underbrace{\dod{m}{t} }_{\dot{m}} W_t +\underbrace{ m\dod{w_t}{t}}_{\SI{0}{\watt\per\kilogram}}=\dot{m} w_t\
\end{equation*}




\noindent\hrulefill
%a) izotermikus T_1=a\'llando\'
\subsubsection{a) izotermikus $T_1$=állandó}
Izotermikus állapotváltozás: ha a hőmérsékletet állandó értéken tartjuk, ekkor $T=állandó$.
\begin{equation}
\dot{m}=\rho \dot{V}=\frac{\dot{V_1}}{v_1}=\SI{0,19}{\kilogram\per\second}
\end{equation}
Fajtérfogat az állapotegyenletből kifejezve:
\begin{equation}
v_1=\frac{R_L T_1}{p_1}=\SI{0,847}{\frac{m^3}{kg}}
\end{equation}
W értéke negatív mert a rendszer végez munkát a környezeten és ezért számolunk $ln\frac{p1}{p2}$ -vel. Számolhatunk $ln\frac{p2}{p1}$-gyel viszont akkor az eredmény mínusz egyszeresét kell venni.
\begin{equation}
w_{t1,2}=R_LT_1ln\frac{p_1}{p_2}=\SI{-176,17}{\kilo\joule\per\kilogram}
\end{equation}
\begin{equation}
P=\dot{m}w_t=\SI{-33,5}{kN}
\end{equation}

\subsubsection{b) adiabatikus q=0}
Adiabatikus állapotváltozás : ha az entrópia állandó.Ilyenkor sem hőkölés sem hőelvonás nem történik.$q=0$ tehát belsőenergia rovására történik munkavégzés.
\begin{equation}
T_{2ad}=T_1\left(\frac{p_2}{p_1}\right)^\frac{\kappa-1}{\kappa}=\SI{534,6}{\kelvin}
\end{equation}
\begin{equation}
w_{t1,2ad}=\kappa\frac{p_1v_1}{\kappa-1}\left(1-\frac{T_{2ad}}{T_1}\right)=\SI{-240,6}{\kilo\joule\per\kilogram}
\end{equation}
\begin{equation}
P_{ad}=\SI{-45,75}{\kilogram\watt}
\end{equation}

\subsubsection{c) politrópikus n=1,3}
Politrópikus állapotváltozás: a legtöbb valós folyamatban a nyomás, hőmérséklet, térfogat bevitt vagy elvont hőenergia is változik.ahol $"n"$politrópikus tényező.A b és c pontokban a számolás azonos módon történik csak a kitevő értékében van különbség.
A $\kappa$ értéke két atomos gáz esetén $\SI{1,4}{}$ egy atomos gáz esetén $\SI{1,3}{}$.
\begin{equation}
w_{t1,2p}=n\frac{p_1v_1}{n-1}\left(1-\frac{T_{2p}}{T_1}\right)=\SI{-226}{\kilo\joule\per\kilogram}
\end{equation}
\begin{equation}
T_{2p}=T_1=T_1\left(\frac{p_2}{p_1}\right)^\frac{n-1}{n}=\SI{-226}{\kilo\joule\per\kilogram}
\end{equation}
\begin{equation}
P_{pol}=\SI{-42,99}{\kilo\watt}
\end{equation}


\begin{figure}[h]
	\centering
	\begin{subfigure}[b]{0.4\textwidth} 
		\begin{tikzpicture}
		%	\draw[->](0,0) -- (6,0)  node[anchor=north] {$v$};
		%	\draw[->](0,0) -- (0,6) node[anchor=east] {$p$};
			\pgfmathsetmacro{\RLEV}{220}
		\pgfmathsetmacro{\TA}{300}
		\pgfmathsetmacro{\TB}{600}
		
			\begin{axis}[
		width=8cm, height=8cm,
		xmin=0, xmax=1.25, ymin=0, ymax=10, 
		axis lines = middle, axis line style={->},
		xlabel=$v \left(\si{\meter\cubed\per\kilogram}\right)$, 
		xlabel style={
			at={(0.0,10.0)}
			%at=(current axis.right of origin), 
			anchor=north east
		}, 
		ylabel=$p \left(\si{\bar}\right)$, 
		ylabel style={
			at=(current axis.above origin), 
			anchor=north east
		},
		xtick={0.25, 0.5, 0.75, 1},
		ytick={1, 2, 3, 4, 5, 6, 7, 8, 9},
		axis on top=true,
		no markers,
		/pgf/number format/.cd, use comma, 1000 sep={}
		]
		
		\addplot[red, thick, domain=0.075:1.1, samples=100] {\RLEV*\TA/x/100000} node[anchor=south west] {$T_1$};
			% p2
		\addplot[black, thick, domain=0.1:0, samples=2]
		{8} node[anchor=west] {$p_2$};
		\addplot[black, thick, domain=1.1:0, samples=2]
		{1} node[anchor=west] {$p_1$};		
		
		\addplot+[ycomb, black, dashed] plot coordinates
			{(0.25, 8)};
		
		\end{axis}
		
		\end{tikzpicture}
		\caption{p-v diagram}
	\end{subfigure}
	\begin{subfigure}[b]{0.4\textwidth} 
		\begin{tikzpicture}
	\draw[->](0,0) -- (6,0)  node[anchor=north] {$s \si{\joule\per\kelvin} $};
	\draw[->](0,0) -- (0,6) node[anchor=east] {$T\mbox(K) $};
		\end{tikzpicture}
		\caption{T-s diaggram}
	\end{subfigure}

\end{figure}


\begin{figure}[h]
	\centering
	\begin{tikzpicture}
	% Rács és vágómaszk
	%\draw[step=1cm, gray!25, very thin] (-1.5, -1) grid (14.5, 11);
	%\clip (-1.5, -1) rectangle (14.5, 11);
	
	\pgfmathsetmacro{\RLEV}{220}
	\pgfmathsetmacro{\TA}{300}
	\pgfmathsetmacro{\TB}{600}
	
	% A tengelykeresztet az axis környezet hozza létre
	\begin{axis}[
		width=16cm, height=12cm,
		xmin=0, xmax=1.25, ymin=0, ymax=10, 
		axis lines = middle, axis line style={->},
		xlabel=$v \left(\si{\meter\cubed\per\kilogram}\right)$, 
		xlabel style={
			at=(current axis.right of origin), 
			anchor=north east
		}, 
		ylabel=$p \left(\si{\bar}\right)$, 
		ylabel style={
			at=(current axis.above origin), 
			anchor=north east
		},
		xtick={0.25, 0.5, 0.75, 1},
		ytick={1, 2, 3, 4, 5, 6, 7, 8, 9},
		axis on top=true,
		no markers,
		/pgf/number format/.cd, use comma, 1000 sep={}
		]
	
	% Izotermák, de a nyomás barban van. Az x a fajtérfogat.
	\addplot[red, thick, domain=0.075:1.1, samples=100] {\RLEV*\TA/x/100000} node[anchor=north west] {$T_A$};
	\addplot[blue, thick, domain=0.15:1.1, samples=100] {\RLEV*\TB/x/100000} node[anchor=north west] {$T_B$};
	
	% A T_A megint, de csak az állapotváltozás szakasza
	\addplot[red, ultra thick, domain=0.25:0.5, samples=25, mid arrow=red] {\RLEV*\TA/x/100000};
	
	% T_B szakasz
	\addplot[blue, ultra thick, domain=0.5:0.25, samples=25, mid arrow=blue] {\RLEV*\TB/x/100000};
	
	% Átlátszó vonalak a kitöltésekhez (a kitöltés és a mid arrow összeakadna)
	\addplot[opacity=0, name path=A0, domain=0.25:0.5, samples=2] {0};
	\addplot[opacity=0, name path=TA, domain=0.25:0.5, samples=25] {\RLEV*\TA/x/100000};
	\addplot[opacity=0, name path=TB, domain=0.5:0.25, samples=25] {\RLEV*\TB/x/100000};
	
	% Kitöltések
	\addplot[red!25] fill between [of=A0 and TA];
	\addplot[blue!25] fill between [of=TA and TB];
	
	% A függőleges tengely felé történő kitöltéshez kell két segédgörbe
	\addplot[opacity=0, name path=TBPF, domain=0:0.5, samples=25] {
		min(\RLEV*\TB/x/100000, \RLEV*\TB/0.25/100000)
	};
	\addplot[opacity=0, name path=TBPA, domain=0:0.5, samples=2] {\RLEV*\TB/0.5/100000};
	\addplot[pattern=north east lines] fill between [of=TBPA and TBPF];
	
	% Pontok
	\node[anchor=south west] at (axis cs: 0.5, {\RLEV*\TB/0.5/100000}) {\pgfcircled{$1$}};
	\filldraw[blue, fill=white] (axis cs: 0.5, {\RLEV*\TB/0.5/100000}) circle (1mm);
	
	\node[anchor=south west] at (axis cs: 0.25, {\RLEV*\TB/0.25/100000}) {\pgfcircled{$2$}};
	\filldraw[blue, fill=white] (axis cs: 0.25, {\RLEV*\TB/0.25/100000}) circle (1mm);
	
	% Felirat a technikai munkának
	\node[fill=white, rounded corners] at (axis cs: 0.1, 4) {$w_{t1,2}$};
	
	\end{axis}
	
	\end{tikzpicture}
	\caption{Ideális gáz $p-v$ diagram}
\end{figure}
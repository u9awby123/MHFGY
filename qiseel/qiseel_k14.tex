\section*{K1/4. feladat: Légsűrítő (kompresszor) teljesítménye} 
\addcontentsline{toc}{section}{K1/4. feladat}
\begin{tabular}{ | p{2cm} | p{14cm} | } 
	\hline
	Szerző & Biró Géza Konrád, 	QISEEL \\ 
	\hline
	Szak & anyagmérnök\\ 
	\hline
	Félév & 2019/2020 II. (tavaszi) félév \\ 
	\hline
\end{tabular}
\vspace{0.5cm}

\noindent  Egy légkompresszor $\dot{V} = \SI{580}{\meter\cubed\per\hour}$ levegőt $p_1=\SI{1}{\bar}$ nyomásról $p_2=\SI{8}{\bar}$ nyomásra sűrít. A levegő hőmérséklete $T_1=\SI{22}{\celsius}$, beszíváskor. Mekkora a kompresszor teljesítmény igénye a/izometrikus, b/adiabatikus  ($=\SI{1,4}{}$), c/politrópikus ($n = \SI{1,3}{} $) kompresszió esetén? 

\vspace{2mm}
\noindent Ábrázolja a folyamatokat $p$-$v$ és $T$-$s$ diagramokban!



\subsubsection{Teljesítmény számítása}
Technikai munka az a munkamennyiség amely egy nyitott rendszerben adott folyamatsor ismételt, ciklikus végrehajtására alkalmas technikai gép működéséhez szükséges.
Teljesítmény számítása:
\begin{equation*}
	W_t=mw_t
\end{equation*}
\begin{equation}
P\approx \dod{W_t}{t}=\underbrace{\dod{m}{t} }_{\dot{m}} W_t + m \overbrace{ \dod{w_t}{t}}^{\SI{0}{\watt\per\kilogram}}=\dot{m} w_t
\label{eq-teljesitmeny}
\end{equation}
A \ref{eq-teljesitmeny} egyenletben a $\dod{w_t}{t}$ tag értéke 0, mert időben állandó.



\noindent\hrulefill
%a) izotermikus T_1=a\'llando\'
\subsubsection{a) izotermikus $T_1$=állandó}
Izotermikus állapotváltozás: ha a hőmérsékletet állandó értéken tartjuk, ekkor $T=\textrm{állandó}$.
\begin{equation}
\dot{m}=\varrho \dot{V}=\frac{\dot{V_1}}{v_1}=\SI{0,19}{\kilogram\per\second}
\end{equation}
Fajtérfogat az állapotegyenletből kifejezve:
\begin{equation}
v_1=\frac{R_L T_1}{p_1}=\SI{0,847}{\frac{m^3}{kg}}
\end{equation}
W értéke negatív mert a rendszer végez munkát a környezeten és ezért számolunk $ln\dfrac{p_1}{p_2}$ -vel. Számolhatunk $ln\dfrac{p_2}{p_1}$-gyel viszont akkor az eredmény mínusz egyszeresét kell venni.
\begin{equation}
w_{t1,2}=R_LT_1ln\frac{p_1}{p_2}=\SI{-176,17}{\kilo\joule\per\kilogram}
\end{equation}
\begin{equation}
P=\dot{m}w_t=\SI{-33,5}{kN}
\end{equation}

\subsubsection{b) adiabatikus q=0}
Adiabatikus állapotváltozás : ha az entrópia állandó.Ilyenkor sem hőközlés sem hőelvonás nem történik.$q=0$ tehát belsőenergia rovására történik munkavégzés.
\begin{equation}
T_{2ad}=T_1\left(\frac{p_2}{p_1}\right)^\frac{\kappa-1}{\kappa}=\SI{534,6}{\kelvin}
\end{equation}
\begin{equation}
w_{t1,2ad}=\kappa\frac{p_1v_1}{\kappa-1}\left(1-\frac{T_{2ad}}{T_1}\right)=\SI{-240,6}{\kilo\joule\per\kilogram}
\end{equation}
\begin{equation}
P_{ad}=\SI{-45,75}{\kilo\watt}
\end{equation}

\subsubsection{c) politrópikus n=1,3}
Politrópikus állapotváltozás: a legtöbb valós folyamatban a nyomás, hőmérséklet, térfogat bevitt vagy elvont hőenergia is változik, ahol $"n"$politrópikus tényező. A b és c pontokban a számolás azonos módon történik csak a kitevő értékében van különbség.
A $\kappa$ értéke két atomos gáz esetén $\SI{1,4}{}$ egy atomos gáz esetén $\SI{1,3}{}$.
\begin{equation}
w_{t1,2p}=n\frac{p_1v_1}{n-1}\left(1-\frac{T_{2p}}{T_1}\right)=\SI{-226}{\kilo\joule\per\kilogram}
\end{equation}
\begin{equation}
T_{2p}=T_1=T_1\left(\frac{p_2}{p_1}\right)^\frac{n-1}{n}=\SI{-240,6}{\kilo\joule\per\kilogram}
\end{equation}
\begin{equation}
P_{pol}=\SI{-42,99}{\kilo\watt}
\end{equation}


\begin{figure}[h]
	\begin{subfigure}[b]{0.45\linewidth} 
		\begin{tikzpicture}
		\pgfmathsetmacro{\RLEV}{220}
		\pgfmathsetmacro{\TA}{300}
		\pgfmathsetmacro{\TB}{600}
		
		\begin{axis}[
		width=8cm, height=8cm,
		xmin=0, xmax=1.25, ymin=0, ymax=10, 
		axis lines = middle, axis line style={->},
		xlabel=$v \left(\si{\meter\cubed\per\kilogram}\right)$, 
		x label style={at={(axis description cs:1,-0.02)},anchor=north},
		ylabel=$p \left(\si{\bar}\right)$, 
		ylabel style={
			at=(current axis.above origin), 
			anchor=north east
		},
		xtick={0.25, 0.5, 0.75, 1},
		ytick={1, 2, 3, 4, 5, 6, 7, 8, 9},
		axis on top=true,
		no markers,
		/pgf/number format/.cd, use comma, 1000 sep={}
		]
		
		\addplot[red, thick, domain=0.075:1.1, samples=100] {\RLEV*\TA/x/100000} node[anchor=south west] {$T_1$};
		\addplot[black, thick, domain=0.1:0, samples=2]	{8}; 
		\addplot[black, thick, domain=1.1:0, samples=2]
		{1};

		
		\addplot+[ycomb, black, dashed] plot coordinates
		{(0.25, 8)};
		
		\end{axis}
		\node[anchor=north] at (-0.7, 0.95) {$p_1$};
		\node[anchor=north] at (-0.7, 5.35) {$p_2$};
		\end{tikzpicture}
		\caption{p-v diagram}
	\end{subfigure}
	\hspace{2cm}
	\begin{subfigure}[b]{0.45\linewidth} 
		\begin{tikzpicture}
		\pgfmathsetmacro{\OFS}{1.5}
		\draw[->](0,0) -- (6,0)  node[anchor=north] {$s \si{\joule\per\kelvin} $};
		\draw[->](0,0) -- (0,6) node[anchor=east] {$T\mbox(K) $};
		\draw[white](0,-\OFS) -- (1,-\OFS);
		\end{tikzpicture}
		\caption{T-s diaggram}
	\end{subfigure}
	
\end{figure}
\pagebreak



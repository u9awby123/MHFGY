\section*{K1/4. feladat: Légsűrítő (kompresszor) teljesítménye} 
\addcontentsline{toc}{section}{K1/4. feladat}
\begin{tabular}{ | p{2cm} | p{14cm} | } 
	\hline
	Szerző & Biró Géza Konrád, 	QISEEL \\ 
	\hline
	Szak & anyagmérnök\\ 
	\hline
	Félév & 2019/2020 II. (tavaszi) félév \\ 
	\hline
\end{tabular}
\vspace{0.5cm}

\noindent  Egy légkompresszor $\dot{V} = \SI{580}{\meter\cubed\per\hour}$ levegőt $p_1=\SI{1}{\bar}$ nyomásról $p_2=\SI{8}{\bar}$ nyomásra sűrít. A levegő hőmérséklete $T_1=\SI{22}{\celsius}$ beszíváskor. Mekkora a kompresszor teljesítmény igénye a/izometrikus, b/adiabatikus  ($=\SI{1,4}{}$), c/politrópikus ($n = \SI{1,3}{} $) kompresszió esetén? 

\vspace{2mm}
\noindent Ábrázolja a folyamatokat $p$-$v$ és $T$-$s$ diagramokban!



\subsubsection{Teljesítmény számítása}
Technikai munka az a munkamennyiség, amely egy nyitott rendszerben adott folyamatsor ismételt, ciklikus végrehajtására alkalmas technikai gép működéséhez szükséges.
Teljesítmény számítása:
\begin{equation*}
	W_t=mw_t
\end{equation*}
\begin{equation}
P\approx \dod{W_t}{t}=\underbrace{\dod{m}{t} }_{\dot{m}} W_t + m \overbrace{ \dod{w_t}{t}}^{\SI{0}{\watt\per\kilogram}}=\dot{m} w_t
\label{eq-teljesitmeny}
\end{equation}
A \ref{eq-teljesitmeny} egyenletben a $\dod{w_t}{t}$ tag értéke 0, mert időben állandó.



\noindent\hrulefill
%a) izotermikus T_1=a\'llando\'
\subsubsection{a) izotermikus $T_1$=állandó}
Izotermikus állapotváltozás: ha a hőmérsékletet állandó értéken tartjuk, ekkor $T=\textrm{állandó}$.
\begin{equation}
\dot{m}=\varrho \dot{V}=\frac{\dot{V_1}}{v_1}=\SI{0,19}{\kilogram\per\second}
\end{equation}
Fajtérfogat az állapotegyenletből kifejezve:
\begin{equation}
v_1=\frac{R_L T_1}{p_1}=\SI{0,847}{\frac{m^3}{kg}}
\end{equation}
\noindent $W$ értéke negatív, mert a rendszer végez munkát a környezeten és ezért számolunk $ln\dfrac{p_1}{p_2}$ -vel. Számolhatunk $ln\dfrac{p_2}{p_1}$-gyel viszont akkor az eredmény mínusz egyszeresét kell venni. $R_L$ a levegőre vonatkoztatott specifikus gázállandó, értéke $R_L=\SI{287}{\joule\per\kilogram\kelvin}$.
\begin{equation}
w_{t1,2}=R_LT_1ln\frac{p_1}{p_2}=\SI{-176,17}{\kilo\joule\per\kilogram}
\end{equation}
\begin{equation}
P=\dot{m}w_t=\SI{-33,5}{kN}
\end{equation}

\subsubsection{b) adiabatikus q=0}
Adiabatikus állapotváltozás: ha az entrópia állandó. Ilyenkor sem hőközlés sem hőelvonás nem történik, így $q=0$ tehát belsőenergia rovására történik munkavégzés.
\begin{equation}
T_{2ad}=T_1\left(\frac{p_2}{p_1}\right)^\frac{\kappa-1}{\kappa}=\SI{534,6}{\kelvin}
\end{equation}
\begin{equation}
w_{t1,2ad}=\kappa\frac{p_1v_1}{\kappa-1}\left(1-\frac{T_{2ad}}{T_1}\right)=\SI{-240,6}{\kilo\joule\per\kilogram}
\end{equation}
\begin{equation}
P_{ad}=\SI{-45,75}{\kilo\watt}
\end{equation}

\subsubsection{c) politrópikus n=1,3}
Politrópikus állapotváltozás: a legtöbb valós folyamatban a nyomás, hőmérséklet, térfogat bevitt vagy elvont hőenergia is változik, ahol $"n"$politrópikus tényező. A b és c pontokban a számolás azonos módon történik csak a kitevő értékében van különbség.
A $\kappa$ értéke két atomos gáz esetén $\SI{1,4}{}$ egy atomos gáz esetén $\SI{1,3}{}$.
\begin{equation}
w_{t1,2p}=n\frac{p_1v_1}{n-1}\left(1-\frac{T_{2p}}{T_1}\right)=\SI{-226}{\kilo\joule\per\kilogram}
\end{equation}
\begin{equation}
T_{2p}=T_1=T_1\left(\frac{p_2}{p_1}\right)^\frac{n-1}{n}=\SI{-240,6}{\kilo\joule\per\kilogram}
\end{equation}
\begin{equation}
P_{pol}=\SI{-42,99}{\kilo\watt}
\end{equation}


\begin{figure}[h]
	\begin{subfigure}[b]{0.45\linewidth} 
		\begin{tikzpicture}
			\pgfmathsetmacro{\RLEV}{287.041}
			\pgfmathsetmacro{\TA}{295}

			\begin{axis}[
				width=8cm, height=8cm,
				xmin=0, xmax=1.25, ymin=0, ymax=10, 
				axis lines = middle, axis line style={->},
				xlabel=$v \left(\si{\meter\cubed\per\kilogram}\right)$, 
				x label style={at={(axis description cs:1,-0.02)},anchor=north},
				ylabel=$p \left(\si{\bar}\right)$, 
				ylabel style={
					at=(current axis.above origin), 
					anchor=north east
				},
				xtick={0.25, 0.5, 0.75, 1},
				ytick={1, 2, 3, 4, 5, 6, 7, 8, 9},
					axis on top=true,
					no markers,
					/pgf/number format/.cd, use comma, 1000 sep={}
				]

				%T1 izobár
				\addplot[ultra thin, domain=0.075:1.1, samples=100] {\RLEV*\TA/x/100000}
				node[anchor=south west] {$T_1$}; node[anchor=south west] {$ $};
				\addplot[ultra thick, blue, <-, domain=0.1058463688:0.847, samples=100] {\RLEV*\TA/x/100000}
				node[anchor=south west] {$ $}; node[anchor=south west] {$ $};

				%1-es pont
				\node[anchor=north east] at (axis cs: 0.847, 1) {\pgfcircled{$1$}};
				\filldraw[black, fill=white] (axis cs: 0.847, 1) circle (1mm);

				%függőleges vonalak
				\draw[very thin] (axis cs: 0.847, 1) -- (axis cs:  0.847, 0);

				%Vízszintesek
				\draw[very thin] (axis cs: 0, 8) -- (axis cs:  0.191786986, 8);
				\draw[very thin] (axis cs: 0, 1) -- (axis cs:  0.847, 1);

				%jelölések
				\node at (axis cs: 0.25, 2.5) {\pgfcircled{$a$}};
				\node at (axis cs: 0.4, 4) {\pgfcircled{$b$}};
				\node at (axis cs: 0.17, 8.25) {\pgfcircled{$c$}};

				%adiabata
				\addplot[ultra thick,->, red] table {./qiseel/pv adiabata.txt};

				%pollitróp
				\addplot[ultra thick,->,dashed] table {./qiseel/pv pollitrop.txt};

			\end{axis}
			\node[anchor=north] at (-0.7, 0.95) {$p_1$};
			\node[anchor=north] at (-0.7, 5.35) {$p_2$};
		\end{tikzpicture}		
		\caption{p-v diagram}
	\end{subfigure}
	\hspace{2cm}
	\begin{subfigure}[b]{0.45\linewidth} 
		\begin{tikzpicture}
			\begin{axis}[
				width=8cm, height=8cm,
				xmin=0, xmax=50, ymin=250, ymax=1150, 
				axis lines = middle, axis line style={->},
				xlabel=$s \left(\si{\kilo\joule\per\kilogram\kelvin}\right)$, 
				x label style={at={(axis description cs:1,-0.02)},anchor=north},
				ylabel=$T \left(\si{\kelvin}\right)$, 
				ylabel style={
					at=(current axis.above origin), 
					anchor=north east
				},
				xtick={\empty},
				extra x ticks = {24.366},
				extra x tick labels = {$s_1$},
				extra y ticks = {534.6},
				extra y tick labels = {$T_1$},
				ytick={\empty},
				axis on top=true,
				no markers,
				/pgf/number format/.cd, use comma, 1000 sep={}
			]
	
			%állapotváltozási görbék
			\addplot [thick, blue, mid arrow = blue] table {./qiseel/adiabata_ts.txt};
			\addplot [thick, black, mid arrow = black] table {./qiseel/izobar1_ts.txt} node[anchor=north west] {$p_1$};
			\addplot [thick, black, mid arrow = black] table {./qiseel/izobar2_ts.txt} node[anchor=north west] {$p_2$};
			\addplot [thick, red, mid arrow = red] table {./qiseel/izoterm_ts.txt};
	
			%szaggatott vonalak
			\draw[dashed] (axis cs: 24.366, 534.6) -- (axis cs: 24.366, 0);
			\draw[dashed] (axis cs: 0, 534.6) -- (axis cs: 12, 534.6);
	
			%jelölések
			\node[circle, draw, inner sep = 0.4mm, red] at (axis cs: 15, 485) {$a$};
			\node[circle, draw, inner sep = 0.4mm, blue] at (axis cs: 27.3, 800) {$b$};
			\node[circle, draw, inner sep = 0.4mm] at (axis cs: 28, 500) {$1$};
			\node[circle, draw, inner sep = 0.4mm] at (axis cs: 10, 580) {$2$};
		\end{axis}
			\draw[white](0,-1.55) -- (1,-1.55);		
	\end{tikzpicture}
	
		\caption{T-s diaggram}
	\end{subfigure}
	
\end{figure}
\pagebreak



\section*{MH9: Vízgőz állapotjelzői}
\addcontentsline{toc}{section}{MH9}
\begin{tabular}{ | p{2cm} | p{14cm} | } 
	\hline
	Név & Szauer János, FLTP3G \\ 
	\hline
	Szak & Mechatronika mérnöki alapszak \\ 
	\hline
	Félév & 2019/2020 II. (tavaszi) félév \\ 
	\hline
\end{tabular}
\vspace{2mm}

\noindent
Határozza meg a $p = \SI{60}{\bar}$ nyomású és $x = \SI{0,95}{}$ fajlagos gőztartalmú vízgőz hőmérsékletét, fajtérfogatát, entrópiáját és belső energiáját, vízgőztáblázat illetve $i-s$, vagy $T-s$ diagram segítségével!

\subsubsection{Vízgőztáblázat által kikeresett adatok:}
\begin{equation*}
	T_s(\SI{60}{bar}) = \SI{275,56}{\celsius},
	\quad
	v' = \SI{0.00132}{\meter\cubed\per\kilogram},
	\quad
	v'' = \SI{0.0324}{\meter\cubed\per\kilogram}
	\quad
	s' = \SI{3.0277}{\kilo\joule\per\kilogram}, 
\end{equation*}
	
\begin{equation*}
	s'' = \SI{5.8878}{\kilo\joule\per\kilogram},
	\quad
	h' = \SI{1213.9}{\kilo\joule\per\kilogram}, 
	\quad
	h'' = \SI{2785}{\kilo\joule\per\kilogram},
\end{equation*}

\noindent\hrulefill

\subsubsection{Az állapotjelzők számítása}
A  $h_x$ hőtartalom, $v_x$ fajtérfogat és $s_x$ entrópia a szélsőértékek és az $x$ fajlagos gőztartalom felhasználásával számolható:

\begin{equation*}
	h_x = \left(1 - x\right) h' + x h'' = \SI{2706.16}{\kilo\joule\per\kilogram}
\end{equation*}
\begin{equation*}
	v_x = \left(1 - x\right) v' + x v'' = \SI{0.03087}{\meter\cubed\per\kilogram}
\end{equation*}
\begin{equation*}
	s_x = \left(1 - x\right) s' + x s'' = \SI{5.745}{\kilo\joule\per\kilogram\kelvin}
\end{equation*}
\begin{equation*}
	 u_x = h - p v_x = \SI{2520.94}{\kilo\joule\per\kilogram}
\end{equation*}


% Oldaltörés
\pagebreak

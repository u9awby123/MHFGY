

\section*{MH31. feladat: Fojtószelepes ammónia hűtőgép}
\addcontentsline{toc}{section}{MH31. feladat: Fojtószelepes ammónia hűtőgép}


\begin{tabular}{ | p{2cm} | p{14cm} | } 
	\hline
	Név & Steiber Árpád \\ 
	\hline
	Szak & Vegyészmérnöki \\ 
	\hline
	Félév & 2019/2020 II. (tavaszi) félév \\ 
	\hline
\end{tabular}
\vspace{0.5cm}

\noindent Egy kompresszoros ammóniás hűtőgép - fojtószeleppel - hűtőteljesítménye \SI{-15}{\celsius} elpárolgási és \SI{25}{\celsius} kondenzációs hőmérséklet mellett  \SI{10}{\kilo\watt}. Határozza meg a sűrítés véghőmérsékletét adiabatikus, reverzibilis esetben, ha a kompresszor száraz telített gőzt szív be. Mennyi a körfolyamatban keringetett ammónia mennyisége, a körfolyamat fenntartásához szükséges villamos teljesítmény, valamint a hűtővíz mennyisége, ha annak felmelegedése 
\begin{equation}
\Delta T = \SI{10}{\celsius}
\end{equation} 
\vspace{0.5cm}
\noindent A rendelkezésre álló gőztáblázati adatok
\vspace{0.2cm}
\begin{center}
\begin{tabular}{p{5cm} p{5cm} p{5cm}}
	$T = \SI{-15}{\celsius}$ &
	$h''=\SI{1587.53}{\kilo\joule\per\kilo\gram}$ & 
	$s''= \SI{6.301}{\kilo\joule\per\kilo\gram}$ \\ 
	$T = \SI{25}{\celsius}$ &
	$h'=\SI{460.82}{\kilo\joule\per\kilo\gram}$ 	
\end{tabular}
\end{center}
\noindent A sűrítés végén a tömegfajlagos hőtartalom $h_2=\SI{1793.0}{\kilo\joule\per\kilo\gram}$ ($p_h$ diagramból). 
A víz fajhője
\begin{equation}
c_V=\SI{4.18}{\kilo\joule\per\kilo\gram\kelvin}
\end{equation}	
Sűrítés utáni nyomás($p_F$) és a sűrítés előtti nyomás($p_A$)
\begin{equation}
P_F=\SI{1}{\mega\pascal} \hspace{1 cm}
P_A=\SI{0.236}{\mega\pascal}
\end{equation}

\vspace{2mm}

Rajzolja le a folyamatot $T-s$ diagramban, jelölje be az elvondandó hőt és a kívülről bevezetett munkát! \vspace{2mm}

A hűtőtérből elvont hő:
\begin{equation}
q_h=h_1-h_4=h_1-h_3=\SI{1587.53}{\kilo\joule\per\kilo\gram}-\SI{460.821}{\kilo\joule\per\kilo\gram} = \SI{1126.709}{\kilo\joule\per\kilo\gram}
\end{equation}
Az ammónia tömegárama:
\begin{equation}
\dot{m}_A=\frac{Q_H}{q_H}=\frac{\SI{100}{\kilo\joule\per\second}}{\SI{1126.709}{\kilo\joule\per\kilo\gram}}=\SI{0.08875}{\kilo\gram\per\second}=\SI{319.51}{\kilo\gram\per\hour}
\end{equation}
A kompresszor technikai munkája:
\begin{equation}
w_k=h_2-h_1=\SI{1793.0}{\kilo\joule\per\kilo\gram}-\SI{1587.53}{\kilo\joule\per\kilo\gram}=\SI{205.47}{\kilo\joule\per\kilo\gram}
\end{equation}
A folyamat fenntartásához szükséges villamos teljesítmény:
\begin{equation}
P=w\dot{m}_A=\SI{205.47}{\kilo\joule\per\kilo\gram}\SI{0.08875}{\kilo\gram\per\second}=\SI{18.24}{\watt}
\end{equation}
A környezet felé leadott hőmennyiség:
\begin{equation}
q_k=h_2-h_3=h_2-h_4=\SI{1793.0}{\kilo\joule\per\kilo\gram}-\SI{460.82}{\kilo\joule\per\kilo\gram}=\SI{1332.18}{\kilo\joule\per\kilo\gram}
\end{equation}
A hűtővíz mennyisége
\begin{equation}
\dot{m}_v=\frac{Q_H}{c_V\Delta T} = \frac{\SI{100}{\kilo\joule\per\second}}{\SI{4.18}{\kilo\joule\per\kilo\gram\kelvin}\SI{10}{\kelvin}}=\SI{2.39}{\kilo\gram\per\second}
\end{equation}
\begin{figure}[h]
	\centering
	\label{TVXE7Y_figure:Ts}
	\begin{tikzpicture}

	
	% A tengelykeresztet az axis környezet hozza létre
	\begin{axis}[
	width=17cm, height=14cm,
	xmin=0, xmax=9,
	ymin=-77.150, ymax=160, 
	axis lines = left ,
	axis line style={->},
	]
	
	
	%  Fázishatár
	\addplot[thick] table {./TVXE7Y/nh3_ts.txt};
	\addplot[dashed] table {./TVXE7Y/P1_2.36.txt};
	\addplot[dashed] table {./TVXE7Y/P2_10.txt};
	\addplot[dashed] table {./TVXE7Y/h2izent.txt};
	\addplot[dashed] table {./TVXE7Y/h3izent.txt};
	

	
	
	\end{axis}
	
	%felső
	\draw[red, ultra thick, mid arrow=red] (9.9, 5.3425) -- (3.25, 5.3425);
	
	\node[anchor=mid] at (9.3, 5.8) {$2.$};
	\draw(9.3, 5.8) circle(0.3);
	
	\node[anchor=mid] at (2.7, 5.3425) {$3.$};
	\draw(2.7, 5.3425) circle(0.3);
	
	%bal
	\draw[red, ultra thick, mid arrow=red] (3.25, 5.3425) -- (3.3, 3.25);
	\draw[red, dashed] (3.25, 5.3425) -- (3.15, 3.25);
	
	
	%alsó
	\draw[red, ultra thick, mid arrow=red] (3.3, 3.25) -- (10.8, 3.25);
	\draw[red, dashed] (3.15, 3.25) -- (5, 3.25);
	
	\node[anchor=mid] at (4, 3.7) {$4.$};
	\draw(4, 3.7) circle(0.3);
	
	\node[anchor=mid] at (10.2, 3.7) {$1.$};
	\draw(10.2, 3.7) circle(0.3);
	
	
	%jobb
	\draw[red, ultra thick, mid arrow=red] (10.8, 3.25) -- (10.8, 8.6);
	\draw[red, ultra thick, mid arrow=red] (10.8,8.6) -- (9.9, 5.3425);	
	%feliratok	
	\node[anchor=mid] at (-0.55, 12.8) {$T \left(\si{\degreeCelsius}\right)$};
	
	\node[anchor=mid] at (16, -0.55) {$s \left(\si{\kilo\joule\per\kilogram\kelvin}\right)$};
	
	%jelölések
	\draw[black,dashed] (3.15,0) -- (3.15,3.25) -- (3.3, 3.25) -- (3.3, 0) -- (0,0);

	\draw[black, dashed] (3.3, 3.25) -- (10.8, 3.25) -- (10.8, 0) -- (3.3,0) -- (3.3,3.25);
		
%h_4
\fill[gray, opacity=0.25] (3.15,0) -- (3.15,3.25) -- (3.3, 3.25) -- (3.3, 0) -- (0,0);
	
	\node[anchor=mid] at (7.05, 1.625) {$q_H$};
	\node[anchor=mid] at (11, 11) {$p_F$};
	\node[anchor=mid] at (12.2, 10) {$p_A$};

	\end{tikzpicture}
	\caption{$T-s$ diagram}
\end{figure}

\pagebreak

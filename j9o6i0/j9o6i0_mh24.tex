\section*{MH 24. feladat: Rankine-Clausius körfolyamat módosítása}
\addcontentsline{toc}{section}{MH 24. feladat}

%feladatleírás
Elméleti Rankine-Clausius körfolyamat alsó folyáshatára $p_k = \SI{0,05}{\bar}$. Mennyivel javul a körfolyamat termikus hatásfoka, ha a felső nyomáshatárt $p_1 = \SI{40}{\bar}$-ról $p^x = \SI{180}{\bar}$-ra emeljük? A túlhevített hőmérsékletet mindkét esetben azonos $T_{TH} = \SI{500}{\celsius}$. A tápszivattyú munkáját figyelmen kívül hagyjuk. Rajzolja meg a körfolyamatot T-s diagramban.

\subsubsection{Ismert adatok:}

\begin{equation*}
	p_k = \SI{0,05}{\bar},
	\quad
	p_1 = \SI{40}{\bar}	
\end{equation*}

\begin{equation*}
	i' = \SI{136}{\kilo\joule\per\kilogram},
	\quad
	i'' = \SI{2560}{\kilo\joule\per\kilogram},
	\quad
	i_1 = \SI{3446}{\kilo\joule\per\kilogram}
\end{equation*}

\begin{equation*}
	s' = \SI{0,46}{\kilo\joule\per\kilogram\kelvin},
	\quad
	s'' = \SI{8,4}{\kilo\joule\per\kilogram\kelvin},
	\quad
	s = \SI{7,1}{\kilo\joule\per\kilogram\kelvin}
\end{equation*}
\begin{equation*}
	p^x = \SI{180}{\bar},
	\quad
	i^x = \SI{3272}{\kilo\joule\per\kilogram},
	\quad
	s^x = \SI{6,2}{\kilo\joule\per\kilogram\kelvin}
\end{equation*}


\subsubsection*{a) Termikus hatásfok számítása $p_1=\SI{40}{\bar}$ esetén.}

A végállapot állapotjelzőinek számításához szükségünk lesz az x fajlagos gőztartalomra. Az állapotváltozás adiabatikus jellegű, így (ha reverzibilisnek tekintjük az állapotváltozást) $s_1 = s_2 = s $.

\begin{equation}
	s = \left(1 - x\right) s' + x s''
	\quad 
	\Rightarrow
	\quad 
	x =	\dfrac{s - s'}{s'' - s'} = \SI{0,836}{}
\end{equation}


\noindent Így ki tudjuk számolni a hőtartalmat a végállapotban:
\begin{equation}
	h_2 = \left(1 - x\right) h' + x h'' = \SI{2163,12}{\kilo\joule\per\kilogram}
\end{equation}

\noindent A végállapotra kapott hőtartalom és a kezdeti hőtartalom segítségével meghatározhatjuk a tömegfajlagos technika munkát.
\begin{equation}
	w_t = h_1 - h_2 = \SI{1282,88}{\kilo\joule\per\kilogram}
\end{equation}

\noindent Ezután meghatározzuk a tömegfajlagos hőt:
\begin{equation}
	q_{BE} = h_1 - h' = \SI{3310}{\kilo\joule\per\kilogram}
\end{equation}

\pagebreak

\noindent A tömegfajlagos technikai munka és tömegfajlagos hő hányadosa a termikus hatásfok:
\begin{equation}
	\eta_{T} = \dfrac{w_t}{q_{BE}} = \SI{0,3876}{}
		\quad 
	\Rightarrow
	\quad 
	\SI{38,76}{\percent}
\end{equation}

\subsubsection*{b) Termikus hatásfok számítása $p_1=\SI{180}{\bar}$ esetén.}

A végállapot állapotjelzőinek számításához szükségünk lesz az $x^x$ fajlagos gőztartalomra. Az állapotváltozás adiabatikus jellegű, így (ha reverzibilisnek tekintjük az állapotváltozást) $s_1 = s_2 = s^x $.

\begin{equation}
s^x = \left(1 - x^x\right) s' + x^x s''
\quad 
\Rightarrow
\quad 
x^x =	\dfrac{s^x - s'}{s'' - s'} = \SI{0,723}{}
\end{equation}


\noindent Így ki tudjuk számolni a hőtartalmat a végállapotban:
\begin{equation}
h_2^x = \left(1 - x^x\right) h' + x h'' = \SI{1888,31}{\kilo\joule\per\kilogram}
\end{equation}

\noindent A végállapotra kapott hőtartalom és a kezdeti hőtartalom segítségével meghatározhatjuk a tömegfajlagos technika munkát.
\begin{equation}
w_t^x = h^x- h_2^x = \SI{1383,69}{\kilo\joule\per\kilogram}
\end{equation}

\noindent Ezután meghatározzuk a tömegfajlagos hőt:
\begin{equation}
q_{BE}^x = h^x - h' = \SI{3310}{\kilo\joule\per\kilogram}
\end{equation}

\noindent A tömegfajlagos technikai munka és tömegfajlagos hő hányadosa a termikus hatásfok:
\begin{equation}
\eta_{T}^x = \dfrac{w_t^x}{q_{BE}^x} = \SI{0,4412}{}
\quad 
\Rightarrow
\quad 
\SI{44,12}{\percent}
\end{equation}

\noindent Tehát $\SI{5,36}{\percent}$-kal javul a körfolyamat hatásfoka, ha a felső nyomáshatárt $p_1 = \SI{40}{\bar}$-ról $p^x = \SI{180}{\bar}$-ra emeljük.

\begin{figure}[h]
	\centering
	\begin{tikzpicture}
	\draw[->] ({0}, {0}) -- ({14}, {0}) node[anchor=base east, shift={(0.4,-0.7)}]{$s$ [\si{\kilo\joule\per\kilogram\kelvin}]};
	\draw[->] ({0}, {0}) -- ({0}, {9}) node[anchor=north east]{$T$ $[\si{\celsius}]$};
	
	\begin{axis}[
	axis lines = middle,
	axis line style = {draw = none},
	xlabel={},
	ylabel={},
	ytick style={draw=none},
	xtick style={draw=none},
	width=15cm, height=10cm, xmin=0, ymin=0, ymax=550,
	]
	
	\addplot [] table {./j9o6i0/ts.txt}; %T-s
	\addplot [thick] table {./j9o6i0/p40.txt};
	\addplot [thick] table {./j9o6i0/izochor_0865.txt};
	\addplot [thick, red] table {./j9o6i0/p180.txt};
	\addplot [thick, red] table {./j9o6i0/izochor_0166.txt};
	
	\draw[fill] (axis cs:7.1, 500) circle [radius = 1mm];
	\draw[fill] (axis cs:6.286, 219.75) circle [radius = 1mm];
	\draw[fill] (axis cs:2.552, 219.75) circle [radius = 1mm];
	\draw[fill] (axis cs:0.46, 32.88) circle [radius = 1mm];
	\draw[fill] (axis cs:7.1, 32.88) circle [radius = 1mm];
	
	\draw[fill, red] (axis cs:6.2, 500) circle [radius = 1mm];
	\draw[fill, red] (axis cs:5.574, 315.88) circle [radius = 1mm];
	\draw[fill, red] (axis cs:3.416, 315.88) circle [radius = 1mm];
 	\draw[fill, red] (axis cs:6.2, 32.88) circle [radius = 1mm];
	
	\draw[thick, red] (axis cs:0.46, 32.88) -- (axis cs:6.2, 32.88);
	\draw[thick, red] (axis cs:3.416, 315.88) -- (axis cs:5.574, 315.88);
	\draw[thick, red] (axis cs:6.2, 32.88) -- (axis cs:6.2, 500);
	
	\draw[thick] (axis cs:0.46, 32.88) -- (axis cs:7.1, 32.88);
	\draw[thick] (axis cs:2.552, 219.75) -- (axis cs:6.286, 219.75);
	\draw[thick] (axis cs:7.1, 32.88) -- (axis cs:7.1, 500);
	
	
	\end{axis}
	
	\end{tikzpicture}
\end{figure}

\pagebreak
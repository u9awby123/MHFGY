
\begin{tabular}{ | p{2cm} | p{14cm} | } 
	\hline
	Név & Simon Ivett Alexandra \\ 
	\hline
	Szak & gépészmérnöki alapszak \\ 
	\hline
	Félév & 2019/2020 II. (tavaszi) félév \\ 
	\hline
\end{tabular}


\section*{MH 24. feladat: Rankine-Clausius körfolyamat módosítása}
\addcontentsline{toc}{section}{MH 24. feladat}

%feladatleírás
Elméleti Rankine-Clausius körfolyamat alsó nyomáshatára $p_k = \SI{0,05}{\bar}$. Mennyivel javul a körfolyamat termikus hatásfoka, ha a felső nyomáshatárt $p_1 = \SI{40}{\bar}$-ról $p^* = \SI{180}{\bar}$-ra emeljük? A túlhevített hőmérsékletet mindkét esetben azonos $T_{TH} = \SI{500}{\celsius}$. A tápszivattyú munkáját figyelmen kívül hagyjuk. Rajzolja meg a körfolyamatot $T-s$ diagramban.

\subsubsection{Ismert adatok:}

\begin{equation*}
	p_k = \SI{0,05}{\bar},
	\quad
	p_1 = \SI{40}{\bar}	
\end{equation*}

\begin{equation*}
	h' = \SI{136}{\kilo\joule\per\kilogram},
	\quad
	h'' = \SI{2560}{\kilo\joule\per\kilogram},
	\quad
	h_1 = \SI{3446}{\kilo\joule\per\kilogram}
\end{equation*}

\begin{equation*}
	s' = \SI{0,46}{\kilo\joule\per\kilogram\kelvin},
	\quad
	s'' = \SI{8,4}{\kilo\joule\per\kilogram\kelvin},
	\quad
	s = \SI{7,1}{\kilo\joule\per\kilogram\kelvin}
\end{equation*}
\begin{equation*}
	p^* = \SI{180}{\bar},
	\quad
	i^* = \SI{3272}{\kilo\joule\per\kilogram},
	\quad
	s^* = \SI{6,2}{\kilo\joule\per\kilogram\kelvin}
\end{equation*}


\subsubsection*{a) Termikus hatásfok számítása $p_1=\SI{40}{\bar}$ esetén.}

A végállapot állapotjelzőinek számításához szükségünk lesz az $x$ fajlagos gőztartalomra. Az állapotváltozás adiabatikus jellegű, így (ha reverzibilisnek tekintjük az állapotváltozást) $s_1 = s_2 = s $.

\begin{equation}
	s = \left(1 - x\right) s' + x s''
	\quad 
	\Rightarrow
	\quad 
	x =	\dfrac{s - s'}{s'' - s'} = \SI{0,836}{}
\end{equation}


\noindent Így ki tudjuk számolni a hőtartalmat a végállapotban:
\begin{equation}
	h_2 = \left(1 - x\right) h' + x h'' = \SI{2163,12}{\kilo\joule\per\kilogram}
\end{equation}

\noindent A végállapotra kapott hőtartalom és a kezdeti hőtartalom segítségével meghatározhatjuk a tömegfajlagos technika munkát.
\begin{equation}
	w_t = h_1 - h_2 = \SI{1282,88}{\kilo\joule\per\kilogram}
\end{equation}

\noindent Ezután meghatározzuk a tömegfajlagos hőt:
\begin{equation}
	q_{BE} = h_1 - h' = \SI{3310}{\kilo\joule\per\kilogram}
\end{equation}

\pagebreak


\noindent A tömegfajlagos technikai munka és tömegfajlagos hő hányadosa a termikus hatásfok:
\begin{equation}
	\eta_{T} = \dfrac{w_t}{q_{BE}} = \SI{0,3876}{}
		\quad 
	\Rightarrow
	\quad 
	\SI{38,76}{\percent}
\end{equation}

\subsubsection*{b) Termikus hatásfok számítása $p_1=\SI{180}{\bar}$ esetén.}

A végállapot állapotjelzőinek számításához szükségünk lesz az $x^*$ fajlagos gőztartalomra. Az állapotváltozás adiabatikus jellegű, így (ha reverzibilisnek tekintjük az állapotváltozást) $s_1 = s_2 = s^* $.

\begin{equation}
s^* = \left(1 - x^*\right) s' + x^* s''
\quad 
\Rightarrow
\quad 
x^* =	\dfrac{s^* - s'}{s'' - s'} = \SI{0,723}{}
\end{equation}


\noindent Így ki tudjuk számolni a hőtartalmat a végállapotban:
\begin{equation}
h_2^* = \left(1 - x^*\right) h' + x h'' = \SI{1888,31}{\kilo\joule\per\kilogram}
\end{equation}

\noindent A végállapotra kapott hőtartalom és a kezdeti hőtartalom segítségével meghatározhatjuk a tömegfajlagos technika munkát.
\begin{equation}
w_t^* = h^*- h_2^* = \SI{1383,69}{\kilo\joule\per\kilogram}
\end{equation}

\noindent Ezután meghatározzuk a tömegfajlagos hőt:
\begin{equation}
q_{BE}^* = h^* - h' = \SI{3310}{\kilo\joule\per\kilogram}
\end{equation}

\noindent A tömegfajlagos technikai munka és tömegfajlagos hő hányadosaként megkapjuk a termikus hatásfokot:
\begin{equation}
\eta_{T}^* = \dfrac{w_t^*}{q_{BE}^*} = \SI{0,4412}{}
\quad 
\Rightarrow
\quad 
\SI{44,12}{\percent}
\end{equation}

\noindent Tehát $\SI{5,36}{\percent}$-kal javul a körfolyamat hatásfoka, ha a felső nyomáshatárt $p_1 = \SI{40}{\bar}$-ról $p^* = \SI{180}{\bar}$-ra emeljük.

\vspace{0.2cm}
\noindent Ábrázoljuk a körfolyamatot $T-s$ diagramon.

%T-s diagram
\begin{figure}[h]
	\centering
	\begin{tikzpicture}

	\begin{axis}[
	axis lines = middle,
	axis line style = {->},
	xlabel={$s \left(\si{\kilo\joule\per\kilogram\kelvin}\right)$},
	ylabel={$T \left(\si{\celsius}\right)$},
	xlabel style={
		at=(current axis.right of origin), 
		anchor=north east
	},
	ylabel style={
		at=(current axis.above origin), 
		anchor=north east
	},
	ytick={32.88, 219.75, 315.88, 500},
	yticklabels={{$32,88$}, {$219,75$}, {$315,88$}, {$500$}},
	xtick={0.46, 2.552, 3.416, 5.574, 6.1, 6.5, 7.1},
	xticklabels={{$0,46$}, {$2,552$}, {$3,416$}, {$5,574$}, {$6,2$}, {$6,286$}, {$7,1$}},
	x tick label style={rotate=90,anchor=east},
	ytick style={draw=none},
	xtick style={draw=none},
	width=15cm, height=10cm, xmin=0, xmax=10, ymin=0, ymax=600,
	]

	
	\addplot [] table {./j9o6i0/ts.txt}; %T-s
	\addplot [ultra thick, red] table {./j9o6i0/p180.txt};
	\addplot [ultra thick, red] table {./j9o6i0/p180_0166.txt};
	\addplot [ultra thick] table {./j9o6i0/p40_0865.txt};
	\addplot [ultra thick, dashed] table {./j9o6i0/p40.txt};

%körfolyamat ábrázolása	
	\draw[fill] (axis cs:7.1, 500) circle [radius = 1mm];
	\draw[fill] (axis cs:6.286, 219.75) circle [radius = 1mm];
	\draw[fill] (axis cs:2.552, 219.75) circle [radius = 1mm];
	\draw[fill] (axis cs:0.46, 32.88) circle [radius = 1mm];
	\draw[fill] (axis cs:7.1, 32.88) circle [radius = 1mm];
	
	\draw[fill, red] (axis cs:6.2, 500) circle [radius = 1mm];
	\draw[fill, red] (axis cs:5.574, 315.88) circle [radius = 1mm];
	\draw[fill, red] (axis cs:3.416, 315.88) circle [radius = 1mm];
 	\draw[fill, red] (axis cs:6.2, 32.88) circle [radius = 1mm];
	
	\draw[ultra thick] (axis cs:0.46, 32.88) -- (axis cs:7.1, 32.88);
	\draw[ultra thick] (axis cs:2.552, 219.75) -- (axis cs:6.286, 219.75);
	\draw[ultra thick] (axis cs:7.1, 32.88) -- (axis cs:7.1, 500);
	
	\draw[ultra thick, dashed, red] (axis cs:0.46, 32.88) -- (axis cs:6.2, 32.88);
	\draw[ultra thick, red] (axis cs:3.416, 315.88) -- (axis cs:5.574, 315.88);
	\draw[ultra thick, red] (axis cs:6.2, 32.88) -- (axis cs:6.2, 500);
	
	
%a körfolyamat iránya
	\draw[mid arrow=red] (axis cs:6.2, 500) -- (axis cs:6.2, 32.88);
	\draw[mid arrow=red] (axis cs:6.2, 32.88) -- (axis cs:0.46, 32.88);
	\draw[mid arrow=red] (axis cs:3.416, 315.88) -- (axis cs:5.574, 315.88);
	
	\draw[mid arrow=black] (axis cs:7.1, 500) -- (axis cs:7.1, 32.88);
	\draw[mid arrow=black] (axis cs:7.1, 32.88) -- (axis cs:0.46, 32.88);
	\draw[mid arrow=black] (axis cs:2.552, 219.75) -- (axis cs:6.286, 219.75);
	
	
%értékek jelölése
	\draw[thick, dashed] (axis cs:0, 32.88) -- (axis cs:0.46, 32.888);
	\draw[thick, dashed] (axis cs:0, 219.75) -- (axis cs:2.552, 219.75);
	\draw[thick, dashed] (axis cs:0, 500) -- (axis cs:7.1, 500);
	\draw[thick, dashed, red] (axis cs:0, 315.88) -- (axis cs:3.416, 315.8);
	
	\draw[thick, dashed] (axis cs: 0.46, 0) -- (axis cs:0.46, 32.888);
	\draw[thick, dashed] (axis cs: 2.552, 0) -- (axis cs:2.552, 219.75);
	\draw[thick, dashed] (axis cs: 6.286, 0) -- (axis cs:6.286, 219.75);
	\draw[thick, dashed] (axis cs: 7.1, 0) -- (axis cs:7.1, 32.888);	
	\draw[thick, dashed, red] (axis cs:3.416, 0) -- (axis cs:3.416, 315.8);
	\draw[thick, dashed, red] (axis cs:5.574, 0) -- (axis cs:5.574, 315.8);	
	\draw[thick, dashed, red] (axis cs:6.2, 0) -- (axis cs:6.2, 32.88);	

	
%nevezetes pontok
	\node[thick, circle, draw, inner sep = 1mm] at (axis cs:7.5, 500) {1};
	\node[thick, circle, draw, inner sep = 1mm] at (axis cs:6.8, 70) {2};
	\node[thick, circle, draw, inner sep = 1mm] at (axis cs:1.3, 60) {3};
	\node[thick, circle, draw, inner sep = 1mm] at (axis cs:2.3, 250) {4};
	\node[thick, circle, draw, inner sep = 1mm] at (axis cs:6.7, 220) {5};
	\node[thick, red, circle, draw, inner sep = 1mm] at (axis cs:6, 530) {$1^*$};
	\node[thick, red, circle, draw, inner sep = 1mm] at (axis cs:5.9, 70) {$2^*$};
	\node[thick, red, circle, draw, inner sep = 1mm] at (axis cs:0.5, 80) {$3^*$};
	\node[thick, red, circle, draw, inner sep = 1mm] at (axis cs:3.1, 350) {$4^*$};
	\node[thick, red, circle, draw, inner sep = 1mm] at (axis cs:5.45, 380) {$5^*$};
	
	
	\end{axis}
	
	\end{tikzpicture}
\end{figure}

\pagebreak
\section*{K7/7. feladat: Cső a csőben hőcserélő} 
\addcontentsline{toc}{section}{K7/7. feladat}

%feladatleírás
Egy "cső a csőben" típusú hőcserélőnél az alábbi adatokat mértük:
\vspace{5mm}

%alapadatok

$t_{1k}=\SI{43,0}{\celsius}$
\quad
$t_{1v}=\SI{28,6}{\celsius}$

$t_{2k}=\SI{12,5}{\celsius}$
\quad
$t_{2v}=\SI{19,6}{\celsius}$

${\dot{V}}_1=\SI{3,923}{\liter\per\min} = \SI{0.23538}{\meter\cubed\per\hour}$

${\dot{V}}_2=\SI{7,95}{\liter\per\min} = \SI{0.477}{\meter\cubed\per\hour} $

Mindkét közeg víz.

$c_{viz}=\SI{4,18}{\kilo\joule\per\kilo\gram\kelvin}$

$A=\SI{0.35}{\meter\squared}$ 	<- a hőcserélő felülete

$\kappa_C=\SI{760}{\watt\per\meter\squared\kelvin}$	<- a hőcserélő úgynevezett tiszta hőátviteli tényezője

$\rho=\SI{998}{\kilo\gram\per\meter\cubed}$


%Rajz
\begin{figure}[h]
	\centering
	\begin{tikzpicture}[every path/.style={line cap=rect}][scale=5]
	%cső
	\draw [ultra thick] (0,0) -- (8.5,0);
	\draw [ultra thick] (8.5,0) -- (8.5,0.5);
	\draw [ultra thick] (8,0.5) -- (10,0.5);
	\draw [ultra thick] (9.5,0.5) -- (9.5,-1);
	\draw [ultra thick] (0,0) -- (0,-6);
	\draw [ultra thick] (0,-6) -- (8.5,-6);
	\draw [ultra thick] (8.5,-6) -- (8.5,-6.5);
	\draw [ultra thick] (8,-6.5) -- (10,-6.5);
	\draw [ultra thick] (9.5,-1) -- (1,-1);
	\draw [ultra thick] (1,-1) -- (1,-5);
	\draw [ultra thick] (1,-5) -- (9.5,-5);
	\draw [ultra thick] (9.5,-5) -- (9.5,-6.5);
	%áramvonal
	\draw [ultra thick][->] (9,0.5) -- (9,1.5) node[anchor=east]{$t_{1v}$};
	\draw [ultra thick][<-] (9,-6.5) -- (9,-7.5) node[anchor=west]{$t_{1k}$} node[anchor=east]{$\dot{V}_1$};
	\draw [ultra thick][<-] (9.5,-0.5) -- (10.5,-0.5) node[anchor=north]{$\dot{V}_2$} node[anchor=south]{$t_{2k}$};
	\draw [ultra thick] (9.5,-0.5) -- (0.5,-0.5);
	\draw [ultra thick][mid arrow] (0.5,-0.5) -- (0.5,-5.5);
	\draw [ultra thick] (9.5,-5.5) -- (0.5,-5.5);
	\draw [ultra thick][->] (9.5,-5.5) -- (10.5,-5.5) node[anchor=north]{$t_{2v}$};
	\end{tikzpicture}
	\caption{Cső a csőben hőcserélő.}
\end{figure}

%kérdések
a/ Rajzolja le a hőmérséklet-hely függvényt!

b/ Állapítsa meg a hőcserélő elpiszkolódásának mértékét!

Ha a lerakódás zömmel mészkőben dús vízkő ($\lambda_k =\SI{2.15}{\watt\per\meter\kelvin} $) mekkora a rétegvastagság? ($\delta_k =?$)

\pagebreak

%kidolgozás
\subsection*{Megoldás:}

\subsection*{a/ feladatrész:}

Írjuk fel az átszármaztatott hőáramot!

\begin{equation}
	{\dot{Q}}_{at}={\dot{V}}_1\cdot\rho\cdot c_{viz}\cdot (t_{1k}-t_{1v})
\end{equation}

illetve:

\begin{equation}
	{\dot{Q}}_{at} = \kappa_D\cdot A\cdot LMTD
\end{equation}

A (7.24)-es egyenletből határozzuk meg ${\dot{Q}}_{at}$ értékét!

\begin{equation}
	{\dot{Q}}_{at} = \SI{0.23538}{\meter\cubed\per\hour} \cdot \SI{998}{\kilo\gram\per\meter\cubed} \cdot \SI{4,18}{\kilo\joule\per\kilo\gram\kelvin} \cdot (\SI{43,0}{\celsius} - \SI{28,6}{\celsius}) 
\end{equation}

(Mivel a zárójelben hőmérséklet különbség szerepel a °C-t nem szükséges átváltani K-re.)

\begin{equation}
	{\dot{Q}}_{at} = \SI{14139.66}{\kilo\joule\per\hour} = \SI {3.93}{\kilo\watt}
\end{equation}

A (7.25)-ös egyenletből fejezzük ki $\kappa_D$ -t!

\begin{equation}
	\kappa_D = \frac{{\dot{Q}}_{at}}{A \cdot LMTD}
\end{equation}

LMTD (logaritmikus hőmérséklet különbség) meghatározása:

\begin{equation}
	LMTD = \dfrac{(t_{1k}-t_{2v}) \cdot (t_{1v}-t_{2k})}{ln(\dfrac{(t_{1k}-t_{2v})}{(t_{1v}-t_{2k})})} 
\end{equation}

\begin{equation}
	LMTD = \dfrac{(\SI{43,0}{\celsius}-\SI{19,6}{\celsius}) \cdot (\SI{28,6}{\celsius}-\SI{12,5}{\celsius})}{ln(\dfrac{(\SI{43,0}{\celsius}-\SI{19,6}{\celsius})}{(\SI{28,6}{\celsius}-\SI{12,5}{\celsius})})} 
\end{equation}

\begin{equation}
	LMTD = \SI {19.52}{\celsius} = \SI {292.52}{\kelvin}
\end{equation}


$\kappa_D$ kiszámítása:

\begin{equation}
	\kappa_D = \dfrac {\SI {3.93}{\kilo\watt}}{\SI{0.35}{\meter\squared}\cdot \SI {19.52}{\celsius}} = \SI {575.23}{\watt\per\meter\squared\kelvin}
\end{equation}

\pagebreak
%A hőmérséklet-hely függvény:

\begin{figure}[h]
	\centering
	\begin{tikzpicture}
	\pgfmathsetmacro{\L}{8}
	\pgfmathsetmacro{\AÖ}{10}
	
	\pgfmathsetmacro{\kelvin}{6}
	\pgfmathsetmacro{\TAK}{43/\kelvin}
	\pgfmathsetmacro{\TAV}{28.6/\kelvin}
	\pgfmathsetmacro{\TBK}{12.5/\kelvin}
	\pgfmathsetmacro{\TBV}{19.6/\kelvin}
	
	% Tengelyek
	\draw[->] (0,-1) -- (0,\L+1) node[anchor=north east]{$T$};
	\draw[->] (-1.25,0) -- (\AÖ+1,0) node[anchor=base east, shift={(0,-0.5)}]{$A$};
	
	% Az összes felület
	\draw[gray, dashed] (\AÖ,0) -- (\AÖ,\L+0.5);
	\draw (\AÖ,-0.1) -- (\AÖ,0.1);
	\node[anchor=base, shift={(0,-0.5)}] at (\AÖ,0) {$A_{\ddot{O}}$};
	
	% A két T(A)
	%\draw[red, ultra thick] (0,\TAK) -- (\AÖ,\TAV);
	%\draw[mid arrow=blue, blue, ultra thick] (\AÖ,\TBK) -- (0,\TBV);
	
	\draw[ultra thick, color=red, mid arrow=red, domain=0:\AÖ, smooth, variable=\A] plot (\A, {\TAK - (\TAK-\TBV)/(270*0.00187555)*(1 - exp(-0.00187555*760*\A*0.35/\AÖ) )});
	\draw[ultra thick, color=blue, mid arrow=blue, domain=\AÖ:0, smooth, variable=\A] plot (\A, {\TBV - (\TAK-\TBV)/(547*0.00187555)*(1 - exp(-0.00187555*760*\A*0.35/\AÖ) )});
	
	% A hőmérséklet értékek
	\draw (-0.1,\TAK) -- (0.1,\TAK);
	\node[anchor=base east] at (0,\TAK) {$t_{1k}$};
	\node[anchor=north east] at (0,\TAK) {$\SI{43}{\celsius}$};
	
	\draw (-0.1,\TBV) -- (0.1,\TBV);
	\node[anchor=base east] at (0,\TBV) {$t_{2v}$};
	
	\draw (-0.1+\AÖ,\TBK) -- (0.1+\AÖ,\TBK);
	\node[anchor=base west] at (\AÖ,\TBK) {$t_{2k}$};
	\node[anchor=north west] at (\AÖ,\TBK) {$\SI{12.5}{\celsius}$};
	
	\draw (-0.1+\AÖ,\TAV) -- (0.1+\AÖ,\TAV);
	\node[anchor=base west] at (\AÖ,\TAV) {$t_{1v}$};
	
	\end{tikzpicture}
	\caption{A hőmérséklet-hely függvények.}
\end{figure}


\subsection*{b/ feladatrész:}

Írjuk fel a termikus ellenállások egyenletét az úgynevezett "tiszta" és "piszkos" -k között!

\begin{equation}
	\frac{1}{\kappa_D}=\frac{1}{\kappa_C}+\frac{\delta_k}{\lambda_k}
\end{equation}

Ebből a lerakódások vastagsága:

\begin{equation}
	\delta_k=(\frac{1}{\kappa_D}-\frac{1}{\kappa_C})\cdot\lambda_k
\end{equation}

\begin{equation}
\delta_k=(\frac{1}{\SI {575.23}{\watt\per\meter\squared\kelvin}}-\frac{1}{\SI{760}{\watt\per\meter\squared\kelvin}})\cdot \SI{2.15}{\watt\per\meter\kelvin}
\end{equation}

\begin{equation}
	\delta_k = \SI {0.0009}{\meter} = \SI {0.9}{\milli\meter}
\end{equation}
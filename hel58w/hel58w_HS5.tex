\begin{tabular}{ | p{2cm} | p{13cm} | } 
	\hline
	Név & Kovács Miklós \\ 
	\hline
	Szak & gépészmérnöki alapszak \\ 
	\hline
	Félév & 2019/2020 II. (tavaszi) félév \\ 
	\hline
\end{tabular}

\section*{HS 5. feladat} 
\addcontentsline{toc}{section}{HS 5. feladat}

%feladatleírás
Egy 160/170 $mm$ átmérőjű gőzvezetéket kétrétegű szigetelés fog körül. A belső réteg vastagsága 30 $mm$, a külsőé 50 $mm$. A cső falának és szigetelő anyagának hővezetési tényezői a következők:
\begin{equation*}
	\lambda_{ac} = \SI{58,15}{\watt\per\meter\kelvin};
	\quad
	\lambda_1 = \SI{0,1745}{\watt\per\meter\kelvin};
	\quad
	\lambda_2 = \SI{0,093}{\watt\per\meter\kelvin}
\end{equation*}
A cső belső felületének hőmérséklete $t_1 =\SI{300}{\celsius}$, a szigetelőréteg külső felületének hőmérséklete $t_4 =\SI{50}{\celsius}$. Határozza meg a csővezeték folyóméterenkénti hőveszteségét, valamint a különböző rétegek határfelületeinek hőmérsékletét!

\subsubsection*{a) A lineáris hőáramsűrűség meghatározása }
Hengeres fal esetén lineáris hőáramsűrűségről beszélhetünk:
\begin{equation*}
 \dot{q}_{lin} = \dfrac {t_1 - t_4}{\dfrac{1}{2 \pi} \left(\dfrac {1}{\lambda_{ac}} \ln ( \dfrac {d_2}{d_1})+\dfrac {1}{\lambda_{1}} \ln ( \dfrac {d_3}{d_2})+\dfrac {1}{\lambda_{2}} \ln ( \dfrac {d_4}{d_3})\right)} 
\end{equation*}

\begin{equation}
\dot{q}_{lin} = \dfrac {300 - 50}{\dfrac{1}{2 \pi} \left(\dfrac {1}{58,15} \ln ( \dfrac {170}{160})+\dfrac {1}{0,1745} \ln ( \dfrac {230}{170})+\dfrac {1}{0,093} \ln ( \dfrac {330}{230})\right)} =\SI{279,74}{\watt\per\meter\squared}
\end{equation}
A csővezeték folyóméterenkénti hővesztesége $\dot{q}_{lin} = \SI{279,74}{\watt\per\meter}$.


\subsubsection*{b) A $t_2$ hőmérséklet meghatározása }
Az előző képletet, a csőfalra felírva és átrendezve:
\begin{equation}
	\dot{q}_{lin} = \dfrac {t_1 - t_2}{\dfrac{1}{2 \pi} \left(\dfrac {1}{\lambda_{ac}} \ln ( \dfrac {d_2}{d_1})\right)} \Rightarrow t_2 = -1 \left(\dot{q}_{lin} \dfrac{1}{2 \pi} \left(\dfrac {1}{\lambda_{ac}} \ln ( \dfrac {d_2}{d_1})\right) - t_1 \right) = 
\end{equation}
behelyettesítve
\begin{equation*}
	= -1 \left(279,74 \dfrac{1}{2 \pi} \left(\dfrac {1}{58,15} \ln ( \dfrac {170}{160})\right) - 300 \right) = \SI{299,954}{\celsius}
\end{equation*}
Tehát a hőmérséklet a csőfal és a belső szigetelőréteg határfelületén $t_2 = \SI{299,954}{\celsius}$.


\subsubsection*{c) A $t_3$ hőmérséklet meghatározása }
A $t_2$ hőmérséklet meghatározásához hasonlóan eljárva, a két szigetelőréteg határfelületére:

\begin{equation}
\dot{q}_{lin} = \dfrac {t_2 - t_3}{\dfrac{1}{2 \pi} \left(\dfrac {1}{\lambda_{1}} \ln ( \dfrac {d_3}{d_2})\right)} \Rightarrow t_3 = -1 \left(\dot{q}_{lin} \dfrac{1}{2 \pi} \left(\dfrac {1}{\lambda_{1}} \ln ( \dfrac {d_3}{d_2})\right) - t_2 \right) = 
\end{equation}
behelyettesítve
\begin{equation*}
	= -1 \left(279,74 \dfrac{1}{2 \pi} \left(\dfrac {1}{0,1745} \ln ( \dfrac {230}{170})\right) - 299,954 \right) = \SI{222,826}{\celsius}
\end{equation*}
A két szigetelőréteg határfelületének hőmérséklete $t_3 =\SI{299,954}{\celsius}$. 

\pagebreak

\subsubsection*{d) A hőmérséklet-hely függvény }
\begin{figure}[!ht]
	\centering
	\begin{tikzpicture}
	% Fiktív értékek a vázlathoz
	\pgfmathsetmacro{\L}{3}
	\pgfmathsetmacro{\LA}{7}
		
	\pgfmathsetmacro{\RA}{0.75}
	\pgfmathsetmacro{\RB}{\RA+0.5}
	\pgfmathsetmacro{\RC}{\RB+(\RB-\RA)*6}
	\pgfmathsetmacro{\RD}{\RC+(\RB-\RA)*10}
	
	\pgfmathsetmacro{\DA}{\RA*2}
	\pgfmathsetmacro{\DB}{\RB*2}
	\pgfmathsetmacro{\DC}{\RC*2}
	\pgfmathsetmacro{\DD}{\RD*2}
	
	\pgfmathsetmacro{\qlin}{279.74}
	\pgfmathsetmacro{\lambdaAC}{58.15}
	\pgfmathsetmacro{\lambdaA}{0.1745}
	\pgfmathsetmacro{\lambdaB}{0.093}
	\pgfmathsetmacro{\kelvin}{50}
	
	% KÖRBEVÁGÁS
	\clip ({-0.345}, {-(\L)-1}) rectangle ({12}, {\L*3});
	
	% Tengelyek
	\draw[->] (0,-0.25) -- (0,\LA+0.5);
	\node[anchor=base] at (-0.2, \LA+0.4) {$T$};
	\draw[->] (-0.25, 0) -- (10, 0) node[anchor=base east, shift={(0,-0.5)}]{$r$};
	
	% A csőfal és a határrétegek
		\fill[gray,opacity=0.5] (\RA,\LA) -- ({(\RA+\RB)/2-0.16}, \LA) -- ({(\RA+\RB)/2-0.08}, \LA-0.12) -- ({(\RA+\RB)/2+0.08}, \LA+0.12) -- ({(\RA+\RB)/2+0.16}, \LA) -- (\RB, \LA) -- (\RB, -0) -- (\RA,-0);
		\draw[] (\RA,\LA) -- ({(\RA+\RB)/2-0.16}, \LA) -- ({(\RA+\RB)/2-0.08}, \LA-0.12) -- ({(\RA+\RB)/2+0.08}, \LA+0.12) -- ({(\RA+\RB)/2+0.16}, \LA) -- (\RB, \LA);
		\draw[ultra thick] (\RA,\LA) -- (\RA, -0);
	
		\fill[red,opacity=0.4] (\RB,\LA) -- ({(\RB+\RC)/2-0.16}, \LA) -- ({(\RB+\RC)/2-0.08}, \LA-0.12) -- ({(\RB+\RC)/2+0.08}, \LA+0.12) -- ({(\RB+\RC)/2+0.16}, \LA) -- (\RC, \LA) -- (\RC, -0) -- (\RB,-0);
		\draw[] (\RB,\LA) -- ({(\RB+\RC)/2-0.16}, \LA) -- ({(\RB+\RC)/2-0.08}, \LA-0.12) -- ({(\RB+\RC)/2+0.08}, \LA+0.12) -- ({(\RB+\RC)/2+0.16}, \LA) -- (\RC, \LA);
		\draw[ultra thick] (\RB,\LA) -- (\RB, -0);
	
		\fill[purple,opacity=0.4] (\RC,\LA) -- ({(\RC+\RD)/2-0.16}, \LA) -- ({(\RC+\RD)/2-0.08}, \LA-0.12) -- ({(\RC+\RD)/2+0.08}, \LA+0.12) -- ({(\RC+\RD)/2+0.16}, \LA) -- (\RD, \LA) -- (\RD, -0) -- (\RC,-0);
		\draw[] (\RC,\LA) -- ({(\RC+\RD)/2-0.16}, \LA) -- ({(\RC+\RD)/2-0.08}, \LA-0.12) -- ({(\RC+\RD)/2+0.08}, \LA+0.12) -- ({(\RC+\RD)/2+0.16}, \LA) -- (\RD, \LA);
		\draw[] (\RD,\LA) -- (\RD, -0);
		\draw[] (\RC,\LA) -- (\RC, -0);
	
	% Az átmérők
	\pgflength[xa={-\RA}, ya={-0.3}, xb={\RA}, yb={-0.3}, alim=0, blim=1, ra=0.6]{$\diameter d_1$};
	\pgflength[xa={-\RB}, ya={-0.6}, xb={\RB}, yb={-0.6}, alim=0, blim=1, ra=1.2]{$\diameter d_2$};
	\pgflength[xa={-\RC}, ya={-0.9}, xb={\RC}, yb={-0.9}, alim=0, blim=1, ra=1.8]{$\diameter d_3$};
	\pgflength[xa={-\RD}, ya={-1.2}, xb={\RD}, yb={-1.2}, alim=0, blim=1, ra=2.4]{$\diameter d_4$};
	\draw[] (\RA, 0) -- (\RA, -0.3);
	\draw[] (\RB, 0) -- (\RB, -0.6);
	\draw[] (\RC, 0) -- (\RC, -0.9);
	\draw[] (\RD, 0) -- (\RD, -1.2);
	
	% A hővezetési tényezők
	\node[anchor=base] at ({(\RA+\RB)/2},{\LA+1.5}) {$\lambda_{ac} =\SI{58.15}{\watt\per\meter\kelvin}$};
		\draw [] ({(\RA+\RB)/2},{\LA-0.5}) -- (\RA-0.9,{\LA+1.25});
		\fill[] ({(\RA+\RB)/2},{\LA-0.5}) circle[radius=0.05];
		
	\node[anchor=base] at (\RC-1,{\LA+0.55}) {$\lambda_1 =\SI{0.1745}{\watt\per\meter\kelvin}$};
		\draw (\RB+0.5,{\LA-0.5}) -- (\RB+0.8,{\LA+0.3});
		\fill[] (\RB+0.5,{\LA-0.5}) circle[radius=0.05];
		
	\node[anchor=base] at ({(\RC+\RD)/2},{\LA+1.5}) {$\lambda_2 =\SI{0.093}{\watt\per\meter\kelvin}$};
		\draw [] (\RC+0.5,{\LA-0.5}) -- (\RC+1.35,{\LA+1.25});
		\fill[] (\RC+0.5,{\LA-0.5}) circle[radius=0.05];
	
	%megnevezesek
	\node[anchor=base] at ({(\RB+\RC)/2},{0.1}) {belső réteg};
	\node[anchor=base] at ({(\RC+\RD)/2},{0.1}) {külső réteg};
	
	%homersekletek
	\node[anchor=base] at (\RD+1.5,6.2) {$T_1 = \SI{300}{\celsius}$};
		\draw [dashed] (-0.1,\L*2 ) -- (\RD+1.5,\L*2);
	\node[anchor=base] at (\RD+1.5,5.5) {$T_2 = \SI{299.95}{\celsius}$};
		\draw [dashed] (-0.1,\L*2) -- (\RD+1.5,\L*2);
	\node[anchor=base] at (\RD+1.5, 4.6) {$T_3 = \SI{222.87}{\celsius}$};
		\draw [dashed] (-0.1,4.456 ) -- ((\RD+1.5, 4.456);
	\node[anchor=base] at (\RD+1.5,1.2) {$T_4 = \SI{50}{\celsius}$};
		\draw [dashed] (-0.1,\L/3 ) -- (\RD+1.5,\L/3);
	
	% A T(r) VALÓS hőmérséklet-hely függvény
	\draw[ultra thick, color=red, domain=\RA:\RB, smooth, variable=\r] plot (\r, {\L*2 - ( \qlin/(2*3.14159*\lambdaAC) * ln(2*\r/\DB))/\kelvin});
	
	\draw[ultra thick, color=red, domain=\RB:\RC, smooth, variable=\r] plot (\r, {\L*2 - ( \qlin/(2*3.14159*\lambdaAC) * ln(\DB/\DA) + \qlin/(2*3.14159*\lambdaA) * ln(2*\r/\DB))/\kelvin});
	
	\draw[ultra thick, color=red, domain=\RC:\RD, smooth, variable=\r] plot (\r, {4.456 - ( \qlin/(2*3.14159*\lambdaB) * ln(2*\r/\DC))/\kelvin});
		
	
			
	\end{tikzpicture}
	\caption{A hőmérséklet-hely függvény méretarányosan ábrázolva.}
\end{figure}

\pagebreak
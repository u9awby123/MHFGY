\begin{tabular}{ | p{2cm} | p{14cm} | } 
	\hline
	Név & Kovács Miklós \\ 
	\hline
	Szak & gépészmérnöki alapszak \\ 
	\hline
	Félév & 2019/2020 II. (tavaszi) félév \\ 
	\hline
\end{tabular}

\section*{HS 32. feladat} 
\addcontentsline{toc}{section}{HS 32. feladat}

%feladatleírás
Az olajhűtőben $\dot{m}_1 = \SI{500}{\kilogram\per\hour}$ mennyiségű olajat $t_{1k} = \SI{120}{\celsius}$ -ról $t_{1v} = \SI{30}{\celsius}$ -ra kell lehűteni. A rendelkezésre álló hűtővíz mennyisége $\dot{m}_2 = \SI{1500}{\kilogram\per\hour}$, hőmérséklete $t_{2k} = \SI{13}{\celsius}$. Meghatározandó a szükséges hőátszármaztató felület egyenáramú ill. ellenáramú kapcsolás esetén! (Veszteséget elhanyagoljuk.)

%további adatok
$c_o = \SI{1,67}{\kilo\joule\per\kilogram\kelvin}$ ...olaj fajhője;
$c_v = \SI{4,18}{\kilo\joule\per\kilogram\kelvin}$ ...víz fajhője;
$\kappa = \SI{75}{\watt\per\meter\squared\kelvin}$ ...hőátszármaztatási tényező

\subsubsection*{a) A vízértékek meghatározása }
A vízértékeket megkapjuk, ha az egyes tömegáramokat megszorozzuk a közegek fajhőjeivel.
\begin{equation}
\dot{w}_1 = \dot{m}_1 c_o = \SI{0,232}{\kilo\watt\per\kelvin}
\end{equation}
\begin{equation}
\dot{w}_2 = \dot{m}_2 c_v = \SI{1,742}{\kilo\watt\per\kelvin}
\end{equation}

\subsubsection*{b) konvektív hőátadás }
A hőáram nagyságát megkaphatjuk, ha az olaj tömegáramát és fajhőjét (tehát az egyes vízérték) az olaj hőmérséklet különbségével összeszorozzuk ($t_{1k} - t_{1v}$).
\begin{equation}
\dot{Q}_{konv.} = \dot{m}_1 c_o (t_{1k} - t_{1v}) = \dot{w}_1  (t_{1k} - t_{1v}) =\SI{20,88}{\kilo\watt} =\SI{20880}{\watt}
\end{equation}

A továbbiakban a hűtővíz felmelegedésének mértékére vagyunk kíváncsiak: 
\begin{equation}
\dot{Q}_{konv.} = \dot{w}_2 \Delta t_2 => \Delta t_2 = \dfrac {\dot{Q}_{konv.}}{\dot{w}_2} =\SI{12}{\celsius}
\end{equation}
így megkaptuk, hogy a hűtővíz $\SI{12}{\celsius}$-kal melegedett, tehát kilépéskor a hőmérséklete:
\begin{equation}
t_{2v} = t_{2k} + \Delta t_2 = 13 + 12 =\SI{25}{\celsius}
\end{equation}

\subsubsection*{c) ellenáramú kapcsolás esete }
Ellenáramú kapcsolás esetén a az olaj és a hűtővíz áramlása egymással ellentétes irányban valósul meg az olajhűtőben. 
\begin{equation}
\Delta t_N =  t_{1k} - t_{2v} = 120 - 25 =\SI{95}{\celsius}
\end{equation}
\begin{equation}
\Delta t_k =  t_{1v} - t_{2k} = 30 - 13 =\SI{17}{\celsius}
\end{equation}
Így megkaptuk a logaritmikus közepes hőmérséklet különbség (LMTD) számításához szükséges  kisebb és nagyobb hőmérséklet különbségeket.
\begin{equation}
\Delta T_{k\ddot{o}z,ln} 
= 
\dfrac{\Delta T_N - \Delta T_K}{\ln\dfrac{\Delta T_N}{\Delta T_K}} 
= 
\dfrac{\SI{95}{\celsius} - \SI{17}{\celsius}}{\ln\dfrac{\SI{95}{\celsius}}{\SI{17}{\celsius}}} 
= 
\SI{45,33}{\celsius}
\end{equation}

Tudjuk, hogy a hőáram egyenlő a hőátszármaztatási tényező, az LMTD és a hőátszármaztató felület nagyságával, az egyenlet átrendezéséből megtudjuk határozni a hőátszármaztatási felület nagyságát:
\begin{equation}
\dot{Q}_{konv.} = \kappa \Delta T_{k\ddot{o}z,ln} A_{ellen} => A_{ellen} = \dfrac{\dot{Q}_{konv.}}{\kappa \Delta T_{k\ddot{o}z,ln}} = \dfrac{\SI{20880}{\watt}}{\SI{75}{\watt\per\meter\squared\kelvin} \SI{45,33}{\celsius}} = \SI{6,14}{\meter\squared}
\end{equation}
Tehát ellenáramú kapcsolás esetében a szükséges hőátszármaztatási felület nagysága $A_{ellen} = \SI{6,14}{\meter\squared}$.

%egyenáramú eset-----------------------------------------------------------------------------
\subsubsection*{d) egyenáramú kapcsolás esete }
Egyenáramú kapcsolás esetén a az olaj és a hűtővíz áramlása egymással megegyező irányban valósul meg az olajhűtőben. 
\begin{equation}
\Delta t_N =  t_{1k} - t_{2k} = 120 - 13 =\SI{107}{\celsius}
\end{equation}
\begin{equation}
\Delta t_k =  t_{1v} - t_{2v} = 30 - 25 =\SI{5}{\celsius}
\end{equation}
Ezen értékek felhasználásával:
\begin{equation}
\Delta T_{k\ddot{o}z,ln} 
= 
\dfrac{\Delta T_N - \Delta T_K}{\ln\dfrac{\Delta T_N}{\Delta T_K}} 
= 
\dfrac{\SI{107}{\celsius} - \SI{5}{\celsius}}{\ln\dfrac{\SI{107}{\celsius}}{\SI{5}{\celsius}}} 
= 
\SI{33,29}{\celsius}
\end{equation}
A hőáram képletébe beírva és átrendezve:
\begin{equation}
\dot{Q}_{konv.} = \kappa \Delta T_{k\ddot{o}z,ln} A_{egyen} => A_{egyen} = \dfrac{\dot{Q}_{konv.}}{\kappa \Delta T_{k\ddot{o}z,ln}} = \dfrac{\SI{20880}{\watt}}{\SI{75}{\watt\per\meter\squared\kelvin} \SI{33,29}{\celsius}} = \SI{8,36}{\meter\squared}
\end{equation}
Tehát egyenáramú kapcsolás esetén a szükséges hőátszármaztatási felület nagysága $A_{egyen} = \SI{8,36}{\meter\squared}$.


\pagebreak
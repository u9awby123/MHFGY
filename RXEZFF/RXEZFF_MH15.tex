\section*{MH 15. Vízgőz fojtása}
\addcontentsline{toc}{section}{MH 15. feladat}

\begin{tabular}{ | p{2cm} | p{14cm} | }
	\hline
	Név & Somodi Mihály András \\
	\hline
	Szak & Mechatronikai mérnöki alapszak\\
	\hline
	Félév & 2019/2020 II. (tavaszi) félév \\ 
	\hline
\end{tabular}

\vspace{0.5cm}

\noindent Az \textbf{MH 14.} számú példában szereplő gőztárolóban lévő közeget $p_{2} = \SI{1}{bar}$, környezeti nyomásra fojtjuk. Mennyi gőz keletkezik az állapotváltozás során és mennyi a fojtás miatt bekövetkező entrópia növekedés? Mennyi munkát kapnánk, ha a fojtás helyett adiabatikus reverzibilis expanziót alkalmaznánk? Ábrázolja \textit{T-s} diagramban a folyamatot!

\subsubsection{Adatok(gőztáblázatból):}
\noindent\textbf{ $p_{1}$ nyomáson:}
\begin{equation*}
	h_{1}' =\SI{990,4}{\dfrac{kJ}{kg}},
	\qquad
	s_{1}' =\SI{2,61}{\dfrac{kJ}{kgK}}
\end{equation*}

\noindent\textbf{ $p_{2}$ nyomáson:}
\begin{equation*}
	h_{2}' = \SI{417,4}{\dfrac{kJ}{kg}},
	\qquad
	s_{2}' = \SI{1,3026}{\dfrac{kJ}{kgK}},
	\qquad
	s_{2}'' = \SI{7,360}{\dfrac{kJ}{kgK}},
	\qquad
	r_{2} = \SI{2258}{\dfrac{kJ}{kg}},
	\qquad
	t_{s} = \SI{99,6}{°C}
\end{equation*}

\noindent\hrulefill

\subsubsection{A fajlagos gőztartalom számítása}
Ennek meghatározásához, először ki kell számolni a $h_{1}$-et: ($h_{1}^{''} = \SI{2803}{\frac{kJ}{kg}}$ gőztáblázatból)
\begin{equation}
	h_{1}
	=
	(1 - x_{1})h_{1}^{'} + x_{1}h_{1}^{''}
	=
	\SI{997,02}{\dfrac{kJ}{kg}}
\end{equation}
A fajlagos gőztartalmat úgy kapjuk meg, hogy figyelembe vesszük, hogy az entalpia a fojtás során nem változik, a két állapot között ezért $h_{1} = h_{2}$. Ehhez először azonban meg kell határoznunk a $h_{2}^{''}$-t, amelyet a fajlagos párolgáshő segítségével ki lehet számolni:
\begin{equation}
	h_{2}^{''} = r_{2} + h_{1}^{'}
	=
	\SI{2675,4}{\dfrac{kJ}{kg}}
\end{equation}
Végül pedig, a 2-es állapothoz tartozó fajlagos gőztartalmat az entalpia képletének átrendezéséből:
\begin{equation}
	h_{2} = \left(1 - x_{2}\right) h_{2}' + x_{2} h_{2}^{''}
	\quad 
	\Rightarrow
	\quad 
	x_{2}
	= 
	\dfrac{h_{2} - h_{2}^{'}}{h_{2}^{''} - h_{2}^{'}} 
	=
	\dfrac{h_{1} - h_{2}^{'}}{h_{2}^{''} - h_{2}^{'}} 
	= 
	0,2567
\end{equation}

\subsubsection{A keletkezett gőz tömegének számítása}
A keletkezett gőz tömegének számításához szükségünk van a kiindulási állapotban lévő tömegre és a 2-es állapot fajlagos gőztartalmára. A fajlagos gőztartalom képletébe illesztve lehet megkapni a keletkezett gőz tömegét:
\begin{equation}
	x_{2} = \dfrac{m_{g}}{m_{telj.}}
	\qquad
	\Rightarrow
	\qquad
	m_{g} = x_{2}m_{telj.}
	=
	\SI{0,2567}{}\cdot\SI{3069,1}{kg}
	=
	\SI{787,8}{kg}
\end{equation}


\subsubsection{Az entrópia növekedés számítása}
Az entrópiaváltozást ki lehet számolni, a két állapot entrópiájának különbségeként, amelyeket a következő módon kapunk:
($s_{1}^{''} = \SI{6,213}{\frac{kJ}{kgK}}$ gőztáblázatból)
\begin{equation}
	s_{1}
	=
	(1-x_{1})s_{1}^{'}+x_{1}s_{1}^{''}
	=
	\SI{2,6232}{\dfrac{kJ}{kgK}}
\end{equation}
\begin{equation}
	s_{2}
	=
	(1-x_{2})s_{2}^{'}+x_{2}s_{2}^{''}
	=
	\SI{2,8575}{\dfrac{kJ}{kgK}}
\end{equation}
Ezekből az entrópia növekedése az állapotváltozás során:
\begin{equation}
	\Delta s = s_{2}-s_{1}\approx \SI{0,2344}{\frac{kJ}{kgK}}
\end{equation}

\subsubsection{A munka számítása adiabatikus reverzibilis expanzió esetén}
A munka kiszámításánál figyelembe kell vennünk, hogy itt már nem fojtással dolgozunk, ezért a feltételek megváltoznak. Adiabatikus tehát $q = 0$ és $s_{1}\approx s_{2}$ (reverzibilis is ezért $s_{1} = s_{2}$). Ennek viszont a következménye, hogy újra kell számolni a fajlagos gőztartalmat, mivel immár az entalpia, az 1-es és a 2-es állapotban nem egyezik meg.
Ebből kifolyólag:
\begin{equation}
	x_{ad} =
	\dfrac{s_{2} - s_{2}^{'}}{s_{2}^{''} - s_{2}^{'}} 
	=
	\dfrac{s_{1} - s_{2}{'}}{s_{2}^{''} - s_{2}^{'}}
	=
	\SI{0,2178}{}
\end{equation}
Innen pedig az expanzió végrehajtása után az entalpia:
\begin{equation}
	h_{2uj} = (1 - x_{ad})h_{2}^{'}+x_{ad}h_{2}^{''}
	=
	\SI{909,1924}{\dfrac{kJ}{kg}}
\end{equation}

\noindent Ebből pedig a munkát:
\begin{equation}
	W_{t} = h_{1} - h_{2uj}
	=
	\SI{87,8276}{\dfrac{kJ}{kg}}
\end{equation}
\begin{equation}
	W_{ad} = m_{g}W_{t} = \SI{69190,58}{kJ} \Rightarrow \SI{69190583}{J}
\end{equation}
\subsubsection{T-s diagram}
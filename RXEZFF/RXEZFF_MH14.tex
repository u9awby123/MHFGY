\section*{MH 14. Gőztároló számítása}
\addcontentsline{toc}{section}{MH 14. feladat}

\begin{tabular}{ | p{2cm} | p{14cm} | }
	\hline
	Név & Somodi Mihály András \\
	\hline
	Szak & Mechatronikai mérnöki alapszak \\
	\hline
	Félév & 2019/2020 II. (tavaszi) félév \\ 
	\hline
\end{tabular}

\vspace{0.5cm}

\noindent Egy gőztároló $\SI{4,5}{\meter\cubed} $ térfogatú dobjában $p_{1} = \SI{28}{\bar}$ telítési nyomáson lévő $V =\SI{3,7}{\meter\cubed}$ víz van. Határozza meg a dobban lévő gőz mennyiségét ($m_{g}$) és fajlagos gőztartalmát (\textit{x})!

\subsubsection{Fajtérfogatok $p_{1}$ nyomáson:}
 \begin{equation*}
	 v^{'} = \SI{0,00121}{\dfrac{\meter\cubed}{\kilogram}},
	 \qquad
	 v^{''} = \SI{0,07141}{\dfrac{\meter\cubed}{\kilogram}}
\end{equation*}

\noindent\hrulefill

\subsubsection{A keletkezett gőz tömegének számítása}
Először meg kell határozni, hogy mennyi helyet foglal el a gőz, a dob teljes térfogatából:
\begin{equation}
	V_{g} = V_{telj.} - V_{v}
	= 
	\SI{4,5}{\meter\cubed} - \SI{3,7}{\meter\cubed} = \SI{0,8}{\meter\cubed}
\end{equation}

\noindent Tudjuk, hogy a keletkezett gőz tömege az azonos állapotra vonatkozó térfogat és a fajtérfogat hányadosa. Ebben az esetben a fajtérfogat szélső értékét kell felhasználni:
\begin{equation}
	m_{g} = \dfrac{V_{g}}{v^{''}}
	=
    \dfrac{\SI{0,8}{\meter\cubed}}{\SI{0,07141}{\dfrac{\meter\cubed}{\kilogram}}}
    = 
	\SI{11,2}{\kilogram}
\end{equation}

\subsubsection{A fajlagos gőztartalom számítása}
A fajlagos gőztartalmat keverékmezőben, meg lehet határozni a két fázis arányával. Általános esetben az egyik fázis tömege és a két fázis együttes tömegének hányadosaként. Ehhez meg kell határoznunk a víz tömegét is, amelyet a 2.2-es egyenlethez hasonlóan meg tudunk tenni:
\begin{equation}
	m_{v} = \dfrac{V_{g}}{v^{'}}
	=
	\dfrac{\SI{3,7}{\meter\cubed}}{\SI{0,00121}{\dfrac{\meter\cubed}{\kilogram}}}
	= 
	\SI{3057,8512}{\kilogram}
\end{equation}
 
Ezek után kiszámolható a fajlagos gőztartalom:
\begin{equation}
	x = \dfrac{m_{g}}{m_{g} + m_{v}}
	=
	\dfrac{\SI{11,2}{\kilogram}}{\SI{11,2}{\kilogram} + \SI{3057,8512}{\kilogram}}
	\approx
	 \SI{0,00365}{}
\end{equation}

\section*{HS34:  Hőcserélő számítása}
\addcontentsline{toc}{section}{}


\begin{tabular}{ | p{2cm} | p{14cm} | } 
	\hline
	Név & Kovács Dániel \\ 
	\hline
	Szak &  Mechatronikai mérnöki alapszak\\
	\hline
	Félév & 2019/2020 II. (tavaszi) félév \\ 
	\hline
\end{tabular}
\vspace{0.5cm}

\noindent Egy $A_{\ddot{O}} = \SI{25}{\meter\squared}$ hőátadó felületű hőcserélőben, amelynek hőátszármaztatási tényezője $\kappa = \SI{175}{\watt\per\meter\squared\kelvin}$, a hőátadó közeg $\dot{w}_m = \SI{2900}{\watt\per\kelvin}$ konvektív vízértékű, a hőfelvevő közeg pedig $\dot{w}_h = \SI{7200}{\watt\per\kelvin}$. \\
A két közeg belépési hőmérséklete $T_{mb} = \SI{185}{\celsius}$ illetve $T_{hb} = \SI{18}{\celsius}$. A hőcsere veszteségmentesnek tekinthető. \\
\\
Meghatározandók: \\
\\
a) egyenáramú kapcsolás esetén a kilépési hőmérsékletek $(T_{mk}$;$ $T_{hk})$ és az átszármaztatott hőáram $(\dot{Q})$ \\
b) ellenáramú kapcsolás esetén ugyanezek \\
\\
\subsubsection*{a) Egyenáramú esetben} \\
A meghatározandó ismeretlenek a $T_{mk}$;$ $T_{hk}$ kilépési hőmérsékletek, emiatt két független egyenletet kell felírnunk. Így az alábbi kétismeretlenes egyenletrendszert tudjuk felírni:\\

\begin{equation*}
	\begin{array}{l}
		-\Delta \dot{Q}_m = \Delta \dot{Q}_h
		\\ \\
		\Delta T\!\left(A_{\ddot{O}}\right) = \Delta T
	\end{array}
	\quad \Rightarrow \quad
	\left.
	\begin{array}{l}
		I.\: -\dot{w}_m \left({T_{mk}} - T_{mb}\right) 
		= \dot{w}_h \left({T_{hk}} - T_{hb}\right) 
		\\ \\
		II.\: \left(T_{mb} - {T_{hb}}\right) \mathrm{e}^{-\kappa \overrightarrow{\overleftarrow{m}} A_{\ddot{O}}} = {T_{mk}} - T_{hk}
	\end{array}
	\right\rbrace
\end{equation}
\\

Az egyenletrendszerbe behelyettesíthetjük az alábbi értékeket: \\
\begin{equation*}
	\varphi = \dfrac{\dot{w}_m}{\dot{w}_h} = 0,4028
	\quad \textrm{és} \quad 
	\eta = \mathrm{e}^{-\kappa \overrightarrow{\overleftarrow{m}} A_{\ddot{O}}} = \mathrm{e}^{-2,1161} = 0,1205
\end{equation}
\\
Ahol a $\varphi$ a konvektív vízértékek hányadosa, az $\eta$ az exponenciális függvény értéke. \\
\\
A behelyettesítés után: \\
\begin{equation*}
	\left.
	\begin{array}{l}
		I.\: \varphi \left( T_{mb} - {T_{mk}} \right) 
		= {T_{hk}} - T_{hb} 
	\\ \\
		II.\: \left(T_{mb} - {T_{hb}}\right) \eta = {T_{mk}} - T_{hk}
	\end{array}
\right\rbrace
\end{equation}
\\
Fejezzük ki ${T_{hk}}$-t az $I.$ egyenletből és helyettesítsük be a $II.$-ba:
\begin{equation*}
	I.\: {T_{hk}} = \varphi T_{mb} + T_{hb} - \varphi {T_{mk}} 
\end{equation}

\begin{equation*}
    II.\: \left(T_{mb} - {T_{hb}}\right) \eta = {T_{mk}} - \varphi T_{mb} - T_{hb} + \varphi {T_{mk}}
\end{equation}

\begin{equation*}
    II.\: T_{mk} = \frac{\left(T_{mb} - {T_{hb}}\right) \eta + \varphi T_{mb} + T_{hb}}{1 + \varphi}
\end{equation}

Ide az adott értékeket behelyettesítve megkapjuk a $T_{mk}$ értéket: \\
\begin{equation*}
    T_{mk} = \frac{\left(\SI{185}{\celsius} - {\SI{18}{\celsius}}\right) 0,1205 + 0,4028 * \SI{185}{\celsius} + {\SI{18}{\celsius}}}{1 + 0,4028} = \SI{80,2976}{\celsius} ≈ \SI{80,3}{\celsius}
\end{equation}

Visszahelyettesítve az $I.$ egyenletbe, megkapjuk $T_{hk}$ értékét: \\
\begin{equation*}
    {T_{hk}} = 0,4028 * \SI{185}{\celsius} + \SI{18}{\celsius} - 0,4028 *  \SI{80,3}{\celsius} = \SI{60,1732}{\celsius} ≈ \SI{60,17}{\celsius}
\end{equation}

Ezután a $\dot{Q}$ = $\dot{w}_h \left({T_{hk}} - T_{hb}\right)$ egyenletet felhasználva, megkapjuk a $\dot{Q}$ értéket: \\
\begin{equation*}
    \dot{Q} = \dot{w}_h \left({T_{hk}} - T_{hb}\right) = \SI{7200}{\watt\per\kelvin} (\SI{333,32}{\kelvin} - \SI{291,15}{\kelvin}) = 3,03647*10^5 W
\end{equation}

Tehát egyenáramú kapcsolás (a)) esetében: \\
\begin{equation*}
    T_{mk} = \SI{80,3}{\celsius};   T_{hk} = \SI{60,17}{\celsius};  \dot{Q} = 3,03647*10^5 W
    \hline
\end{equation}
\pagebreak

\subsubsection*{b) Ellenáramú esetben} \\
A meghatározandó ismeretlenek a $T_{mk}$;$ $T_{hk}$ kilépési hőmérsékletek, emiatt két független egyenletet kell felírnunk. Így az alábbi kétismeretlenes egyenletrendszert tudjuk felírni:\\
\begin{equation*}
	\begin{array}{l}
		-\Delta \dot{Q}_m = \Delta \dot{Q}_h
		\\ \\
		\Delta T\!\left(A_{\ddot{O}}\right) = \Delta T
	\end{array}
	\quad \Rightarrow \quad
	\left.
	\begin{array}{l}
		I.\: -\dot{w}_m \left({T_{mk}} - T_{mb}\right) 
		= \dot{w}_h \left({T_{hk}} - T_{hb}\right) 
		\\ \\
		II.\: \left(T_{mb} - {T_{hk}}\right) \mathrm{e}^{-\kappa \overrightarrow{\overleftarrow{m}} A_{\ddot{O}}} = {T_{mk}} - T_{hb}
	\end{array}
	\right\rbrace
\end{equation}
\\

Az egyenletrendszerbe behelyettesíthetjük az alábbi értékeket: \\
\begin{equation*}
	\varphi = \dfrac{\dot{w}_m}{\dot{w}_h} = 0,4028
	\quad \textrm{és} \quad 
	\eta = \mathrm{e}^{-\kappa \overrightarrow{\overleftarrow{m}} A_{\ddot{O}}} = \mathrm{e}^{-0,9009} = 0,4062
\end{equation}
\\
Ahol a $\varphi$ a konvektív vízértékek hányadosa, az $\eta$ az exponenciális függvény értéke. \\
\\
A behelyettesítés után: \\
\begin{equation*}
	\left.
	\begin{array}{l}
		I.\: \varphi \left( T_{mb} - {T_{mk}} \right) 
		= {T_{hk}} - T_{hb} 
	\\ \\
		II.\: \left(T_{mb} - {T_{hk}}\right) \eta = {T_{mk}} - T_{hb}
	\end{array}
\right\rbrace
\end{equation}
\\
Fejezzük ki ${T_{hk}}$-t az $I.$ egyenletből és helyettesítsük be a $II.$-ba:
\begin{equation*}
	I.\: {T_{hk}} = \varphi T_{mb} + T_{hb} - \varphi {T_{mk}} 
\end{equation}

\begin{equation*}
	II.\: {T_{mk}} + \eta \left(\varphi T_{mb} + T_{hb} - \varphi {T_{mk}}\right) = \eta T_{mb} + T_{hb}
\end{equation}

\begin{equation*}
	II.\: {T_{mk}} = \dfrac{\eta T_{mb} + T_{hb} - \eta \left(\varphi T_{mb} + T_{hb}\right)}{1-\eta \varphi}
\end{equation}

Ide az adott értékeket behelyettesítve megkapjuk a $T_{mk}$ értéket: \\
\begin{equation*}
    {T_{mk}} = \dfrac{0,4062 * \SI{185}{\celsius} + \SI{18}{\celsius} - 0,4062 \left(0,4028 * \SI{185}{\celsius} + \SI{18}{\celsius}\right)}{1-0,4062 * 0,4028} = \SI{66,4363}{\celsius} ≈ \SI{66,44}{\celsius}
\end{equation}

Visszahelyettesítve az $I.$ egyenletbe, megkapjuk $T_{hk}$ értékét: \\
\begin{equation*}
    {T_{hk}} = 0,4028 * \SI{185}{\celsius} + \SI{18}{\celsius} - 0,4028 * \SI{66,44}{\celsius} = \SI{65,756}{\celsius} ≈ \SI{65,76}{\celsius}
\end{equation}

Ezután a $\dot{Q}$ = $\dot{w}_h \left({T_{hk}} - T_{hb}\right)$ egyenletet felhasználva, megkapjuk a $\dot{Q}$ értéket: \\
\begin{equation*}
    \dot{Q} = \dot{w}_h \left({T_{hk}} - T_{hb}\right) = \SI{7200}{\watt\per\kelvin} (\SI{338,91}{\kelvin} - \SI{291,15}{\kelvin}) = 3,43872*10^5 W
\end{equation}

Tehát ellenáramú kapcsolás (b)) esetében: \\
\begin{equation*}
    T_{mk} = \SI{66,44}{\celsius};   T_{hk} = \SI{65,76}{\celsius};  \dot{Q} = 3,43872*10^5 W
    \hline
\end{equation}